\part{Wahlprogramm}
\section{Präambel}
\wahlprogramm{Präambel}\label{praeambel:unglow}
\antrag{Unglow}\konkurrenz{praeambel:trier}\version{17:14, 18. Jun. 2010}
\subsubsection{Absatz 1}
\abstimmung
Im Zuge der Digitalen Revolution aller Lebensbereiche sind trotz aller Lippenbekenntnisse die Würde und die Freiheit des Menschen in bisher ungeahnter Art und Weise gefährdet. Dies geschieht zudem in einem Tempo, das die gesellschaftliche Meinungsbildung und die staatliche Gesetzgebung ebenso überfordert wie den Einzelnen selbst. Gleichzeitig schwinden die Möglichkeiten, diesen Prozess mit demokratisch gewonnenen Regeln auf der Ebene eines einzelnen Staates zu gestalten, dahin.
\subsubsection{Absatz 2}
\abstimmung
Die Globalisierung des Wissens und der Kultur der Menschheit durch Digitalisierung und Vernetzung stellt deren bisherige rechtliche, wirtschaftliche und soziale Rahmenbedingungen ausnahmslos auf den Prüfstand. Nicht zuletzt die falschen Antworten auf diese Herausforderung leisten einer entstehenden totalen und totalitären, globalen Überwachungsgesellschaft Vorschub. Die Angst vor internationalem Terrorismus lässt Sicherheit vor Freiheit als wichtigstes Gut erscheinen – und viele in der Verteidigung der Freiheit fälschlicherweise verstummen.
\subsubsection{Absatz 3}
\abstimmung
Die Globalisierung des Wissens und der Kultur der Menschheit durch Digitalisierung und Vernetzung stellt deren bisherige rechtliche, wirtschaftliche und soziale Rahmenbedingungen ausnahmslos auf den Prüfstand. Nicht zuletzt die falschen Antworten auf diese Herausforderung leisten einer entstehenden totalen und totalitären, globalen Überwachungsgesellschaft Vorschub. Die Angst vor internationalem Terrorismus lässt Sicherheit vor Freiheit als wichtigstes Gut erscheinen – und viele in der Verteidigung der Freiheit fälschlicherweise verstummen.
\subsubsection{Absatz 4}
\abstimmung
Informationelle Selbstbestimmung, freier Zugang zu Wissen und Kultur und die Wahrung der Privatsphäre sind die Grundpfeiler der zukünftigen Informationsgesellschaft. Nur auf ihrer Basis kann eine demokratische, sozial gerechte, freiheitlich selbstbestimmte, globale Ordnung entstehen. Die Piratenpartei versteht sich daher als Teil einer weltweiten Bewegung, die diese Ordnung zum Vorteil aller mitgestalten will.
\subsubsection{Absatz 5}
\abstimmung
Die Piratenpartei will sich auf die im Programm genannten Themen konzentrieren, da wir nur so die Möglichkeit sehen, diese wichtigen Forderungen in Zukunft durchzusetzen. Gleichzeitig glauben wir, dass diese Themen für Bürger aus dem gesamten traditionellen politischen Spektrum unterstützenswert sind, und dass eine Positionierung in diesem Spektrum uns in unserem gemeinsamen Streben nach Wahrung der Privatsphäre und Freiheit für Wissen und Kultur hinderlich sein würde.

\wahlprogramm{Präambel}\label{praeambel:trier}
\antrag{KV Trier/Trier-Saarburg}\konkurrenz{praeambel:unglow}\version{17:14, 18. Jun. 2010}
\subsubsection{Wer sind die Piraten}
\abstimmung
\textit{Freiheitsrechte und die Gestaltung der modernen Informations- und Wissensgesellschaft sind die Kernanliegen der Piratenparteien in ganz Europa und weltweit – und natürlich auch bei uns in Rheinland-Pfalz.}

Seit ihrer Gründung 2006 in Berlin wirkt die Piratenpartei Deutschland gemäß ihrer grundgesetzlichen Pflichtenan der "Willensbildung des Volkes" mit. Während des Wahlkampfs zur Europawahl und Bundestagswahl 2009 erlebte die Piratenpartei einen raschen Mitgliederzuwachs. Bei der Bundestagswahl konnte sie als neue Partei sofort 2\% der Stimmen erreichen. Für die schwedische Schwesterpartei sitzen zwei Abgeordnete im Europaparlament.

Der uralte Traum, alles Wissen und alle Kultur der Menschheit zusammenzutragen, zu speichern und heute und in der Zukunft verfügbar zu machen, ist durch die rasante technische Entwicklung der vergangenen Jahrzehnte in greifbare Nähe gerückt. Wie jede bahnbrechende Neuerung erfasst diese vielfältige Lebensbereiche und führt zu tiefgreifenden Veränderungen. Die Piratenpartei möchte die Chancen dieser Situation nutzen und vormöglichen Gefahren warnen.
Informationelle Selbstbestimmung, freier Zugang zu Wissen und Kultur und die Wahrung der Privatsphäre sind die Grundpfeiler der zukünftigen Informationsgesellschaft. Nur auf dieser Basis kann eine selbstbestimmte, sozial gerechte, freiheitlich-demokratische Grundordnung erhalten bleiben. Die Piratenpartei ist Teil einer weltweiten Bewegung, die diese Ordnung zum Vorteil aller mitgestalten will.

\subsubsection{Unsere Ziele}
\abstimmung
\textit{Grundrechte verteidigen}
Die Piratenpartei setzt sich für einen stärkeren Schutz und eine unbedingte Beachtung der Menschen- und Bürgerrechte ein. Die gesamte Politik muss sich an ihnen orientieren. 

\textit{Informationelle Selbstbestimmung}
Das Recht des Einzelnen, die Nutzung seiner persönlichen Daten zu kontrollieren, muss garantiert werden. Dies gilt dem Staat gegenüber ebenso wie im Wirtschaftsbereich. Wir wollen weder den gläsernen Bürger noch den gläsernen Konsumenten.

\textit{Transparenz}
Alles staatliche Handeln muss transparent und für jeden nachvollziehbar sein. Nach unserer Überzeugung ist dies unabdingbare Voraussetzung für eine moderne Wissensgesellschaft in einer freiheitlichen und demokratischen Ordnung.

\textit{Bildung ermöglichen}
Jeder Mensch hat das Recht auf freien Zugang zu Information und Bildung. Dies ist notwendig, um jedem Menschen unabhängig von seiner sozialen Herkunft ein größtmögliches Maß an gesellschaftlicher Teilhabe zu ermöglichen. Bildung ist eine der wichtigsten Ressourcen der Gesellschaft und der Wirtschaft, da nur durch den Erhalt, die Weitergabe und die Vermehrung von Wissen auf Dauer Fortschritt und gesellschaftlicher Wohlstand gesichert werden können.

\textit{Patente}
Wir lehnen Patente auf Lebewesen und Gene, auf Geschäftsideen und auch auf Software ab, weil sie unzumutbare und unverantwortliche Konsequenzen haben, weil sie die Entwicklung der Wissensgesellschaft behindern, weil sie allgemeine Güter ohne angemessene Gegenleistung und ohne Not privatisieren und weil sie kein Erfindungspotenzial im ursprünglichen Sinne besitzen.

\textit{Open Access}
Aus dem Staatshaushalt wird eine Vielzahl schöpferischer Tätigkeiten finanziert. Da diese Werke von der Allgemeinheit finanziert werden, sollten sie auch der Allgemeinheit kostenlos zur Verfügung stehen.

\textit{Urheberrecht fair gestalten}
Das Urheberrecht muss auf die Anforderungen der sich entwickelnden Informationsgesellschaft angepasst werden und muss die Bedürfnisse von Konsumenten und Produzenten gleichermaßen berücksichtigen, auch in Hinblick darauf, dass die Grenzen zwischen Konsument und Produzent immer mehr verschwi

\subsubsection{Die Piraten in Rheinland-Pfalz}
\abstimmung
Die Piratenpartei Deutschland hat Landesverbände in allen Bundesländern. In Rheinland-Pfalz wurde der Landesverband 2008 in Koblenz gegründet.

Die Forderungen des Piratenprogramms spielen auch auf Landesebene eine große Rolle. Wir setzen uns in unserem Bundesland deshalb für bessere Bildungschancen, mehr Transparenz in der Politik, mehr Mitbestimmung und Wahrung der Grundrechte ein.
Die folgenden Vorschläge für eine zukünftige Politik in Rheinland-Pfalz haben wir auf Basis der piratigen Grundsätze und des Parteiprogramms der Piratenpartei Deutschland erstellt.
