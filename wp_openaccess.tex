\section{Open Access}

\subsection*{Open Access - Zugang zu Wissen schaffen}
\wahlprogramm{Open Access}\label{wp:oa:zugang}
\antrag{Unglow}\version{03:36, 20. Jun. 2010}

\subsubsection{Modul 1}
\abstimmung 	Wissenschaft und Forschung sind zentrale Bausteine für ein zukunftsfähiges Deutschland und Rheinland-Pfalz. Wissenschaftliche Großprojekte und Grundlagenforschung lassen sich oft nur noch staatlich oder sogar im Verbund von mehreren Staaten durchführen.

\subsubsection{Modul 2}
\abstimmung
Mit öffentlichen Geldern geförderte Arbeit muss aber auch der Öffentlichkeit zugute kommen. Noch immer sind viele wissenschaftliche Erkenntnisse nur gegen Bezahlung erhältlich, und das, obwohl dank moderner Technik die Reproduktion der Werke praktisch kostenlos erfolgen kann. Dieses Problem ist auch vielen Wissenschaftlern bewusst, die daher zunehmend dazu übergehen, Arbeiten als Open-Access-Publikationen zu veröffentlichen und damit einen dauerhaften kostenfreien Zugang zu den Ergebnissen ihrer Forschung sicherzustellen. Diesen Trend möchten wir PIRATEN unterstützen, da wir glauben, dass ein leichterer Zugang zu Wissen zu erfolgreicherer Forschung und mehr Innovation führen wird und darüberhinaus sogar weltweit eine wohlstandsfördernde Wirkung entfaltet.

\subsubsection{Modul 3}
\abstimmung
Open Access heißt daher für uns, dass mit öffentlichen Geldern geförderte wissenschaftliche Arbeit und daraus resultierende Publikationen für jeden Menschen kostenfrei zugänglich sein müssen.

\subsubsection{Modul 4}
\abstimmung
Gleichzeitig muss eine Infrastruktur geschaffen werden, die digitale Archivierung und den dauerhaften einfachen Zugang zu Publikationen ermöglicht. Diese Aufgabe wird heute vorrangig von den etablierten Verlagen übernommen. Für Open-Access-Publikationen entwickeln sich entsprechende Mechanismen erst, oft in loser Kooperation von Bibliotheken und Universitäten. Derartige Initiativen wollen die PIRATEN auch finanziell fördern.

\subsubsection{Modul 5}
\abstimmung
In Rheinland-Pfalz soll jede Universität ein eigenes Open-Access-Repository führen in dem alle ihre Fachbereiche unterkommen. Dies vermeidet eine Zersplitterung in zu kleine Einheiten. Die Repositories sollen zwischen den Universitäten vernetzt werden, um die Durchsuchbarkeit und Verfügbarkeit von Wissen zu erhöhen. Es braucht einheitliche APIs (Zugangs- und Nutzungsschnittstellen der Software) auf der Serverseite der Repositories, um die Anschluss- und Verwendungsmöglichkeiten der Repositories zu erhöhen.

\subsubsection{Modul 6}
\abstimmung
Zur allgemeinen Förderung von Open Access sollten bei der Beurteilung von Anträgen auf Forschungsgelder nur noch Publikationen herangezogen werden, die auch öffentlich verfügbar sind.
 
\wahlprogramm{Open Access}
\antrag{KV Trier/Trier-Saarburg}\zusatz{wp:oa:zugang}\version{03:36, 20. Jun. 2010}

\subsubsection{Open Access}
\abstimmung
Die Publikationen aus staatlich finanzierter oder geförderter Forschung und Lehre werden oft in kommerziellen Verlagen publiziert, deren Qualitätssicherung von ebenfalls meist staatlich bezahlten Wissenschaftlern im Peer-Review-Prozess übernommen wird. Die Publikationen werden jedoch nicht einmal den Bibliotheken der Forschungseinrichtungen kostenlos zur Verfügung gestellt. Der Steuerzahler kommt also mehrfach für die Kosten der Publikationen auf. Wir unterstützen die Berliner Erklärung der Open-Access-Bewegung und verlangen die Zugänglichmachung des wissenschaftlichen und kulturellen Erbes der Menschheit nach dem Prinzip des Open Access. Wir sehen es als Aufgabe des Staates an, dieses Prinzip an den von ihm finanzierten und geförderten Einrichtungen durchzusetzen.
 
\paragraph{Anmerkung}: In \ref{wp:oa:zugang}.1 ''aber auch'' dann streichen.

\newpage
\subsection*{Open Access in der öffentlichen Verwaltung}
\wahlprogramm{Open Access in der Verwaltung}
\antrag{Unglow}\version{03:36, 20. Jun. 2010}

\subsubsection{Modul 1}
\abstimmung
Wir fordern die Einbeziehung von Software und anderen digitalen Gütern, die mit öffentlichen Mitteln produziert werden, in das Open-Access-Konzept. Werke, die von oder im Auftrag von staatlichen Stellen erstellt werden, sollen der Öffentlichkeit zur freien Verwendung zur Verfügung gestellt werden. Der Quelltext von Software muss dabei Teil der Veröffentlichung sein.

\subsubsection{Modul 2}
\abstimmung
Dies ist nicht nur zum direkten Nutzen der Öffentlichkeit, sondern die staatlichen Stellen können auch im Gegenzug von Verbesserungen durch die Öffentlichkeit profitieren (Open-Source-Prinzip/Freie Software). Weiterhin wird die Nachhaltigkeit der öffentlich eingesetzten IT-Infrastruktur verbessert und die Abhängigkeit von Softwareanbietern verringert.
 
\subsection*{Digitalisierung von Büchern}
\wahlprogramm{Digitalisierung von Büchern}
\antrag{KV Trier/Trier-Saarburg}\version{03:36, 20. Jun. 2010}

\subsubsection{Digitalisierung von Büchern}
\abstimmung
Wir planen die konsequente Digitalisierung der Werke, die in den Landesbibliotheken vorhanden sind und nicht mehr durch Verwertungsrechte geschützt sind. Die Werke sollen unter einer freien Lizenz veröffentlicht und im Internet der Öffentlichkeit frei zugänglich gemacht werden.
 
\newpage
\subsection*{Dauerhafte Verfügbarkeit öffentlich-rechtlicher Berichterstattung}
\wahlprogramm{Dauerhafte Verfügbarkeit öffentlich-rechtlicher Berichterstattung}\label{wp:oa:dauerhaft}
\antrag{Pirat aus RLP}\version{03:36, 20. Jun. 2010}

\subsubsection{Dauerhafte Verfügbarkeit öffentlich-rechtlicher Berichterstattung}
\abstimmung
Eine der Aufgaben des gebührenfinanzierten öffentlich-rechtlichen Rundfunks besteht in der Versorgung der Bevölkerung mit unabhängiger Berichterstattung. Die dabei erstellten Inhalte sind seit Umsetzung des 12. Rundfunkänderungsstaatsvertrags nur kurze Zeit in den Mediatheken der Rundfunkanstalten abrufbar, obwohl sie auch dauerhaft von öffentlichem Interesse sind, da sie beispielsweise als Quelle für die politische Diskussion dienen. Sie sollten deshalb zeitlich unbegrenzt zur Verfügung gestellt werden.
 
\wahlprogramm{Dauerhafte Verfügbarkeit öffentlich-rechtlicher Berichterstattung}
\antrag{Thomas Heinen}\zusatz{wp:oa:dauerhaft}\version{03:36, 20. Jun. 2010}

\subsubsection{ }
\abstimmung
Wir fordern die sofortige Überarbeitung des Staatsvertrages mit dem Ziel, die Inhalte, die durch die Bürger finanziert werden, langfristig für jeden Menschen frei verfügbar zu machen. Jeder Bürger hat einen Anspruch auf diese Inhalte. Die gesetzlichen Verweildauerregelungen müssen daher genauso wie der Drei-Stufen-Test umgehend auf den Prüfstand.
 
\subsection*{Freie Lizenzen für Inhalte der öffentlich-rechtlichen Rundfunkanstalten}
\wahlprogramm{Freie Lizenzen für Inhalte der öffentlich-rechtlichen Rundfunkanstalten}
\antrag{Pirat aus RLP}\version{03:36, 20. Jun. 2010}

\subsubsection{Freie Lizenzen für Inhalte der öffentlich-rechtlichen Rundfunkanstalten}
\abstimmung
Wenn die Allgemeinheit Fernseh- und Rundfunkprogramme bezahlt, soll sie diese auch uneingeschränkt nutzen können. Überwiegend aus deutschen Rundfunkgebühren finanzierte Inhalte sollen deshalb unter freie Lizenzen gestellt werden.
