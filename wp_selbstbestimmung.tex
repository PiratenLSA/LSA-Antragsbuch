\section{Für ein selbstbestimmtes Leben}

\wahlprogramm{Präambel}
\antrag{Thomas Heinen}\version{03:31, 20. Jun. 2010}

\subsubsection{Grundlagen}
\abstimmung
Jeder Mensch hat das Grundrecht auf freien Zugang zu Information und Bildung. Dies ist in einer freiheitlich-demokratischen Gesellschaft essentiell, um jedem Menschen, unabhängig von seiner sozialen Herkunft, ein größtmögliches Maß an Selbstbestimmung zu ermöglichen. Eine freiheitliche Gesellschaft lebt von der Teilhabe ihrer Bürger. Voraussetzung dafür ist die selbständige Entscheidung über die eigene Lebensgestaltung und über die Art der Teilhabe an der gesellschaftlichen Entwicklung. Die Grundlagen dafür sind Bildung und der Zugang zu Kultur.
 
\wahlprogramm{Selbstbestimmung}
\antrag{Unglow}\version{03:31, 20. Jun. 2010}

\subsubsection{Modul 1}
\abstimmung
Immer mehr Bereiche des täglichen Lebens werden vom Staat durch Vorschriften und Gesetze reguliert und reglementiert. Dadurch werden die Bürger vom Staat bevormundet und daran gehindert, ihren individuellen Lebensstil zu führen und sich frei zu entfalten. Die Piratenpartei setzt sich für die Reforum und ggf. Abschaffung von Gesetzen ein, die den Bürger unverhältnismäßig bevormunden.

\subsubsection{Modul 2}
\abstimmung
In den letzten Jahren werden zunehmend Gesetze diskutiert und teilweise auch verabschiedet, welche die Bürger bevormunden und sie in ihren Gewohnheiten oder ihrer Freizeitgestaltung einschränken. Die Freiräume der Bürger zur individuellen Gestaltung und Entfaltung ihres Lebens werden immer stärker beeinträchtigt. Die Piratenpartei stellt sich ungerechtfertigten Bevormundungen der Bürger durch den Staat entgegen.
 
\wahlprogramm{Für Paintball, Computer- und Videospiele}
\antrag{Unglow}\version{03:31, 20. Jun. 2010}

\subsubsection{Modul 1}
\abstimmung
Es wird diskutiert Paintball und so genannte ''Killerspiele'' zu verbieten, um Amokläufe zu verhindern. Dabei ist weder definiert, was genau unter ''Killerspielen'' zu verstehen ist, noch gibt es gesicherte Erkenntnisse, dass diese Spiele Amokläufe verursachen oder fördern. Computer- und Videospiele sind ein wesentlicher Teil der Jugendkultur. Sie sind künstlerisches und kulturelles Gut, vergleichbar mit Filmen oder Büchern. Die Piraten halten es für falsch, Spieler zu kriminalisieren, statt die eigentlichen gesellschaftlichen Probleme zu lösen. Wir setzen uns deshalb dafür ein, dass Computer- und Videospiele als Kulturgüter anerkannt werden und wenden uns gegen Verbote dieser Form von Kultur.

\subsubsection{Modul 2}
\abstimmung
Paintball ist ein Mannschaftssport und eine legitime Freizeitbeschäftigung Erwachsener. Es gibt keine Anzeichen, dass dieser Sport die Gewahlbereitschaft erhöht oder sogar Amokläufe verursacht. Die Forderungen nach einem Verbot von Paintball betrachtet die Piratenpartei als populistisch und stellt sich ihnen entgegen.
 
\wahlprogramm{Computerspiele}
\antrag{KV Trier/Trier-Saarburg}\version{03:31, 20. Jun. 2010}


\subsubsection{Gegen Stigmatisierung von eSport und Action-Computerspielen als ''Killerspiele''} 
\abstimmung
Die Bezeichnung ''Killerspieler'' diskreditiert in völlig inakzeptabler Weise Spieler, eSportler und sogar Jugendliche, die gerne ihrem Hobby nachgehen. Wir lehnen das von der Innenministerkonferenz geforderte generalisierende Verbot zur Herstellung und Verbreitung von Computerspielen strikt ab. Populistische Verbotsforderungen lösen vor allem dort keine Probleme, wo sie völlig andere Ursachen, als die vordergründig unterstellten, haben. Spieler sind keine Mörder und Gewalttäter. Nicht Verbote, sondern präventive Maßnahmen und die Stärkung medienkompetenten Handelns stärken den verantwortungsvollen Umgang mit elektronischen Medien.
 
\wahlprogramm{Computerspiele}
\antrag{KV Trier/Trier-Saarburg}\version{03:31, 20. Jun. 2010}
\subsubsection{eSport-Vereine anerkennen}
\abstimmung
Immer mehr Spieler organisieren sich in eSport-Vereinen, um gemeinsam ihrem Hobby nachzugehen. Ebenso wie traditionelle Sportvereine leisten sie dabei einen wichtigen Beitrag zum gesellschaftlichen Zusammenhalt. Viele bemühen sich beispielsweise um die Vermittlung von Medienkompetenz gegenüber Jugendlichen und Eltern. Wir möchten diese Arbeit honorieren und dafür sorgen, dass eSport-Vereine genauso wie andere Sportvereine als gemeinnützig anerkannt werden.

\wahlprogramm{Poker}
\antrag{Unglow}\version{03:31, 20. Jun. 2010}

\subsubsection{Modul 1}
\abstimmung
Das Pokerspielen hat in den letzten Jahren an Popularität gewonnen. In Rheinland-Pfalz hat dies leider dazu geführt, dass ein weitreichendes Pokerverbot geschaffen wurde. Öffentliche Pokerveranstaltungen und Internet-Poker wurden weitgehend verboten. Das Pokerverbot in Rheinland-Pfalz halten wir für ungerechtfertigt und werden uns deshalb für die Abschaffung dieses Gesetzes einsetzen.
 
\subsection*{Mehr Freiheit für Raucher und ein sinnvoller Nichtraucherschutz}
\wahlprogramm{Mehr Freiheit für Raucher und ein sinnvoller Nichtraucherschutz}\label{wp:selbst:raucher1}
\antrag{Unglow}\version{03:31, 20. Jun. 2010}
\begin{itemize}
\item \konkurrenz{wp:selbst:raucher2}
\item \konkurrenz{wp:selbst:raucher3}
\item \konkurrenz{wp:selbst:raucher4}
\end{itemize}

\subsubsection{Modul 1}
\abstimmung
Ein großer Teil der Deutschen raucht. Da Rauchen schädlich ist und Krebs auslösen kann, wurde versucht die Raucher durch übergroße Warnhinweise und höhere Tabaksteuern zwangszubelehren, ohne großen Erfolg. Da Raucher durch das Passivrauchen auch die Gesundheit von Nichtrauchern gefährden, wurden von den Bundesländern Nichtraucherschutzgesetze verabschiedet. An Arbeitsplätzen und in öffentlichen Gebäuden, bei denen keine Entscheidungsmöglichkeit besteht, sich dem Rauch zu entziehen, sind solche Maßnahmen nachvollziehbar und sinnvoll.

\subsubsection{Modul 2}
\abstimmung
Jedoch wurde auch das Rauchen in Gaststätten größtenteils verboten. In Nebenräumen von Gaststätten darf noch geraucht werden. Ob in Einraumgaststätten geraucht werden darf hängt in Rheinland-Pfalz von der Größe der Gaststätte ab, oder davon in welchem Umfang in der Gaststätte Speisen serviert werden. Das Risiko an Krebs zu erkranken sinkt jedoch nicht dadurch, dass man nur eine Kleinigkeit isst, statt einer richtigen Mahlzeit. Auch kleine Räume senken dass Risiko für Nichtraucher nicht. In der Regel dürfte die Luft in kleinen Gaststätten sogar schlechter sein, als in großen Gaststätten, womit das Risiko für Nichtraucher ansteigt.

\subsubsection{Modul 3}
\abstimmung
Die Regelungen zum Nichtraucherschutz in Rheinland-Pfalz sind für uns nicht nachvollziehbar. Die Freiheit der Bürger und der Gaststättenbetreiber auf nicht nachvollziehbare Weise beschnitten. Gleichzeitig findet aber kein effektiver Schutz der Angestellten im Gastronomiegewerbe statt. Wir setzen uns für eine nachvollziehbare und effektive Regelung zum Nichtraucherschutz ein, in der überall in Gaststätten geraucht werden darf, außer in Bereichen in denen Angestellte arbeiten.
 
\wahlprogramm{Mehr Freiheit für Raucher und ein sinnvoller Nichtraucherschutz}\label{wp:selbst:raucher2}
\antrag{Unglow}\version{03:31, 20. Jun. 2010}
\begin{itemize}
\item \konkurrenz{wp:selbst:raucher1}
\item \konkurrenz{wp:selbst:raucher3}
\item \konkurrenz{wp:selbst:raucher4}
\end{itemize}

\subsubsection{Modul 1}
\abstimmung
Wir werden uns dafür einsetzen, dass bei bestehenden Gastronomiebetrieben der Betreiber selbst entscheiden kann ob er eine Raucher- oder Nichtraucher- Lokalität betreiben möchte, soweit eine Trennung in einen abgeschlossenen Raucher- und Nichtraucherbereich nicht möglich ist. Hierbei ist gleichgültig ob Speisen angeboten werden oder nicht. Gleichzeitig muss bei neu erteilten Betriebserlaubnissen eine klare Regelung eine Trennung in einen separaten abgeschlossenen Bereich für beide Gruppen beinhalten. Ansonsten kann keine Betriebserlaubnis erteilt werden.

\subsubsection{Modul 2}
\abstimmung
Der Betreiber muss im Rahmen des Mitarbeiterschutzes sicherstellen, dass kein nichtrauchender Mitarbeiter in Bereichen in denen geraucht werden darf bedienen muss. Im Umkehrschluss bedeutet dies dass ein Mitarbeiter der Raucher ist zwar im Nichtraucherbereich bedienen kann, dies aber sicherlich den nichtrauchenden Gast stört, wenn die Bedienung nach Nikotin riecht.

\subsubsection{Modul 3}
\abstimmung
Für kleine Gastronomiebetriebe ohne Speiseangebot entscheidet der Betreiber ob er eine Raucher- oder Nichtraucherkneipe betreiben möchte. Genauso wie der Nichtraucher sich entscheiden kann dort zu verweilen oder auch das Lokal zu meiden.
 
\wahlprogramm{Mehr Freiheit für Raucher und ein sinnvoller Nichtraucherschutz}\label{wp:selbst:raucher3}
\antrag{MatthiasK}\version{03:31, 20. Jun. 2010}
\begin{itemize}
\item \konkurrenz{wp:selbst:raucher1}
\item \konkurrenz{wp:selbst:raucher2}
\item \konkurrenz{wp:selbst:raucher4}
\end{itemize}

\subsubsection{Modul 1}
\abstimmung
Die Regelungen zum Nichtraucherschutz in Rheinland-Pfalz sind für uns nicht nachvollziehbar, die Freiheit der Bürger und der Gaststättenbetreiber auf nicht nachvollziehbare Weise beschnitten. Gleichzeitig findet aber kein effektiver Schutz der Angestellten im Gastronomiegewerbe statt. Die staatliche Bevormundung von Bürgern und Gastronomiebetrieben muss ein Ende haben.

\subsubsection{Modul 2}
\abstimmung
Gastronomiebetreiber müssen selbst entscheiden können, ob sie eine Raucher- oder Nichtraucher- Lokalität betreiben möchten. Hierbei darf ein etwaiges Speisenangebot keine Rolle spielen.

\subsubsection{Modul 3}
\abstimmung
Ferner fordern wir eine einheitliche Kenntlichmachung an der Außenseite aller Betriebe, die auf die jeweiligen Verhältnisse hinweist. So ist jedem Bürger die Freiheit gegeben ein Lokal ohne Raucher- bzw. Nichtraucherbereich zu meiden.

\subsubsection{Modul 4}
\abstimmung
Der Betreiber muss im Rahmen des Mitarbeiterschutzes sicherstellen, dass kein nichtrauchender Mitarbeiter in Raucherbereichen bedienen muss.
 
\wahlprogramm{Mehr Freiheit für Raucher und ein sinnvoller Nichtraucherschutz}\label{wp:selbst:raucher4}
\antrag{Silberpappel}\version{03:31, 20. Jun. 2010}
\begin{itemize}
\item \konkurrenz{wp:selbst:raucher1}
\item \konkurrenz{wp:selbst:raucher2}
\item \konkurrenz{wp:selbst:raucher3}
\end{itemize}

\subsubsection{Einleitung}
\abstimmung
Die Regelungen zum Nichtraucherschutz in Rheinland-Pfalz sind für uns nicht nachvollziehbar. Die Freiheit der Bürger und der Gaststättenbetreiber wird auf nicht nachvollziehbare Weise beschnitten. Gleichzeitig findet aber kein effektiver Schutz der Angestellten im Gastronomiegewerbe statt.

\subsubsection{Rein inhaberbetriebene Lokalitäten}
\abstimmung
In Betrieben, in denen nur der / die Inhaber arbeiten, sollen selbst entscheiden können, ob sie eine Raucher- oder Nichtraucher- Lokalität betreiben möchten. Hierbei darf ein etwaiges Speisenangebot keine Rolle spielen.

\subsubsection{Angestellte}
\abstimmung
In Betrieben, in denen auch Angestellte arbeiten, gilt Rauchverbot.

\subsubsection{Raucherbereich}
\abstimmung
Betriebe, in denen Rauchverbot gilt, können einen abgeschlossenen Raucherbereich einrichten, in dem die Angestellten dann aber nicht arbeiten.

\subsubsection{Kennzeichnung}
\abstimmung
Ferner fordern wir eine einheitliche Kennzeichnung an der Außenseite aller Betriebe, die auf die jeweiligen Verhältnisse hinweist. So ist jedem Bürger die Freiheit gegeben, ein Lokal ohne Raucher- bzw. Nichtraucherbereich zu meiden.
 
\subsection*{Waffenkontrollen}
\wahlprogramm{Waffenkontrollen einschränken}
\antrag{KV Trier/Trier-Saarburg}\version{03:31, 20. Jun. 2010}

\subsubsection{ }
\abstimmung
Verdachtsunabhängige, unangekündigte Waffenkontrollen in privaten Wohnräumen verletzen das Grundrecht der Unverletzlichkeit der Wohnung. Deswegen möchten wir diese einschränken.
 
\wahlprogramm{Zuständigkeit für Waffenkontrolle}
\antrag{KV Trier/Trier-Saarburg}\version{03:31, 20. Jun. 2010}

\subsubsection{ }
\abstimmung
Soweit Kontrollen in Wohnungen unumgänglich sind, sollten diese nur von Polizisten durchgeführt werden, da diese im Gegensatz zu anderen Personen über die hierfür nötige Ausbildung und Routine im Umgang mit Schusswaffen verfügen.
 
\newpage
\wahlprogramm{Keine Hausdurchsuchungen zur Waffenkontrolle}
\antrag{Piraten aus RLP}\version{03:31, 20. Jun. 2010}

\subsubsection{Keine Hausdurchsuchungen zur Waffenkontrolle}
\abstimmung
Gegen den Willen von Bürgerinnen und Bürgern dürfen ohne richterlichen Durchsuchungsbeschluss keine Kontrollen von Wohnungen oder sonstigen Liegenschaften erfolgen.
 
\subsection*{Rechtliche Gleichstellung}
\wahlprogramm{Rechtliche Gleichstellung}
\antrag{KV Trier/Trier-Saarburg}\version{03:31, 20. Jun. 2010}

\subsubsection{ }
\abstimmung
Wir werden uns dafür einsetzen, dass das Land Rheinland-Pfalz sich im Bundesrat dafür stark macht, die rechtliche Gleichstellung aller Menschen unabhängig von ihrer sexuellen Identität voranzutreiben.

\subsubsection{ }
\abstimmung
Insbesondere sind Familien für uns all jene Lebenskonstellationen, in denen Verantwortung für Kinder und Eltern übernommen wird. Dabei ist die Anzahl der Verantwortlichen, deren Beziehung zueinander und deren Geschlecht unerheblich.
