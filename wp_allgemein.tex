\section{Wahlprogramm}
\wahlprogramm{Vergaberegister zur Korruptionsbekämpfung - Positivliste}\label{wpa:vergaberegister1}
\antrag{Stephan Schurig}\\
\version{17:32, 9. Apr. 2012}
\begin{itemize}
\item \konkurrenz{wpa:vergaberegister2}
\end{itemize}

\paragraph{Antragstext}:

Der Landesparteitag möge beschließen, folgende Formulierung im Wahlprogramm zu ändern:

\einruecken{Wir wollen ein Vergaberegister schaffen, mit dessen Hilfe bereits \textbf{positiv} auffällig gewordene Firmen künftig von der Vergabe öffentlicher Aufträge \textbf{bevorzugt} werden können \textbf{(Positivliste). Zu diesem Zwecke werden alle Unternehmen aufgeführt, die sich bei vorangegangenen Projekten als vertrauenswürdig erwiesen haben.} Diese Informationen sollen nicht nur Behörden zur Verfügung stehen, sondern auch der interessierten Öffentlichkeit.}

\paragraph{Alte Fassung}:

\einruecken{Wir wollen ein Vergaberegister schaffen, mit dessen Hilfe bereits \textbf{negativ} auffällig gewordene Firmen künftig von der Vergabe öffentlicher Aufträge \textbf{ausgeschlossen} werden können. Diese Informationen sollen nicht nur Behörden zur Verfügung stehen, sondern auch der interessierten Öffentlichkeit.}

\paragraph{Begründung}:

Die aktuelle Formulierung macht keine Aussage über die Form des Vergaberegisters (Blacklist, Whitelist, Greylist etc.).

% -----

\wahlprogramm{Landärztemangel entgegenwirken}
\antrag{Martin Otto}\\
\version{17:51, 9. Apr. 2012}

\paragraph{Antragstext}:

Der Landesparteitag (LPT) möge beschließen, dem Landärztemangel aktiv gegenzusteuern.

\paragraph{Begründung}:

Seit Jahren ist Zahl der praktizierenden Ärzte auf dem Land Rückläufig. Das führt zu einer gravierenden Unterversorgung der fachärztlichen Betreuung in ländlichen Regionen.


Erläuterung

Um diesem Mangel an Ärzten in ländlichen Regionen entgegenzusteuern bedarf es umfangreicher struktureller Maßnahmen. Dazu gehört die Einführung eines nichtrückzahlbaren Zusatzstipendiums. Dieses geht einher mit der Verpflichtung, für Dauer der Zahlung anschließend auf dem Land zu arbeiten. Zusätzlich müssen dafür weitere Anreize geschaffen werden. Die angehenden Landärzte erfahren Unterstützung bei der Einrichtung einer Praxis und erhalten ein Grundgehalt. Des weiteren ist eine Neuordnung des Bereitschaftsdienstes unabdinglich. Die Unterstützung von Familienmitgliedern bei der Erwerbstätigkeit, ist ein weiterer wichtiger Anreiz. Zur Umsetzung dieser Maßnahmen, sind umgehend Kommissionen einzusetzen, bestehend aus Fachleuten vom KVSA, dem Hausärzteverband Sachsen-Anhalt e.V., sowie dem Hartmannbund, die die anstehenden Probleme benennen, damit die Politik die zeitnahe Behebung auf den Weg bringen kann. 

% -----

\wahlprogramm{Vergaberegister zur Korruptionsbekämpfung - Negativliste}\label{wpa:vergaberegister2}
\antrag{Stephan Schurig}\\
\version{17:54, 9. Apr. 2012}
\begin{itemize}
\item \konkurrenz{wpa:vergaberegister1}
\end{itemize}

\paragraph{Antragstext}:

Der Landesparteitag möge beschließen, folgende Formulierung im Wahlprogramm zu ändern:

\einruecken{Wir wollen ein Vergaberegister schaffen, mit dessen Hilfe bereits negativ auffällig gewordene Firmen künftig von der Vergabe öffentlicher Aufträge ausgeschlossen werden können \textbf{(Negativliste). Zu diesem Zwecke werden alle Unternehmen aufgeführt, die sich bei vorangegangenen Projekten als nicht vertrauenswürdig erwiesen haben. Dazu muss im Vorfeld eine Negativliste mit Verhaltensverboten aufgestellt werden, welche sich an der Definition {\Gu}Unzulässige geschäftliche Handlungen{\Go} im Gesetz gegen den unlauteren Wettbewerb (UWG) orientiert. Bei Zuwiderhandlung kommt es zur Eintragung des Unternehmens in eine schwarze Liste. Die Löschung dieser Daten ist erst nach einer bestimmten Zeit möglich.} Diese Informationen sollen nicht nur Behörden zur Verfügung stehen, sondern auch der interessierten Öffentlichkeit.}

\paragraph{Alte Fassung}:

\einruecken{Wir wollen ein Vergaberegister schaffen, mit dessen Hilfe bereits negativ auffällig gewordene Firmen künftig von der Vergabe öffentlicher Aufträge ausgeschlossen werden können. Diese Informationen sollen nicht nur Behörden zur Verfügung stehen, sondern auch der interessierten Öffentlichkeit.}

\paragraph{Begründung}:

Die Formulierung macht keine Aussage über die Form des Vergaberegisters (Blacklist, Whitelist, Greylist etc.). Die Definition {\Gu}Unzulässiger geschäftlicher Handlungen{\Go} \href{http://www.gesetze-im-internet.de/uwg_2004/anhang_26.html}{ist im UWG geklärt} und \href{http://anwalt-im-netz.de/archiv/2008/uwg-verhaltensverbote.html}{verständlichere Beispiele sind hier aufgeführt}.

% -----

\wahlprogramm{Abschaffung der 5\%-Hürde bzw. Sperrklausel auf Landesebene}\label{wpa:prozenthuerde1}
\antrag{Stephan Schurig}\\
\version{18:03, 9. Apr. 2012}
\begin{itemize}
\item \konkurrenz{wpa:prozenthuerde2}
\item \konkurrenz{wpa:prozenthuerde3}
\item \konkurrenz{wpa:prozenthuerde4}
\item \konkurrenz{wpa:prozenthuerde5}
\item \konkurrenz{wpa:prozenthuerde6}
\end{itemize}

\paragraph{Antragstext}:

Der Landesparteitag möge folgenden Punkt an geeigneter Stelle im Wahlprogramm einfügen:

\einruecken{Die Piratenpartei Sachsen-Anhalt fordert die Abschaffung der Sperrklausel bei Landtagswahlen in Sachsen-Anhalt.}

\paragraph{Begründung}:

Zu mehr Bürgerbeteiligung und Mitbestimmung gehört, dass auch kleinere gewählte Fraktionen an den demokratischen Prozessen teilhaben dürfen. Der Nutzen wiegt dabei größer als die Gefahren, durch den Einzug radikaler Parteien, da die Mehrheit der Unter-5\%-Parteien nicht als radikal eingestuft werden kann.

% -----

\wahlprogramm{Herabsetzung der 5\%-Hürde bzw. Sperrklausel auf Landesebene auf 1\%}\label{wpa:prozenthuerde2}
\antrag{Stephan Schurig}\\
\version{18:06, 9. Apr. 2012}
\begin{itemize}
\item \konkurrenz{wpa:prozenthuerde1}
\item \konkurrenz{wpa:prozenthuerde3}
\item \konkurrenz{wpa:prozenthuerde4}
\item \konkurrenz{wpa:prozenthuerde5}
\item \konkurrenz{wpa:prozenthuerde6}
\end{itemize}

\paragraph{Antragstext}:

Der Landesparteitag möge folgenden Punkt an geeigneter Stelle im Wahlprogramm einfügen:

\einruecken{Die Piratenpartei Sachsen-Anhalt fordert die Herabsetzung der Sperrklausel in Sachsen-Anhalt bei Landtagswahlen auf 1\%.}

\paragraph{Begründung}:

Zu mehr Bürgerbeteiligung und Mitbestimmung gehört, dass auch kleinere gewählte Fraktionen an den demokratischen Prozessen teilhaben dürfen. Der Nutzen wiegt dabei größer als die Gefahren, durch den Einzug radikaler Parteien, da die Mehrheit der Unter-5\%-Parteien nicht als radikal eingestuft werden kann.

Die staatliche Parteienfinanzierung aus Steuergeldern erhalten Parteien bereits mit {\Gu}mindestens 1,0 Prozent (Bundestags- oder Europawahl) bzw. 0,5 Prozent (Landtagswahlen) der gültigen Stimmen{\Go} (\href{http://www.bpb.de/themen/513F3I,0,Staatliche_Parteienfinanzierung.html}{Quelle}). 

% -----

\wahlprogramm{Herabsetzung der 5\%-Hürde bzw. Sperrklausel auf Landesebene auf 2\%}\label{wpa:prozenthuerde3}
\antrag{Stephan Schurig}\\
\version{18:07, 9. Apr. 2012}
\begin{itemize}
\item \konkurrenz{wpa:prozenthuerde1}
\item \konkurrenz{wpa:prozenthuerde2}
\item \konkurrenz{wpa:prozenthuerde4}
\item \konkurrenz{wpa:prozenthuerde5}
\item \konkurrenz{wpa:prozenthuerde6}
\end{itemize}

\paragraph{Antragstext}:

Der Landesparteitag möge folgenden Punkt an geeigneter Stelle im Wahlprogramm einfügen:

\einruecken{Die Piratenpartei Sachsen-Anhalt fordert die Herabsetzung der Sperrklausel in Sachsen-Anhalt bei Landtagswahlen auf 2\%.}

\paragraph{Begründung}:

Zu mehr Bürgerbeteiligung und Mitbestimmung gehört, dass auch kleinere gewählte Fraktionen an den demokratischen Prozessen teilhaben dürfen. Der Nutzen wiegt dabei größer als die Gefahren, durch den Einzug radikaler Parteien, da die Mehrheit der Unter-5\%-Parteien nicht als radikal eingestuft werden kann.

Die staatliche Parteienfinanzierung aus Steuergeldern erhalten Parteien bereits mit {\Gu}mindestens 1,0 Prozent (Bundestags- oder Europawahl) bzw. 0,5 Prozent (Landtagswahlen) der gültigen Stimmen{\Go} (\href{http://www.bpb.de/themen/513F3I,0,Staatliche_Parteienfinanzierung.html}{Quelle}). 

Dänemark besitzt auf nationaler Ebene lediglich eine Sperrklausel von 2\%. (\href{https://de.wikipedia.org/wiki/Sperrklausel}{Quelle}) 

% -----

\wahlprogramm{Herabsetzung der 5\%-Hürde bzw. Sperrklausel auf Landesebene auf 3\%}\label{wpa:prozenthuerde4}
\antrag{Stephan Schurig}\\
\version{18:07, 9. Apr. 2012}
\begin{itemize}
\item \konkurrenz{wpa:prozenthuerde1}
\item \konkurrenz{wpa:prozenthuerde2}
\item \konkurrenz{wpa:prozenthuerde3}
\item \konkurrenz{wpa:prozenthuerde5}
\item \konkurrenz{wpa:prozenthuerde6}
\end{itemize}

\paragraph{Antragstext}:

Der Landesparteitag möge folgenden Punkt an geeigneter Stelle im Wahlprogramm einfügen:

\einruecken{Die Piratenpartei Sachsen-Anhalt fordert die Herabsetzung der Sperrklausel in Sachsen-Anhalt bei Landtagswahlen auf 3\%.}

\paragraph{Begründung}:

Zu mehr Bürgerbeteiligung und Mitbestimmung gehört, dass auch kleinere gewählte Fraktionen an den demokratischen Prozessen teilhaben dürfen. Der Nutzen wiegt dabei größer als die Gefahren, durch den Einzug radikaler Parteien, da die Mehrheit der Unter-5\%-Parteien nicht als radikal eingestuft werden kann.

Die staatliche Parteienfinanzierung aus Steuergeldern erhalten Parteien bereits mit {\Gu}mindestens 1,0 Prozent (Bundestags- oder Europawahl) bzw. 0,5 Prozent (Landtagswahlen) der gültigen Stimmen{\Go} (\href{http://www.bpb.de/themen/513F3I,0,Staatliche_Parteienfinanzierung.html}{Quelle}). 

Dänemark besitzt auf nationaler Ebene lediglich eine Sperrklausel von 2\%. (\href{https://de.wikipedia.org/wiki/Sperrklausel}{Quelle}) 

3\% sind ca. 3 Sitze und gleichzeitig die Mindestgröße einer Fraktion im Landesparlament.

% -----

\wahlprogramm{Herabsetzung der 5\%-Hürde bzw. Sperrklausel auf Landesebene auf 4\%}\label{wpa:prozenthuerde5}
\antrag{Stephan Schurig}\\
\version{18:10, 9. Apr. 2012}
\begin{itemize}
\item \konkurrenz{wpa:prozenthuerde1}
\item \konkurrenz{wpa:prozenthuerde2}
\item \konkurrenz{wpa:prozenthuerde3}
\item \konkurrenz{wpa:prozenthuerde4}
\item \konkurrenz{wpa:prozenthuerde6}
\end{itemize}

\paragraph{Antragstext}:

Der Landesparteitag möge folgenden Punkt an geeigneter Stelle im Wahlprogramm einfügen:

\einruecken{Die Piratenpartei Sachsen-Anhalt fordert die Herabsetzung der Sperrklausel in Sachsen-Anhalt bei Landtagswahlen auf 4\%.}

\paragraph{Begründung}:

Zu mehr Bürgerbeteiligung und Mitbestimmung gehört, dass auch kleinere gewählte Fraktionen an den demokratischen Prozessen teilhaben dürfen. Der Nutzen wiegt dabei größer als die Gefahren, durch den Einzug radikaler Parteien, da die Mehrheit der Unter-5\%-Parteien nicht als radikal eingestuft werden kann.

Die staatliche Parteienfinanzierung aus Steuergeldern erhalten Parteien bereits mit {\Gu}mindestens 1,0 Prozent (Bundestags- oder Europawahl) bzw. 0,5 Prozent (Landtagswahlen) der gültigen Stimmen{\Go} (\href{http://www.bpb.de/themen/513F3I,0,Staatliche_Parteienfinanzierung.html}{Quelle}). 

4\% sind ca. 4 Sitze und gleichzeitig etwas mehr als die Mindestgröße einer Fraktion (3) im Landesparlament. 

% -----

\wahlprogramm{Beibehalten der 5\%-Hürde bzw. Sperrklausel auf Landesebene}\label{wpa:prozenthuerde6}
\antrag{Stephan Schurig}\\
\version{18:11, 9. Apr. 2012}
\begin{itemize}
\item \konkurrenz{wpa:prozenthuerde1}
\item \konkurrenz{wpa:prozenthuerde2}
\item \konkurrenz{wpa:prozenthuerde3}
\item \konkurrenz{wpa:prozenthuerde4}
\item \konkurrenz{wpa:prozenthuerde5}
\end{itemize}

\paragraph{Antragstext}:

Der Landesparteitag möge folgenden Punkt an geeigneter Stelle im Wahlprogramm einfügen:

\einruecken{Die Piratenpartei Sachsen-Anhalt setzt sich dafür ein, die Sperrklausel bei Landtagswahlen in Sachsen-Anhalt von 5\% beizubehalten.}

\paragraph{Begründung}:

Die momentane Regelung ist absolut ausreichend.

% -----

\wahlprogramm{Herabsetzung des aktiven Wahlalters bei Landtagswahlen auf 0 Jahre}\label{wpa:wahlalter1}
\antrag{Stephan Schurig}\\
\version{18:13, 9. Apr. 2012}
\begin{itemize}
\item \konkurrenz{wpa:wahlalter2}
\end{itemize}

\paragraph{Antragstext}:

Es wird beantragt ins Wahlprogramm folgende Forderung einzufügen:

\einruecken{Die Piratenpartei fordert die vollständige Aufhebung des notwendigen Mindestalters zur Wahrnehmung des aktiven Wahlrechts bei Landtagswahlenund damit eine Anpassung des § 42 Abs. 2 der Verfassung des Landes Sachsen-Anhalt. Das aktive Wahlrecht soll ab der Geburt von jedem Bürger wahrgenommen werden können. Die erstmalige Ausübung dieses Wahlrechts erfordert für Unter-16-Jährige die selbständige Eintragung in eine Wählerliste. Eine Stellvertreterwahl durch Erziehungsberechtigte lehnen wir ab.}

\paragraph{Begründung}:

Text ist zu lang, \href{http://wiki.piratenpartei.de/LSA:Landesverband/Organisation/Mitgliederversammlung/2012.1/Antragsfabrik/Herabsetzung_des_aktiven_Wahlalters_bei_Landtagswahlen_auf_0_Jahre}{siehe Antragsfabrik}

% -----

\wahlprogramm{Herabsetzung des aktiven Wahlalters bei Landtagswahlen auf 12 Jahre}\label{wpa:wahlalter2}
\antrag{Stephan Schurig}\\
\version{18:14, 9. Apr. 2012}
\begin{itemize}
\item \konkurrenz{wpa:wahlalter1}
\end{itemize}

\paragraph{Antragstext}:

Es wird beantragt ins Wahlprogramm folgende Forderung einzufügen:

\einruecken{Die Piratenpartei fordert die Senkung des notwendigen Alters zur Wahrnehmung des aktiven Wahlrechts bei Landtagswahlen auf 12 Jahre und damit eine Anpassung des § 42 Abs. 2 der Verfassung des Landes Sachsen-Anhalt. Die erstmalige Ausübung dieses Wahlrechts erfordert für Unter-16-Jährige die selbständige Eintragung in eine Wählerliste. Eine Stellvertreterwahl durch Erziehungsberechtigte lehnen wir ab.}

\paragraph{Begründung}:

Text ist zu lang, \href{http://wiki.piratenpartei.de/LSA:Landesverband/Organisation/Mitgliederversammlung/2012.1/Antragsfabrik/Herabsetzung_des_aktiven_Wahlalters_bei_Landtagswahlen_auf_12_Jahre}{siehe Antragsfabrik}

% -----

\wahlprogramm{Aufhebung von §5 FeiertG LSA (Tanzverbot u.a. an Feiertagen)}
\antrag{Stephan Schurig}\\
\version{18:15, 9. Apr. 2012}

\paragraph{Antragstext}:

Der Landesparteitag möge beschließen, folgenden Abschnitt an geeigneter Stelle in das Wahlprogramm aufzunehmen:

\einruecken{\textbf{Aufhebung des §5 FeiertG LSA}

Die Piratenpartei Sachsen-Anhalt strebt die Aufhebung des §5 Gesetz über die Sonn- und Feiertage(FeiertG LSA) an. Die Trennung von Religion und Staat bzw. die Selbstbestimmung des Individuums ist höher zu bewerten, als der erhöhte Schutz religiöser Bräuche. Durch Beibehalten von §4 bleibt der besondere Schutz von Gottesdiensten jedoch bestehen.}

\paragraph{Begründung}:

In \href{http://st.juris.de/st/FeiertG_ST_P5.htm}{§5 FeiertG} ist festgelegt, dass an speziellen christlichen Feiertagen neben den Einschränkungen nach §4 zusätzlich untersagt sind:

\begin{enumerate}
\item Veranstaltungen in Räumen mit Schankbetrieb, die über den Schank- und Speisebetrieb hinausgehen,
\item öffentliche sportliche Veranstaltungen sowie
\item alle sonstigen öffentlichen Veranstaltungen, außer wenn sie der Würdigung des Feiertages oder der Kunst, Wissenschaft oder Volksbildung dienen und auf den Charakter des Tages Rücksicht nehmen.
\end{enumerate}

Dieser Abschnitt ist zu streichen, da er eine Einschränkung insbesondere für alle Nicht-Christen darstellt. Im Sinne der Trennung von Kirche und Staat ist das Gesetz nicht mehr zeitgemäßg. Christen können allerdings weiterhin ihrem Glauben und Gottesdiensten uneingeschränkt nachgehen, da der §4 bestehen bleibt, welcher sicherstellt, dass keine Veranstaltungen erlaubt sind, die einen Gottesdienst stören.

% -----

\wahlprogramm{Flächendeckendes barrierefreies Notruf- und Informationssystem per Mobilfunk (SMS-Notruf)}\label{wpa:smsnotruf1}
\antrag{Stephan Schurig}\\
\version{18:19, 9. Apr. 2012}
\begin{itemize}
\item \konkurrenz{wpa:smsnotruf2}
\end{itemize}

\paragraph{Antragstext}:

Der Landesparteitag möge beschließen, in das Wahlprogramm für die kommende Landtagswahl an geeigneter Stelle aufzunehmen:

\einruecken{Die Piratenpartei Sachsen-Anhalt setzt sich für die zeitnahe Einführung eines flächendeckenden barrierefreien Notruf- und Informationssystem per Mobilfunk in Sachsen-Anhalt ein. Weiterhin unterstützen wir nach Möglichkeit alle Bemühungen für eine bundesweite Umsetzung.}

\paragraph{Begründung}:

Text ist zu lang, siehe \href{http://wiki.piratenpartei.de/LSA:Landesverband/Organisation/Mitgliederversammlung/2012.1/Antragsfabrik/Fl\%C3\%A4chendeckendes_barrierefreies_Notruf-_und_Informationssystem_per_Mobilfunk_\%28SMS-Notruf\%29}{Antragsfabrik}

% -----

\wahlprogramm{Verbandsklagerecht}
\antrag{Alexander Magnus}\\
\version{18:21, 9. Apr. 2012}

\paragraph{Antragstext}:

Wir setzen uns für die Einführung eines Verbandsklagerechtes für anerkannte Tierschutzorganisationen im Sachsen-Anhalt ein. Tiere können als Lebewesen nicht selbst für ihre Rechte eintreten bzw. diese verteidigen. Daher sind sie auf Vertreter in Form von Verbänden angewiesen. Obwohl Tier- und Umweltschutz nach Art. 20a GG denselben Verfassungsrang haben, werden die beiden Staatsziele ungleich behandelt, wenn es um das Verbandsklagerecht geht. Erfahrungen in Bremen, wo es die Tierschutzverbandsklage inzwischen gibt, zeigen zudem, dass die von den Gegnern der Verbandsklage befürchtete Klageflut ausgeblieben ist. Da auf Bundesebene keine Lösung in Sicht ist, ist die Einführung des Verbandsklagerechts auf Landesebene geboten.

\paragraph{Begründung}:

Aus dem Antragsportal LTW2012 des Saarlandes übernommen

% -----

\wahlprogramm{Mehr Polizeibeamte, weniger Überwachung}
\antrag{Alexander Magnus}\\
\version{18:23, 9. Apr. 2012}

\paragraph{Antragstext}:

Statt den Bürgern Sicherheit durch mehr Überwachungsmaßnahmen vorzuspiegeln, sollten die Gelder dafür in die Beschäftigung von mehr Polizeibeamten investiert werden. Eine Kamera kann - sofern sie überhaupt von einem Beamten überwacht wird - keine Hilfe leisten oder herbeirufen. Ein vor Ort patrouillierender Polizei erhöht die subjektive und die tatsächliche Sicherheit, er kennt die Bewohner {\Gu}seines{\Go} Stadtteiles und kann, noch vor der Notwendigkeit von Sanktionen, auf Mitglieder der Gesellschaft einwirken, die auf die schiefe Bahn zu geraten drohen.

Allerdings lehnen wir einen Polizeistaat ab. Mehr Personal sollte lediglich in problematischen Regionen, Orten bzw. Plätzen bereit gestellt werden, oder dort, wo laufende Ermittlungen durch mangelndes Personal behindert oder gar unmöglich gemacht werden.

% -----

\wahlprogramm{Verbesserte Ausstattung der Polizei}
\antrag{Alexander Magnus}\\
\version{18:24, 9. Apr. 2012}

\paragraph{Antragstext}:

Um der Polizei die Erfüllung ihrer Aufgaben in einem vernünftigen Maße zu ermöglichen, muss die materielle und personelle Ausstattung verbessert werden. Die Anschaffung von Ausrüstung wie z. B. Schutzwesten darf nicht dem einzelnen Polizisten aufgebürdet werden. Gleichzeitig müssen ausreichend Beamte beschäftigt werden, um die Polizeiarbeit angemessen bewältigen zu können. 

\paragraph{Begründung}:

Eine entsprechende Präsenz einer gut ausgerüsteten Polizei auf unseren Straßen erhöht die Sicherheit des Einzelnen weit mehr als jede Videoüberwachung.

Quelle: \href{http://www.piratenpartei-bw.de/wahl/wahlprogramm/inneres-und-justiz/}{Wahlprogramm BW} und \href{http://wiki.piratenpartei.de/SH:Landtagswahl_2012/Wahlprogramm#Inneres_und_Justiz}{Wahlprogramm SH}

% -----

\wahlprogramm{Mehr und besser ausgestatte Polizeibeamte statt mehr Überwachung}
\antrag{Alexander Magnus}\\
\version{18:25, 9. Apr. 2012}

\paragraph{Antragstext}:

Statt den Bürgern Sicherheit durch mehr Überwachungsmaßnahmen vorzuspiegeln, sollten die Gelder dafür in die Beschäftigung von mehr Polizeibeamten investiert werden. Eine Kamera kann - sofern sie überhaupt von einem Beamten überwacht wird - keine Hilfe leisten oder herbeirufen. Ein vor Ort patrouillierender Polizist erhöht die subjektive und die tatsächliche Sicherheit, er kennt die Bewohner {\Gu}seines{\Go} Stadtteiles und kann, noch vor der Notwendigkeit von Sanktionen, auf Mitglieder der Gesellschaft einwirken, die auf die schiefe Bahn zu geraten drohen.

Allerdings lehnen wir einen Polizeistaat ab. Mehr Personal sollte lediglich in problematischen Regionen, Orten bzw. Plätzen bereit gestellt werden, oder dort, wo laufende Ermittlungen durch mangelndes Personal behindert oder gar unmöglich gemacht werden.

Um der Polizei die Erfüllung ihrer Aufgaben in einem vernünftigen Maße zu ermöglichen, muss die materielle und personelle Ausstattung verbessert werden. Die Anschaffung von Ausrüstung wie z. B. Schutzwesten darf nicht dem einzelnen Polizisten aufgebürdet werden. 

\paragraph{Begründung}:

Zusammenlegung von {\Gu}Mehr Polizeibeamte, weniger Überwachung{\Go} und {\Gu}Verbesserte Ausstattung der Polizei{\Go} Begründung des Antrages zweite Zeile etc. 

% -----

\wahlprogramm{Flächendeckendes barrierefreies Notruf- und Informationssystem per Mobilfunk (SMS-Notruf) - Zielgruppe präzisiert}\label{wpa:smsnotruf2}
\antrag{Stephan Schurig}\\
\version{18:27, 9. Apr. 2012}
\begin{itemize}
\item \konkurrenz{wpa:smsnotruf1}
\end{itemize}

\paragraph{Antragstext}:

Der Landesparteitag möge beschließen, in das Wahlprogramm für die kommende Landtagswahl an geeigneter Stelle aufzunehmen: 

\einruecken{Die Piratenpartei Sachsen-Anhalt setzt sich für die zeitnahe Einführung eines flächendeckenden barrierefreien Notruf- und Informationssystem per Mobilfunk in Sachsen-Anhalt ein. Davon profitieren insbesondere gehörlose und schwerhörige Menschen in Gefahrensituationen. Weiterhin unterstützen wir nach Möglichkeit alle Bemühungen für eine bundesweite Umsetzung.}

\paragraph{Begründung}:

Text ist zu lang, siehe \href{http://wiki.piratenpartei.de/LSA:Landesverband/Organisation/Mitgliederversammlung/2012.1/Antragsfabrik/Fl\%C3\%A4chendeckendes_barrierefreies_Notruf-_und_Informationssystem_per_Mobilfunk_\%28SMS-Notruf\%29_-_Zielgruppe_pr\%C3\%A4zisiert}{Antragsfabrik}

% -----

\wahlprogramm{Klare Trennung von Kirche und Staat}
\antrag{Prof. Dr. Michael Rost, Biederitz}\\
\version{18:29, 9. Apr. 2012}

\paragraph{Antragstext}:

Die Piratenpartei setzt sich für eine klare Trennung von Kirche und Staat ein. Die Piratenpartei ist für Religionsfreiheit und Gleichberechtigung aller Religionen. Jeder Mensch hat das Recht eine Religion auszuüben, aber jede Religion ist reine Privatsache jedes Menschen. Die Piratenpartei ist gegen weitere Alimentierung der Kirchen und Religionsgemeinschaften vom Staat, gegen das Eintreiben der Kirchensteuer durch den Staat, gegen vom Staat alimentierte kirchliche Hochschulen, gegen finanzielle Zuschüsse an Kirchen und Religionsgemeinschaften, gegen Religionsunterricht an staatlichen Schulen, gegen religiöse Zeichen in Schulen. Im Sinne eines evolutionären Humanismus dürfen Menschen ohne Religionsbindung nicht gegenüber anderen Menschen benachteiligt werden und umgekehrt. Die Piratenpartei setzt sich insbesondere auch für die Ablösung der historisch bedingten Finanztransfers an die Kirchen ein.

\paragraph{Begründung}:

37,20\% der Deutschen Bevölkerung, und fast 81\% der Bevölkerung Sachsen Anhalts ist konfessionsfrei, es ist deshalb in höchstem Maße ungerecht, wenn dieser überwiegende Teil der Bevölkerung über Steuern und Abgaben für jene aufkommen muss die einer der großen Religionsgemeinschaften angehören, zumal kleine Religionsgemeinschaft dabei ohnehin benachteiligt werden.

% -----

\wahlprogramm{Änderung der öffentlichen Vergabepraxis}
\antrag{Andreas Rieger}\\
\version{13:21, 11. Apr. 2012}

\paragraph{Antragstext}:

Die Piratenpartei treten für eine Änderung der Vergabepraxis für öffentliche Aufträge ein. Dabei soll nicht das günstigste Angebot, sondern das Angebot, das am nächsten an den kalkulierten Kosten liegt verwendet werden. Vergabeverträge sollen weiterhin immer so gestrickt werden, dass Änderung bei den Kosten zu Lasten der Firmen geht, die die Auftäge erhalten haben.

\paragraph{Begründung}:

die bisherige Praxis zeigt dass bei öffentlichen Aufträgen am Ende häufig die 2 -3 fachen Kosten erreicht werden als in den Planfeststellungverfahren geplant. Dies ist vor allem deshalb der Fall, weil die Vergabepraxis über das Günstigkeitsprinzip häufig dazu führt, das kartelleartige Strukturen mit Sub-Subfirmen die Auftragsvergabeverfahren gewinnen und anschließende Kostensteigerungen über erpressungartige Verfahren durchgedrückt werden ( bewirkte Pleite von Firmenteilen oder der Vertragsfirma anschließende Erpressung der Poltik nach dem Motto {\Gu}wenn Ihr nicht mehr zahlt wirds nie fertig{\Go}), Hinzu kommen Häufig Korruption und Ungerechtfertigte Einsichtnahme der Vergabeunterlagen.

% -----

\wahlprogramm{Rechtsextremismus}
\antrag{Alexander Magnus}\\
\version{18:30, 9. Apr. 2012}

\paragraph{Antragstext}:

In unserer Gesellschaft darf kein Platz für Rechtsextremismus, Rassismus und Antisemitismus sein. Rechtsextreme Propaganda muss als solche bloßgestellt und unsere demokratischen Werte ihr gegenübergestellt werden. Die Morde der sich selbst als {\Gu}Nationalsozialistischer Untergrund{\Go} bezeichnenden Vereinigung haben auf besonders erschreckende Art und Weise verdeutlicht, wie groß das Problem des Rechtsextremismus und die von ihm ausgehende Gefahr ist. In den vergangenen Jahren wurde dieses Problem allzu oft verkannt, ignoriert oder kleingeredet. Präventionsarbeit in diesen Bereichen wurde durch Budgetkürzungen erschwert und mitunter unmöglich gemacht. Diese Schritte müssen rückgängig gemacht werden, sodass diese Programme nicht nur ihre alte Stärke zurückgewinnen, sondern darüber hinaus weiter ausgebaut werden können.

\paragraph{Begründung}:

Übernommen von \href{http://jandoerrenhaus.de/2012/04/06/distanz.zu.rechts.punkt/}{Jan Doerrenhaus/NRW}

Ich möchte, dass sich der Landesverband klar zum Selbstverständnis der Partei und den auf dem letzten BPT angenommenen Anträgen gegen Rechtsextremismus bekennt. Dabei geht es nicht(!) um eine generelle Abkehr von Extremismus - die ich befürworte! - sondern ganz speziell und insbesondere um Rechtsextremismus. Wer der Meinung ist, dass wir auch ebenso klar und deutlich gegen andere Formen von Extremismus Stellung beziehen sollten, darf gern einen solchen Antrag stellen. Ich bitte daher von Kommentaren, die eine Änderung des Antrages in diese Richtung vorschlagen, abzusehen. Das ist NICHT Thema dieses Antrages.

Zur weiteren Argumentationsunterstützung sei auf das oben verlinkte Blog sowie das von \href{http://tarzun.de/archives/423-Das-Problem-heisst-nicht-Kevin.html}{Tarzun} verwiesen.

% -----

\wahlprogramm{Geschlechter- und Familienpolitik}
\antrag{Stephan Schurig}\\
\version{13:24, 11. Apr. 2012}

\paragraph{Antragstext}:

Der Landesparteitag möge beschließen folgenden Abschnitt im Wahlprogramm unter dem Punkt {\Gu}Geschlechter- und Familienpolitik{\Go} einzufügen:

\einruecken{Die Piratenpartei steht für eine zeitgemäße Geschlechter- und Familienpolitik. Diese basiert auf dem Prinzip der freien Selbstbestimmung über Angelegenheiten des persönlichen Lebens. Die Piraten setzen sich dafür ein, dass Politik der Vielfalt der Lebensstile gerecht wird. Jeder Mensch muß sich frei für den selbstgewählten Lebensentwurf und für die individuell von ihm gewünschte Form gleichberechtigten Zusammenlebens entscheiden können. Das Zusammenleben von Menschen darf nicht auf der Vorteilnahme oder Ausbeutung Einzelner gründen.

\textbf{Freie Selbstbestimmung von geschlechtlicher und sexueller Identität bzw. Orientierung}

Die Piratenpartei steht für eine Politik, die die freie Selbstbestimmung von geschlechtlicher und sexueller Identität bzw. Orientierung respektiert und fördert. Fremdbestimmte Zuordnungen zu einem Geschlecht oder zu Geschlechterrollen lehnen wir ab. Diskriminierung aufgrund des Geschlechts, der Geschlechterrolle, der sexuellen Identität oder Orientierung ist Unrecht. Gesellschaftsstrukturen, die sich aus Geschlechterrollenbildern ergeben, werden dem Individuum nicht gerecht und sind zu überwinden.

Die Piratenpartei lehnt die Erfassung des Merkmals “Geschlecht” durch staatliche Behörden ab. Übergangsweise kann die Erfassung seitens des Staates durch eine von den Individuen selbst vorgenommene Einordnung erfolgen.

\textbf{Freie Selbstbestimmung des Zusammenlebens}

Die Piraten bekennen sich zum Pluralismus des Zusammenlebens. Politik muss der Vielfalt der Lebensstile gerecht werden und eine wirklich freie Entscheidung für die individuell gewünschte Form des Zusammenlebens ermöglichen. Eine bloß historisch gewachsene strukturelle und finanzielle Bevorzugung ausgewählter Modelle lehnen wir ab.

\textbf{Freie Selbstbestimmung und Familienförderung}

Die Piratenpartei setzt sich für die gleichwertige Anerkennung von Lebensmodellen ein, in denen Menschen füreinander Verantwortung übernehmen. Unabhängig vom gewählten Lebensmodell genießen Lebensgemeinschaften, in denen Kinder aufwachsen oder schwache Menschen versorgt werden, einen besonderen Schutz. Unsere Familienpolitik ist dadurch bestimmt, dass solche Lebensgemeinschaften als gleichwertig und als vor dem Gesetz gleich angesehen werden müssen.}

\paragraph{Begründung}:

Übernahme aus dem \href{http://berlin.piratenpartei.de/wp-content/uploads/2011/08/PP-BE-wahlprogramm-v1screen.pdf}{Wahlprogramm der Berliner Piraten}

% -----

\wahlprogramm{Ablehnung von Fracking}
\antrag{Stephan Schurig}\\
\version{13:30, 11. Apr. 2012}

\paragraph{Antragstext}:

Der Landesparteitag möge beschließen folgenden Abschnitt an geeigneter Stelle in das Wahlprogramm aufzunehmen:

\einruecken{\textbf{Ablehnung von Fracking}

Die Piratenpartei Sachsen-Anhalt lehnt Hydraulic Fracturing, auch Fracking genannt, als Gasfördermethode ab. Durch diese Methode werden wir und zukünftige Generationen einem kaum kalkulierbaren Risiko ausgesetzt. Das Einbringen zahlreicher, zum Teil hochtoxischer Stoffe mit unkontrollierter Ausbreitung ist abzulehnen. Daher setzen wir uns für ein Verbot von Fracking auf allen politischen Ebenen ein. Um den Energiebedarf zu decken, setzen wir stattdessen auf Effizienzverbesserungen, Einsparungen und generative Energien mit modernen Speichertechniken zum Ausgleich von Fluktuationen bei Energieproduktion und -verbrauch.}

\paragraph{Begründung}:

Übernommen von den \href{https://wiki.piratenpartei.de/NRW-Web:Grundsatzprogramm#Ablehnung_von_Fracking}{Piraten NRW} bzw. aus dem \href{https://lqfb.piratenpartei.de/pp/initiative/show/2104.html}{Bundes-LQFB}, Text korrigiert und leicht abgeändert

siehe dortige Begründungen

% -----

\wahlprogramm{Kulturerhalt und -förderung (inkl. kulturelle Vielfalt vs. Prestigeobjekte)}
\antrag{Stephan Schurig}\\
\version{13:31, 11. Apr. 2012}

\paragraph{Antragstext}:

Der Landesparteitag möge beschließen folgenden Abschnitt in das Wahlprogramm aufzunehmen: 

\einruecken{\textbf{Kulturerhalt und -förderung}

Wie ein demokratisches Gemeinwesen verfasst ist, wird treffend durch die Worte Friedrich Schillers beschrieben: „Die Kunst ist eine Tochter der Freiheit.“ Durch die Kulturförderung werden nicht nur die Kreativen geschützt, sondern auch unsere Haltung und Freiheitsrechte. Eine verantwortliche, transparente, anregende und nachhaltig gestaltende Kulturpolitik kräftigt eine zukunftsorientierte, vielfältige und humane Gesellschaft. Diese Politik muss die notwendigen Rahmenbedingungen für eine freie Entfaltung von Kunst und Kultur schaffen – sie darf diese nicht bewerten oder vereinnahmen.

Die kulturelle Freizügigkeit und Vielfalt sollen durch geförderten Freiraum und unter Berücksichtigung der Rechte der Anwohner verteidigt werden. Behörden sollen ihre Ermessensspielräume nutzen, um zugunsten von Kunst- und Kulturinitiativen zu entscheiden. Das Kulturleben soll sich auch als Wirtschaftsfaktor und Vernetzungsplattform lebendig weiterentwickeln. Kulturentwicklungsplanung ist vielschichtig und muss die kulturelle Bildung, Betätigung und Mitwirkung des Bürgers sowie die Künste und die Kulturwirtschaft aufeinander abstimmen und die dafür notwendigen Ressourcen und Verfahren definieren. Die Piratenpartei ist bestrebt, die Förderstruktur von Kunst und Kultur möglichst stabil zu halten. Bei einzelnen Sparten sollte auch in Wirtschaftskrisen nicht so stark gekürzt werden, dass ihre jeweilige Existenz gefährdet ist, denn im Gegensatz zu materiellen Werten kann eine verlorene kulturelle Infrastruktur nur langsam wieder aufgebaut werden.

Für die PIRATEN steht die Förderung kultureller Vielfalt über der einzelner Prestigeobjekte. Kleine Kulturprojekte sind meist ehrenamtlich organisiert, erreichen und beziehen in ihrer Gesamtheit aber deutlich mehr Menschen mit ein.

Der Zugang zu Kultureinrichtungen muss für alle Gesellschaftsschichten offen gehalten werden, damit diese Institutionen gesellschaftlich verankert sind. Des Weiteren müssen größtenteils öffentlich finanzierte Einrichtungen auch für die gesamte Bevölkerung zugänglich sein.}

\paragraph{Begründung}:

\begin{itemize}
\item übernommen aus dem Grundsatzprogramm LV Berlin (Dank an Alex)
\item Vorletzter Abschnitt zur Inititative von alexkid hinzugefügt ({\Gu}Für die PIRATEN...{\Go})
\item Dank an Lennstar und zig fürs Korrekturlesen!
\end{itemize}
