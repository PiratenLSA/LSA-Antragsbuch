\section{Positionspapier}
\positionspapier{Ablehnung von Versuchen der Firma BASF mit gentechnisch veränderten Kartoffeln}
\antrag{Spearmind}\\
\version{18:50, 12. Apr. 2012}

\paragraph{Antragstext}:

Der Landesparteitag möge beschließen:

\einruecken{Die Piraten lehnen Versuche der Firma BASF mit gentechnisch veränderten Kartoffeln im Lande entschieden ab und fordern die Einstellung betreffender Planungen. Arbeitsplätze, die im Zusammenhang mit der Verbreitung gentechnisch veränderten Saatguts stehen, erklären wir als unerwünscht.}

\begruendung{{\Gu}Die Gentechnik-Tochter des Chemiekonzerns BASF will trotz der Verlagerung ihres Sitzes in die USA in Europa gentechnisch veränderte Kartoffeln testen. In diesem Jahr gebe es in Sachsen-Anhalt sowie in Schweden und in den Niederlanden Feldversuche mit drei Sorten, für die EU-Genehmigungsverfahren liefen, kündigte das Chemieunternehmen am Donnerstag in Ludwigshafen an.{\Go} Mitteldeutsche Zeitung vom 10.04.2012

Trotz der Aussage zu Anfang des Jahres, Europa sei kein guter Markt für die Grüne Gentechnik und der Verlagerung von Geschäftstätigkeit der Sparte "Plant Science" in die USA plant BASF Versuche mit gentechnisch veränderten Kartoffeln auf Feldern in Sachsen Anhalt. Die Firmen Syngenta und Bayer stellten bereits 2004 sämtliche Gentech-Saatgut Feldversuche auf deutschen Äckern ein. Dazu gehört auch das Bekenntnis, dass man bestimmte Arbeitsplätze schlicht nicht möchte, weil Gentechnik nach aller Erkenntnis nicht verantwortbar ist und weithin von den Menschen abgelehnt wird. Das Bundesverfassungsgericht stufte die Grüne Gentechnik 2010 als Hochrisikotechnologie ein.

Unser Spitzenkandidat für die Landtagswahlen 2012 in Schleswig-Holstein, Torge Schmidt, setzt sich für eine "gentechnikfreie Zone" in seinem Bundesland ein. Bescheren wir den Piraten im Norden Rückenwind!

Das erfolgreiche Wahlprogramm der Piratenpartei Saarland liest sich wie folgt:

{\Gu}Keine Gentechnik in der Landwirtschaft Wir lehnen den Einsatz gentechnisch veränderter Nutzpflanzen in der saarländischen Landwirtschaft ab. Die Wechselwirkungen der veränderten Pflanzen mit der Umwelt und die Langzeitfolgen für Natur, Mensch und Tier lassen sich nicht abschätzen. Sicherheit und Gesundheit der Bürger haben Vorrang vor den Profiten Einzelner.{\Go}

Die Berliner Piraten haben die Gentechnik satt und sind der Meinung dass {\Gu}Massentierhaltung und Genmanipulation Artikel 20a des Grundgesetzes aushöhlen{\Go}.}
