\section{Positionspapier}
\positionspapier{Ablehnung von Versuchen der Firma BASF mit gentechnisch veränderten Kartoffeln}
\antrag{Spearmind}\\
\version{18:50, 12. Apr. 2012}

\paragraph{Antragstext}:

Der Landesparteitag möge beschließen:

\einruecken{Die Piraten lehnen Versuche der Firma BASF mit gentechnisch veränderten Kartoffeln im Lande entschieden ab und fordern die Einstellung betreffender Planungen. Arbeitsplätze, die im Zusammenhang mit der Verbreitung gentechnisch veränderten Saatguts stehen, erklären wir als unerwünscht.}

\begruendung{{\Gu}Die Gentechnik-Tochter des Chemiekonzerns BASF will trotz der Verlagerung ihres Sitzes in die USA in Europa gentechnisch veränderte Kartoffeln testen. In diesem Jahr gebe es in Sachsen-Anhalt sowie in Schweden und in den Niederlanden Feldversuche mit drei Sorten, für die EU-Genehmigungsverfahren liefen, kündigte das Chemieunternehmen am Donnerstag in Ludwigshafen an.{\Go} Mitteldeutsche Zeitung vom 10.04.2012

Trotz der Aussage zu Anfang des Jahres, Europa sei kein guter Markt für die Grüne Gentechnik und der Verlagerung von Geschäftstätigkeit der Sparte "Plant Science" in die USA plant BASF Versuche mit gentechnisch veränderten Kartoffeln auf Feldern in Sachsen Anhalt. Die Firmen Syngenta und Bayer stellten bereits 2004 sämtliche Gentech-Saatgut Feldversuche auf deutschen Äckern ein. Dazu gehört auch das Bekenntnis, dass man bestimmte Arbeitsplätze schlicht nicht möchte, weil Gentechnik nach aller Erkenntnis nicht verantwortbar ist und weithin von den Menschen abgelehnt wird. Das Bundesverfassungsgericht stufte die Grüne Gentechnik 2010 als Hochrisikotechnologie ein.

Unser Spitzenkandidat für die Landtagswahlen 2012 in Schleswig-Holstein, Torge Schmidt, setzt sich für eine "gentechnikfreie Zone" in seinem Bundesland ein. Bescheren wir den Piraten im Norden Rückenwind!

Das erfolgreiche Wahlprogramm der Piratenpartei Saarland liest sich wie folgt:

{\Gu}Keine Gentechnik in der Landwirtschaft Wir lehnen den Einsatz gentechnisch veränderter Nutzpflanzen in der saarländischen Landwirtschaft ab. Die Wechselwirkungen der veränderten Pflanzen mit der Umwelt und die Langzeitfolgen für Natur, Mensch und Tier lassen sich nicht abschätzen. Sicherheit und Gesundheit der Bürger haben Vorrang vor den Profiten Einzelner.{\Go}

Die Berliner Piraten haben die Gentechnik satt und sind der Meinung dass {\Gu}Massentierhaltung und Genmanipulation Artikel 20a des Grundgesetzes aushöhlen{\Go}.}

% -----

\positionspapier{Geldreformen als Schlüssel gegen die Finanzkrise}
\antrag{Prof. Dr. Michael Rost, Johannes Paul, Andreas Rieger}\\
\version{16:36, 14. Apr. 2012}

\paragraph{Antragstext}:

Die Piratenpartei Sachsen-Anhalt setzt sich für Reformen im Geldsystem ein. Insbesondere sollen durch parallele umlaufgesicherte Geldsysteme alternative Finanzierungsmöglichkeiten für die Kommunen geschaffen werden, regionale Wirtschaftskreisläufe unterstützt werden und insgesamt damit die Umverteilung von Arm zu Reich durch unser Geldsystem unterbrochen und schrittweise abgelöst werden.

\begruendung{Noch nie gab es gleichzeitig eine so hohe Verschuldung und so explodierende Geldvermögen. Die Konsequenz ist eine zunehmende Ratlosigkeit in den Verwaltungen, in der Politik, bei Regierungen und etablierten Parteien. Notwendige Arbeit (wie die Sanierung von Schulen, Lehrertätigkeit, Infrastruktur ist zunehmend weniger finanzierbar), während gleichzeitig Menschen, die diese Tätigkeit ausführen könnten, in Arbeitslosigkeit geschickt werden.}

% ------

\positionspapier{Herabsetzung des aktiven Wahlalters bei Landtagswahlen auf 0 Jahre}
\antrag{Stephan Schurig}\\
\version{18:50, 12. Apr. 2012}

\paragraph{Antragstext}:

Es wird beantragt ins Positionspapier folgende Forderung einzufügen:

\einruecken{Die Piratenpartei fordert die vollständige Aufhebung des notwendigen Mindestalters zur Wahrnehmung des aktiven Wahlrechts bei Landtagswahlenund damit eine Anpassung des § 42 Abs. 2 der Verfassung des Landes Sachsen-Anhalt. Das aktive Wahlrecht soll ab der Geburt von jedem Bürger wahrgenommen werden können. Die erstmalige Ausübung dieses Wahlrechts erfordert für Unter-16-Jährige die selbständige Eintragung in eine Wählerliste. Eine Stellvertreterwahl durch Erziehungsberechtigte lehnen wir ab.}

\begruendung{
Das Wahlrecht ist ein fundamentales Menschenrecht, kein freundlicherweise gewährtes Privileg. Dieses Recht ist in Artikel 21 der allgemein Erklärung der Menschenrechte verbrieft. Aus der Rechtsprechung des Bundesverfassungsgerichts geht hervor, dass Kinder ab ihrer Geburt zum Staatsvolk zählen und ihnen die Grund- und Bürgerrechte des Grundgesetzes in vollem Umfang zustehen. Einschränkungen dieser Grundrechte müssen sorgfältig begründet werden. Die Rechtsprechung des Verfassungsgerichts steht in dieser Hinsicht im Einklang mit der UN-Konvention für die Rechte des Kindes$^1$, der Gesetzgeber hinkt diesem Anspruch aber weiterhin hinterher. Für uns ist es nicht nachvollziehbar, warum es zum Schutz der Demokratie notwendig ist, Minderjährige von der Wahl auszuschließen und ihnen ihr Abstimmungsrecht zu nehmen. Im Gegenteil stellt ihre Beteiligung in unseren Augen eine Bereicherung dar. Vor diesem Hintergrund ist die Beschränkung des Wahlrechts in Art. 38 II GG auf Menschen über 18 Jahre nicht hinnehmbar.

Demokratie ist kein Instrument zur Wahrheitsfindung, sondern trägt der Idee Rechnung, dass wir nur dann Macht über Menschen ausüben dürfen, wenn sie darüber mitentscheiden und ihre eigenen Interessen in die Waagschale werfen durften, wer diese Macht wie ausübt. Der Gedanke, z.B. Menschen das Wahlrecht zu entziehen, die im Gespräch Beeinflussbarkeit oder politische Unkenntnis zeigen, erscheint uns daher unangemessen. Ebensowenig dürfen wir daher Kindern und Jugendlichen mit dem Argument, ihnen fehlte es noch an politischer Kenntnis oder sie seien zu beeinflussbar, das Wahlrecht vorenthalten: Dies gilt erstens nicht für alle (und zudem für viele Erwachsene), zweitens geht es bei Demokratie eben um die Berücksichtigung des Willens aller im gleichen Maße und nicht etwa darum, die {\Gu}politische Wahrheit{\Go} herauszufinden. Einen Willen und politische Interessen haben Kinder und Jugendliche aber sehr wohl$^2$. Eine Regierung, die von ca. 20\% derer, über deren Rechte und Pflichten sie bestimmen darf, nicht mitgewählt werden durfte, ist nicht demokratisch legitimiert.

Dass uns ein Kinderwahlrecht auf den ersten Blick merkwürdig vorkommt, ist unserer historischen Situation geschuldet und ging vielen Menschen bezüglich des Frauenwahlrechts einmal ebenso. Die Jungen Piraten behaupten von sich, unvoreingenommen neue Wege zu durchdenken und zu beschreiten, wenn die besten Argumente für sie sprechen. Das ist hier der Fall.

Die Grenzziehung zwischen Kind und Jugendlichen ist wissenschaftlich nicht einheitlich definiert$^3$. {\Gu}Kindheit{\Go} ist eine historische Konstruktion der gesellschaftlichen Verhältnisse während der Industrialisierung. Die Unterscheidung von Gesellschaftsmitgliedern nach ihrem Alter ist kein biologischer Diskurs, sondern ein Erziehungsdiskurs und hängt mit gesellschaftlichen Machtverhältnissen zusammen. Kinder werden nicht als Subjekte anerkannt, deren Interessen in der Gegenwart zu berücksichtigen sind, sondern nur im Hinblick auf ihre Zukunft und ihr Potential, zum vollwertigen Mitglied der Gesellschaft zu werden, betrachtet. Der gesellschaftliche Blick auf Kinder ist damit fast immer erwachsenenzentriert$^4$.

Bei der Bewertung des aktuellen Wahlrechts ab 18 - bzw. in einigen Fällen ab 16 Jahren - gilt es zu bedenken, dass alle Beschränkungen des Wahlrechts historische Relikte sind und eine Koppelung des Wahlrechts an die Volljährigkeit keinesfalls die einzig denkbare Möglichkeit ist. Die ersten {\Gu}Demokratien{\Go} schlossen Frauen, Nichtathener und Sklaven aus. Das Wahlrecht zur ersten Wahl im Deutschen Reich im Jahre 1871 besaßen lediglich Männer ab 25 Jahre, was zur damaligen Zeit den Ausschluss eines hohen Bevölkerungsanteils zur Folge hatte. Im Jahr 1970 wurde das aktive Wahlrecht in der Bundesrepublik Deutschland von 21 Jahren auf 18 Jahre abgesenkt. Das Wahlrecht ist historisch gewachsen und nicht an objektiven Kriterien festgemacht. Die Grenze von 18 Jahren ist willkürlich.

Wer wählen darf, interessiert sich mehr für Politik. Durch das fehlende Wahlrecht werden Kinder und Jugendliche zu spät an der demokratischen Kultur beteiligt und somit die Chance vertan, sie früh für Politik zu begeistern und einzubinden. Es ist daher wünschenswert, Kindern und Jugendlichen eine möglichst frühe Beteiligung an Wahlen zu ermöglichen. Politisches Desinteresse würde nicht mehr 18 Jahre eingeübt, stattdessen könnten sich Kinder und Jugendliche demokratisch einbringen, würden sich mehr informieren und es bestünden mehr Anreize, ihnen politische Informationsangebote zu machen. Die politische Bildung der Bevölkerung würde nachhaltig besser. Den durch eine Senkung des Wahlalters auftretenden politischen Fragen von Kindern und Jugendlichen ist auch durch ein stärkeres Gewicht der politischen Bildung im Schulalltag Rechnung zu tragen. NGOs wie z.B. die Greenpeace-Jugend ermöglichen eine Mitgliedschaft ab 14 Jahren, die Jugendfeuerwehr ab 10 Jahren und das Deutsche Jugendrotkreuz ab 6 Jahren. Bereits im Kindesalter werden Menschen also in gesellschaftlich verantwortungsvolle (zukünftige) Positionen einbezogen und begleitet. Es gibt bereits viele kommunale Beteiligungsprojekte mit Kindern und Jugendlichen, beispielsweise Bürgerhaushalte oder Projekte zur Gestaltung der eigenen Stadt bzw. Gemeinde$^5$. Österreich ermöglichte mit der Wahlrechtsreform 2007 allen Bürgerinnen und Bürgern bereits ab 16 Jahren eine Teilnahme an allen Wahlen im Land$^6$.

Die Nicht-Anerkennung von Kindern und Jugendlichen als politische Subjekte basiert auf mehreren Faktoren, die große Parallelen zum Ausschluss von Frauen aufzeigen$^1$:

\begin{itemize}
\item Kinder und Jugendliche sind im beruflichen Umfeld als Partner unbekannt und werden dadurch nicht akzeptiert, bzw. es fehlt die Erfahrung, mit ihnen umzugehen und sie in Entscheidungsprozesse einzubinden,
\item es herrscht ein Adultismus (analog zum Sexismus oder Rassismus), der aus der gesellschaftlichen Realität der Erwachsenenherrschaft hervorgeht,
\item Kinder und Jugendliche werden kaum als öffentliche Personen wahrgenommen und vornehmlich der Privatsphäre (Familie) zugeschrieben, mit der Ausnahme, wenn sie ein öffentliches Ärgernis darstellen,
\item Exklusion von der politischen Partizipation wird häufig als {\Gu}Schutz{\Go} vor sich selbst (z.B. wegen Empfänglichkeit für rassistische und totalitäre Positionen) oder Überforderung begründet.
\end{itemize}

Empfänglichkeit für Rassismus und Totalitarismus ist trotz landläufiger Meinung kein Phänomen, das nur unter Jugendlichen und jungen Erwachsenen auftritt. Andererseits kann politische Partizipation hier sogar präventiv wirken$^1$. Über 75\% aller Jugendlichen bezeichnen die Demokratie als geeignetste Staatsform. Sie sprechen sich für das Grundgesetz aus, sind aber mit der Realisierung demokratischer Ideale und Strukturen unzufrieden$^7$. Insgesamt sind die Ansprüche der Jugendlichen gegenüber der Politik hoch, so erwarten sie von Politikern Ehrlichkeit, Kompromissbereitschaft, Mitbestimmungsrechte, die Fähigkeit zur Durchsetzung politischer Entscheidungen und eine stärkere Einbindung der Interessen Jugendlicher$^3$. Nichtsdestotrotz bleiben viele Jugendliche gegenüber dem Parteiensystem skeptisch und Politikern gegenüber misstrauisch, was teilweise ihre generelle Zurückhaltung beim Wählen erklärt. So erklären beispielsweise 35-40\% aller Jugendlichen zwischen 12 und 17 Jahren in einer Umfrage, dass es keine Partei gebe, die ihre Interessen vertrete und sie deswegen auch nicht wählen gehen würden$^7$.

Ein häufig formulierter Einwand gegen die Absenkung oder Aufhebung des Wahlalters ist, vielen Kindern und Jugendlichen fehle die notwendige Reife. Man kann allerdings nicht abstreiten, dass Kinder und Jugendliche bereits in der Lage sind, sich eigenständige Gedanken zu vielgestaltigen Problemen zu machen und ihre eigenen Wertungen zu finden. Es ist anmaßend, eine zwar womöglich mit geringer Lebenserfahrung getroffene, aber dennoch durchaus überlegte Entscheidung oder Wertung aus einem erwachsenen Blickwinkel per se als unreif zu deklarieren, zumal das Reifekriterium bei der Wahlentscheidung Erwachsener keine Rolle spielt. Selbst wenn eine Senkung des Wahlalters mitunter zu naiven und unsachgemäßen Entscheidungen führte - angenommen, eine objektive Bewertung wäre hier möglich - muss Kindern und Jugendlichen auch die Möglichkeit eingeräumt werden, Fehler zu machen und aus ihnen zu lernen. Eine Gefahr für die Demokratie wäre aus dieser Möglichkeit nicht abzuleiten, zumal die Unter-18-Jährigen nur einen geringen Teil der gesamten Wählerschaft ausmachen würden. Daher ist die Sorge über die Beschädigung der Demokratie durch massenhaft unreife Wähler unbegründet, zumal sie zu dem gewonnenen rechtlichen Gehör der Betroffenen in keinem Verhältnis stünde.

Teilhaberechte bedeuten immer auch, Macht abzugeben, in diesem Fall aus den Händen der Erwachsenen in die Hände junger Menschen. Der Ausschluss von Kindern und Jugendlichen vom Wahlrecht bedeutet nicht zuletzt, dass es keine Verpflichtung bzw. keine Verantwortlichkeit der politischen Akteure gibt, die Interessen dieser Altersgruppe zu berücksichtigen und sich vor ihr zu rechtfertigen. Artikel 20 GG formuliert, dass alle Staatsgewalt vom Volke ausgeht, Abgeordnete sollen nach Artikel 38 GG Vertreter des ganzen Volkes sein. In der Praxis stellt sich die Situation allerdings anders dar, wenn rund 15 Millionen Unter-18-Jährige keine Möglichkeit besitzen ihre Stimme abzugeben. Solange Kinder und Jugendliche nicht wählen können, werden ihre Interessen weniger berücksichtigt. Generationengerechtigkeit, Klimaschutz etc. können so schlecht erreicht werden und Probleme werden auf die junge Generation abgeschoben.

Die Absenkung des Wahlalters erfordert auch eine besondere Sorgfalt der Wahlämter und Wahlhelfer im Umgang mit den Jungwählern. Um einem potentiellen Mißbrauch vorzubeugen, müssen die zuständigen Sachbearbeiter entsprechend unterwiesen und vorbereitet werden. Eine Missbrauchsgefahr von Rechten besteht in einer Demokratie immer und unabhängig vom Alter, eine wehrhafte und wertstabile Demokratie ficht das aber nicht an.

Erstwähler, die unter 16 Jahre alt sind, müssen selbständig einmalig ihren Willen zu wählen persönlich in dem für Sie zuständigen Wahlamt beurkunden. Sobald sie als Wähler erfasst sind, erhalten sie zu jeder anstehenden Wahl, zu der sie wahlberechtigt sind, eine Einladung. Eine vollautomatische Erfassung aller Erstwähler unter 16 findet nicht statt. Wahlrecht ist keine Wahlpflicht. Dieses Recht wahrzunehmen, ist die Entscheidung des einzelnen Wählers, der damit auch eine Verantwortung übernimmt.

Es ist jedoch klar, dass allein die Herabsetzung des aktiven Wahlrechts nur ein kleines Glied in einer ganzen Kette von Maßnahmen sein kann, um Jugendliche politisch zu involvieren, ihnen damit die Chance zu geben ihre und unsere Gesellschaft von heute und von morgen zu gestalten. Diese Forderung kann damit lediglich als Anfang einer deutlichen Wendung in der Politik dienen. Kinder und Jugendliche brauchen mehr Begleitung und Ansprechpartner als Erwachsene, um ihre Interessen in politisches Wissen zu transformieren und dieses schließlich für politische Partizipation zu verwenden. Dabei müssen auch politische Diskussionen in Schulen geführt werden, demokratische Mitbestimmungsrechte an Schulen ausgebaut werden und Kinder und Jugendliche in allen Lebensbereichen die Chance erhalten, ihre Lebenswelt fair und ihrem Alter entsprechend zu gestalten.

\textbf{Quellen:}

$^1$ Bertelsmann Stiftung (Hrsg.) (2007): Mehr Partizipation wagen. Argumente für eine verstärkte Beteiligung von Kindern und Jugendlichen. 2. Aufl., Gütersloh.

$^2$ van Deth, J. W., Abendschön, S., Rathke, J. \& M. Vollmar (2007): Kinder und Politik. Politische Einstellungen von jungen Kindern im ersten Grundschuljahr. Wiesbaden.

$^3$ Maßlo, J. (2010): Jugendliche in der Politik. Chancen und Probleme einer institutionalisierten Jugendbeteiligung am Beispiel des Kinder- und Jugendbeirats der Stadt Reinbek. Wiesbaden.

$^4$ Abels, H. \& A. König (2010): Sozialisation. Soziologische Antworten auf die Frage, wie wir werden, was wir sind, wie gesellschaftliche Ordnung möglich ist und wie Theorien der Gesellschaft und der Identität ineinander spielen. Wiesbaden.

$^5$ Gernbauer, K. (2008): Geleitwort. Beteiligung von Jugendlichen als politische Herausforderung. In: Ködelpeter, T. \& U. Nitschke (Hrsg.): Jugendliche planen und gestalten Lebenswelten. Partizipation als Antwort auf den gesellschaftlichen Wandel. Wiesbaden.

$^6$ {\Gu}Parlamentskorrespondenz Nr. 510 vom 21.06.2007. \href{http://www.parlinkom.gv.at/PAKT/PR/JAHR_2007/PK0510/index.shtml}{Wahlrechtsreform 2007 passiert den Bundesrat}{\Go} (Abruf am 22.01.2012)

$^7$ Hurrelmann, K. (o.J.): \href{http://gedankensex.de/2011/08/23/jugendliche-an-die-wahlurnen/}{Jugendliche an die Wahlurnen} (Abruf am 22.01.2012)
}
