\section{Programmänderungsanträge}
\programm{Zivilklausel für Hochschulen}
\antrag{Stephan Schurig}
\lqfb{http://lqfb.piraten-lsa.de/lsa/initiative/show/151.html}{18}{0}{2}

\paragraph{Antragstext}:

Der Landesparteitag möge beschließen folgenden Programmpunkt in das Wahlprogramm aufzunehmen:

\einruecken{\textbf{Zivilklausel für Hochschulen}

Die Piratenpartei Sachsen-Anhalt setzt sich dafür ein, eine Zivilklausel in das Landeshochschulgesetz (HSG LSA) aufzunehmen. Forschung für die Rüstungsindustrie, für militärische Zwecke oder durch Drittmittel von Einrichtungen mit militärischem und rüstungsspezifischen Hintergrund (z.B. dem Bundesministerium der Verteidigung) lehnen wir ab. Sogenannte {\Gu}Dual-Use{\Go}-Güter (doppelter Verwendungszweck), wie z.B. Satellitentechnologie, müssen dabei besonders bewertet werden. Der Studierendenschaft und den Angestellten der Hochschule muss es möglich sein, eine Urabstimmungen mit möglichst kleiner Hürde einfordern zu können.

Die Freiheit der Lehre kann nur dann gegeben sein, wenn sie zur zivilen und friedlichen Forschung eingesetzt wird. Daher fordern wir zusätzlich alle Informationen zu früheren, momentanen und zukünftig geplanten Forschungsprojekten mit militärischem Hintergrund zu veröffentlichen, um das Ausmaß bzw. den aktuellen Einfluss auf die Hochschulen festzustellen.

Die PIRATEN sind sich allerdings im Klaren, dass eine Zivilklausel für Hochschulen keine allgemeine Problemlösung darstellt, dass kriegerische Handlungen oder Kriegsmaterial in der Welt existieren. Stattdessen soll die Kriegs-, Friedens- bzw. Konfliktforschung, d.h. die Analyse von internationalen Konflikten verstärkt an Hochschulen angeboten und gefördert werden, aber dahingehend keine Politikberatung zur Optimierung militärischer Interventionen darstellen.}

\begruendung{Wir setzen uns für eine Zivilklausel ein, die nicht lediglich eine Selbstverpflichtung einer Hochschule, sondern gesetzlich verankert ist. Dies war bereits zwischen 1993 und 2002 im Landeshochschulgesetz des Landes Niedersachsen verankert (Rot-Grün), wurde von der schwarz-gelben Regierung aber wieder abgeschafft.

Hinweis zur Freiheit der Wissenschaftler\_innen:

\begin{quote}
Der weisungsfrei, selbständig arbeitende (Hochschul-)Wissenschaftler ist durch sein individuelles Grundrecht davor geschützt, in Bezug auf sein Forschungsthema und seine Methodik unmittelbar verpflichtende Ge- oder Verbote befolgen zu müssen. Jedoch muss er unter Umständen in Kauf nehmen, dass das Gros der zu verteilenden Mittel dem von ihm abgelehnten, vom zuständigen Organ jedoch rechtmäßig beschlossenen Forschungsschwerpunkt zufließt.
\end{quote}

Quelle: \url{http://www.boeckler.de/pdf/mbf\_gutachten\_denninger\_2009.pdf} (Seite 21)

\textbf{Informationen}

\begin{itemize}
\item \href{http://al.blogsport.de/zivilklausel/zivilklausel-was-ist-das/}{Zivilklausel -- Was ist das?}
\item \href{http://www.zivilklausel.uni-koeln.de/faz\_wenn-sie-dir-morgen-befehlen.htm}{FAZ v. 12.01.2011}
\item \href{http://www.boeckler.de/pdf/mbf\_gutachten\_denninger\_2009.pdf}{Zur Zulässigkeit einer so genannten „Zivilklausel” im Errichtungsgesetz für das geplante Karlsruher Institut für Technologie (KIT)}
\item \href{http://www.gew.de/Binaries/Binary79165/Gemeinsame\%20Erkl\%C3\%A4rung\_HS\%20f\%20Frieden\_Z.pdf}{Gemeinsame Erklärung der Initiative {\Gu}Hochschule für den Frieden -- ja zur Zivilklausel{\Go}}
\item \href{http://de.wikipedia.org/wiki/Zivilklausel}{Zivilklause}
\item \href{http://de.wikipedia.org/wiki/Dual\_Use}{Dual Use}
\item \href{http://zivilklausel-ffm.info/}{AK Zivilklausel Uni Frankfurt}
\item \href{http://freie-radios.net/43872}{Forschung für den Unfrieden: Wer betreibt wo Rüstungsforschung in Deutschland?}
\item \href{http://freie-radios.net/48009}{Zur Zivilklausel an Hochschulen und Universitäten}
\item \href{http://freie-radios.net/41584}{Effizienter Krieg führen: Friedensforschung als Politikberatung}
\end{itemize}
}

% -----

\programm{Kennzeichnungspflicht von Medikamenten mit Sucht - bzw. Abhängigkeitspotential }
\antrag{Stephan Schurig}
\lqfb{http://lqfb.piraten-lsa.de/lsa/initiative/show/139.html}{15}{0}{1}

\paragraph{Antragstext}:

Der Landesparteitag möge beschließen folgenden Antrag ins Wahlprogramm unter dem Punkt {\Gu}Drogen- und Suchtpolitik{\Go} aufzunehmen:

\einruecken{\textbf{Kennzeichnungspflicht von Medikamenten mit Sucht - bzw. Abhängigkeitspotential}

Um auf die Suchtgefahr bei bestimmten Medikamenten aufmerksam zu machen, müssen die Pharmahersteller in die Pflicht genommen werden. Wie bei Zigaretten üblich, sollten vereinheitlichte Warnhinweise auf die Medikamentenverpackungen aufgedruckt werden. Aus diesen muss hervorgehen, dass es sich bei dem Medikament um eine Arznei handelt, die ein Suchtrisiko birgt. Diese Warnhinweise sollen Patienten sensibilisieren und auf die Gefahr einer Sucht hinweisen.

Der Grund: Die oft sehr versteckt in der Packungsbeilage beschriebenen Hinweise werden allzu leicht nicht wahrgenommen. Darüber hinaus könnte von Apotheken bei der Ausgabe des Medikamentes ein Informationsblatt zum Thema "Suchtgefahren bei Medikamenten" angeboten werden. Dieses soll ein Angebot mit weiteren Informationen darstellen und Hilfe beim Auffinden von geeigneten Stellen für Hilfesuchende bieten.}

\begruendung{Übernommen aus dem \href{http://www.piratenpartei-nrw.de/politik/drogenpolitik/kennzeichnungspflicht-von-medikamenten-mit-sucht-bzw-abhangigkeitspotenzial/}{Wahlprogramm von NRW}. Begründungstext nicht vorhanden, der Antrag ist aber ziemlich selbsterklärend.}

% -----

\programm{Kulturelle Förderung für Suchtgefährdete und Suchtkranke}
\antrag{Stephan Schurig}
\lqfb{http://lqfb.piraten-lsa.de/lsa/initiative/show/140.html}{11}{2}{1}

\paragraph{Antragstext}:

Der Landesparteitag möge beschließen folgenden Antrag ins Wahlprogramm unter dem Punkt {\Gu}Drogen- und Suchtpolitik{\Go} aufzunehmen:

\einruecken{\textbf{Kulturelle Förderung für Suchtgefährdete und Suchtkranke}

Insbesondere Konsumenten illegaler Drogen sind häufig gefangen im Kreislauf der Drogenbeschaffung, des Drogenkonsums und der Bewältigung ihres Tagesablaufs. Ein Ausbruch aus diesem Kreislauf ist ohne fremde Unterstützung oft nicht möglich. Suchtberatungsstellen in den einzelnen Kommunen tragen bereits seit Jahren ihren Teil dazu bei, einen organisierten Tagesablauf zu ermöglichen - sei es durch Streetworker, Szenetreffpunkte in Krankenhäusern und/oder JVAs oder beispielsweise Kontaktläden, in denen es den Abhängigen ermöglicht wird ihren Alltag zu organisieren.

Einige kommunale Beratungsstellen gehen noch einen Schritt weiter. Sie bieten ihrer Klientel die Möglichkeit sich am kulturellen Austausch zu beteiligen. Es werden Literatur-Lesungen von und für Abhängige, mit anschließender Diskussion angeboten; ebenso können eigene Texte präsentiert werden. Des Weiteren werden Kunst-Ausstellungen von Werken Abhängiger, sei es Malerei oder Fotografie in den Räumen der Beratungsstellen durchgeführt. Theaterprojekte sind nur ein weiteres Spektrum. Unterstützt werden Einrichtungen zudem durch lokale Kulturvereinigungen. Der Effekt dieser Maßnahme ist unbestritten, den Abhängigen werden neue Wege aufgezeigt, das Selbstbewusstsein wird gestärkt und eine Resozialisierung wird vorangetrieben.}

\begruendung{Übernommen aus dem Wahlprogramm von NRW (WP234 - Kulturelle Förderung für Suchtgefährdete und Suchtkranke). Begründungstext nicht vorhanden, der Antrag ist aber ziemlich selbsterklärend.}

% -----

\programm{Förderung gemeinfreier Werke und Kulturgüter}
\antrag{Stephan Schurig}
\lqfb{http://lqfb.piraten-lsa.de/lsa/initiative/show/135.html}{24}{0}{0}

\paragraph{Antragstext}:

Der Landesparteitag möge beschließen folgenden Abschnitt in das Grundsatz- oder Wahlprogramm (bitte Anregungen wohin):

\einruecken{\textbf{Förderung gemeinfreier Werke und Kulturgüter}

Die Piratenpartei Sachsen-Anhalt fordert den besseren Zugang zu Kulturgütern und geschichtlichen Dokumenten, deren Urheberrechte nach deutschem Urheberrecht (UrhG) ausgelaufen und welche damit gemeinfrei sind. Alle gemeinfreien Werke (Audio, Video, Bild, Text etc.) an denen das Land Sachsen-Anhalt die Urheberrechte besaß, sollen auf einer Online-Plattform kostenlos und in digitaler Form allen Menschen zur Verfügung gestellt werden. Dabei sollen möglichst alle Optionen ausgeschöpft werden, Bild und Ton barrierefrei zugänglich zu machen (z.B. durch Untertitel, Transkripte o.ä.). Weiterhin setzen sich die PIRATEN für bundesweite Projekte ein, die zum Ziel haben, die Bereitstellung gemeinfreier Werke voranzutreiben und für alle frei verfügbar zu machen.}

\begruendung{Gemeinfreie Kulturgüter sind momentan kaum zu finden bzw. werden nicht gefördert. Im Gegensatz zu den USA, in denen viele Werke unter Public Domain zur Verfügung stehen (z.B. unter \url{http://www.archive.org}), erlaubt das momentane Urheberrecht nur wenig Spielraum. So werden bspw. Filme erst gemeinfrei (d.h. das Urheberrecht läuft aus), wenn Hauptregisseur, Urheber des Drehbuchs, Urheber der Dialoge und Komponist der Musik 70 Jahre verstorben sind [1]. Eine Förderung solcher gemeinfreien Kulturgüter ist anzustreben.

[1] \url{http://www.schulen-ans-netz.de/uploads/tx\_templavoila/Urheberrecht.pdf} (Seite 8)}

% -----

\programm{Zufällige Reihenfolge der Parteien und Kandidaten auf Wahlstimmzetteln}
\antrag{Stephan Schurig}
\lqfb{http://lqfb.piraten-lsa.de/lsa/initiative/show/158.html}{17}{2}{0}

\paragraph{Antragstext}:

Der Landesparteitag möge beschließen folgenden Punkt in das Wahlprogramm aufzunehmen:

\einruecken{\textbf{Zufällige Reihenfolge der Parteien auf Wahlstimmzetteln}

Die PIRATEN setzen sich für eine Änderung der Wahlgesetze auf allen Ebenen ein, um eine zufällige Positionierung der Parteien und Kandidaten auf jeweils allen Stimmzetteln zu gewährleisten. Die Reihenfolge soll dabei nicht mehr durch die Anzahl der Stimmen bei der letzten Wahl festgelegt sein, sondern zufällig ausgelost werden. Dies soll den sogenannten {\Gu}Primacy-Effect{\Go} bzw. Primäreffekt (frühere Informationen haben einen stärkeren Effekt, als spätere) verhindern, welcher der Partei zugute kommt, die bei der letzten Wahl die meisten Stimmen bekommen hat. Der Unterschied zwischen der oberen und einer unteren Position erbrachte laut einigen Studien im Schnitt zwischen 2 bis 5 Prozent mehr Wählerstimmen. Eine schrittweise Einführung beginnend auf kommunaler Ebene soll weitere Erfahrungswerte für eine landes- und bundesweite Umsetzung liefern.}

\begruendung{Signifikante Effekte der Namensreihenfolge tauchten in 43\% der Test auf. 89\% dieser Effekte waren Primäreffekte, wobei Kandidat\_innen an oberster Stelle bevorzugt wurden. Im Durchschnitt stiegen die Stimmanteile um 2,33\% im Vergleich zu einer Platzierung ganz hinten (siehe Miller \& Krosnick, 1998, S. 315).

\begin{quote}
Bei Wahlen, in deren Vorfeld wenig Informationen zur Verfügung stehen oder Informationsaufnahme mit hohen Anstrengungen oder Kosten verbunden ist, hat der primacy Effekt einen hoch signifikanten Einfluss auf die Wahlentscheidung, selbst wenn doch vorhandene cues wie Geschlecht der Kandidaten oder der Amtsinhabervorteil mit in die Betrachtung einbezogen werden. So konnte Brockington (2003: 15) zeigen, dass bei general elections, die von primary elections unterschieden werden, '(...) on average, a candidate would suffer a vote share loss of 5.2 percent at the precinct level for each place down on the ballot his or her name was found.' (...) Wie bereits weiter oben angedeutet und wie im Vorfeld der Untersuchung unterstellt, könnte das Auftreten oder die Berücksichtigung der Stimmzettelpositionierung bei der Wahlentscheidung abnehmen, je mehr Informationen und je mehr Wissen im Vorfeld einer Entscheidung erlangt wird und für die Entscheidung herangezogen werden kann
\end{quote}

(Jungwirth, 2007, S. 53).

Wir streben an, dass Machtstrukturen sich nicht verfestigen bzw. reproduzieren. Daher ist eine zufällige Reihenfolge der Parteien auf einem Stimmzettel anzustreben.

\href{https://lqfb.piratenpartei.de/pp/initiative/show/3435.html}{Der Antrag wurde auch auf Bundesebene im LQFB eingestellt.}

\textbf{Quellen:}

\begin{itemize}
\item \href{http://comm.stanford.edu/faculty/krosnick/docs/1998/1998\%20impact\%20of\%20candidate\%20name\%20order\%20on\%20election\%20outcomes.pdf}{Miller \& Krosnick (1998): The Impact of Candidate Name Order on Election Outcomes, Public Opinion Quarterly, Vol. 62, No. 3, pp. 291-330.}
\item \href{http://books.google.de/books?id=cWrUEPisWOUC\&lpg=PA53\&ots=ikhsT59MAz\&dq=primacy\%20effect\%20wahlen\&hl=de\&pg=PA54\#v=onepage\&q=primacy\%20effect\%20wahlen\&f=false}{Jungwirth (2007): Politisches Wissen - Möglichkeiten und Grenzen der empirischen Erfassung und gesellschaftliche Bedeutung.}
\item \url{http://de.wikipedia.org/wiki/Prim\%C3\%A4reffekt}
\item \url{http://www.sueddeutsche.de/wissen/wahlen-und-psychologie-bizarre-waehlerentscheidungen-1.48456-4}
\end{itemize}
}

% -----

\programm{Kulturerhalt und -förderung - Fördertopf}
\antrag{Alexander Magnus}
\lqfb{http://lqfb.piraten-lsa.de/lsa/initiative/show/112.html}{19}{2}{2}

\paragraph{Antragstext}:

Der Landesparteitag möge beschließen, folgenden Text in das laufende Parteiprogramm aufzunehmen:

\einruecken{Die Piratenpartei Sachsen-Anhalt setzt sich dafür ein, einen Fördertopf einzurichten, um Initiativen, Organisationen und Vereinen, die der Förderung von Kultur, Bildung, politischer Aufklärung und Gleichbereichtigung dienen, zu unterstützen. Die Budgetgröße soll dabei von Mitteln des Wirtschafts- und Innenministeriums abgezogen werden. Die Förderung soll prinzipiell allen Vereinigungen, die der freiheitlich-demokratischen Grundordnung nicht entgegenstehen, möglich sein. Um auch kleineren Initiativen die Möglichkeit zu geben Mittel zu beantragen, müssen die Bedingungen für die Förderung möglichst niedrigschwellig sein. Sie dürfen nicht an politische Forderungen oder eine Gesinnungsprüfung gebunden sein. Eine paritätisch besetzte Komission aus Vertretern des Bildungs-, Kultus- und Innenministeriums sowie Vertretern von oben genannten Vereinen soll die Förderbedingungen nach den vorgenannten Maßgaben erarbeiten und über Förderungsanträge entscheiden. Die Gesamtgröße der Kommission soll in der parlamentarischen Verhandlung festgelegt werden. Vereinsvertretern, die sich um einen Sitz in der Kommission bewerben, darf dieser Sitz nicht verwehrt werden, es sei denn, die Ausrichtung des Vereins widerspricht in seinen Grundsätzen der freiheitlich-demokratischen Ordnung. Die Kommission soll mehrmals jährlich öffentlich tagen, um die Förderung und die Förderbedingungen kontinuierlich zu evaluieren und zu verbessern. Sämtliche Ergebnisse und Förderungsbescheide der Kommissionstagungen müssen in maschinenlesbarer Form und barrierefrei kostenlos im Internet zur Verfügung gestellt werden.}

\begruendung{Mit diesem Antrag soll konkrete Hilfe und Förderung gerade für kleinere Initiativen und Vereine geboten werden, die durch die nur schleppend bearbeitende und tendentiell auf Prestigeprojekte ausgelegte Verwaltung keine oder unzureichende Finanzierung erfahren.}

% -----

\programm{Präventionsmaßnahmen gegen menschenverachtende Meinungen}
\antrag{Alexander Magnus}
\lqfb{http://lqfb.piraten-lsa.de/lsa/initiative/show/111.html}{22}{2}{3}

\paragraph{Antragstext}:

Der Landesparteitag möge beschließen, folgenden Text in das laufende Parteiprogramm aufzunehmen: 

\einruecken{Menschenverachtene Weltbilder und Meinungen werden in der Gesellschaft immer salonfähiger. Um dieser Entwicklung entgegen zu wirken ist es notwendig, Engagement für politische Aufklärung und Präventation zu fördern. Dazu zählen unter anderem Aussteigerprogramme, schulische Informationsveranstaltungen, Konzerte sowie öffentliche Feste zur Förderung von Toleranz und Gleichberechtigung. Diese Initiativen wurden in den letzten Jahren durch Budgetkürzungen -- unter anderem seitens des Bundesfamilienministeriums -- erheblich behindert und mitunter unmöglich gemacht. Die Piratenpartei Sachsen-Anhalt setzt sich dafür ein, dass diese Schritte rückgängig gemacht werden, damit diese Programme nicht nur ihre alte Stärke zurückgewinnen, sondern darüber hinaus weiter ausgebaut werden können.}

\begruendung{Nach dem erfolgreichen Abstimmen des Antrages zum Bekenntnis gegen menschenverachtende Meinungen soll dieser Antrag erste konkrete Maßnahmen anbieten, um gegen die vorgenannten Weltbilder Stellung zu beziehen und Hilfe zu bieten.}

% -----

\programm{Waffenrecht}
\antrag{Alexander Magnus}
\lqfb{http://lqfb.piraten-lsa.de/lsa/initiative/show/73.html}{9}{4}{2}

\paragraph{Antragstext}:

Der Landesparteitag möge beschließen, folgenden Text in das laufende Parteiprogramm aufzunehmen:

\einruecken{Die Verschärfungen der Waffengesetze in den letzten Jahren dienten vor allem dazu, Sicherheit vorzutäuschen und einfache und schnelle Antworten auf komplizierte Probleme zu geben. Die Piraten setzen sich für eine Reformierung der Waffengesetze ein, welche die sorgfältige Aufbewahrung von Schusswaffen regeln und dadurch die Sicherheit aller Bürger gewährleisten. Wir lehnen es aber ab, beispielsweise Sportschützen und Softair-Spieler zu Sündenböcken für gesellschaftliche Probleme zu machen.}

\begruendung{Angepasster Antrag nach \url{http://www.piratenpartei-bw.de/wp-content/uploads/wahlprogramm\_screen.pdf} (PDF-Seite 32). Die PIRATEN als Bürgerrechtspartei treten dafür ein, Bürgerrechte Einzelner nur aufgrund klarer, nachweisbarer Fakten einer tatsächlichen Gefährdung einzuschränken. WaffR-Verschärfungen sind in der Regel aus populistischem Aktionismus nach einzelnen Ereignissen geschehen; nationale wie internationale Datenlagen geben viele der Verschärfungen nicht her. Die AG Waffenrecht bearbeitet auf Bundesebene das Thema und schließt sich Forderungen nach wirksamen und der Sicherheit dienenden Maßnahmen an (z.B. \url{http://www.dsb.de/media/PDF/Recht/Waffenrecht/Neues\%20Waffenrecht/DSB\_Poster\_A4.pdf} oder eine verbindliche Regelung der Anforderungen auf Alarmanlagen in Tresorräumen). Die auf Landesebene und darunter geregelte Ausführung des WaffR macht das Thema auch für die Landespolitik wichtig (einheitliche Regeln zur Bewilligung von Anträgen und zu Kontrollen). Für eine Bürgerrechtspartei - im Gegensatz zu einer populistischen und paternalistischen Verbotspartei) - ist die Frage, warum Privatleute Waffen besitzen wollen, nicht relevant, ebenso wenig, warum Bürger Paintball spielen wollten oder statt Büchern digitale Medien verwenden (\#Heveling)}

% -----

\programm{Aufhebung des ESM Vertrages}
\antrag{Lutz Kölbl}

\paragraph{Antragstext}:

Es wird beantragt im Grundsatzprogramm und im Wahlprogramm der Piratenpartei Deutschland zur kommenden Bundestagswahl 2013, die Aufhebung des ESM Vertrages an geeigneter Stelle einzufügen. Der Landesverband der Piratenpartei Deutschland -- Sachsen Anhalt möge dazu eine Stellung beziehen.

\einruecken{Der wesentliche Inhalt des ESM-Vertrages und Begründung des Antrages:

1. Die Regierungen gründen die erste europäische, supranationale, ESM-Bank. Diese Mega-Bank braucht keine Banklizenz (Art. 1, Art. 32, Abs. 9).

2. Die ESM-Bank hat Blankovollmacht für unbeschränkte Geschäfte jeder Art (Art. 3).

3. Die schwachen Euro-Ländern haben Stimmrechtsvorteile (Art. 4).

4. Die jeweiligen Finanzminister (für die BRD: Dr. W. Schäuble) bilden den Gouverneursrat (BoG) der ESM-Bank. Der BoG und die Räte sind rechtlich unantastbar, haben die totale Kontrolle und letzte Entscheidungsmacht in allen finanziellen, sachlichen und personellen Dingen der ESM-Bank. (Art. 5).

5. Die Gouverneure setzen sich ihr Gehalt und das ihrer Direktoren geheim in unbekannter Millionenhöhe selbst fest (Art. 5 Abs. 7 (n), Art 34).

6. Das Aktien-Haftungs-Kapital der ESM-Bank beträgt (zunächst) \euro{} 700 Mrd. aufgeteilt in (a) \euro{} 80 Milliarden einzuzahlende Aktien und (b) \euro{} 620 Milliarden abrufbare Aktien. (Art.

8 Abs. 1). Das Haftungs-Kapital kann ggf. durch Ausgabe neuer Aktien (auch höheren Nennwerts!) bis in Billionenhöhe (c) beliebig erhöht werden (Art. 8 Abs. 2, Art. 10 Abs. 1).

7. Im Ernstfall muss ESM-Haftungskapital binnen 7 Tagen eingezahlt werden oder wird auf die übrigen Aktionäre umgelegt (Art. 9, Art. 10, Art. 25 Abs. 1 c, 2).

8. Die Deutschen haften (Ziff. 6), für (Minimum) 27\% - 100\% (Maximum) aus \euro{} 700 Mrd. Wird das Aktien-Haftungs-Kapital erhöht (Art. 8, Art 10), kann sich daraus erhöhte Haftung über \euro{} 700 Mrd. hinaus ergeben (Art. 9, Art. 10, Art. 25 Abs. 1 c, 2).

9. Die ESM-Bank kann: (A) Überziehungskreditlinien einräumen, Art 14 ; (B) Banken finanzieren, Art. 15; (C) Kredite geben, Art. 16; (D) direkt Staatsanleihen ankaufen, Art. 17; (E) indirekt Staatsanleihen ankaufen, Art. 18; (F) diese Liste ändern, also auch erweitern, Art. 19; (G) Zinspolitik betreiben, Art. 20; (H) Eurobonds herausgeben, Art. 21. - Summa summarum Finanzgeschäfte jeder Art und Höhe betreiben. (Art. 14 – 21).

10. Die ESM-Bank kann unbegrenzt Kredit/Geld aufnehmen um Schulden schwacher Länder/ Banken zu finanzieren. Diese neuen ESM-Schulden werden durch das Aktienkapital der ESM-Bank (mindestens \euro{} 700 Mrd.) gedeckt, für dessen Einzahlung am Ende die Bürger mit ihrem ganzen Vermögen haften. Wegen des Dominoeffektes beträgt die Haftung im Extremfall \euro{} 700 Mrd. (ggf. erhöht gem. Art.10!) für alle in Europa verteilten Gelder/ Kredite. Die Regierung führt so heimlich Eurobonds ein, ohne dies auszusprechen, Art. 21.

11. Die ESM-Kredite (Art. 14, 15, 16) haben im Konkurs eines Eurolandes Nachrang gegenüber IWF-Krediten. Daraus folgt ein massiv erhöhtes Haftungs-Risiko (Präambel, Abs. 13, 14).

12. Die zahlenden und haftenden Bürger haben keine Möglichkeit die Geschäfte der ESM-Bank durch Bestellung unabhängiger externer Prüfer auf ordnungsgemäße, sachliche und rechnerische Richtigkeit zu prüfen. Solche Prüfungen sind ausgeschlossen (Art. 26 - 30).

13. Die ESM-Bank samt Vermögen ist immun, von Kontrollen und Lizenzen jeder Art befreit, kann nicht vor Gericht belangt werden. Gerichtliche/gesetzgeberische Maßnahmen gelten für sie nicht. Die Bank ihrerseits hat Klagerecht gegen jedermann. (Art. 32, 32 Abs. 9).

14. Die Gouverneure (incl. Dr. Schäuble) und Mitarbeiter der ESM-Bank haben Schweigerecht und sichern so die Geheimhaltung (a) der Operationen der Bank, (b) eigenen Aktivitäten innerhalb der ESM-Bank und insbesondere (c) die Bestimmungen von Art. 32, 34 - 36 ab.

15. In ihrem Interesse genießen alle Gouverneure (incl. Dr. Schäuble), Direktoren etc. pp Immunität hinsichtlich ihrer geschäftlichen Tätigkeit für die ESM-Bank, gleich ob hunderte Milliarden Euro verschleudert, vernichtet, oder veruntreut werden (Art. 35).

16. Die Gehälter der Gouverneure (s.o. Ziff. 5), Direktoren, sonstigen Mitarbeiter der Bank sind von allen (auch indirekten) Steuern und Abgaben vollständig befreit und unterliegen nur einer internen (!) Steuer an die ESM-Bank, Art. 36 Abs. 5.

17. Das Volumen der (konsolidierten) Darlehensvergabe von ESM und EFSF ist unbegrenzt und nur in der Übergangsphase auf 500 Milliarden EUR beschränkt (Art. 39, Art 10).

18. Da der jeweilige Regierungschef/Kanzler den Finanzminister/Gouverneur bestimmt, wird es zu Machtkämpfen um dieser Posten kommen.

19. Mit Ratifizierung des ESM-Vertrages besiegeln die deutschen Bundestagsabgeordneten das Ende ihrer eigenen demokratischen, nationalen Rechte, (Art. 47 Abs. 1).


Ab 1999 haben die Euro-Regierungen durch unprofessionelle Finanzpolitik Kreditorgien und die Eurokrise ausgelöst. Über den ESM sollen nun die Schulden der {\Gu}Club-Med.-Länder{\Go} in Höhe von Billionen klammheimlich u.a. deutschen Bürgern aufgeladen werden, während schuldige Finanzminister/Politiker mit haarsträubenden Privilegien, einem Quantensprung im Einkommen und unvorstellbarem Machtzuwachs belohnt und alte Fehlentscheidungen vertuscht werden.

Der ESM-Vertrag ist eine Verhöhnung und Verspottung des gesunden Menschenverstandes und der europäischen Rechtstradition schlechthin. Schon das Ansinnen der Regierenden, die Einrichtung der ESM-Bank durch das deutsche Parlament absegnen zu lassen, ist der schwerste Anschlag gegen die Demokratie und die deutsche Nation seit 1933. Mit dem ESM-Vertrag putscht eine kleine Gruppe von Regierenden gegen ihr eigenes Volk.}

\begruendung{eingereicht per eMail am 13.09./Begründung siehe Antragstext}

% -----

\programm{Antrag auf Entfernung des Positionspapier PP002 mit dem Titel 'Geldreformen als Schlüssel gegen die Finanzkrise'}
\antrag{Thomas Hübner}

\paragraph{Antragstext}:

Antrag auf Entfernung des Positionspapier PP002, eingefügt auf dem LPT2012.1, mit dem Titel Geldreformen als Schlüssel gegen die Finanzkrise.

\paragraph{Alte Fassung}:

\einruecken{\textbf{Geldreformen als Schlüssel gegen die Finanzkrise}

Die Piratenpartei Sachsen-Anhalt setzt sich für Reformen im Geldsystem ein. Insbesondere sollen durch parallele umlaufgesicherte Geldsysteme alternative Finanzierungsmöglichkeiten für die Kommunen geschaffen werden, regionale Wirtschaftskreisläufe unterstützt werden und insgesamt damit die Umverteilung von Arm zu Reich durch unser Geldsystem unterbrochen und schrittweise abgelöst werden.}

\begruendung{Begründung: Weder vor noch nach dem LPT2012.1, auf dem die Aufnahme dieses Positionspapiers beschlossen wurde, gab bzw. gibt es eine anwendbare Argumentation für diese Position. Den Mitgliedern des Landesverbandes konnte bisher nicht vermittelt werden, wie diese Position in der Öffentlichkeit vertreten werden soll. Auf Nachfragen in den Mailinglisten kamen von den Initiatoren bisher nie schlüssige Argumentationen, die an Infoständen und in der Öffentlichkeitsarbeit genutzt werden könnten, sondern lediglich Hinweise auf youtube-Videos über Vorträge von Dritten. Auf Kritik (siehe auch \url{http://de.wikipedia.org/wiki/Umlaufgesichertes\_Geld\#Kritik}) wird dabei nicht eingegangen.

Da das Thema an sich, gerade im Hinblick auf Auswirkungen der Umlaufsicherung, schwer an die Bevölkerung vermittelbar ist, fehlen hier eine handfeste Argumentationskette und wissenschaftliche Argumente. Es gibt zu diesem Thema keinerlei Ausarbeitungen seitens des Landesverbandes, der Vor- und Nachteile von umlaufgesicherten Geldsystemen auf einfache Weise beschreibt. Insbesondere fehlt es an einer Begründung, inwieweit umlaufgesicherte Geldsysteme und Regionalwährungen als Schlüssel gegen die Finanzkrise wirken sollen.

Erschwerend kommt hinzu, das eine Liquid Feedback Abstimmung vor dem LPT2012.1 mehrheitlich ergab, keine Positionen zum Geldsystem einzunehmen noch diese in ein Programm aufzunehmen. (\url{http://lqfb.piraten-lsa.de/lsa/issue/show/4.html}) Das LQFB-Meinungsbild wurde der Mitgliederversammlung während der Verlesung des Antrages nicht bekannt gemacht und die Abstimmung über das Positionspapier fand zu einem Zeitpunkt statt, an dem der LPT weit fortgeschritten war, so dass nur noch ein kleiner Teil der zu Beginn akkreditierten Piraten im Saal war.

Wir empfehlen daher, die Rücknahme des Positionspapieres und evtl. Neuabstimmung, sobald eine schlüssige, mit wissenschaftlichen Argumenten hinterlegte und vermittelbare Begründung des Antrages vorliegt.}

% -----