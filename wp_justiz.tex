\section{Inneres und Justiz}
\version{03:19, 20. Jun. 2010}

\subsection*{Einleitung}
\wahlprogramm{Grundlagen einer piratigen Innenpolitik}
\antrag{Silberpappel (abgeschrieben von BaWü)}\version{03:19, 20. Jun. 2010}
\subsubsection{Grundlagen einer piratigen Innenpolitik}
\abstimmung
Wir setzen uns für eine effektive Sicherheitspolitik ein. Ein sicheres Leben ist ein wichtiges und unabdingbares Gut für die Bewohner unseres Landes. Uns geht es um angemessene und vor allem wirksame Methoden. In der Vergangenheit wurden im Bereich der Sicherheitspolitik ineffektive Gesetze erlassen, aber gleichzeitig die Mittel für Polizei und andere Behörden gekürzt. Grundrechte wurden dadurch eingeschränkt und die Überwachung hat zugenommen, dabei soll in einer Demokratie der Bürger den Staat überwachen und nicht umgekehrt.
 
\subsection*{Lockerung der Residenzpflicht}
\wahlprogramm{Asyl- und Flüchtlingspolitik}
\antrag{KV Trier/Trier-Saarburg}\version{03:19, 20. Jun. 2010}
\subsubsection{Lockerung der Residenzpflicht}
\abstimmung
Die allgemeine Erklärung der Menschenrechte garantiert in Artikel 13 das Recht auf Freizügigkeit. Deutschland ist der einzige Staat in Europa, der dieses für Asylsuchende und anerkannte Flüchtlinge einschränkt.

Die Residenzpflicht macht Menschen zu Kriminellen, nur weil sie sich frei bewegen wollen. Polizei, Gerichte und Behörden werden zusätzlich unnötig belastet.

Ähnlich wie bereits in Bayern und Brandenburg wollen wir daher auch in unserem Bundesland die Residenzpflicht lockern.
 

\wahlprogramm{Abschaffung von Ausreisezentren}
\antrag{Thomas Heinen}\version{03:19, 20. Jun. 2010}
\subsubsection{Abschaffung von Ausreisezentren}
\abstimmung
Wir unterstützen die Forderung der Bewohner des Ausreisezentrums (LUFA) in Trier nach einer ersatzlosen Schließung der Einrichtung.

Die Bewohner der LUFA sind Flüchtlinge, die durch starken Druck zu einer "freiwilligen" Ausreise gebracht werden sollen. Sinn und Zweck von Ausreisezentren ist es, den Willen von Menschen zu brechen.

Diese Zielsetzung und die Art und Weise der Unterbringung sind mit dem Menschenrecht nicht vereinbar.

Wir werden uns für eine menschenwürdige Flüchtlingspolitik einsetzen.
 
\subsection*{Polizei im öffentlichen Raum Internet}
\wahlprogramm{Polizei im öffentlichen Raum Internet}
\antrag{KV Trier/Trier-Saarburg}\version{03:19, 20. Jun. 2010}
\subsubsection{Modul 1}
\abstimmung
Auch im Internet müssen die Grundrechte und das Prinzip der Verhältnismäßigkeit gewahrt bleiben. Eingriffe in private Kommunikation, etwa das Mitlesen von E-Mails, dürfen nur nach richterlicher Anordnung möglich sein. Das Einschleusen von Software in private Computer lehnen wir vollständig ab.
 
\subsection*{Keine religiösen Symbole in öffentlichen Gebäuden}
\wahlprogramm{Keine religiösen Symbole in öffentlichen Gebäuden}\label{justiz:symbole1}
\antrag{KV Trier/Trier-Saarburg}\konkurrenz{justiz:symbole2}\version{03:19, 20. Jun. 2010}
\subsubsection{Keine religiösen Symbole in öffentlichen Gebäuden}
\abstimmung
Das Anbringen von religiösen Symbolen in öffentlichen Gebäuden verletzt die Religionsfreiheit von Angehörigen anderer Religionen und Menschen ohne Religion. Dies wurde vom Bundesverfassungsgericht sowie vom Europäischen Gerichtshof für Menschenrechte festgestellt. Wir möchten daher dafür sorgen, dass diese religiösen Symbole aus öffentlichen Gebäuden, vor allem auch den öffentlichen Schulen, entfernt werden.
 
\wahlprogramm{Keine religiösen Symbole in öffentlichen Gebäuden}\label{justiz:symbole2}
\antrag{marcus}\konkurrenz{justiz:symbole1}\version{03:19, 20. Jun. 2010}
\subsubsection{Keine religiösen Symbole in öffentlichen Gebäuden}
Das Anbringen von religiösen Symbolen in öffentlichen Gebäuden kann die religiösen Gefühle von Menschen anderer Religion verletzen. In diesen Fällen ist basisdemokratisch eine Entscheidung über den Verbleib oder die Entfernung dieser Symbole zu entscheiden. In öffentlichen Schulen haben ausserhalb des Religionsunterrichtes religiöse Symbole nichts zu suchen, da wir im Zeitalter der Aufklärung für eine strikte Trennung von Kirche und Staat eintreten.
 
\newpage
\subsection*{Gewaltmonopol}
\wahlprogramm{Gewaltmonopol}
\antrag{KV Trier/Trier-Saarburg}\version{03:19, 20. Jun. 2010}
\subsubsection{Keine Privatisierung hoheitlicher Aufgaben}
Das Gewaltmonopol des Staates darf nicht an Privatfirmen delegiert werden. Polizeiaufgaben, das Beaufsichtigen von Gefängnissen und ähnliches müssen vollständig in staatlicher Hand bleiben.
 
[Bearbeiten] Justiz
Meta
Antragsteller: 	KV Trier/Trier-Saarburg
Thema: 	Justiz
Id: 	17.8
Module
17.8.1 Unabhängigkeit der Staatsanwaltschaften 	Staatsanwälte sind an dienstliche Anweisungen ihrer Vorgesetzten gebunden. Dadurch besteht die Gefahr, dass politisch unerwünschte Strafverfahren beeinträchtigt werden. Um die Unabhängigkeit der Justiz und den Rechtsstaat zu stärken, wollen wir, dass die Landesregierung sich gesetzlich verpflichtet, von ihrem Weisungsrecht gegenüber den Landesstaatsanwälten keinen Gebrauch mehr zu machen. Insbesondere soll es keine Dienstanweisungen mehr geben, die sich auf einzelne Verfahren beziehen.
17.8.2 Öffentlichkeitsarbeit der Staatsanwaltschaften 	Wir sehen mit Sorge, wie durch eine nicht zu verantwortende Öffentlichkeitsarbeit einiger Staatsanwaltschaften die im Rechtsstaat verankerte Unschuldsvermutung zunehmend zu Lasten von Beschuldigten ausgehebelt wird. Deshalb wollen wir dienst- und strafrechtliche Sanktionsmöglichkeiten gegenüber Staatsanwälten bei entsprechenden Verstößen verschärfen.
 
Meta
Antragsteller: 	marcus
Thema: 	Justiz
Id: 	17.9
Module
17.9.1 Unabhängigkeit der Staatsanwaltschaften 	Wir fordern die Unabhängigkeit der Staatsanwälte von der politischen Führung und deshalb ein Verbot des Weisungsrechtes der vorgesetzten Behörden auf Einstellung von Verfahren.
17.9.2 Unabhängigkeit der Staatsanwaltschaften 	Alle Verfahren, die aufgrund von Weisungsrechten übergeordneter Behörden gegenüber Staatsanwälten von diesen eingestellt werden, sind jährlich unter Namensnennung des Anweisenden zu veröffentlichen. Nur sofern die Verfahren gegen Privatpersonen liefen, sind deren Daten zu anonymisieren.
 
Meta
Antragsteller: 	Limbo
Thema: 	Strafmaß
Id: 	17.10
Module
17.10.1 Strafmaßverschärfung 	Die Würde jedes Menschen ist bekanntermaßen unantastbar. Deswegen sollte jeder, der diese Würde wissentlich (psychologisch geprüft) zerstört (z.B. durch Vergewaltigung, Misshandlung, Freiheitsberaubung) mit lebenslanger Strafe rechnen. Dies ist nicht immer der Fall, da die Sicherungsverwahrung oftmals nachträglich auf Antrag gändert werden. Dadurch verliert diese Strafe an abschreckender Wirkung. Deswegen wäre eine Möglichkeit Anträge zu streichen durch verpflichtende regelmäßge Gutachten über Gefahrepotential zu ersetzen. Darauf sollte die weitere Haftplanung fußen.
 
[Bearbeiten] Polizei
Meta
Antragsteller: 	Silberpappel
Thema: 	Unabhängige Kontrolle der Polizei
Id: 	17.11
Module
17.11.1 who watches the watchmen? 	Wir treten für eine demokratische, transparente und unabhängige Kontrolle der Polizei ein. Das wollen wir erreichen durch die Einführung einer unabhängigen Polizeikommission. Die Polizeikommission soll sicherstellen, dass die Exekutivorgane nach rechtsstaatlichen Prinzipien arbeiten. Die Kommission soll als Ansprechpartner von Beschwerdeführern und aktiver Ermittler tätig werden. Sie soll polizeiinterne Beschwerden und Beschwerden von Bürgern entgegennehmen. Um die Unabhängigkeit der Kommission zu gewährleisten, sollte sie nicht den Innenministerien, sondern einem anderen Ressort unterstellt sein.