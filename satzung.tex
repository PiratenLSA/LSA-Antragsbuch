\section{Satzungsänderungen}
\satzung{Feststellung der Beschlussfähigkeit während einer Mitgliederversammlung}
\antrag{Stephan Schurig}

\paragraph{Antragstext}:

Der Landesparteitag möge beschließen folgenden Absatz in die Satzung am Ende des {\Gu}\textbf{§ 9b - Der Landesparteitag (2)}{\Go} einzufügen: 

\einruecken{(2) Der Landesparteitag tagt mindestens einmal jährlich. (...) Spätestens \textbf{eine Woche} vor dem Parteitag sind die Tagesordnung in aktueller Fassung, die geplante Tagungsdauer und alle bis dahin dem Vorstand eingereichten Anträge im Wortlaut zu veröffentlichen. \textbf{Der Landesparteitag kann dann abge- bzw. unterbrochen werden, wenn festgestellt wird, dass weniger als die Hälfte der zu Beginn der Veranstaltung akkreditierten Piraten bei einer Abstimmung anwesend sind. Er wird nur dann fortgesetzt werden, wenn mindestens die Hälfte der Akkreditierten bei einer Abstimmung teilnehmen.}}

\paragraph{Alte Fassung}:

\einruecken{(2) Der Landesparteitag tagt mindestens einmal jährlich. (...) Spätestens \textbf{1 Wochen} vor dem Parteitag sind die Tagesordnung in aktueller Fassung, die geplante Tagungsdauer und alle bis dahin dem Vorstand eingereichten Anträge im Wortlaut zu veröffentlichen.}

\begruendung{Dies soll verhindern, dass insbesondere zu Ende eines Landesparteitages zu wenige Piraten vor Ort sind, um eine ausreichend {\Gu}kritische Masse{\Go} zu darzustellen. Je weniger Piraten anwesend sind, umso größer sind die Abweichungen von der {\Gu}wirklichen{\Go} Position eines Landesparteitages.

Ein \href{http://lqfb.piraten-lsa.de/lsa/initiative/show/167.html}{Meinungsbild} diesbzgl. fiel eindeutig positiv aus (18 dafür, 0 enthalten, 0 dagegen).}

% -----

\satzung{Der Parteitag möge beschließen §1 Abs. (1) der Landesfinanzordnung wie folgt zu ändern: }
\antrag{Alexander Zinser}

\paragraph{Antragstext}:

Der Parteitag möge beschließen §1 Abs. (1) der Landesfinanzordnung wie folgt zu ändern:

\einruecken{(1) Abschlagszahlungen aus der Parteienfinanzierung verbleiben als Rücklage beim Landesverband. Sollten Rückzahlungen fällig werden, können diese aus der Rücklage beglichen werden. Sind Rückzahlungen für ein Jahr beglichen oder wurden keine Rückzahlungen fällig, so wird die Rücklage entsprechend folgendem Schlüssel umgelegt.}

\paragraph{Alte Fassung}:

\einruecken{(1) 10\% der Parteienfinanzierung verbleibt bis zur nächsten Abschlagszahlung, mindestens jedoch für ein Jahr, als Rücklage beim Landesverband. Aufgelöste Rücklagen werden zur aktuellen Abschlagszahlung addiert und entsprechend diesem Schüssel umgelegt.}

\begruendung{Im {\Gu}worst case{\Go} müssen alle Abschlagszahlungen für ein Jahr zurückgezahlt werden. Eine Rücklage von 10\% ist da ggf. nicht ausreichend diese zu begleichen.}

% -----

\satzung{Streichung Landesgeschäftsstelle }
\antrag{Christoph Giesel}

\paragraph{Antragstext}:

Der Landesparteitag möge beschließen, folgenden Satz von §1 Abs. 3 der Landessatzung zu streichen:

{\Gu}Dort befindet sich auch die Landesgeschäftsstelle.{\Go}

\paragraph{Neue Fassung}:

\einruecken{(3) Der Sitz des Landesverbandes ist Halle. Untergeordnete Gliederungen des Landesverbandes Sachsen-Anhalt der Piratenpartei Deutschland führen den Namen Piratenpartei Deutschland verbunden mit ihrer Organisationsstellung und dem Namen der Gliederung.}

\paragraph{Alte Fassung}:

\einruecken{(3) Der Sitz des Landesverbandes ist Halle. \sout{Dort befindet sich auch die Landesgeschäftsstelle.} Untergeordnete Gliederungen des Landesverbandes Sachsen-Anhalt der Piratenpartei Deutschland führen den Namen Piratenpartei Deutschland verbunden mit ihrer Organisationsstellung und dem Namen der Gliederung.}

\begruendung{Der Sitz der Landesgeschäftsstelle gehört nicht in die Satzung.}

% -----

\satzung{Beschlussunfähigkeit, wenn die GO nicht nach 4 Wochen vorliegt }
\antrag{Christoph Giesel}

\paragraph{Antragstext}:

Der Landesparteitag möge folgenden neuen Absatz dem §7 der Landessatzung hinzufügen:

\einruecken{(7) Liegt die Kopie der Geschäftsordnung des Vorstandes einer Untergliederung nicht 4 Wochen nach Wahl gemäß Absatz (6) beim Landesvorstand vor, so gilt der Vorstand der Untergliederung dauerhaft als beschlussunfähig. In diesem Fall hat der Landesvorstand unverzüglich eine neue Mitgliederversammlung einzuberufen, bei der sich nur mit der Neuwahl des Vorstands oder Auflösung des Verbands befasst werden darf. Bis die Neuwahl des Vorstandes zustande kommt, führt der Landesvorstand oder vom Landesvorstand beauftragte Personen kommissarisch die Geschäfte der Untergliederung.}

\begruendung{Der Antrag ist selbsterklärend.}

% -----

\satzung{Gebiets- und Aufstellungsversammlungen}\label{saa:gebietsversammlung1}
\antrag{Christoph Giesel}
\begin{itemize}
\item \konkurrenz{saa:gebietsversammlung2}
\end{itemize}

\paragraph{Antragstext}:

Der Landesparteitag möge folgende Satzungsänderungen beschließen:

Der Paragraph §9 Absatz 1 wird wie folgt geändert:

\einruecken{(1) Organe sind der Vorstand, der Landesparteitag, das Landesschiedsgericht, \textbf{die Gebietsversammlung, die Aufstellungsversammlung} und die Gründungsversammlung.}

Die Paragraphen §9c und §9d werden mit folgenden Inhalt neu eingefügt: 

\einruecken{\textbf{§ 9c - Gebietsversammlung}

(1) Eine Gebietsversammlung ist die Versammlung aller Piraten eines Landkreises, einer Gemeinde, einer Stadt, eines Ortsteils oder Stimm- bzw. Wahlkreises im Bundesland Sachsen-Anhalt.

(2) Die Gebietsversammlung ist ein Organ der untersten existierenden Gliederung, die das Gebiet vollständig umfasst. Diese Gliederung wird im folgenden als {\Gu}zuständige Gliederung{\Go} bezeichnet. Ist das Gebiet identisch mit dem Gebiet der zuständigen Gliederung, so ist eine Mitgliederversammlung stattdessen durchzuführen.

(3) Der Vorstand der zuständigen Gliederung vertritt die Interessen der Gebietsversammlung nach Maßgabe ihrer Beschlüsse, sofern die Gebietsversammlung keine Personen aus ihrer Mitte damit beauftragt.

(4) Die Gebietsversammlung entscheidet über

1. ausschließlich das Gebiet betreffende politische Fragen

2. gegebenenfalls weitere ihr nach der Satzung der zuständigen Gliederung zukommende Aufgaben

(5) Stimmberechtigt ist jeder Pirat, dessen angegebener Wohnsitz im Gebiet der Gebietsversammlung liegt. Die Bestimmungen in §4 (4) der Bundessatzung gelten entsprechend.

(6) Eine Gebietsversammlung wird vom Vorstand der zuständigen Gliederung einberufen, wenn

1. der betreffende Vorstand es beschließt

2. mindestens 10\%, aber mindestens drei Mitglieder, des Gebiets es verlangen

(7) Gibt sich die Gebietsversammlung keine eigene Wahl- und Geschäftsordnung, gilt die aktuelle Wahl- und Geschäftsordnung der zuständigen Gliederung.

(8) Eine Gebietsversammlung ist beschlussfähig, wenn mindestens 5\%, aber mindestens drei Piraten, des Gebiets akkreditiert sind.

(9) Für die Einladung zu einer Gebietsversammlung gelten die gleichen Regelungen wie zur Mitgliederversammlung der zuständigen Gliederung. Die Satzung der zuständigen Gliederung kann jedoch abweichende Regelungen beschließen.

\textbf{§ 9d - Aufstellungsversammlung}

(1) Die Aufstellungsversammlung ist die Versammlung zur Bewerberaufstellung für die Wahlen zu Volksvertretungen.

(2) Die Bewerberaufstellung für die Wahlen zu Volksvertretungen erfolgt nach den Regularien der einschlägigen Gesetze sowie den Parteisatzungen der Gliederungen, die den betreffenden Stimm- bzw. Wahlkreis vollständig umfassen.

(3) Die Regelungen in § 9c mit Ausnahme von Absatz 4 und 5 gelten entsprechend auch für Aufstellungsversammlungen.

(4) Stimmberechtigt ist jedes Mitglied, dass zum Zeitpunkt der Wahl der Volksvertretung wahlberechtigt ist.}

Der Paragraph §10 Absatz 1 wird wie folgt geändert:

\einruecken{(1) Die Bewerberaufstellung für die Wahlen zu Volksvertretungen wird durch die Aufstellungsversammlung durchgeführt. Näheres regelt § 9d.}

Absatz 2 von §10 wird gestrichen. 

\begruendung{Momentan ist es nicht möglich, dass Gebiete ohne einer gegründeten Untergliederung (Kreisverband, Regionalverband etc.) eine Mitgliederversammlung einberufen können, um ein Programm zu entwickeln, Kandidat aufzustellen o.ä. Lediglich zur Gründung einer Untergliederung können sich solche Versammlungen zusammenfinden. Dies soll regionale Gruppen stärken und eine Alternative für die Gründung eines Verbandes darstellen.}

% -----

\satzung{Ständige Mitgliederversammlung}
\antrag{Karl}

\paragraph{Antragstext}:

Der Landesparteitag möge beschließen die Satzung der Piratenpartei Sachsen-Anhalt wie folgt zu ergänzen:

\einruecken{§ 9b - Der Landesparteitag

(1) Der Landesparteitag ist die Mitgliederversammlung auf Landesebene.

(2) Der Landesparteitag tagt mindestens einmal jährlich als \textbf{Realversammlung}. Die Einberufung erfolgt aufgrund eines Vorstandsbeschlusses. Wenn ein Zehntel der Piraten, mindestens aber zehn Piraten es beim Vorstand beantragen, muss dieser binnen 2 Wochen einen Parteitag einberufen. Der Vorstand lädt jedes Mitglied schriftlich (Brief, Email oder Fax) mindestens 4 Wochen vorher ein. Die Einladung hat Angaben zum Tagungsort, Tagungsbeginn, vorläufiger Tagesordnung und der Angabe, wo weitere, aktuelle Veröffentlichungen gemacht werden, zu enthalten. Spätestens 1 Wochen vor dem Parteitag sind die Tagesordnung in aktueller Fassung, die geplante Tagungsdauer und alle bis dahin dem Vorstand eingereichten Anträge im Wortlaut zu veröffentlichen.

(3) Ist der Vorstand handlungsunfähig, kann einen außerordentlichen Landesparteitag einberufen werden. Dies geschieht schriftlich mit einer Frist von zwei Wochen unter Angabe der Tagesordnung und des Tagungsortes. Er dient ausschließlich der Wahl eines neues Vorstandes.

(4) Der Landesparteitag nimmt den Tätigkeitsbericht des Vorstandes entgegen und entscheidet daraufhin über seine Entlastung.

(5) Über den Landesparteitag, die Beschlüsse und Wahlen wird ein Ergebnisprotokoll gefertigt, das von der Protokollführung, der Versammlungsleitung und dem neu gewählten Vorsitzenden oder dem stellvertretenden Vorsitzenden unterschrieben wird. Das Wahlprotokoll wird durch den Wahlleiter und mindestens zwei Wahlhelfer unterschrieben und dem Protokoll beigefügt.

(6) Der Landesparteitag wählt für die anstehende Amtsperiode des Vorstandes mindestens zwei Rechnungsprüfer, die den finanziellen Teil des Tätigkeitsberichtes des Vorstandes vor der Beschlussfassung über ihn prüfen. Das Ergebnis der Prüfung wird dem Landesparteitag verkündet und zu Protokoll genommen. Danach sind die Rechnungsprüfer aus ihrer Funktion entlassen.

(7) Es können außerordentliche Landesparteitage statt finden. Die Einberufung erfolgt aufgrund eines Vorstandsbeschlusses. Wenn ein Zehntel der Piraten, mindestens aber zehn Piraten es beim Vorstand beantragen, muss dieser binnen 2 Wochen einen Parteitag einberufen. Dies geschieht schriftlich mit einer Frist von zwei Wochen unter Angabe der Tagesordnung und des Tagungsortes.

\textbf{(8) Der Landesparteitag tagt daneben online und nach den Prinzipien von Abschnitt D: Liquid Democracy als Ständige Mitgliederversammlung. Jeder Pirat im Landesverband Sachsen-Anhalt hat das Recht, an der Ständigen Mitgliederversammlung teilzunehmen.}

\textbf{(9) Die Ständige Mitgliederversammlung kann für den Landesverband verbindliche Stellungnahmen und Positionspapiere beschließen. Entscheidungen über die Parteiprogramme, die Satzung, die Beitragsordnung, die Schiedsgerichtsordnung, die Auflösung sowie die Verschmelzung mit anderen Parteien (§ 9 Abs. 3 Parteiengesetz) sind ausgeschlossen, insoweit kann die Ständige Mitgliederversammlung nur Empfehlungen abgeben.}

\textbf{(10) Der Landesparteitag beschließt die Geschäftsordnung der Ständigen Mitgliederversammlung, in der auch die Konstituierung der Ständigen Mitgliederversammlung geregelt ist.}}

\begruendung{Dieser Antrag befasst sich mit der Aufnahme des Modells der ständigen Mitgliederversammlung in die Satzung der Piratenpartei Sachsen-Anhalt. Die ständige Mitgliederversammlung ist ein Konzept, welches der Parteibasis ermöglichen würde, zwischen Landesparteitagen \textbf{verbindliche Beschlüsse} in Form von Positionspapieren über ein Onlineabstimmungstool wie z.B. LiquidFeedback zu verabschieden. Was zuerst einmal sehr verführerisch klingt, wirft nach einigem Nachdenken viele Fragen auf, z.B. wie mit einem solchen Modell geheime Abstimmung durchführbar sein sollten, ohne den demokratischen Wahlgrundsatz der Nachvollziehbarkeit zu vernachlässigen (siehe Wahlcomputer). Der Pirat Niels Lohmann aus dem Landesverband Mecklenburg-Vorpommern hat sich zu diesem Thema Gedanken gemacht und schon ein Konzept einer möglichen \href{http://dl.dropbox.com/u/3658551/SMVMV.pdf}{Geschäftsordung} für die ständige Mitgliederversammlung erarbeitet. Diese Geschäftsordnung würde nach dem Verabschieden dieses Antrages auf einem nächsten Landesparteitag verabschiedet. \textbf{Die ständige Mitgliederversammlung tritt nicht mit der Annahme dieses Antrages auf einem Landesparteitags in Kraft. Sie tritt erst in Kraft, wenn sie sich auf einen darauf folgenden Landesparteitag mit der Verabschiedung einer Geschäftsordnung konstituiert.} Deswegen kann dieser Antrag nicht auf alle Bedenken bezüglich des Konzeptes eingehen, da es sich bei diesen Antrag schlicht um eine formelle Satzungsänderung handelt, die wir brauchen, um das Modell der ständigen Mitgliederversammlung parteigesetzkonform durchführen zu können. \textbf{Darum lautet die Frage dieser Initiative nicht {\Gu}Wollen wir die ständige Mitgliederversammlung einführen?{\Go}, sondern vielmehr {\Gu}Wollen wir ein Konzept zur Durchführung einer ständigen Mitgliederversammlung erarbeiten und auf dem nächsten Landesparteitag, wenn es unseren Ansprüchen genügt, verabschieden?{\Go}, denn wenn sich der Landesverband strikt gegen die Idee aussprechen sollte, über ein Onlinetool verbindliche Beschlüsse zu fassen, macht es keinen Sinn über genaue Konzepte zur Durchführung nachzudenken.}

Da ich weiß, dass viele aufgrund der oben beschriebenen Wahlcomputerproblematik die ständige Mitgliederversammlung grundsätzlich ablehnen, werde ich dennoch kurz auf diesen Punkt eingehen.

Die meisten Abstimmungen auf Landesparteitagen sind öffentlich Wahlen. Nur bei äußerst kontroversen Themen wird geheim abgestimmt. Deswegen ist es im oben verlinkten Geschäftsordnungsvorschlag von Niels Lohmann vorgesehen, dass bei verbindlichen Abstimmungen im LiquidFeedback die Funktion eingebaut werden soll, eine geheime Abstimmung zu beantragen. Ähnlich wie auf Bundesparteitagen sind dafür die Stimmen von 10\% der abstimmenden Piraten erforderlich. \textbf{Ist ein Antrag auf geheime Abstimmung angenommen, wird dieser auf dem nächsten Landesparteitag geheim abgestimmt.} Dort kann dann der Antrag unter Wahrung der der Wahlprinzipien abgestimmt werden. So ist es möglich bei den Themen bei denen sich die Piraten des Landesverbandes einig sind, zwischen den Parteitagen offizielle Positionen zu verabschieden. Äußerst kontroverse Diskussionen werden dann auf den Landesparteitag verlagert und dort wie gehabt abgestimmt. }

% -----

\satzung{Gebietsversammlungen}\label{saa:gebietsversammlung2}
\antrag{Stephan Schurig}
\begin{itemize}
\item \konkurrenz{saa:gebietsversammlung1}
\end{itemize}

\paragraph{Antragstext}:

Der Landesparteitag möge folgende Satzungsänderung (§ 9) und Erweiterung (§ 9c) beschließen: 

\einruecken{\textbf{§ 9 - Organe des Landesverbands}

(1) Organe sind der Vorstand, der Landesparteitag, das Landesschiedsgericht, \textbf{die Gebietsversammlung} und die Gründungsversammlung.

(...)

\textbf{§ 9c - Gebietsversammlung}

(1) Eine Gebietsversammlung ist die Versammlung aller Piraten eines Landkreises, einer Gemeinde, einer Stadt oder eines Ortsteils im Bundesland Sachsen-Anhalt.

(2) Die Gebietsversammlung ist ein Organ der niedrigsten Untergliederung, der das Gebiet vollständig umfasst. Diese Untergliederung wird im Folgenden als "zuständige Gliederung" bezeichnet. Ist das Gebiet der Gebietsversammlung identisch mit dem Gebiet der zuständigen Untergliederung, so ist die Gebietsversammlung zugleich das höchste Organ dieser Untergliederung.

(3) Der Vorstand des zuständigen Verbands vertritt die Interessen der Gebietsversammlung nach Maßgabe ihrer Beschlüsse, sofern die Gebietsversammlung keine Personen aus ihrer Mitte damit beauftragt. Der Vorsitzende oder der stellvertretende Vorsitzende des zuständigen Verbands sind befugt, die Wahlvorschläge für Wahlen zu Volksvertretungen einzureichen und zu unterzeichnen, soweit hierüber keine gesetzlichen Vorschriften bestehen. Wahlvorschläge werden von der jeweils größten Gebietsversammlung bestimmt, die nach dem Wahlgesetz möglich ist.

(4) Die Gebietsversammlung entscheidet je nach Gebietsart über

1. die Aufstellung von Kandidaten für die Wahl zum Landrat

2. die Aufstellung von Direktkandidaten für die Wahl zum Landtag entsprechend den gesetzlichen Regelungen.

3. Die Aufstellung von Direktkandidaten für Bundestagswahlkreise

4. Wichtige, ausschließlich das Gebiet betreffende politische Fragen

5. über die Gründung eines Kreis- oder Regionalverbandes entsprechend § 7 (3)

6. gegebenenfalls weitere ihr nach der Satzung des zuständigen Verbands zukommende Aufgaben

(5) Stimmberechtigt ist jeder nach dem Landes- oder Bundeswahlgesetz im Gebiet wahlberechtigte Pirat, der nicht länger als 3 Monate mit seinem Mitgliedsbeitrag im Rückstand ist. Ist die Gebietsversammlung höchstes Organ des zuständigen Verbands, so haben auch die vom Verband aufgenommenen Mitglieder ohne Wahlrecht im Gebiet ein Stimmrecht in allen Wahlen und Abstimmungen, bei denen dies nicht vom Wahlgesetz ausgeschlossen ist.

(6) Eine Gebietsversammlung wird vom Vorstand des zuständigen Verbands einberufen, wenn

1. der betreffende Vorstand es beschließt

2. mindestens 10\% der und mindestens drei Mitglieder des Gebiets es verlangen

3. Entscheidungen nach Absatz 4 dieses Paragrafen anstehen

(7) Gibt sich die Gebietsversammlung keine eigene Wahl- und Geschäftsordnung, gilt die aktuelle Wahl- und Geschäftsordnung des zuständigen Verbands.

(8) Gebietsversammlungen können mit anderen Gebietsversammlungen oder der Landesmitgliederversammlung örtlich und zeitlich zusammengelegt werden und an einem beliebigen Ort innerhalb der Landesgrenzen von Sachsen-Anhalt stattfinden.

(9) Eine Gebietsversammlung ist beschlussfähig, wenn mindestens drei Piraten und mindestens 5\% der Piraten des Gebiets akkreditiert sind.

(10) Für die Einladung zu einer Gebietsversammlung gelten grundsätzlich die gleichen Regelungen wie zur Mitgliederversammlung des zuständigen Verbands. Der Vorstand des zuständigen Verbands kann jedoch abweichende Regelungen beschließen, wenn die jeweilige Gebietsversammlung nicht zugleich das höchste Organ des Verbands ist.}

\begruendung{Momentan ist es nicht möglich, dass Gebiete ohne einer gegründeten Untergliederung (Kreisverband, Regionalverband etc.) eine Mitgliederversammlung einberufen können, um ein Programm zu entwickeln, Kandidat\_innen aufzustellen o.ä. Lediglich zur Gründung einer Untergliederung können sich solche Versammlungen zusammenfinden. Dies soll regionale Gruppen stärken und eine Alternative für die Gründung eines Verbandes darstellen.

Von einem Piraten (sigi) habe ich dazu vereinfacht folgende (verständlichere) Aussage erhalten: {\Gu}in Berlin gibts nur den LV, auf Bezirksebene haben wir keine Gliederungen. In solchen Fällen, wie Du es schilderst gibt es hier Gebietsversammlungen, zu denen der Vorstand des LV einlädt. Ein Vertreter des LaVo ist zwingend anwesend. Hier können Beschlüsse gefasst werden, die dann auch nur für den Bezirk gelten (können){\Go}

\begin{itemize}
\item \href{http://www.piraten-lsa.de/satzung}{Satzung der Piratenpartei Sachsen-Anhalt}
\item \href{http://wiki.piratenpartei.de/BE:Satzung\#.C2.A7\_9\_GEBIETSVERSAMMLUNGEN}{Satzung der Piratenpartei Berlin zu den Gebietsversammlungen}
\end{itemize}
}

% -----

\satzung{Gründung von Untergliederungen}\label{saa:neogruendung1}
\antrag{Stephan Schurig}
\begin{itemize}
\item \konkurrenz{saa:neogruendung2}
\item \konkurrenz{saa:neogruendung3}
\item \konkurrenz{saa:neogruendung4}
\item \konkurrenz{saa:synexgruendung1}
\item \konkurrenz{saa:synexgruendung2}
\item \konkurrenz{saa:synexgruendung3}
\end{itemize}

\paragraph{Antragstext}:

Der Landesparteitag möge beschließen folgenden Abschnitt in der Satzung des Landesverbandes zu ändern: 

\einruecken{§ 7 - Gliederung

(2) Regionalverbände sind Kreisverbände im Sinne der Bundessatzung, deren Gebiet sich über mehr als einen \textbf{Landkreis und/oder einer kreisfreien Stadt} erstreckt. Eine Koexistenz von Kreis- und Regionalverband auf dem selben Gebiet ist nicht zulässig.

\textbf{(3) Auf Verlangen von mindestens drei gründungswilligen Piraten lädt der Landesvorstand alle Piraten mit angezeigtem Wohnsitz im Gebiet des künftigen Kreisverbands zu einer Gründungsversammlung ein. Ort und Zeit der Gründungsversammlung werden von den gründungswilligen Piraten bestimmt, wobei die Ladungsfrist mindestens vier Wochen beträgt. Die Gründungsversammlung ist beschlussfähig, wenn mindestens zehn stimmberechtigte Piraten erschienen sind. Der Kreisverband ist gegründet, wenn auf der Gründungsversammlung dessen Satzung beschlossen worden ist. Für den Beschluss ist eine Mehrheit von 2/3 der abgegebenen Stimmen erforderlich. Über die Versammlung ist ein Protokoll anzufertigen und zu binnen eines Monats zu veröffentlichen.}

(4) Sofern der zuständige Kreisverband keine anderen Regelungen getroffen hat, gilt für die Gründung von Ortsverbänden Absatz (3).}

\paragraph{Alte Fassung}:

\einruecken{§ 7 - Gliederung

(2) Regionalverbände sind Kreisverbände im Sinne der Bundessatzung, deren Gebiet sich über mehr als einen \textbf{politischen Kreis} erstreckt. Eine Koexistenz von Kreis- und Regionalverband auf dem selben Gebiet ist nicht zulässig.

\textbf{(3) Der Gründung eines Kreisverbandes müssen mindestens drei akkreditierte Piraten aus jedem politischen Kreis mehrheitlich zustimmen. Insgesamt müssen der Gründung mindestens zehn akkreditierte Piraten mehrheitlich zustimmen.}

(4) Sofern der zuständige Kreisverband keine anderen Regelungen getroffen hat, gilt für die Gründung von Ortsverbänden Absatz (3) \textbf{Satz 2.}}

\begruendung{Der momentane Passus ist nicht eindeutig. Der Vorschlag zur Änderung ist von der \href{http://wiki.piratenpartei.de/MV:Satzung\#.C2.A7\_7\_-\_Gliederung}{Satzung der Piratenpartei MV} übernommen und leicht abgeändert.}

% -----

\satzung{Gründung von Untergliederungen (Mindestzahl 5 Gründungswillige)}\label{saa:neogruendung2}
\antrag{Stephan Schurig}
\begin{itemize}
\item \konkurrenz{saa:neogruendung1}
\item \konkurrenz{saa:neogruendung3}
\item \konkurrenz{saa:neogruendung4}
\item \konkurrenz{saa:synexgruendung1}
\item \konkurrenz{saa:synexgruendung2}
\item \konkurrenz{saa:synexgruendung3}
\end{itemize}

\paragraph{Antragstext}:

Der Landesparteitag möge beschließen folgenden Abschnitt in der Satzung des Landesverbandes zu ändern: 

\einruecken{\textbf{§ 7 - Gliederung}

(2) Regionalverbände sind Kreisverbände im Sinne der Bundessatzung, deren Gebiet sich über mehr als einen \textbf{Landkreis und/oder einer kreisfreien Stadt} erstreckt. Eine Koexistenz von Kreis- und Regionalverband auf dem selben Gebiet ist nicht zulässig.

(3) Auf Verlangen von mindestens \textbf{fünf} gründungswilligen Piraten lädt der Landesvorstand alle Piraten mit angezeigtem Wohnsitz im Gebiet des künftigen Kreis- oder Regionalverbands zu einer Gründungsversammlung ein. Ort und Zeit der Gründungsversammlung werden von den gründungswilligen Piraten bestimmt, wobei die Ladungsfrist mindestens vier Wochen beträgt. Die Gründungsversammlung ist beschlussfähig, wenn mindestens \textbf{zwanzig} stimmberechtigte Piraten erschienen sind. Der Kreis- oder Regionalverband ist gegründet, wenn auf der Gründungsversammlung dessen Satzung beschlossen worden ist. Für den Beschluss ist eine Mehrheit von 2/3 der abgegebenen Stimmen erforderlich. Über die Versammlung ist ein Protokoll anzufertigen und zu binnen eines Monats zu veröffentlichen.

(4) Sofern der zuständige Kreisverband keine anderen Regelungen getroffen hat, gilt für die Gründung von Ortsverbänden Absatz (3). }

\paragraph{Alte Fassung}:

\einruecken{\textbf{§ 7 - Gliederung}

(2) Regionalverbände sind Kreisverbände im Sinne der Bundessatzung, deren Gebiet sich über mehr als einen \textbf{politischen Kreis} erstreckt. Eine Koexistenz von Kreis- und Regionalverband auf dem selben Gebiet ist nicht zulässig.

(3) Der Gründung eines Kreisverbandes müssen mindestens \textbf{drei} akkreditierte Piraten aus jedem politischen Kreis mehrheitlich zustimmen. Insgesamt müssen der Gründung mindestens \textbf{zehn} akkreditierte Piraten mehrheitlich zustimmen.

(4) Sofern der zuständige Kreisverband keine anderen Regelungen getroffen hat, gilt für die Gründung von Ortsverbänden Absatz (3) \textbf{Satz 2.}}

\begruendung{Der momentane Passus ist nicht eindeutig. Der Vorschlag zur Änderung ist von der \href{http://wiki.piratenpartei.de/MV:Satzung\#.C2.A7\_7\_-\_Gliederung}{Satzung der Piratenpartei MV} übernommen und leicht abgeändert. Andere Zahlen sind fett markiert.

Basisdemokratie ist abhängig von einer kritischen Masse. Die kritische Masse steigt idealerweise mit der realen Masse von Personen. Je höher die Grundgesamtheit ist, die notwendig ist für eine Untergliederungsgründung, desto besser ist sie geschützt vor Missbrauch einzelner Gruppen, Lobbys etc.

\textbf{Anstatt weitere Strukturen zu schaffen, die keine kritische Masse erreichen, sollten wir uns stattdessen auf andere Konzepte konzentrieren, um Basisdemokratie sicherzustellen, aber auch möglichst liquide zu halten. So empfehle ich allen die Diskussion um eine \href{http://lqfb.piraten-lsa.de/lsa/initiative/show/235.html}{Ständige Mitgliederversammlung} und den \href{http://lqfb.piraten-lsa.de/lsa/initiative/show/219.html}{Gebietsversammlungen} zu verfolgen und sich mit einzubringen.}

Weitere Lektüre unter: \href{http://wiki.piratenpartei.de/Strukturdebatte}{Strukturdebatte}}

% -----

\satzung{Deadline auch für Programmanträge}\label{saa:deadline1}
\antrag{Karl}
\begin{itemize}
\item \konkurrenz{saa:deadline2}
\end{itemize}

\paragraph{Antragstext}:

Der Landesparteitag möge beschließen die Satzung der Piratenpartei Sachsen-Anhalt wie folgt zu ändern: 

\einruecken{(1) Änderungen der Landessatzung \textbf{und dem Grundsatzprogramm} können nur von einem Landesparteitag mit einer 2/3 Mehrheit beschlossen werden. Besteht das dringende Erfordernis einer Satzungsänderung zwischen zwei Landesparteitagen, so kann die Satzung auch geändert werden, wenn mindestens 2/3 der Piraten sich mit dem Antrag/den Anträgen auf Änderung schriftlich (Brief, Email oder Fax) einverstanden erklären.


(2) Über einen Antrag auf \textbf{Satzungs- oder Programmänderung} auf einem Landesparteitag kann nur abgestimmt werden, wenn er mindestens drei Wochen vor Beginn des Landesparteitages beim Vorstand eingegangen ist.


\textbf{(3) Ausgenommen von dieser Frist sind Änderungsanträge, die sich auf nach Punkt (2) beantragte Programmanträge beziehen. Diese können auch vor Ort gestellt werden.}}

\paragraph{Alte Fassung}:

\einruecken{(1) Änderungen der Landessatzung können nur von einem Landesparteitag mit einer 2/3 Mehrheit beschlossen werden. Besteht das dringende Erfordernis einer Satzungsänderung zwischen zwei Landesparteitagen, so kann die Satzung auch geändert werden, wenn mindestens 2/3 der Piraten sich mit dem Antrag/den Anträgen auf Änderung schriftlich (Brief, Email oder Fax) einverstanden erklären.

(2) Über einen Antrag auf \textbf{Satzungsänderung} auf einem Landesparteitag kann nur abgestimmt werden, wenn er mindestens zwei Wochen vor Beginn des Landesparteitages beim Vorstand eingegangen ist.}

\begruendung{Um die Qualität der Anträge zu erhöhen, halte ich eine Regelung ähnlich der für den Bundesparteitag für sinnvoll. Dadurch ist gewährleistet, dass sowohl die Antragsteller als auch die anwesenden stimmberechtigten Piraten besser vorbereitet sind.}

% -----

\satzung{Gründung von Untergliederungen (Mindestzahl 10 Gründungswillige)}\label{saa:neogruendung3}
\antrag{Stephan Schurig}
\begin{itemize}
\item \konkurrenz{saa:neogruendung1}
\item \konkurrenz{saa:neogruendung2}
\item \konkurrenz{saa:neogruendung4}
\item \konkurrenz{saa:synexgruendung1}
\item \konkurrenz{saa:synexgruendung2}
\item \konkurrenz{saa:synexgruendung3}
\end{itemize}

\paragraph{Antragstext}:

Der Landesparteitag möge beschließen folgenden Abschnitt in der Satzung des Landesverbandes zu ändern: 

\einruecken{\textbf{§ 7 - Gliederung}

(2) Regionalverbände sind Kreisverbände im Sinne der Bundessatzung, deren Gebiet sich über mehr als einen \textbf{Landkreis und/oder einer kreisfreien Stadt} erstreckt. Eine Koexistenz von Kreis- und Regionalverband auf dem selben Gebiet ist nicht zulässig.

(3) Auf Verlangen von mindestens \textbf{zehn} gründungswilligen Piraten lädt der Landesvorstand alle Piraten mit angezeigtem Wohnsitz im Gebiet des künftigen Kreis- oder Regionalverbands zu einer Gründungsversammlung ein. Ort und Zeit der Gründungsversammlung werden von den gründungswilligen Piraten bestimmt, wobei die Ladungsfrist mindestens vier Wochen beträgt. Die Gründungsversammlung ist beschlussfähig, wenn mindestens \textbf{zwanzig} stimmberechtigte Piraten erschienen sind. Der Kreis- oder Regionalverband ist gegründet, wenn auf der Gründungsversammlung dessen Satzung beschlossen worden ist. Für den Beschluss ist eine Mehrheit von 2/3 der abgegebenen Stimmen erforderlich. Über die Versammlung ist ein Protokoll anzufertigen und zu binnen eines Monats zu veröffentlichen.

(4) Sofern der zuständige Kreisverband keine anderen Regelungen getroffen hat, gilt für die Gründung von Ortsverbänden Absatz (3). }

\paragraph{Alte Fassung}:

\einruecken{\textbf{§ 7 - Gliederung}

(2) Regionalverbände sind Kreisverbände im Sinne der Bundessatzung, deren Gebiet sich über mehr als einen \textbf{politischen Kreis} erstreckt. Eine Koexistenz von Kreis- und Regionalverband auf dem selben Gebiet ist nicht zulässig.

(3) Der Gründung eines Kreisverbandes müssen mindestens \textbf{drei} akkreditierte Piraten aus jedem politischen Kreis mehrheitlich zustimmen. Insgesamt müssen der Gründung mindestens \textbf{zehn} akkreditierte Piraten mehrheitlich zustimmen.

(4) Sofern der zuständige Kreisverband keine anderen Regelungen getroffen hat, gilt für die Gründung von Ortsverbänden Absatz (3) \textbf{Satz 2.}}

\begruendung{Der momentane Passus ist nicht eindeutig. Der Vorschlag zur Änderung ist von der \href{http://wiki.piratenpartei.de/MV:Satzung\#.C2.A7\_7\_-\_Gliederung}{Satzung der Piratenpartei MV} übernommen und leicht abgeändert. Andere Zahlen sind fett markiert.

Basisdemokratie ist abhängig von einer kritischen Masse. Die kritische Masse steigt idealerweise mit der realen Masse von Personen. Je höher die Grundgesamtheit ist, die notwendig ist für eine Untergliederungsgründung, desto besser ist sie geschützt vor Missbrauch einzelner Gruppen, Lobbys etc.

\textbf{Anstatt weitere Strukturen zu schaffen, die keine kritische Masse erreichen, sollten wir uns stattdessen auf andere Konzepte konzentrieren, um Basisdemokratie sicherzustellen, aber auch möglichst liquide zu halten. So empfehle ich allen die Diskussion um eine \href{http://lqfb.piraten-lsa.de/lsa/initiative/show/235.html}{Ständige Mitgliederversammlung} und den \href{http://lqfb.piraten-lsa.de/lsa/initiative/show/219.html}{Gebietsversammlungen} zu verfolgen und sich mit einzubringen.}

Weitere Lektüre unter: \href{http://wiki.piratenpartei.de/Strukturdebatte}{Strukturdebatte}}

% -----

\satzung{Gründung von Untergliederungen (Mindestzahl 20 Gründungswillige)}\label{saa:neogruendung4}
\antrag{Stephan Schurig}
\begin{itemize}
\item \konkurrenz{saa:neogruendung1}
\item \konkurrenz{saa:neogruendung2}
\item \konkurrenz{saa:neogruendung3}
\item \konkurrenz{saa:synexgruendung1}
\item \konkurrenz{saa:synexgruendung2}
\item \konkurrenz{saa:synexgruendung3}
\end{itemize}

\paragraph{Antragstext}:

Der Landesparteitag möge beschließen folgenden Abschnitt in der Satzung des Landesverbandes zu ändern: 

\einruecken{\textbf{§ 7 - Gliederung}

(2) Regionalverbände sind Kreisverbände im Sinne der Bundessatzung, deren Gebiet sich über mehr als einen \textbf{Landkreis und/oder einer kreisfreien Stadt} erstreckt. Eine Koexistenz von Kreis- und Regionalverband auf dem selben Gebiet ist nicht zulässig.

(3) Auf Verlangen von mindestens \textbf{zwanzig} gründungswilligen Piraten lädt der Landesvorstand alle Piraten mit angezeigtem Wohnsitz im Gebiet des künftigen Kreis- oder Regionalverbands zu einer Gründungsversammlung ein. Ort und Zeit der Gründungsversammlung werden von den gründungswilligen Piraten bestimmt, wobei die Ladungsfrist mindestens vier Wochen beträgt. Die Gründungsversammlung ist beschlussfähig, wenn mindestens \textbf{40} stimmberechtigte Piraten erschienen sind. Der Kreis- oder Regionalverband ist gegründet, wenn auf der Gründungsversammlung dessen Satzung beschlossen worden ist. Für den Beschluss ist eine Mehrheit von 2/3 der abgegebenen Stimmen erforderlich. Über die Versammlung ist ein Protokoll anzufertigen und zu binnen eines Monats zu veröffentlichen.

(4) Sofern der zuständige Kreisverband keine anderen Regelungen getroffen hat, gilt für die Gründung von Ortsverbänden Absatz (3). }

\paragraph{Alte Fassung}:

\einruecken{\textbf{§ 7 - Gliederung}

(2) Regionalverbände sind Kreisverbände im Sinne der Bundessatzung, deren Gebiet sich über mehr als einen \textbf{politischen Kreis} erstreckt. Eine Koexistenz von Kreis- und Regionalverband auf dem selben Gebiet ist nicht zulässig.

(3) Der Gründung eines Kreisverbandes müssen mindestens \textbf{drei} akkreditierte Piraten aus jedem politischen Kreis mehrheitlich zustimmen. Insgesamt müssen der Gründung mindestens \textbf{zehn} akkreditierte Piraten mehrheitlich zustimmen.

(4) Sofern der zuständige Kreisverband keine anderen Regelungen getroffen hat, gilt für die Gründung von Ortsverbänden Absatz (3) \textbf{Satz 2.}}

\begruendung{Der momentane Passus ist nicht eindeutig. Der Vorschlag zur Änderung ist von der \href{http://wiki.piratenpartei.de/MV:Satzung\#.C2.A7\_7\_-\_Gliederung}{Satzung der Piratenpartei MV} übernommen und leicht abgeändert. Andere Zahlen sind fett markiert.

Basisdemokratie ist abhängig von einer kritischen Masse. Die kritische Masse steigt idealerweise mit der realen Masse von Personen. Je höher die Grundgesamtheit ist, die notwendig ist für eine Untergliederungsgründung, desto besser ist sie geschützt vor Missbrauch einzelner Gruppen, Lobbys etc.

\textbf{Anstatt weitere Strukturen zu schaffen, die keine kritische Masse erreichen, sollten wir uns stattdessen auf andere Konzepte konzentrieren, um Basisdemokratie sicherzustellen, aber auch möglichst liquide zu halten. So empfehle ich allen die Diskussion um eine \href{http://lqfb.piraten-lsa.de/lsa/initiative/show/235.html}{Ständige Mitgliederversammlung} und den \href{http://lqfb.piraten-lsa.de/lsa/initiative/show/219.html}{Gebietsversammlungen} zu verfolgen und sich mit einzubringen.}

Weitere Lektüre unter: \href{http://wiki.piratenpartei.de/Strukturdebatte}{Strukturdebatte}}

% -----

\satzung{Einladung LPT (nur an stimmberechtigte Piraten)}
\antrag{Stephan Schurig}

\paragraph{Antragstext}:

Der Landesparteitag beschließt, § 9b Abs. (2) der Landessatzung wie folgt zu ändern:

\einruecken{Der Landesparteitag tagt mindestens einmal jährlich. Die Einberufung erfolgt aufgrund eines Vorstandsbeschlusses. Wenn ein Zehntel der Piraten, mindestens aber zehn Piraten es beim Vorstand beantragen, muss dieser binnen 2 Wochen einen Parteitag einberufen. Der Vorstand lädt jedes Mitglied schriftlich (Brief, Email oder Fax) mindestens 4 Wochen vorher ein. \textbf{Eine Einladung entfällt, wenn ein Mitglied weder stimmberechtigt ist, noch eine Email-Adresse oder Postanschrift vorliegt. Zusätzlich wird die Einladung auf der Website des Landesverbandes publiziert.} Die Einladung hat Angaben zum Tagungsort, Tagungsbeginn, vorläufiger Tagesordnung und der Angabe, wo weitere, aktuelle Veröffentlichungen gemacht werden, zu enthalten. Spätestens 1 Wochen vor dem Parteitag sind die Tagesordnung in aktueller Fassung, die geplante Tagungsdauer und alle bis dahin dem Vorstand eingereichten Anträge im Wortlaut zu veröffentlichen.}

\paragraph{Alte Fassung}:

\einruecken{Der Landesparteitag tagt mindestens einmal jährlich. Die Einberufung erfolgt aufgrund eines Vorstandsbeschlusses. Wenn ein Zehntel der Piraten, mindestens aber zehn Piraten es beim Vorstand beantragen, muss dieser binnen 2 Wochen einen Parteitag einberufen. Der Vorstand lädt jedes Mitglied schriftlich (Brief, Email oder Fax) mindestens 4 Wochen vorher ein. Die Einladung hat Angaben zum Tagungsort, Tagungsbeginn, vorläufiger Tagesordnung und der Angabe, wo weitere, aktuelle Veröffentlichungen gemacht werden, zu enthalten. Spätestens 1 Wochen vor dem Parteitag sind die Tagesordnung in aktueller Fassung, die geplante Tagungsdauer und alle bis dahin dem Vorstand eingereichten Anträge im Wortlaut zu veröffentlichen.}

\begruendung{Wenn ein Mitglied weder stimmberechtigt ist, noch eine Email-Adresse oder Postanschrift vorliegt, wird keine Einladung verschickt. Ansonsten werden auch Mitglieder, die nicht stimmberechtigt sind zu einem LPT eingeladen.}

% -----

\satzung{keine Reallife-Vorstandssitzungen, aber sechs Mal Treffen während der reg. Amtsperiode}
\antrag{Stephan Schurig}

\paragraph{Antragstext}:

Der LPT beschließt § 9a Abs. (4) Satz 1 der Landessatzung wie folgt zu ändern:

\einruecken{Der Vorstand tritt in seiner regulären Amtsperiode mindestens sechsmal zusammen.}

\paragraph{Alte Fassung}:

\einruecken{Der Vorstand tritt in seiner regulären Amtsperiode mindestens zweimal auf einem persönlichen Treffen zusammen.}

\begruendung{RL-Treffen sollten nicht per Satzung fest sein, allerdings die Anzahl der Vorstandstreffen auf mindestens 6 pro Amtsperiode (ca. 1 Jahr), d.h. mindestens ein Vorstandstreffen alle 2 Monate.}

% -----

\satzung{Deadline auch für Programmanträge UND Positionspapiere}\label{saa:deadline2}
\antrag{Karl}
\begin{itemize}
\item \konkurrenz{saa:deadline1}
\end{itemize}

\paragraph{Antragstext}:

Der Antrag steht in Konkurrenz zu SÄA010 - 'Deadline auch für Programmanträge'. Der Landesparteitag möge beschließen die Satzung der Piratenpartei Sachsen-Anhalt wie folgt zu ändern:

\einruecken{(1) Änderungen der \textbf{Landessatzung und dem Grundsatzprogramm} können nur von einem Landesparteitag mit einer 2/3 Mehrheit beschlossen werden. Besteht das dringende Erfordernis einer Satzungsänderung zwischen zwei Landesparteitagen, so kann die Satzung auch geändert werden, wenn mindestens 2/3 der Piraten sich mit dem Antrag/den Anträgen auf Änderung schriftlich (Brief, Email oder Fax) einverstanden erklären.

(2) Über einen Antrag auf \textbf{Annahme eines Positionspapiers, Satzungs- oder Programmänderung} auf einem Landesparteitag kann nur abgestimmt werden, wenn er mindestens zwei Wochen vor Beginn des Landesparteitages beim Vorstand eingegangen ist.

\textbf{(3) Ausgenommen von dieser Frist sind Änderungsanträge, die sich auf nach Punkt (2) beantragte Programmanträge beziehen, sowie nach Punkt (2) beantragte Programmanträge, die nachträglich als Positionspapiere eingereicht wurden. Diese können auch vor Ort gestellt werden.}}

\paragraph{Alte Fassung}:

\einruecken{(1) Änderungen der \textbf{Landessatzung} können nur von einem Landesparteitag mit einer 2/3 Mehrheit beschlossen werden. Besteht das dringende Erfordernis einer \textbf{Satzungsänderung} zwischen zwei Landesparteitagen, so kann die Satzung auch geändert werden, wenn mindestens 2/3 der Piraten sich mit dem Antrag/den Anträgen auf Änderung schriftlich (Brief, Email oder Fax) einverstanden erklären.

(2) Über einen Antrag auf Satzungsänderung auf einem Landesparteitag kann nur abgestimmt werden, wenn er mindestens zwei Wochen vor Beginn des Landesparteitages beim Vorstand eingegangen ist.}

\begruendung{Um die Qualität der Anträge zu erhöhen, halte ich eine Regelung ähnlich der für den Bundesparteitag für sinnvoll. Dadurch ist gewährleistet, dass sowohl die Antragsteller als auch die anwesenden stimmberechtigten Piraten besser vorbereitet sind.

Ich sehe keinen Grund, warum diese Regelung nicht auch für Positionspapiere gelten sollte.}

% -----

\satzung{Wiedereinführung des Vorstandsamtes 'Generalsekretär'}
\antrag{Christoph Giesel}

\paragraph{Antragstext}:

Der Landesparteitag möge §9a (1) wie folgt ändern:

\einruecken{(1) Dem Vorstand gehören mindestens vier Piraten an: Der Vorsitzende, der stellvertretende Vorsitzende, \textbf{der Schatzmeister und der Generalsekretär}. Der Landesparteitag kann zusätzlich bis zu \textbf{fünf} Beisitzer zu Vorstandsmitgliedern wählen.}

\paragraph{Alte Fassung}:

\einruecken{(1) Dem Vorstand gehören mindestens drei Piraten an: Der Vorsitzende, der stellvertretende Vorsitzende \textbf{und der Schatzmeister}. Der Landesparteitag kann zusätzlich bis zu \textbf{sechs} Beisitzer zu Vorstandsmitgliedern wählen.}

\begruendung{Der Generalsekretär ist zuständig für die Mitgliederverwaltung und Kommunikation mit den Mitgliedern. Die Mitglieder des Landesverbandes sollten einen festen Ansprechpartner für alle Belange bezüglich ihrer Mitgliedschaft haben. Zudem brauchen die Untergliederungen ohne Zugriff auf Mitgliedsdaten (Stammtische, Crews) einen Ansprechpartner für Weiterleitung von Informationen. Die Einrichtung 'Generalsekretariat' hat sich in meinen Augen nicht bewährt.}

% -----

\satzung{Gründung von Untergliederungen (Alternative 02: 20 insgesamt - 5/Kreis)}\label{saa:synexgruendung1}
\antrag{Christoph Giesel}
\begin{itemize}
\item \konkurrenz{saa:neogruendung1}
\item \konkurrenz{saa:neogruendung2}
\item \konkurrenz{saa:neogruendung3}
\item \konkurrenz{saa:neogruendung4}
\item \konkurrenz{saa:synexgruendung2}
\item \konkurrenz{saa:synexgruendung3}
\end{itemize}

\paragraph{Antragstext}:

Der Landesparteitag möge beschließen, die Absätze 2, 3 und 4 von §7 in der Satzung des Landesverbandes zu ändern: 

\einruecken{(2) Regionalverbände sind Kreisverbände im Sinne der Bundessatzung, deren Gebiet sich über mehr als einen \textbf{Landkreis und/oder kreisfreien Stadt} erstreckt. Eine Koexistenz von Kreis- und Regionalverband auf dem selben Gebiet ist nicht zulässig.

(3) Der Gründung eines Kreis\textbf{- oder Regionalverbandes} müssen mindestens \textbf{2/3, mindestens aber zwanzig, der akkreditierten Piraten zustimmen. Erstreckt sich der zu gründende Verband über mehrere Landkreise und/oder kreisfreien Städte, so müssen außerdem für jeden Landkreis bzw. kreisfreien Stadt mindestens 2/3, mindestens aber fünf, der im Landkreis bzw. kreisfreien Stadt wohnenden} akkreditierten Piraten zustimmen.

(4) \textbf{Kreisverbände können eigene Regelungen für die Gründung von Ortsverbänden festlegen. Sofern keine Regelungen getroffen wurden, gilt für die Gründung von Ortsverbänden Absatz (3) entsprechend.}}

\paragraph{Alte Fassung}:

\einruecken{(2) Regionalverbände sind Kreisverbände im Sinne der Bundessatzung, deren Gebiet sich über mehr als einen \textbf{politischen Kreis} erstreckt. Eine Koexistenz von Kreis- und Regionalverband auf dem selben Gebiet ist nicht zulässig.

(3) Der Gründung eines Kreis\textbf{verbandes} müssen mindestens \textbf{drei akkreditierte Piraten aus jedem politischen Kreis mehrheitlich zustimmen. Insgesamt müssen der Gründung mindestens zehn} akkreditierte Piraten \textbf{mehrheitlich} zustimmen.

(4) \textbf{Sofern der zuständige Kreisverband keine anderen Regelungen getroffen hat, gilt für die Gründung von Ortsverbänden Absatz (3) Satz 2.}}

\begruendung{Ich habe versucht den Paragraphen ein bisschen verständlicher zu formulieren. Inhaltlich hat sich einmal das {\Gu}mehrheitlich{\Go} geändert. Außerdem wird jetzt festgelegt, dass jeweils 2/3 zustimmen müssen.}

% -----

\satzung{Gründung von Untergliederungen (Alternative 01: 10 insgesamt - 3/Kreis)}\label{saa:synexgruendung2}
\antrag{Christoph Giesel}
\begin{itemize}
\item \konkurrenz{saa:neogruendung1}
\item \konkurrenz{saa:neogruendung2}
\item \konkurrenz{saa:neogruendung3}
\item \konkurrenz{saa:neogruendung4}
\item \konkurrenz{saa:synexgruendung1}
\item \konkurrenz{saa:synexgruendung3}
\end{itemize}

\paragraph{Antragstext}:

Der Landesparteitag möge beschließen, die Absätze 2, 3 und 4 von §7 in der Satzung des Landesverbandes zu ändern: 

\einruecken{(2) Regionalverbände sind Kreisverbände im Sinne der Bundessatzung, deren Gebiet sich über mehr als einen \textbf{Landkreis und/oder kreisfreien Stadt} erstreckt. Eine Koexistenz von Kreis- und Regionalverband auf dem selben Gebiet ist nicht zulässig.

(3) Der Gründung eines Kreis\textbf{- oder Regionalverbandes} müssen mindestens \textbf{2/3, mindestens aber zehn, der akkreditierten Piraten zustimmen. Erstreckt sich der zu gründende Verband über mehrere Landkreise und/oder kreisfreien Städte, so müssen außerdem für jeden Landkreis bzw. kreisfreien Stadt mindestens 2/3, mindestens aber drei, der im Landkreis bzw. kreisfreien Stadt wohnenden} akkreditierten Piraten zustimmen.

(4) \textbf{Kreisverbände können eigene Regelungen für die Gründung von Ortsverbänden festlegen. Sofern keine Regelungen getroffen wurden, gilt für die Gründung von Ortsverbänden Absatz (3) entsprechend.}}

\paragraph{Alte Fassung}:

\einruecken{(2) Regionalverbände sind Kreisverbände im Sinne der Bundessatzung, deren Gebiet sich über mehr als einen \textbf{politischen Kreis} erstreckt. Eine Koexistenz von Kreis- und Regionalverband auf dem selben Gebiet ist nicht zulässig.

(3) Der Gründung eines Kreis\textbf{verbandes} müssen mindestens \textbf{drei akkreditierte Piraten aus jedem politischen Kreis mehrheitlich zustimmen. Insgesamt müssen der Gründung mindestens zehn} akkreditierte Piraten \textbf{mehrheitlich} zustimmen.

(4) \textbf{Sofern der zuständige Kreisverband keine anderen Regelungen getroffen hat, gilt für die Gründung von Ortsverbänden Absatz (3) Satz 2.}}

\begruendung{Ich habe versucht den Paragraphen ein bisschen verständlicher zu formulieren. Inhaltlich hat sich einmal das {\Gu}mehrheitlich{\Go} geändert. Außerdem wird jetzt festgelegt, dass jeweils 2/3 zustimmen müssen.}

% -----

\satzung{Gründung von Untergliederungen (Alternative 02: 40 insgesamt - 10/Kreis)}\label{saa:synexgruendung3}
\antrag{Christoph Giesel}
\begin{itemize}
\item \konkurrenz{saa:neogruendung1}
\item \konkurrenz{saa:neogruendung2}
\item \konkurrenz{saa:neogruendung3}
\item \konkurrenz{saa:neogruendung4}
\item \konkurrenz{saa:synexgruendung1}
\item \konkurrenz{saa:synexgruendung2}
\end{itemize}

\paragraph{Antragstext}:

Der Landesparteitag möge beschließen, die Absätze 2, 3 und 4 von §7 in der Satzung des Landesverbandes zu ändern: 

\einruecken{(2) Regionalverbände sind Kreisverbände im Sinne der Bundessatzung, deren Gebiet sich über mehr als einen \textbf{Landkreis und/oder kreisfreien Stadt} erstreckt. Eine Koexistenz von Kreis- und Regionalverband auf dem selben Gebiet ist nicht zulässig.

(3) Der Gründung eines Kreis\textbf{- oder Regionalverbandes} müssen mindestens \textbf{2/3, mindestens aber vierzig, der akkreditierten Piraten zustimmen. Erstreckt sich der zu gründende Verband über mehrere Landkreise und/oder kreisfreien Städte, so müssen außerdem für jeden Landkreis bzw. kreisfreien Stadt mindestens 2/3, mindestens aber zehn, der im Landkreis bzw. kreisfreien Stadt wohnenden} akkreditierten Piraten zustimmen.

(4) \textbf{Kreisverbände können eigene Regelungen für die Gründung von Ortsverbänden festlegen. Sofern keine Regelungen getroffen wurden, gilt für die Gründung von Ortsverbänden Absatz (3) entsprechend.}}

\paragraph{Alte Fassung}:

\einruecken{(2) Regionalverbände sind Kreisverbände im Sinne der Bundessatzung, deren Gebiet sich über mehr als einen \textbf{politischen Kreis} erstreckt. Eine Koexistenz von Kreis- und Regionalverband auf dem selben Gebiet ist nicht zulässig.

(3) Der Gründung eines Kreis\textbf{verbandes} müssen mindestens \textbf{drei akkreditierte Piraten aus jedem politischen Kreis mehrheitlich zustimmen. Insgesamt müssen der Gründung mindestens zehn} akkreditierte Piraten \textbf{mehrheitlich} zustimmen.

(4) \textbf{Sofern der zuständige Kreisverband keine anderen Regelungen getroffen hat, gilt für die Gründung von Ortsverbänden Absatz (3) Satz 2.}}

\begruendung{Ich habe versucht den Paragraphen ein bisschen verständlicher zu formulieren. Inhaltlich hat sich einmal das {\Gu}mehrheitlich{\Go} geändert. Außerdem wird jetzt festgelegt, dass jeweils 2/3 zustimmen müssen.}

% -----

\satzung{Regeln zur Gründung von Untergliederungen}
\antrag{Christoph Giesel}
\begin{itemize}
\item \zusatz{saa:synexgruendung1}
\item \zusatz{saa:synexgruendung2}
\item \zusatz{saa:synexgruendung3}
\end{itemize}

\paragraph{Antragstext}:

Der Landesparteitag möge beschließen, folgenden Absatz in §7 nach Absatz 3 einzuschieben und die nachfolgenden Absätze entsprechend der Nummerierung anzupassen:

\einruecken{(4) Die übergeordnete Gliederung lädt zur Gründung einer Untergliederung ein, wenn mindestens die Hälfte der nach Absatz 3 für die Gründung geforderten Anzahl stimmberechtigter Piraten des Gebietes der zukünftigen Gliederung dies beantragen. Besitzt das Gebiet der zu gründenden Gliederung zum Zeitpunkt des Antrags nicht so viele stimmberechtigte Mitglieder wie in Absatz 3 für die Gründung gefordert, so kann der Antrag abgelehnt werden. Der Vorstand der übergeordneten Gliederung kann hierzu weitere Regeln beschließen.}

Der aktuelle Absatz 4 bzw. zukünftige Absatz 5 des §7 wird wie folgt geändert:

\einruecken{(5) Kreisverbände können eigene Regelungen für die Gründung von Ortsverbänden festlegen. Sofern keine Regelungen getroffen wurden, gilt für die Gründung von Ortsverbänden Absatz (3) und (4) entsprechend.}

\begruendung{Dieser SÄA gilt für meine (Christophs) SÄA bzgl. der Untergliederungen.}

% -----

\satzung{Gliederung im LV }
\antrag{Henning Lübbers}

\paragraph{Antragstext}:

Es wird beantragt in der Landesssatzung den Paragraphen §7 Absatz 1 wie folgt zu ändern:

\einruecken{(1) Der Landesverband Sachsen-Anhalt \textbf{kann} sich in Orts-, Kreis- und Regionalverbände \textbf{gliedern}.}

\paragraph{Alte Fassung}:

\einruecken{(1) Der Landesverband Sachsen-Anhalt \textbf{gliedert} sich in Orts-, Kreis- und Regionalverbände.}

% -----

\satzung{Presseabteilungen}
\antrag{Henning Lübbers}

\paragraph{Antragstext}:

Der Landesparteitag möge folgenden Text als neuen Abschnitt in die Satzung einfügen:

\einruecken{Pressemitteilungen im Namen des Landesverbandes, der Kreisverbände oder Gruppierungen innerhalb von politischen Grenzen werden nur nach Überprüfung und Freigabe durch die SG Presse, der Presseabteilungen der Kreise oder des Pressebeauftragten der jeweiligen Gruppierung, an die Medien zugestellt. Dies soll uns vor Schnellschüssen und peinlichen Auftritten durch, nicht von der Mehrheit gedeckten, Einzelmeinungen bewahren. Es soll keine Zensur darstellen sondern eine Überprüfung auf handwerkliche Richtigkeit. Ablehnungen von Pressemitteilungen durch die SG Presse, der Presseabteilungen der Kreise oder des Pressebeauftragten der jeweiligen Gruppierung, müssen schriftlich begründet werden. 

Die SG Presse wird durch den Landesvorstand organisiert, Kreisverbände organisieren ihre Presseabteilungen Selbständig und benennen die berichtigten, Gruppierungen innerhalb von politischen Kreisen benennen dem Landesvorstand oder den zuständigen Kreisvorständen ein oder mehrere Personen die offiziell durch den zuständigen Vorstand beauftragt werden. Presseabteilungen geben sich eine Geschäftsordnung in der die Kompetenzen geregelt werden.}

% -----

\satzung{Werbung}
\antrag{Henning Lübbers}

\paragraph{Antragstext}:

An geeigneter Stelle der Satzung soll folgender Passus eingefügt werden:

\einruecken{Der Landesverband soll keine Werbung mit kommerziellem Hintergrund oder solche, die mit den Programminhalten der Piratenpartei nicht vereinbar ist, auf Parteimedien erlauben. Dies betrifft insbesondere die Homepages des Landesverbandes und deren Untergliederungen.}

% -----

\satzung{Parteiübertritt mit Mandat}
\antrag{Henning Lübbers}

\paragraph{Antragstext}:

Der Landesparteitag möge folgenden neuen Absatz in Paragraph §3 der Landessatzung einfügen:

\einruecken{Der Landesverband möge sich gegen die Mitgliedsaufnahme aussprechen, wenn die betreffende Person ein Kommunal-, Landtags- oder Bundestagsmandat bzw. Amt (Stadtrat etc.) inne hat, bis das zukünftige Mitglied dieses Mandat niedergelegt hat.\\
Das soll auf alle Personen, die über eine Liste in solche Ämter gekommen sind, angewandt werden.\\
Eine Person mit einem Mandat das direkt gewonnen wurde soll mitsamt dem Mandat aufgenommen werden.\\
Parteilose können auch aufgenommen werden.}

% -----