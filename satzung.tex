\part{Satzungsänderungen}
\section{Eingereichte Satzungsänderunganträge}
\satzung{§ 2 (2) - Mitgliedschaft (Änderung)}
\antrag{Stephan Schurig}

\paragraph{Antragstext}:

Der Landesparteitag möge beschließen folgende Änderung des \href{http://wiki.piratenpartei.de/LSA:Satzung#.C2.A7_2_-_Mitgliedschaft}{§2 (2)} der Satzung des LV Sachsen-Anhalt zu vollziehen:

\textit{(2) Der Landesverband und jede \textbf{untergeordnete} Gliederung führt ein Piratenverzeichnis auf entsprechender Ebene.}

\paragraph{Alte Fassung}:

\textit{(2) Der Landesverband und jede \textbf{niedere} Gliederung führt ein Piratenverzeichnis auf entsprechender Ebene.}

\paragraph{Begründung}:

Wertneutralere Formulierung. 

% -----

\satzung{Ladungsmodalitäten - Anpassung §9b (2) (Der Landesparteitag) an Bundessatzung}
\antrag{René Emcke}

\paragraph{Antragstext}:

Der Landesparteitag möge beschließen, \href{http://wiki.piratenpartei.de/LSA:Satzung#.C2.A7_9b_-_Der_Landesparteitag}{§9b (2) (Der Landesparteitag)} der Landessatzung wie folgt zu ändern:

\textit{(2) Der Landesparteitag tagt mindestens einmal jährlich. Die Einberufung erfolgt aufgrund eines Vorstandsbeschlusses. Wenn ein Zehntel der Piraten, mindestens aber zehn Piraten es beim Vorstand beantragen, muss dieser binnen 2 Wochen einen Parteitag einberufen. Der Vorstand lädt jedes Mitglied schriftlich \textbf{per Brief oder Fax} mindestens 4 Wochen vorher ein. \textbf{Es gilt per Brief das Datum des Poststempels, per Fax der mit Datum und Unterschrift vom Versender bestätigte Sendebericht. Ist eine E-Mail-Adresse bekannt, so kann vorher per E-Mail eingeladen werden. Die reguläre Einladung kann entfallen, wenn das Mitglied den Empfang der E-Mail spätestens 4 Wochen vor dem Landesparteitag bestätigt hat.} Die Einladung hat Angaben zum Tagungsort, Tagungsbeginn, vorläufiger Tagesordnung und der Angabe, wo weitere, aktuelle Veröffentlichungen gemacht werden, zu enthalten. Spätestens 1 Wochen vor dem Parteitag sind die Tagesordnung in aktueller Fassung, die geplante Tagungsdauer und alle bis dahin dem Vorstand eingereichten Anträge im Wortlaut zu veröffentlichen.}

\paragraph{Alte Fassung}:

\textit{(2) Der Landesparteitag tagt mindestens einmal jährlich. Die Einberufung erfolgt aufgrund eines Vorstandsbeschlusses. Wenn ein Zehntel der Piraten, mindestens aber zehn Piraten es beim Vorstand beantragen, muss dieser binnen 2 Wochen einen Parteitag einberufen. Der Vorstand lädt jedes Mitglied schriftlich \textbf{(Brief, Email oder Fax)} mindestens 4 Wochen vorher ein. Die Einladung hat Angaben zum Tagungsort, Tagungsbeginn, vorläufiger Tagesordnung und der Angabe, wo weitere, aktuelle Veröffentlichungen gemacht werden, zu enthalten. Spätestens 1 Wochen vor dem Parteitag sind die Tagesordnung in aktueller Fassung, die geplante Tagungsdauer und alle bis dahin dem Vorstand eingereichten Anträge im Wortlaut zu veröffentlichen.}

\paragraph{Begründung}:

\begin{itemize}
\item Anpassung der Landessatzung an die \href{https://wiki.piratenpartei.de/Satzung#.C2.A7_9b_-_Der_Bundesparteitag}{Bundessatzung} in Bezug auf Ladungsmöglichkeiten
\item Verankerung der bereits praktizierten Einladungsmöglichkeit per Email in der Satzung (Rechtssicherheit)
\item Verankerung von Definitionen über fristgerecht erfolgte Zustellung (Rechtssicherheit)
\end{itemize}

% -----

\satzung{Landesfinanzordnung}
\antrag{Alexander Zinser}

\paragraph{Antragstext}:

Der Landesparteitag möge beschließen die \href{http://wiki.piratenpartei.de/LSA:Satzung#Abschnitt_B:_Finanzordnung}{derzeitge Landesfinanzordnung} durch folgenden Text zu ersetzen: 

\textit{§1 Allgemeines\\
(1) Es gilt im Wesentlichen die Bundesfinanzordnung.\\
(2) Der Vorstand ist dem Vier-Augen-Prinzip verpflichtet. Jede Transaktion muß von zwei Vorstandsmitgliedern unterzeichnet werden, wobei der übrige Vorstand unverzüglich in Kenntnis zu setzen ist, oder durch einen Vorstandsbeschluss gedeckt sein.\\
(3) Der Schatzmeister des Landesverbandes kann gegen Transaktionen sein Veto einlegen, wenn es die Finanzlage erfordert.\\
(4) Der Schatzmeister des Landesverbandes kann von untergeordneten Gliederungen alle für den Rechenschaftsbericht notwendigen Daten einfordern. Sollte dies nicht möglich sein, hat er zeitnah Ordnungsmaßnahmen zu beantragen.\\\\
§2 Mitgliedsbeitrag\\
(1) Der Mitgliedsbeitrag wird zum Jahresbeginn vollständig an die für das Mitglied zuständige Gliederung überwiesen. Der Mitgliedsbeitrag wird von der für das Mitglied zuständigen Gliederung quartalsweise über die Umlage an die höheren Gliederungen überwiesen.\\
(2) Der Mitgliedsbeitrag wird abzüglich des Bundesanteils wie folgt aufgeteilt: 50\% an den Landesverband, 25\% an den zuständigen Kreisverband und 25\% an den zuständigen Ortsverband. Sofern eine Gliederung nicht existiert, gehen die Gelder an die jeweils übergeordnete Gliederung.\\\\
§3 Virtuelle Kreisverbände\\
(1) Basierend auf den politischen Grenzen werden für alle Kreise ohne existierenden Kreisverband Konten in der Buchhaltung geschaffen (virtuelle Kreisverbände). Auf diese Konten werden alle Finanzen gebucht, die einem tatsächlich existierenden Kreisverband zustünden.}

\paragraph{Alte Fassung}:

\textit{1. Es gilt im Wesentlichen die Bundesfinanzordnung.\\
2. Der Vorstand ist dem Vier-Augen-Prinzip verpflichtet. Jede Transaktion muß von zwei Vorstandsmitgliedern unterzeichnet werden, wobei der übrige Vorstand unverzüglich in Kenntnis zu setzen ist, oder durch einen Vorstandsbeschluss gedeckt sein.\\
3. Der Schatzmeister des Landesverbandes kann gegen Transaktionen sein Veto einlegen, wenn es die Finanzlage erfordert.\\
4. Der Schatzmeister des Landesverbandes kann von untergeordneten Gliederungen alle für den Rechenschaftsbericht notwendigen Daten einfordern. Sollte dies nicht möglich sein, hat er zeitnah Ordnungsmaßnahmen zu beantragen.}

\paragraph{Begründung}:

Erweiterung der derzeitigen Landesfinanzordnung. §1 ist die alte Finanzordnung, §2 regelt Umlage der Mitgliedsbeiträge (50\% Land, 25\% Kreis, 25\% Ort oder Gesamtschlüssel mit Bund: 40\% Bund, 30\% Land, 15\% Kreis, 15\% Ort), §3 Unterkonten beim LV für Kreise ohne KV.

% -----

\satzung{Umlage PartFin}\label{satzung:partfin1}
\antrag{Alexander Zinser}
\begin{itemize}
\item \konkurrenz{satzung:partfin2}
\item \konkurrenz{satzung:partfin3}
\end{itemize}

\paragraph{Antragstext}:

Der Landesparteitag möge beschließen, folgenden Abschnitt in die \href{http://wiki.piratenpartei.de/LSA:Satzung#Abschnitt_B:_Finanzordnung}{Landesfinanzordnung} aufzunehmen: 

\textit{§XX - Umlage Parteienfinanzierung\\
Die Gelder aus der Parteienfinanzierung werden auf Landesebene nach folgendem Schlüssel umgelegt:\\
(1) 10\% der Parteienfinanzierung verbleibt bis zur nächsten Abschlagszahlung, mindestens jedoch für ein Jahr, als Rücklage beim Landesverband. Aufgelöste Rücklagen werden zur aktuellen Abschlagszahlung addiert und entsprechend diesem Schüssel umgelegt.\\
(2) Vom verbleibenden Betrag gehen 50\%, mindestens jedoch ein Sockelbetrag von 3600 EUR per anno, an den Landesverband. Der Restbetrag geht an die untergliederten Kreisverbände.\\
(3) Die Verteilung des Anteils der Kreisverbände erfolgt zu je einem Drittel nach Sockel, nach Einwohner und nach Fläche der Kreisverbände.\\
(3a) Der Sockelanteil eines Kreisverbandes berechnet sich aus dem Verhältnis Anzahl der politischen Kreise des Kreisverbandes zu Anzahl der politischen Kreise des Landes.\\
(3b) Der Anteil nach Einwohner berechnet sich aus dem Verhältnis Einwohnerzahl des Gebietes des Kreisverbandes zu Einwohnerzahl des Landes.\\
(3c) Der Anteil nach Fläche berechnet sich aus dem Verhältnis Fläche des Gebietes des Kreisverbandes zu Fläche des Landes.\\
(4) Sofern in einem politischen Kreis noch kein Kreisverband existiert, wird der entsprechende Betrag gegen ein virtuelles Unterkonto des Landesverbandes gebucht. Von diesem Unterkonto sollen primär Aktionen in dem jeweiligen Gebiet finanziert werden. Der Landesvorstand ist berechtigt diesen Betrag begründet anderweitig zu verwenden.\\
(5) Anspruch auf Auszahlung aus der Parteienfinanzierung besteht ab dem Monat der Gründung eines Kreisverbandes.}

% -----

\satzung{Umlage PartFin BaWü}\label{satzung:partfin2}
\antrag{Alexander Zinser}
\begin{itemize}
\item \konkurrenz{satzung:partfin1}
\item \konkurrenz{satzung:partfin3}
\end{itemize}

\paragraph{Antragstext}:

Der LPT möge beschließen, den folgenden Text an geeigneter Stelle in die \href{http://wiki.piratenpartei.de/LSA:Satzung#Abschnitt_B:_Finanzordnung}{Landesfinanzordnung} aufzunehmen:

\textit{§X Parteienfinanzierung\\
(1) Die Parteienfinanzierung für den Landesverband und all seine Untergliederungen werden nach folgendem Schlüssel unter den Gliederungen verteilt.\\
(2) Dem Landesverband stehen 50\%, den Kreisverbänden 25\% und den Ortsverbänden 25\% der Parteienfinanzierung zu.\\
(3) Unter den Gliederungen gleicher Ebene wird die Parteienfinanzierung durch die Anzahl der stimmberechtigten Mitglieder des Landesverbandes geteilt. Anschließend wird mit der Anzahl der stimmberechtigten Mitglieder der betroffenen Gliederung multipliziert. Die Anzahl der stimmberechtigten Mitglieder jeder Gliederung wird durch den Landesvorstand festgestellt. Stichtag ist jeweils der 31.12. des Vorjahres.\\
(4) Sofern eine Gliederung nicht existiert, gehen die Gelder an die jeweils übergeordnete Gliederung.\\
(5) Der Landesverband verteilt die Parteienfinanzierung quartalsweise auf seine Gliederungen.}

\paragraph{Begründung}:

Umlageschlüssel wesentlich einfacher als LQFB-Sieger (KISS-Prinzip, wa?)

% -----

\satzung{Umlage PartFin BaWü Sicher}\label{satzung:partfin3}
\antrag{Alexander Zinser}
\begin{itemize}
\item \konkurrenz{satzung:partfin1}
\item \konkurrenz{satzung:partfin2}
\end{itemize}

\paragraph{Antragstext}:

Der LPT möge beschließen, den folgenden Text an geeigneter Stelle in die \href{http://wiki.piratenpartei.de/LSA:Satzung#Abschnitt_B:_Finanzordnung}{Landesfinanzordnung} aufzunehmen:

\textit{§X Parteienfinanzierung\\
(1) Die Parteienfinanzierung für den Landesverband und all seine Untergliederungen werden nach folgendem Schlüssel unter den Gliederungen verteilt.\\ 
(2) Dem Landesverband stehen 50\%, den Kreisverbänden 25\% und den Ortsverbänden 25\% der Parteienfinanzierung zu.\\
(3) Unter den Gliederungen gleicher Ebene wird die Parteienfinanzierung durch die Anzahl der stimmberechtigten Mitglieder des Landesverbandes geteilt. Anschließend wird mit der Anzahl der stimmberechtigten Mitglieder der betroffenen Gliederung multipliziert. Die Anzahl der stimmberechtigten Mitglieder jeder Gliederung wird durch den Landesvorstand festgestellt. Stichtag ist jeweils der 31.12. des Vorjahres.\\
(4) Sofern eine Gliederung nicht existiert, gehen die Gelder an die jeweils übergeordnete Gliederung.\\
(5) Abschlagszahlungen werden zurückgelegt und am 01.01. des Folgejahres ausgeschüttet.}

\paragraph{Begründung}:

Einfacher Umlageschlüssel und im worst case vollständig resistent gegenüber Rückzahlungen an Landtag et al. da Abschlagszahlungen erst mal vollständig zurückgelegt werden.

% -----

\satzung{Finanzrat}
\antrag{Alexander Zinser}

\paragraph{Antragstext}:

Der Landesparteitag möge beschließen, folgenden Abschnitt in die \href{http://wiki.piratenpartei.de/LSA:Satzung#Abschnitt_B:_Finanzordnung}{Landesfinanzordnung} aufzunehmen:

\textit{§X - Finanzrat\\
(1) Der Landesparteitag wählt einmal jährlich zwei Piraten des Landesverbandes in den Finanzrat der Piratenpartei Deutschland.}

\paragraph{Begründung}:

Es existiert keine entsprechende Regelung auf Landesebene. Über die Frequenz der Wahl zum Finanzrat macht auch die Bundesfinanzordnung keine Aussage.

% -----

\satzung{Gliederungen}\label{satzung:gliederungen1}
\antrag{Alexander Zinser}
\begin{itemize}
\item \konkurrenz{satzung:gliederungen2}
\item \konkurrenz{satzung:gliederungen3}
\end{itemize}

\paragraph{Antragstext}:

Der Landesparteitag möge beschließen, \href{http://wiki.piratenpartei.de/LSA:Satzung#.C2.A7_7_-_Gliederung}{§7 (Gliederung)} der Landessatzung wie folgt zu ändern

\textit{(1) Der Landesverband Sachsen-Anhalt gliedert sich in Orts-, Kreis- und Regionalverbände.\\
(2) Regionalverbände sind Kreisverbände im Sinne der Bundessatzung, deren Gebiet sich über mehr als einen politischen Kreis erstreckt. Eine Koexistenz von Kreis- und Regionalverband auf dem selben Gebiet ist nicht zulässig.\\
(3) Gründet sich eine Untergliederung oder ändert ihre Satzung, so muss dem Landesvorstand die aktuelle Satzung vorgelegt werden.\\
(4) Die Geschäftsordnung des Vorstandes einer Untergliederung ist von allen Vorstandsmitgliedern zu unterschreiben und dem Landesvorstand in Kopie vorzulegen. Die Geschäftsordnung ist an geeigneter Stelle online zu stellen. Änderungen an der Geschäftsordnung sind dem Landesvorstand unverzüglich zu melden sowie in der Onlineversion zu aktualisieren.}

\paragraph{Alte Fassung}:

\textit{(1) Die Gliederung des Landesverbands regelt die Bundessatzung.}

\paragraph{Begründung}:

\begin{itemize}
\item Die Landessatzung verweist bisher nur auf die Bundessatzung.
\item Der Begriff Regionalverband ist bisher nicht definiert.
\end{itemize}

% -----

\satzung{Gliederungen (Alternative mit Gründungsklausel)}\label{satzung:gliederungen2}
\antrag{René Emcke}
\begin{itemize}
\item \konkurrenz{satzung:gliederungen1}
\item \konkurrenz{satzung:gliederungen3}
\end{itemize}

\paragraph{Antragstext}:

Der Landesparteitag möge beschließen, \href{http://wiki.piratenpartei.de/LSA:Satzung#.C2.A7_7_-_Gliederung}{§7 (Gliederung)} der Landessatzung wie folgt zu ändern

\textit{(1) Der Landesverband Sachsen-Anhalt gliedert sich in Orts-, Kreis- und Regionalverbände.\\
(2) Regionalverbände sind Kreisverbände im Sinne der Bundessatzung, deren Gebiet sich über mehr als einen politischen Kreis erstreckt. Eine Koexistenz von Kreis- und Regionalverband auf dem selben Gebiet ist nicht zulässig.\\
(3) Der Gründung eines Kreisverbandes müssen mindestens drei akkreditierte Piraten aus jedem politischen Kreis mehrheitlich zustimmen. Insgesamt müssen der Gründung mindestens zehn akkreditierte Piraten mehrheitlich zustimmen.\\
(4) Sofern der zuständige Kreisverband keine anderen Regelungen getroffen hat, gilt für die Gründung von Ortsverbänden Absatz (3) Satz 2.\\
(5) Gründet sich eine Untergliederung oder ändert ihre Satzung, so muss dem Landesvorstand die aktuelle Satzung vorgelegt werden.\\
(6) Die Geschäftsordnung des Vorstandes einer Untergliederung ist von allen Vorstandsmitgliedern zu unterschreiben und dem Landesvorstand in Kopie vorzulegen. Die Geschäftsordnung ist an geeigneter Stelle online zu stellen. Änderungen an der Geschäftsordnung sind dem Landesvorstand unverzüglich zu melden sowie in der Onlineversion zu aktualisieren.}

\paragraph{Alte Fassung}:

\textit{(1) Die Gliederung des Landesverbands regelt die Bundessatzung.}

\paragraph{Begründung}:

\begin{itemize}
\item Die Landessatzung verweist bisher nur auf die Bundessatzung.
\item Der Begriff Regionalverband ist bisher nicht definiert.
\item gemäß \href{http://lqfb.piraten-lsa.de/lsa/initiative/show/33.html}{erfolgreicher LF-Initiative} mit beschränkender Klausel für (Neu)Gründungen
\item Satzungsverankerung der notwendigen separaten Zustimmung der Mitglieder aus allen beteiligten politischen Kreisen bei Gründung von kreisübergreifenden Regionalverbänden
\item Sicherstellung der Arbeitsfähigkeit und Legitimation von Untergliederungen durch eine ausreichende Anzahl zustimmender Mitglieder
\item Verhinderung von Gründungen durch lediglich 3 (Mindestanzahl für einen Vorstand) Mitglieder, die sich bei Wahl des Vorstandes gegenseitig wählen (Legitimation) 
\end{itemize}

% -----

\satzung{§ 11 - Satzungs- und Programmänderung (3)}
\antrag{Roman Ladig}

\paragraph{Antragstext}:

Die Landesmitgliederversammlung möge beschließen, \href{http://wiki.piratenpartei.de/LSA:Satzung#.C2.A7_11_-_Satzungs-_und_Programm.C3.A4nderung}{§11 (3)} der Landessatzung wie folgt zu ändern:

\textit{(3) Vom Landesparteitag kann ein eigenes Grundsatzprogramm für den Landesverband sowie Wahlprogramme für Kommunal- und Landtagswahlen verabschiedet werden. Diese dürfen dem Grundsatzprogramm der Piratenpartei Deutschland nicht widersprechen.}

\paragraph{Alte Fassung}:

\textit{(3) Das Grundsatzprogramm der Piratenpartei Deutschland wird vom Landesverband übernommen. Ein eigenes Wahlprogramm basierend auf den Werten des Grundsatzprogrammes kann auf Landesebene für Kommunal- und Landtagswahlen bei Bedarf vom Landesparteitag verabschiedet werden. }

\paragraph{Begründung}:

Ein Landesgrundsatzprogramm hilft, die Position von Sachsen-Anhalt innerhalb des Bundes besser zu beschreiben, regionale Unterschiede aufzuzeigen und sich gegenüber anderen Landesverbänden falls nötig abzugrenzen. Darüberhinaus können von einem entwickelten Grundsatzprogramm leichter Wahlprogramme und Schlüsselpapiere abgeleitet und Anregungen für die Weiterentwicklung des Grundsatzprogramms der Piratenpartei Deutschland gefunden werden.

% -----

\satzung{§ 10 - Bewerberaufstellung für die Wahlen zu Volksvertretungen }
\antrag{Roman Ladig}

\paragraph{Antragstext}:

Die Landesmitgliederversammlung möge beschließen, \href{http://wiki.piratenpartei.de/LSA:Satzung#.C2.A7_10_-_Bewerberaufstellung_f.C3.BCr_die_Wahlen_zu_Volksvertretungen}{§10} der Landessatzung wie folgt zu ändern: 

\textit{(1) Die Bewerberaufstellung für die Wahlen zu Volksvertretungen erfolgt nach Maßgabe der Wahlgesetze und den Vorgaben der Bundessatzung. Soweit die Vorschriften der Wahlgesetze nicht vorgehen oder ein anderes vorschreiben, gilt im Übrigen das Prozedere in den nachfolgenden Absätzen.\\
(2) Landeslisten werden von der Mitgliederversammlung des Landesverbandes aufgestellt, sofern nicht eine gemeinsame Liste zusammen mit dem Bundesverband zur Europawahl aufgestellt wird.\\
(3) Die Mitglieder werden nach § 9b dieser Satzung zur Wahl geladen. Lassen die Wahlgesetze kürzere Ladungsfristen zu, so genügt deren Einhaltung. In der Einladung wird ausdrücklich auf die Bewerberaufstellung hingewiesen. Die Mitgliederversammlung ist beschlussfähig, wenn mindestens 10 \% der stimmberechtigten Mitglieder anwesend sind.\\
(4) Wahlkreisbewerber werden\\
1. In Wahlkreisen, deren Grenzen deckungsgleich mit denen eines oder mehrerer Regional- bzw. Kreisverbände sind, von den existierenden Gliederungen selbst aufgestellt,\\
2. in sonstigen Fällen beruft der Landesvorstand die Wahlkreisversammlung ein. In diesen Versammlungen wählen jeweils die in einem gemeinsamen Wahlkreis wohnhaften Piraten einen gemeinsamen Wahlkreisbewerber,\\
3. falls Punkt 2. nicht möglich ist zu Landtagswahlen auch in einer Landesversammlung der zur Wahl des Landtages wahlberechtigten Piraten gewählt.\\
(5) Die Bewerberaufstellung zu Kommunalwahlen nach dem Kommunalwahlgesetz regeln die Gliederungen unterhalb des Landesverbandes selbst.}

\paragraph{Alte Fassung}:

\textit{(1) Die Bewerberaufstellung für die Wahlen zu Volksvertretungen erfolgt nach den Regularien der einschlägigen Gesetze sowie den Vorgaben der Bundessatzung.\\
(2) Die Aufstellung kann sowohl als Mitgliederversammlung des zuständigen Stimm- bzw. Wahlkreises als auch im Rahmen einer anderen Mitgliederversammlung stattfinden, sofern gewährleistet wird, dass alle Stimmberechtigten in angemessener Zeit und Form eingeladen wurden und nur die Stimmberechtigten an der Wahl teilnehmen. Die Einladung muss dabei explizit auf die Bewerberaufstellung hinweisen.}

\paragraph{Begründung}:

In der aktuellen Satzung ist in §10 die Bewerberaufstellung für die Wahl zu Volksvertretung nicht genügend geregelt. So fehlte z.b. ein Passus über die Abgrenzung der Gliederungshoheiten (Land vs. Kommunal) beim Aufstellen der Wahlkreisbewerber.

% -----

\satzung{Liquid Democracy}
\antrag{Karl}

\paragraph{Antragstext}:

Der Landesparteitag möge beschließen den folgenden Abschnitt an passender Stelle in die Satzung des Landesverbandes Sachsen-Anhalt einzufügen.

\textit{Liquid Democracy\\\\
(1) Die Piratenpartei Deutschland Sachsen-Anhalt nutzt zur Willensbildung über das Internet eine geeignete Software. Diese muss die “Anforderungen für den Liquid Democracy Systembetrieb” erfüllen, welche vom Vorstand beschlossen werden.\\\\
Die Mindestanforderungen sind:\\\\
a) Jedes Mitglied muss die Möglichkeit haben, Anträge im System zu stellen. Zulassungsquoren und Antragskontingente sind zulässig, müssen jedoch für alle Mitglieder gleich sein.\\
b) Das System muss ohne Moderatoren auskommen.\\
c) In das System eingebrachte Anträge dürfen nicht gegen den Willen des Antragsstellers von anderen Mitgliedern verändert oder gelöscht werden können.\\
d) Jedem Mitglied muss es innerhalb eines bestimmten Zeitraums möglich sein, Alternativanträge einzubringen.\\
e) Das eingesetzte Abstimmungsverfahren darf Anträge, zu denen es ähnliche Alternativanträge gibt, nicht prinzipbedingt bevorzugen oder benachteiligen. Mitgliedern muss es möglich sein, mehreren konkurrierenden Anträgen gleichzeitig zuzustimmen. Der Einsatz eines Präferenzwahlverfahrens ist hierbei zulässig.\\
f) Es muss möglich sein, die eigene Stimme mindestens themenbereichsbezogen durch Delegation an ein anderes Mitglied zu übertragen. Diese Delegationen müssen jederzeit widerrufbar sein und übertragenes Stimmgewicht muss weiter übertragen werden können. Selbstgenutztes Stimmgewicht darf nicht weiter übertragen werden.\\
(2) Der Vorstand stellt den dauerhaften und ordnungsgemäßen Betrieb des Systems sicher.\\
(3) Jedem Mitglied ist Einsicht in den abstimmungsrelevanten Datenbestand des Systems zu gewähren. Während einer Abstimmung darf der Zugriff auf die jeweiligen Abstimmdaten anderer Mitglieder vorübergehend gesperrt werden.\\
(4) Die Organe sind gehalten, das Liquid Democracy System zur Einholung von Empfehlungen zur Grundlage ihrer Beschlüsse zu nutzen und vom diesen Empfehlungen abweichende Entscheidungen zu begründen. Das Schiedsgericht ist davon ausgenommen.\\
(5) Die Organe der Partei sind angehalten, die Anträge, die im Liquid Democracy System positiv beschieden wurden, vorrangig zu behandeln.\\
(6) Teilnahmeberechtigt ist jeder Pirat, der nach der Satzung Mitglied der Piratenpartei Sachsen-Anhalt ist. Jeder Pirat erhält genau einen persönlichen Zugang, der nur von ihm genutzt werden darf.\\
(7) Verstößt ein Nutzer wiederholt und in erheblichem Maße gegen die Nutzungsbedingungen des Systems, so kann der Vorstand als Ordnungsmaßnahme dem Nutzer auf Zeit das Recht entziehen, Anträge oder andere Texte in das System einzustellen. Im Falle technischer Angriffe auf das System, die von einem angemeldeten Benutzer ausgehen, kann dieses Benutzerkonto durch Administratoren vorübergehend gesperrt werden.}

\paragraph{Begründung}:

Das Konzept der Liqiud Democracy und deren Umsetzung in der Piratenpartei in Form von Liquid Feedback, bilden zusammen wohl eines der vielversprechendsten Projekte innerhalb der Partei und haben ein gewaltiges Potential die Art, wie Demokratie praktiziert wird, zu verändern. Daher ist es wichtig diese besondere Stellung innerhalb der Partei auch in der Satzung abzubilden.

Bis jetzt werden die Ergebnisse von Liquid Feedback meist als Meinungsbilder interpretiert, doch diese Aussage wird der tatsächlichen Relevanz der ausgearbeiteten Anträge nicht mehr gerecht. Damit diese nicht mehr übergangen oder ignoriert werden können, sollen Anträge als Empfehlungen an die Parteiorgane gelten.

Eine abweichende Entscheidung sollte von den Organen begründet werden, hierzu ist ausreichend, dass eine Begründung im Rahmen des Protokolls des jeweiligen Organs festgehalten wird. Die Begründung dient zur Nachvollziehbarkeit der getroffenen Entscheidung und somit zur innerparteilichen Transparenz.

% -----

\satzung{Landesvorstand Piraten LSA - Amtszeitbegrenzung/Wiederwahl}
\antrag{Markus Hünniger}

\paragraph{Antragstext}:

Ich beantrage der LPT möge beschließen den \href{http://wiki.piratenpartei.de/LSA:Satzung#.C2.A7_9a_-_Der_Vorstand}{Punkt (3) in §9a} zu ergänzen durch: 

\textit{Der Vorsitzende und der 2. Vorsitzende können nur 3x in Folge für diese Ämter kandidieren und dürfen dieses Amt maximal 6 Jahre übernehmen. Danach ist eine Kandidatur für ein Vorstandsamt, für die Dauer der geleisteten Amtszeit (in Jahren, aufgerundet) nicht zulässig.}

\paragraph{Begründung}:

Die Mitglieder, welche diese Ämter ausfüllen, sind sich so bewusst das es für Sie nur eine zeitliche begrenzte Aufgabe ist. Man richtet sich auf diesen Posten nicht ein. Es erzeugt kein Beharrungsvermögen und lässt Sie danach abkühlen und aus der Schusslinie nehmen, sowie begrenzt auf einfach Art und Weise das Amt. Es lässt Zeit um sich politisch abzukühlen, in sein normales Leben zurückzukehren, sich neue zu Besinnen und Kraft zu schöpfen!

% -----

\satzung{Kandidatur, Amtszeit, Wiederwahl, Landtag, Kreistage, Stadträte}
\antrag{Markus Hünniger}

\paragraph{Antragstext}:

Der Landesparteitag möge beschließen, \href{http://wiki.piratenpartei.de/LSA:Satzung#.C2.A7_10_-_Bewerberaufstellung_f.C3.BCr_die_Wahlen_zu_Volksvertretungen}{§10} der Landessatzung durch folgenden neuen Absatz zu ergänzen:

\textit{Piraten, die zwei Legislaturperioden (unabhängig von der Dauer) Mitglieder von Volksvertretungen waren, können für die Dauer der geleisteten Zeit als Mitglied dieser Vertretung nicht wieder für das gleiche Gremium kandidieren.}

\paragraph{Begründung}:

Die Mitglieder, welche diese Ämter ausfüllen, sind sich so bewusst, dass es für sie nur eine zeitliche begrenzte Aufgabe ist. Man richtet sich auf diesen Posten nicht ein. Es erzeugt kein Beharrungsvermögen und lässt sie danach abkühlen, aus der Schusslinie nehmen und begrenzt auf einfach Art und Weise das Amt. Es lässt Zeit um sich politisch abzukühlen, in sein normales Leben zurückzukehren, sich neue zu Besinnen und Kraft zu schöpfen! Nach Ablauf der Ruhezeit, kann man dann wieder kandidieren!

% -----

\satzung{ein Pirat - ein Mandat}
\antrag{Markus Hünniger}

\paragraph{Antragstext}:

Der Landesparteitag möge beschließen, \href{http://wiki.piratenpartei.de/LSA:Satzung#.C2.A7_10_-_Bewerberaufstellung_f.C3.BCr_die_Wahlen_zu_Volksvertretungen}{§10} der Landessatzung durch folgenden neuen Absatz zu ergänzen:

\textit{Ein Pirat darf nicht mehr als ein Mandat innehaben.}

\paragraph{Begründung}:

Eine Person, ein Mandat. Keine Häufung.

z.B. entweder Landtag oder Stadtrat, nicht beides, Entw. Bundestag oder LT nicht beides.

% -----

\satzung{Gliederungen\_3}\label{satzung:gliederungen3}
\antrag{Kevin Oelze}
\begin{itemize}
\item \konkurrenz{satzung:gliederungen1}
\item \konkurrenz{satzung:gliederungen2}
\end{itemize}

\paragraph{Antragstext}:

Der Landesparteitag möge beschließen, \href{http://wiki.piratenpartei.de/LSA:Satzung#.C2.A7_7_-_Gliederung}{§7 (Gliederung)} der Landessatzung wie folgt zu ändern

\textit{(1) Der Landesverband Sachsen-Anhalt gliedert sich in Orts-, Kreis- und Regionalverbände.\\
(2) Regionalverbände sind Kreisverbände im Sinne der Bundessatzung, deren Gebiet sich über mehr als einen politischen Kreis erstreckt. Eine Koexistenz von Kreis- und Regionalverband auf dem selben Gebiet ist nicht zulässig.\\
(3) Gründet sich eine Untergliederung oder ändert ihre Satzung, so muss dem Landesvorstand die aktuelle Satzung vorgelegt werden.}

\paragraph{Alte Fassung}:

\textit{(1) Die Gliederung des Landesverbands regelt die Bundessatzung.}

\paragraph{Begründung}:

\begin{itemize}
\item Die Landessatzung verweist bisher nur auf die Bundessatzung.
\item Der Begriff Regionalverband ist bisher nicht definiert.
\item Die Bekanntgabe der GO ist obligatorisch, auch das Verfahren ist selbstverständlich, eine Meldung an den LaVo bei jeder Änderung ist nicht zielführend.
\end{itemize}

% -----

\satzung{Maximale Spendenhöhe von 5000 Euro}
\antrag{Christian Kunze}

\paragraph{Antragstext}:

Der Landesparteitag möge beschließen, folgenden neuen Punkt der \href{http://wiki.piratenpartei.de/LSA:Satzung#Abschnitt_B:_Finanzordnung}{Finanzordnung} des Landesverbands hinzuzufügen:

\textit{Die maximale Spendenhöhe an den Landesverband von juristischen Personen (Firmen, Konzerne) und Privatpersonen darf 5000 Euro nicht überschreiten.}

\paragraph{Begründung}:

Es soll eine Maßnahme zur Prävention von zu starker Einflussnahme auf den Landesverband sein. Da wir bis jetzt noch keine so hohe Spende hatten, sollte es uns stutzig machen, wenn jemand z. B. vor der nächsten Wahl einen höheren Betrag spenden will und was er damit zu erreichen versucht. Weitere Begründungen folgen auf dem Parteitag.
