\section{Infrastrukturmonopole}

\subsection*{Infrastrukturmonopole}
\wahlprogramm{Infrastruktur}
\antrag{Unglow}\version{03:32, 20. Jun. 2010}

\subsubsection{Modul 1}
\abstimmung
Eine gute Infrastruktur ist eine grundlegende Voraussetzung, um Wirtschaftswachstum zu ermöglichen und Rheinland-Pfalz als Standort für Unternehmen attraktiv zu halten. Zudem wird eine zuverlässige und neutrale Infrastruktur benötigt, um freien Informationszugang und die Teilhabe am gesellschaftlichen Leben zu ermöglichen.

\subsubsection{Modul 2}
\abstimmung
Die Piratenpartei möchte verhindern, dass durch privatwirtschaftliche Interessen Infrastrukturen wettbewerbsverzerrend und auf Kosten der Gesellschaft beeinflusst werden.

\subsubsection{Modul 3}
\abstimmung
Die Infrastrukturen sind nicht nur die Basis für die Marktwirtschaft, sondern für das generelle Miteinander der Menschen. Durch dieses zentrale Element des Zusammenlebens entscheidet sich, wer aktiv an der Wirtschaft und dem kulturellen Leben teilhaben kann.

\subsubsection{Modul 4}
\abstimmung
Die Struktur und die Funktionsweise von Infrastrukturen muss transparent sein, um eine Nachvollziehbarkeit von außen zu ermöglichen. Der Zugang zu Infrastrukturen muss allen Teilen der Gesellschaft offen stehen.

\subsubsection{Modul 5}
\abstimmung
Der Staat ist für Verfügbarkeit und Zuverlässigkeit verantwortlich, um hohe Versorgungssicherheit, Effizienz und Nachhaltigkeit zu garantieren. Die Zugänge zu jeglicher Infrastruktur müssen sowohl für Produzenten und Anbieter als auch für Nutzer und Konsumenten möglichst unlimitiert und barrierefrei sein. Durch gleiche Zugangsmöglichkeiten wird der freie Wettbewerb zwischen den verschiedenen privaten Anbietern gefördert.

\subsubsection{Modul 6}
\abstimmung
Wir werden durch geeignete, öffentlich kontrollierbare und transparente Kontrollinstanzen dafür sorgen, dass die für Infrastruktur geltenden Regeln eingehalten werden. In Fällen, in denen diese Kontrollinstanzen versagen und Abhilfe auch nicht durch Auflagen, Verordnungen und Gesetze mit einem verhältnismäßigen und endlichen Aufwand erreicht werden kann, werden wir diese Infrastruktur verstaatlichen.
 
\wahlprogramm{Verkehrs- und Stromnetze}\label{wp:monopol:netz1}
\antrag{Unglow}\konkurrenz{wp:monopol:netz2}\version{03:32, 20. Jun. 2010}

\subsubsection{Modul 1}
\abstimmung
Straßen-, Schienen- und Stromnetze sowie Wasserwege gelten als natürliche Infrastrukturmonopole. Der Zugang zu diesen Teilen der Infrastruktur ist für unsere Gesellschaft überlebenswichtig. Gleichzeitig sind sie extrem anfällig für Wettbewerbsverzerrung. Nur wenn der Staat, als einzig öffentlich kontrollierbare Instanz, der Betreiber solcher Netze ist, kann sichergestellt werden, dass die von uns geforderten Ansprüche erfüllt werden.
 
\wahlprogramm{Verkehrs- und Stromnetze}\label{wp:monopol:netz2}
\antrag{marcus}\konkurrenz{wp:monopol:netz1}\version{03:32, 20. Jun. 2010}

\subsubsection{Alternativantrag}
\abstimmung
Straßen-, Schienen- und Stromnetze sowie Wasserwege gelten als natürliche Infrastrukturmonopole. Der Zugang zu diesen Teilen der Infrastruktur ist für unsere Gesellschaft überlebenswichtig. Gleichzeitig sind sie extrem anfällig für Wettbewerbsverzerrung. Nur wenn der Staat, als einzig öffentlich konrollierbare Instanz der \textbf{Besitzer} (= Eigentümer) solcher Netze ist, kann sichergestellt werden, dass die von uns geforderten Ansprüche erfüllt werden. Da er als Besitzer jederzeit geltende Pachtverträge bei Nichteinhaltung durch den Betreiber widerrufen kann. Darüber hinaus profitiert der Staat bei Wertsteigerungen der Infrastruktur durch die Möglichkeit der Pachtpreiserhöhung (Modell Public private Partnership) bei Begrenzung des Unternehmerischen Risikos.
 
\subsection*{Infrastruktur Internet}
\wahlprogramm{Infrastruktur Internet}\label{wp:monopol:inet1}
\antrag{Unglow}\konkurrenz{wp_monopol:inet2}\version{03:32, 20. Jun. 2010}

\subsubsection{Modul 1}
\abstimmung
Im Informationszeitalter ist das Internet als Infrastruktur von besonderer Bedeutung. Es ist Grundlage für den freien Meinungsaustausch, die Teilhabe am kulturellen und sozialen Leben, für Wissenschaft und politische Partizipation. Aufgrund dieser Relevanz muss die Verfügbarkeit des Netzes im Rahmen einer unpfändbaren Grundversorgung wie bei Radio und TV gewährleistet werden. Der gleichberechtigte Zugang jedes einzelnen Bürger muss besonders geschützt werden. Das Netz muss sich neutral gegenüber den transportierten Inhalten verhalten. Die Netzbetreiber tragen keine Verantwortung für die übertragenen Daten.

\subsubsection{Modul 2}
\abstimmung
Die Installation von Filtern in die Infrastruktur des Internets lehnen wir ab. Der Kampf gegen rechtswidrige Angebote im Internet muss jederzeit mit rechtsstaatlichen Mitteln geführt werden. Allein die Etablierung einer Zensurinfrastruktur ist bereits inakzeptabel. Die Beurteilung der Rechtswidrigkeit muss gemäß der in Deutschland geltenden Gewaltenteilung und Zuständigkeit getroffen werden.

\subsubsection{Modul 3}
\abstimmung
Der Ausschluss von Bürgern aus dem Internet ist nach Ansicht der Piratenpartei eine eklatante Bürgerrechtsverletzung. Eine Three-Strikes-Regelung nach französischem Vorbild oder ähnliche Maßnahmen lehnen wir deshalb strikt ab.

\subsubsection{Modul 4}
\abstimmung
Volks- und betriebswirtschaftlich sind Regionen mit schneller Internetanbindung stark aufgewertet. Daher werden wir den Ausbau schneller Internetverbindungen fördern und erleichtern.

\wahlprogramm{Infrastruktur Internet}\label{wp:monopol:inet2}
\antrag{KV Trier/Trier-Saarburg}\konkurrenz{wp_monopol:inet1}\version{03:32, 20. Jun. 2010}

\subsubsection{Breitbandausbau - Einleitung}
\abstimmung
Regionen ohne Breitbandtechnologie sind nicht nur wirtschaftlich benachteiligt und haben einen Standortnachteil, sie drohen auch von der kulturellen, politischen und technischen Entwicklung abgehängt zu werden.

\subsubsection{Breitbandausbau - Verfügbarkeit}
\abstimmung
Breitband-Internetverbindungen sollen wie Strom, Straßen, Telefon und andere Infrastruktur flächendeckend verfügbar sein.

\subsubsection{Neue Definition von Breitband}
\abstimmung
Die zur Zeit vom Bundeswirtschaftsministerium genannte untere Grenze der Breitbandgeschwindigkeit von 128 KBit/s ist dabei nicht ausreichend. Die Definition von Breitbandgeschwindigkeit soll in Zukunft der aktuellen technischen Entwicklung angepasst werden.

\subsubsection{Breitbandausbau - vorausschauender Ausbau}
\abstimmung
Beim Bau und der Sanierung von Straßen müssen vorausschauend Leerrohre gelegt werden, um einen kostengünstigen Breitbandausbau zu ermöglichen.
\subsubsection{Breitbandausbau - Ausbauförderung}
\abstimmung
Wir wollen unterversorgte Gebiete finanziell fördern, um den Ausbau voranzutreiben. Das Land soll einen Beauftragten einsetzen, dessen Aufgabe es ist, in den Kommunen gezielt über die Fördermittel zu informieren und für den Breitbandausbau zu werben.