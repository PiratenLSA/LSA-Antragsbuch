\section{Wirtschaft}

\wahlprogramm{Präambel}
\antrag{KV Trier/Trier-Saarburg}\version{03:27, 20. Jun. 2010}

\subsubsection{Präambel Wirtschaft Teil 1}
\abstimmung
Die Piratenpartei Rheinland-Pfalz steht für eine nachhaltige und soziale Wirtschaftspolitik.

\subsubsection{Präambel Wirtschaft Teil 2}
\abstimmung
Wir setzen uns für fairen Wettbewerb, für die Förderung von Innovationen sowie gegen privatwirtschaftliche Monopole und übermäßige staatliche Regulierung der Unternehmen ein.
 
\wahlprogramm{Privatisierung öffentlicher Einrichtungen}\label{wp:wirt:privat1}
\antrag{Acamir}\version{03:27, 20. Jun. 2010}
\begin{itemize}
\item \konkurrenz{wp:wirt:privat2}
\item \konkurrenz{wp:wirt:privat3}
\end{itemize}

\subsubsection{Privatisierung öffentlicher Einrichtungen}
\abstimmung
Die Privatisierung von öffentlichen Einrichtungen, die für die Grundversorgung der Bevölkerung notwendig sind (z.B. ÖPNV, Müllabfuhr, Wasserversorgung, Krankenhäuser), ist zu stoppen. Der Verkauf solcher Einrichtungen an Investoren bei anschließendem Zurückmieten ( Sale and Lease Back Verträge) ist zu verbieten. Öffentliche Einrichtungen, die nicht für die Grundversorgung der Bevölkerung notwendig sind (z.B. Schwimmbäder oder andere Freizeiteinrichtungen), sollten nur privatisiert werden dürfen, wenn dadurch keine Monopolstellungen entstehen.
 
\wahlprogramm{Privatisierung öffentlicher Einrichtungen}\label{wp:wirt:privat2}
\antrag{Unbekannt}\version{03:27, 20. Jun. 2010}
\begin{itemize}
\item \konkurrenz{wp:wirt:privat1}
\item \konkurrenz{wp:wirt:privat3}
\end{itemize}

\subsubsection{Privatisierungen öffentlicher Einrichtungen}
\abstimmung
Über die Privatisierung öffentlicher Einrichtungen der kommunalen Grundversorgung sollen die Bürger vor Ort und damit die Kunden der Einrichtung direkt entscheiden dürfen.

\paragraph{Begründung}: Privatisierungen sind nicht grundsätzlich gut oder schlecht, egal was Parteien mit neo-liberaler oder sozialistischer Ideologie weismachen wollen. Privatisierungen sind meist schlecht durchgeführt, da die betreffenden Unternehmen jeweils eine kleine Gruppe von Politikern überzeugen müssen, was viel Raum für Korruption lässt. Andererseits sind auch viele Einrichtungen in staatlichem Besitz schlecht geführt und bieten nur suboptimalen Service bzw. überhöhte Preise.

Die Anbieter öffentlicher Dienstleistungen sollen direkt um ihre Kunden, also die Bürger des betreffenden Gebietes werben, nicht um die Gunst Einzelner in hohen Ämtern. Eine Vertragsklausel zur Rückführung in die öffentliche Hand bei ungenügenden Leistungen, wieder festgestellt durch den Bürger, ist wirksamer, als einem privaten oder staatlichen Dienstleister ein dauerhaftes Monopol zu geben.
 
\wahlprogramm{Privatisierung öffentlicher Einrichtungen}\label{wp:wirt:privat3}
\antrag{Marcus}\version{03:27, 20. Jun. 2010}
\begin{itemize}
\item \konkurrenz{wp:wirt:privat1}
\item \konkurrenz{wp:wirt:privat2}
\end{itemize}

\subsubsection{Privatisierung öffentlicher Einrichtungen}
\abstimmung
Sale and Lease Back Verträge (bei Immobilien+Sachanlagen) sind generell zu verbieten, weil 1. diese Finanzierungsform über die Gesamtlaufzeit grundsätzlich teurer ist, wie eine direkte Kreditaufnahme: Es muß der Unternehmergewinn über das Leasing mitbezahlt werden, dazu fallen Transaktionskosten für die Kommune an. 2. Sofern nicht ein Restkaufpreis gemäß dem steuerlichen Restwert (nach Abschreibungen) vereinbart wird, fallen Wertsteigerungen dem Privatinvestor zu. 3. Die Kommune verliert die Handlungshoheit über das Objekt und ist meistens langfristig gebunden.

\subsubsection{ }
\abstimmung
Cross Border Leasing ist grundsätzlich zu verbieten, weil 1.Ethisch verwerflich: es wird auf Steuervergünstigungen des Investors in den USA spekuliert. 2. CBL-Verträge werden grundsätzlich nach US Recht New York City abgeschlossen, dadurch völlige Intransparenz(hochkomplizierte Verträge, die nur von spezialisierten Rechtsanwälten geprüft werden können.3. Dadurch fallen Transaktionskosten von bis zu 10\% an, die von der Kommune zu tragen sind, ebenso wie das Risiko, dass die Steuervorteile von einem US-Finanzamt und Finanzgericht nicht anerkannt werden. Letzteres ist häufig der Fall.

\subsubsection{ }
\abstimmung
Die Privatisierung, also Auslagerung und Verkauf von öffentlichen Einrichtungen, die für die Grundversorgung der Bürger notwendig sind (Wasserwerke, städtische Energieversorger Gas,Strom, städtische Schienennetze,Stromleitungen, Krankenhäuser, Schulen, Verwaltungsgebäude, sonstige Anlagen, sind zu stoppen, da 1.die Kommune die Planungshoheit und Verfügungsgewalt verliert,2.langfristig finanzielle Nachteile zu erwarten sind.

\subsubsection{ }
\abstimmung
Die Autragsvergabe für die Übernahme der Aufgaben der Müllabfuhr, des ÖPNV, für den Betrieb von Krankenhäusern und anderen Dienstleistungen ist gemäß EU-Richtlinie unter den Bedingungen (Auschreibung) zu gestatten, daß keine Monopole/Oligopole entstehen, bzw. diese durch zeitliche Befristungen (im Falle Müllabfuhr) begrenzt werden.
 
\wahlprogramm{Wirtschaftsförderung überprüfen}\label{wp:wirt:wirt1}
\antrag{Acamir}\konkurrenz{wp:wirt:wirt2}\version{03:27, 20. Jun. 2010}

\subsubsection{Subventionen}
Subventionen sollen nur gewährt werden, wenn sie eine positive Wirkung für die Allgemeinheit entfalten, die durch andere Maßnahmen nicht zu erreichen wäre. Die Wirkung einer Subvention ist durch eine regelmäßige Erfolgskontrolle zu überprüfen. Subventionen für Unternehmen sollen nur befristet als Anschubfinanzierung gewährt werden. Aus Gründen der Transparenz sind direkte Subventionen indirekten (wie z.B. Steuererleichterungen) vorzuziehen.
 
\wahlprogramm{Wirtschaftsförderung überprüfen}\label{wp:wirt:wirt2}
\antrag{KV Trier/Trier-Saarburg}\konkurrenz{wp:wirt:wirt1}\version{03:27, 20. Jun. 2010}

\subsubsection{Wirtschaftsförderung überprüfen}
\abstimmung
Ausgaben, im besonderen Maße Subventionen, werden auf den Prüfstand gestellt. Subventionen sollen nur dort eingesetzt werden, wo wichtige wirtschafts- und forschungspolitische Ziele anders nicht erreicht werden können. Darüber hinaus müssen alle Subventionen degressiv oder zeitlich befristet sein, so dass man nach einem festen Zeitraum den Sinn dieser Subvention wieder prüfen muss.
 
\newpage
\wahlprogramm{Zwangsmitgliedschaft in Kammern und Verbänden abschaffen}\label{wp:wirt:zwang1}
\antrag{Salorta}\konkurrenz{wp:wirt:zwang2}\version{03:27, 20. Jun. 2010}

\subsubsection{Kammerzwang}
\abstimmung
Die Zwangsmitgliedschaft in Kammern und Verbänden in Deutschland wie in der Industrie- und Handelskammer (IHK) oder den Handwerkskammern ist ein Beispiel für unnötige Bürokratie. Viele Unternehmer und Selbständige haben kein Interesse an deren Leistungen und kennen diese oft nicht einmal. Trotzdem ist jeder Gewerbetreibende und jeder Gründer einer Firma ab dem ersten Tag zur Beitragszahlung an die IHK verpflichtet. Zwar kostet die Zwangsmitgliedschaft in der IHK nicht viel, dieser Beitrag ist jedoch nach Ansicht vieler Unternehmer der sinnloseste Beitrag für die Verwaltung. Diese Zwangsregelung trifft besonders kleine Gewerbetreibende oder Handwerker hart, die keine Leistungen in Anspruch nehmen. Selbst inaktive Firmen oder Betriebe, die sich in Auflösung befinden, sind zu dieser Abgabe verpflichtet. Für Selbständige kommt erschwerend hinzu, dass deren private Einkünfte an die IHK beziehungsweise die Handwerkskammer übermittelt werden, da sich nach deren Höhe die Abgabenhöhe an die Kammern bemisst. Dies stellt nach Auffassung der Piraten eine eklatante Verletzung der Privatsphäre von Selbständigen dar. Die vielfach praktizierte Zwangsmitgliedschaft in Kammern und Verbänden in Deutschland schränkt Unternehmer und Betriebe in ihrer Freiheit ein und bieten nicht durchgängig für den Zwangsbeitrag äquivalente Leistungen.

Wir treten daher für die Abschaffung der Zwangsmitgliedschaft mit Zwangsbeiträgen in Kammern und Verbänden ein, um diese durch eine freiwillige Beitrittsmöglichkeit zu ersetzen. Damit wird auch die Übermittlung der privaten Einkünfte von Selbständigen an die IHK beziehungsweise die Handwerkskammern beendet.
 
\wahlprogramm{Zwangsmitgliedschaft in Kammern und Verbänden abschaffen}\label{wp:wirt:zwang2}
\antrag{KV Trier/Trier-Saarburg}\konkurrenz{wp:wirt:zwang1}\version{03:27, 20. Jun. 2010}

\subsubsection{Kammerzwang}
\abstimmung
Wir planen, die Zwangsmitgliedschaft mit Zwangsbeiträgen in Kammern und Verbänden abzuschaffen und durch eine freiwillige Beitrittsmöglichkeit zu ersetzen. Hierzu wollen wir eine Bundesratsinitiative anregen.
 
\wahlprogramm{Begrenzung der Leiharbeit}
\antrag{KV Trier/Trier-Saarburg}\version{03:27, 20. Jun. 2010}

\subsubsection{Begrenzung der Leiharbeit}
\abstimmung
Nach französischem Vorbild sollen Leiharbeiter nicht eine billige Verfügungsmasse sein, mit der reguläre Beschäftigte unter Druck gesetzt werden können, sondern für die ihnen abverlangte Flexibilität mit einem Lohnzuschlag entschädigt werden. Wir wollen, dass das Land Rheinland-Pfalz dazu eine entsprechende Initiative im Bundesrat startet.
 
\wahlprogramm{Missbrauch von Praktika verhindern}
\antrag{KV Trier/Trier-Saarburg}\version{03:27, 20. Jun. 2010}

\subsubsection{Missbrauch von Praktika verhindern}
\abstimmung
Arbeitgeber, die Praktikanten als billige Arbeitskräfte ausbeuten, verhalten sich nicht nur unfair gegenüber den Praktikanten sondern auch gegenüber ihren Mitbewerbern und den sozialen Sicherungssystemen.

Darum wollen wir die Regelungen für Praktika verschärfen.

Probezeit, Werkstudententätigkeit und befristete Arbeitsverträge sind ausreichende Werkzeuge des Arbeitsmarkts, um Berufsanfängern den Start in das Berufsleben zu erleichtern.

\subsubsection{Praktika nur während Ausbildung}
\abstimmung
Ein Praktikumsvertrag soll nur im Rahmen von Schule, Studium oder Berufsausbildung geschlossen werden können.

\subsubsection{Praktika nur während Ausbildung}
\abstimmung
Praktikumsstellen müssen öffentlich ausgeschrieben werden, verpflichtend ist dabei die Angabe einer Mindestvergütung.
