\section{Sozialpolitik}

\wahlprogramm{Präambel}
\antrag{KV Trier/Trier-Saarburg}\version{03:11, 20. Jun. 2010}

\subsubsection{Präambel}
\abstimmung
Freiheit hat auch mit dem Recht jedes Menschen zu tun, ein möglichst selbstbestimmtes Leben bis ins hohe Alter zu führen. Dazu ist man aber oft auch auf die Solidarität anderer angewiesen. Deshalb wollen wir, dass auch künftig Gesunde für die Kranken, Arbeitende für Arbeitslose, Jung für Alt und Alt für Jung eintreten. So kann eine gerechte Gesellschaft bestehen, die Freiheit für jeden verheißt.
 
\wahlprogramm{Sozialpolitik im Bundesrat}
\antrag{KV Trier/Trier-Saarburg}\version{03:11, 20. Jun. 2010}
\subsubsection{Sozialpolitik im Bundesrat}
\abstimmung
Wir wollen, dass sich das Land auch bei seiner Mitwirkung an der sozial- und gesundheitspolitischen Gesetzgebung im Bundesrat am Ideal einer gerechten Gesellschaft orientiert.
 
\wahlprogramm{Sozialpolitik als Grundrecht und Grundpflicht}
\antrag{Limbo}\version{03:11, 20. Jun. 2010}

\subsubsection{Sozialpolitik als Grundrecht und Grundpflicht}
\abstimmung
Die Piratenpartei sollte sehen, dass im Zuge des Demografischenwandels jeder Mensch zu sozialen Handeln herangeführt und erzogen werden sollte. Dies kann sich in Pflichtpraktika während der Schulzeit aber auch in Zwangsleistunge durch Arbeitslose ausdrücken. Jeder Mensch in Rheinland-Pfalz soll lernen gerne und mit Freude anderen zu Helfen. So wird die Piratenpartei das Land ein Stück gerechter machen. Wer Anreize (finanzielle oder materielle) schafft, wird auch den Menschen überzeugen können, seine Freizeit für mehr "Ehrenamt" zu nutzen.
 
\newpage
\wahlprogramm{Handlungsfreiheit und Würde von finanzschwachen Bürgern sicherstellen}
\antrag{Thomas Heinen}\version{03:11, 20. Jun. 2010}

\subsubsection{Handlungsfreiheit und Würde von finanzschwachen Bürgern sicherstellen}
\abstimmung
Gerade in der aktuellen Situation, in der Regierende die Bürger- und Menschenrechte nach und nach zu erodieren versuchen, brauchen wir eine wachsame und politisch aktive Zivilgesellschaft. In einem modernen Sozialstaat muss die Möglichkeit der Teilnahme am politischen und kulturellen Leben für alle Menschen sichergestellt werden. Diese Freiheit darf nicht durch staatliche Kürzungen, die eine mangelnde soziale Sicherung oder gar Existenzängste nach sich ziehen, eingeschränkt werden.

Aus finanzieller Notlage und Zukunftsängsten heraus kann keine Freiheit für politisches Handeln erwachsen. Das Schaffen von Zwangslagen führt bei den Betroffenen zu einer Radikalisierung der politischen Forderungen. Dies gefährdet die Demokratie in unserer Gesellschaft.

Daher wird sich die Piratenpartei Rheinland-Pfalz dafür einsetzen, dass die Handlungsfreiheit auch und gerade von finanzschwachen Bürgern sichergestellt und deren Würde nicht als Folge von bestimmten Kürzungen oder Änderungen im Sozialbereich verletzt wird.