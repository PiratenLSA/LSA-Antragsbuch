\section{Verbraucherschutz}

\subsection*{Mehr Transparenz beim Einkauf}
\wahlprogramm{Mehr Transparenz beim Einkauf}
\antrag{unbekannt}\version{03:16, 20. Jun. 2010}
\subsubsection{Mehr Transparenz beim Einkauf}
\abstimmung
Die Piratenpartei setzt sich für einen besseren Verbraucherschutz und mehr Transparenz, nicht nur in der Politik und Verwaltung, sondern auch im Alltag ein.
 
\wahlprogramm{Mehr Transparenz beim Einkauf}
\antrag{unbekannt}\version{03:16, 20. Jun. 2010}
\subsubsection{Mehr Transparenz beim Einkauf}
\abstimmung
Nach dem Scheitern der Lebensmittelampel will die Piratenpartei einen alternativen Weg testen, um für mehr Transparenz beim Einkauf zu sorgen. Die Piratenpartei will ein System testen, bei dem die Verbraucher ihre individuellen Kriterien an ein Produkt, also zum Beispiel auch Unverträglichkeit gegen bestimmte Stoffe, in ein individuelles Profil auf einer Karte speichern können. An Barcodescannern soll so jeder Kunde schnell erkennen können, ob das Produkt seinen individuellen Anforderungen entspricht. Der Datenschutz hat für uns in Verbindung mit diesem Projekt natürlich höchste Priorität.
 
\subsection*{Verbraucherschutz}
\wahlprogramm{Verbraucherinformationsgesetz stärken}
\antrag{KV Trier/Trier-Saarburg}\version{03:16, 20. Jun. 2010}
\subsubsection{Verbraucherinformationsgesetz stärken}
\abstimmung
Wir wollen das Landesgesetz zur Ausführung des Verbraucherinformationsgesetzes (AGVIG) so stärken, dass Verbraucher Informationen, beispielsweise zu belasteten Lebensmitteln, auf gut zugänglichen Plattformen rasch und einfach auffinden können, ohne sie erst in langen Auskunftsprozessen anfordern zu müssen.
 
\subsection*{Verbraucherzentrale stärken}
\antrag{KV Trier/Trier-Saarburg}\version{03:16, 20. Jun. 2010}
\subsubsection{Verbraucherzentrale stärken}
\abstimmung
Verbraucherzentralen spielen eine wichtige Rolle in der Beratung von Verbrauchern und im Schutz von Verbraucherinteressen. Die Einschränkung der Arbeit der Verbraucherzentrale Rheinland-Pfalz durch restriktive Mittelzuweisungen lehnen wir ab.

Wir unterstützen insbesondere den Einsatz der Verbraucherzentralen für den Datenschutz der Verbraucher und ihren Kampf gegen das Modell des „Gläsernen Konsumenten“.

Wir wollen einen Verbraucherschutz, der das Recht auf umfassende Information verbindet mit einem Verbandsklagerecht zur Durchsetzung von Verbraucherinteressen.
 
\subsection*{Verbraucherinformation vor Ort durch Smiley-System}
\antrag{KV Trier/Trier-Saarburg}\version{03:16, 20. Jun. 2010}
\subsubsection{Verbraucherinformation vor Ort durch Smiley-System}
\abstimmung
Die Ergebnisse von Lebensmittelkontrollen werden anhand unterschiedlicher Smileys zeitnah und gut sichtbar an der Eingangstür angebracht, um den Verbraucher zusätzlich zum Informationssystem im Internet direkt vor Ort zu informieren. Das in Dänemark etablierte und sehr erfolgreiche Smiley-System soll auch in Rheinland-Pfalz eingeführt werden. So ist für den Kunden direkt, beispielsweise vor Restaurants, Eisdielen oder Supermärkten, ersichtlich, ob Hygienevorschriften und Lebensmittelgesetze eingehalten werden. Auf Hygienesünder kann reagiert werden, was bisher meistens nicht möglich ist. Negativ bewertete Betriebe haben durch die Kundenreaktion und Folgekontrollen die Möglichkeit und vor allem die Motivation, Mängel zu beseitigen und sich positive Smileys zu verdienen.
 
\subsection*{Transparente Kennzeichnung von Lebensmitteln}
\antrag{KV Trier/Trier-Saarburg}\version{03:16, 20. Jun. 2010}
\subsubsection{Transparente Kennzeichnung von Lebensmitteln}
\abstimmung
Auf der Vorderseite von Verpackungen muss statt Prozentangaben und beliebig wählbarer Portionsgrößen eine einheitliche, differenzierte und transparente Kennzeichnung dem Verbraucher eine schnelle und verlässliche Orientierung geben. Die von der Lebensmittelindustrie auf der Vorderseite von Verpackungen bevorzugte Nährwertkennzeichnung trägt nicht dazu bei, dem Verbraucher sinnvolle Informationen an die Hand zu geben. Besonders irreführend ist die Angabe des prozentualen Anteils am Tagesbedarf. Da sie prinzipiell vielen Personengruppen wie zum Beispiel Kindern nicht gerecht werden kann, ist sie durch eine sinnvolle, verpflichtende Kennzeichnung zu ersetzen. Diese muss sich auf feste Portionsgrößen von 100g/ml entsprechend der Nährwertangaben auf der Rückseite beziehen.
