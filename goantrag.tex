\section{Geschäftsordnungsanträge}
\goantrag{Wiedereinführung der zwei Stimmkarten}
\antrag{Alexander Zinser \& Christoph Giesel}\\
\version{19:21, 12. Apr. 2012}

\paragraph{Antragstext}:

Der LPT möge beschließen, § 5 Abs. (2) der Geschäftsordnung wie folgt zu ändern:

\einruecken{(2) Für offene Wahlen und Abstimmungen erhält jeder Stimmberechtigte zwei Stimmkarten, die durch Farbe, Symbol und Beschriftung als {\frqq}Ja{\flqq} und {\frqq}Nein{\flqq} gekennzeichnet sind. Bei Abstimmungen wird in einer Abfrage gleichzeitig nach Ja- und Nein-Stimmen gefragt, es ist die jeweils gewünschte Stimmkarte zu zeigen. Enthaltungen werden nicht gezählt.}

\paragraph{Alte Fassung}:

\einruecken{(2) Für offene Wahlen und Abstimmungen erhält jeder Stimmberechtigte eine Stimmkarte. Bei Abstimmungen wird in einer Abfrage nacheinander nach Ja- und Nein-Stimmen gefragt, bei Zustimmung ist die Stimmkarte zu zeigen. Enthaltungen werden nicht gezählt.}

\begruendung{Laut Geschäftsordnung ist vorgeschrieben, dass es zwei Stimmkarten für Ja und Nein geben muss (die man hochhält). Wir haben aus ökologischen Gründen keine unterschiedlichen Stimmkarten - also jeder bekommt nur eine. Dieser Antrag ist Alternativlos. :D}
