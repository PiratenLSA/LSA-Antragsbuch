\section{Privatsphäre, Datenschutz und Bürgerrechte - Grundpfeiler der freiheitlichen Informationsgesellschaft}

\subsection*{Bedeutung von Datenschutz und Privatsphäre}
\wahlprogramm{Bedeutung von Datenschutz und Privatsphäre}\label{datenschutz:bedeutung}
\antrag{Unglow}\version{22:51 18. Jun. 2010}
\abstimmung
Der Schutz der Privatsphäre und der Datenschutz gewährleisten Würde und Freiheit des Menschen, die freie Meinungsäußerung, demokratische Teilhabe und in der Folge unsere freiheitlich-demokratische Gesellschaftsform, die in der Vergangenheit auch unter Einsatz zahlloser Menschenleben erkämpft und verteidigt wurde.

Jeder einzelne Schritt auf dem Weg zum Überwachungsstaat mag noch so überzeugend begründet sein - doch als Deutsche und Europäer wissen wir aus Erfahrung, wohin dieser Weg führt. Diesen Entwicklungen stellen wir uns entschieden entgegen und sagen dem Überwachungsstaat den Kampf an.

Das Recht auf Wahrung der Privatsphäre ist ein unabdingbares Fundament einer demokratischen Gesellschaft. Die Meinungsfreiheit und das Recht auf persönliche Entfaltung sind ohne diese Voraussetzung nicht zu verwirklichen.

Die Piratenpartei hat das Ziel, die Privatsphäre der Bürger vor unberechtigten und unverhältnismäßigen Eingriffen durch Staat und Wirtschaft zu schützen. Die Überwachungspolitik der letzten Jahre wollen wir umkehren, um eine freiheitsfreundliche Sicherheitspolitik unter Achtung der Bürgerrechte zu gewährleisten.

\subsection*{Kein Überwachungsstaat - das Recht in Ruhe gelassen zu werden}
\wahlprogramm{Datenschutz}\label{datenschutz:datenschutz}
\antrag{Piraten aus RLP}\version{22:51 18. Jun. 2010}
\subsubsection{Kein Überwachungsstaat - das Recht in Ruhe gelassen zu werden}
\abstimmung
Systeme und Methoden, die der Staat gegen seine Bürger einsetzen kann, müssen der ständigen Bewertung und genauen Prüfung durch gewählte Mandatsträger unterliegen. Wenn die Regierung Bürger beobachtet, ohne dass sie eines Verbrechens verdächtig sind, ist dies eine fundamental inakzeptable Verletzung des Bürgerrechts auf Privatsphäre.

Die pauschale Verdächtigung und anlasslose Überwachung aller Bürger hat generell zu unterbleiben. Eine als 'präventive Strafverfolgung' verschleierte Abschaffung der Unschuldsvermutung lehnen wir unbedingt ab.

Die flächendeckende Überwachung des öffentlichen Raums durch Videokameras oder andere Maßnahmen darf nicht zugelassen werden. Wir fordern ein allgemeines Verbot der Überwachung des öffentlichen Raums, von dem nur einzelne, richterlich angeordnete Ausnahmen zulässig sind.
Jedem Bürger muss das Recht auf Anonymität garantiert werden, das unserer Verfassung innewohnt. Die Weitergabe personenbezogener Daten vom Staat an die Privatwirtschaft hat in jedem Falle zu unterbleiben.

\subsubsection{Informationelle Selbstbestimmung}
\abstimmung
Das Recht des Einzelnen, die Verwendung seiner persönlichen Daten zu kontrollieren, muss gestärkt werden. Jegliche kommerzielle Nutzung persönlicher Daten muss verboten sein, solange sie nicht ausdrücklich vom Betroffenen erlaubt wird. Dazu müssen insbesondere die Datenschutzbeauftragten völlig unabhängig agieren können. Neue Methoden wie das Scoring machen es erforderlich, nicht nur die persönlichen Daten kontrollieren zu können, sondern auch die Nutzung aller Daten, die zu einem Urteil über eine Person herangezogen werden können. Jeder Bürger muss gegenüber den Betreibern zentraler Datenbanken einen durchsetzbaren und wirklich unentgeltlichen Anspruch auf Selbstauskunft, Korrektur, Sperrung oder Löschung der Daten haben. Ausgenommen davon sind Fälle, in denen ein öffentliches Interesse zur Erfüllung der staatlichen Aufgaben vorliegt.

\subsubsection{Biometrische Daten und Gentests}
\abstimmung
Erhebung und Nutzung biometrischer Daten und Gentests erfordern aufgrund des hohen Missbrauchspotentials eine besonders kritische Bewertung und Kontrolle von unabhängiger Stelle. Der Aufbau zentraler Datenbanken mit solchen Daten muss unterbleiben. Die Verwendung biometrischer Merkmale in Passdokumenten hat zu unterbleiben oder auf Freiwilligkeit zu beruhen. Es ist gegenüber Drittstaaten durchzusetzen, dass diese Pässe unabhängig von biometrischen Merkmalen vollständig gültig sind. Massengentests für polizeiliche Zwecke, bei denen die Vorgeladenen nicht individuell verdächtigt werden, müssen als anlasslose Verdächtigungen gewertet und entsprechend untersagt werden.

\subsubsection{Besonderheiten digitaler Daten}
\abstimmung
Generell müssen die Bestimmungen zum Schutze personenbezogener Daten die Besonderheiten digitaler Daten, wie etwa mögliche Langlebigkeit und schwer kontrollierbare Verbreitung, stärker berücksichtigen. Gerade weil die Piratenpartei für eine stärkere Befreiung von Information, Kultur und Wissen eintritt, fordern wir Datensparsamkeit, Datenvermeidung und unabhängige Kontrolle von Stellen, die personenbezogene Daten verwenden. Wenn diese nämlich für wirtschaftliche oder Verwaltungszwecke genutzt werden, können sie die Freiheit und die informationelle Selbstbestimmung des Bürgers unnötig einschränken und den Überwachungsdruck verstärken. Zu einem effektivem Datenschutz gehört aus Sicht der Piratenpartei ausserdem das Recht des Bürgers, über ungewollte Datenabflüsse personenbezogener Daten aus Unternehmen und Behörden unverzüglich und lückenlos informiert zu werden.

\newpage
\subsubsection{Konkrete Maßnahmen}
\abstimmung
Konkrete Forderungen im Bereich der Privatsphäre und der Inneren Sicherheit sind:
\begin{itemize}
\item Durchsetzung des Folterverbots
\item Bessere, wirksame Kontrolle von Geheimdiensten und Polizei national und europaweit
\item Solange kein europaweiter einheitlicher Datenschutz auf hohem Niveau existiert, dürfen die Hürden für den Informationsaustausch zwischen der deutschen Polizei und der anderer Mitgliedsstaaten nicht weiter abgesenkt werden.
\item Kein Informationsaustausch mit Staaten ohne wirksamen Datenschutz
\item Einführung einer Informations- und Auskunftspflicht gegenüber den Betroffenen beim Datenaustausch zwischen Polizeien der EU-Länder
\item Rücknahme der EU-Richtlinie über die Vorratsdatenspeicherung und Stopp aller Planungen zur Wiedereinführung des Gesetzes
\item keine Vorratsspeicherung von Flug-, Schiff- und sonstigen Passagierdaten (PNR: Passenger Name Records)
\item Stopp der anlasslosen Übermittlung von Passagierdaten
\item keine Weitergabe von solchen Passagierdaten an Dritte
\item Abschaffung des ELENA-Systems zum elektronischen Einkommensnachweis
\item Stopp des Jugendmedienschutz-Staatsvertrags
\item Rücknahme des ZugangsErschwerungsgesetzes
\item Stopp des SWIFT-Abkommens mit den USA
\item Stopp der Volkszählung 2011 und Rücknahme des Zensus-Gesetzes zur
\item Rücknahme der Auslandskopfüberwachung
\item kein automatisiertes KFZ-Kennzeichen-Scanning
\item Abschaffung der biometrischen Daten in Pässen und Ausweisen. Verzicht auf RFID-Chips in Ausweisdokumenten.
\item Einrichtung einer unabhängigen deutschen Datenschutzbehörde mit Sanktions-Recht
\item keine 'präventive' Strafverfolgung (keine Aufhebung der Unschuldsvermutung)
\item keine Internierungslager (Gefängnis ohne Aburteilung) in Deutschland
\item Abbau von Echelon-Abhörzentralen auf deutschem Boden
\item Abschaffung der "Anti-Terror-Datei", der "Visa-Warndatei" und anderer unrechtmäßiger Datenbanken
\item Stärkung des allgemeinen Informantenschutzes
\item Abschaffung der Beugehaft für Zeugen
\item Wiederherstellung der Trennung von Polizei und Geheimdiensten. Rücknahme der geheimdienstlichen Befugnisse für das BKA.
\item Schutz von Ermittlungsdaten vor automatischem Austausch zwischen Polizeien verschiedener Staaten
\item Einführung eines eindeutigen, gut sichtbaren Identifikationsmerkmals (Nummer oder Name) für Polizisten bei Einsätzen zur Identifikation
\item Verzicht auf Videoüberwachung von öffentlichen Plätzen etc., Videoüberwachung generell verstärkt ersetzen durch unbewaffnete Polizeistreifen.
\item Keine automatische Gesichts- oder Verhaltenskontrolle
\item Ausweitung des Persönlichkeits-Kernbereichs auf elektronische Medien (z.B. Mail bei Webmailern, Laptop)
\item Keine geheimen Durchsuchungen - weder online noch offline!
\item Überprüfung und ggf. Aufhebung der unter dem Namen 'Anti-Terror-Maßnahme' eingeführten Regelungen, die seit dem 11.9.2001 installiert wurden
\item Einführung einer Meldepflicht von Unternehmen und Behörden bei Datenpannen
\end{itemize}

\subsection*{Überwachung abbauen, Befugnisse evaluieren}
\wahlprogramm{Überwachung abbauen, Befugnisse evaluieren}\label{datenschutz:ueberwachung}
\antrag{Piraten aus RLP}\version{22:51 18. Jun. 2010}
\subsubsection{Überwachung abbauen}
\abstimmung
Gemeinsam mit dem Bürgerrechtsbündnis 'Freiheit statt Angst' fordern wir:
\begin{itemize}
\item Abschaffung der flächendeckenden Protokollierung der Kommunikation und unserer Standorte (Vorratsdatenspeicherung)
\item Abschaffung der flächendeckenden Erhebung biometrischer Daten, sowie von RFID-Ausweisdokumenten
\item Schutz vor Bespitzelung am Arbeitsplatz durch ein Arbeitnehmerdatenschutzgesetz
\item Berücksichtigung des Datenschutzes für Bürger- und Arbeitnehmer/innen bereits in der Konzeptionsphase aller öffentlicher eGovernment-Projekte
\item Keine einheitliche Schülernummer (Berliner SchülerID)
\item Keine Weitergabe von Informationen über Menschen ohne triftigen Grund und konkreten Anlass
\item Keine europaweite Vereinheitlichung staatlicher Informationssammlungen (Stockholmer Programm)
\item Keine systematische Überwachung des Zahlungsverkehrs oder sonstige Massendatenanalyse (Stockholmer Programm)
\item Kein Informationsaustausch mit den USA und anderen Staaten ohne wirksamen Grundrechtsschutz
\item Abbau von Videoüberwachung und Verbot des Einsatzes von Verhaltenserkennungssystemen
\item Keine pauschale Registrierung aller Flug- und Schiffsreisenden (PNR-Daten)
\item Keine geheime Durchsuchung von Privatcomputern, weder online noch offline
\item Keine Einführung der Elektronischen Gesundheitskarte in der derzeit geplanten Form
\end{itemize}

\subsubsection{Evaluierung der bestehenden Überwachungsbefugnisse}
\abstimmung
Wir fordern eine unabhängige Überprüfung aller bestehenden Überwachungsbefugnisse im Hinblick auf ihre Wirksamkeit, Kosten, schädliche Nebenwirkungen und Alternativen.

\subsubsection{Moratorium für neue Überwachungsbefugnisse}
\abstimmung
Nach der inneren Aufrüstung der letzten Jahre fordern wir einen sofortigen Stopp neuer Gesetzesvorhaben auf dem Gebiet der inneren Sicherheit, die mit weiteren Grundrechtseingriffen verbunden sind.

\subsubsection{Gewährleistung der Meinungsfreiheit und des freien Meinungs- und Informationsaustauschs über das Internet}
\abstimmung
\begin{itemize}
\item Keine Beschränkung des Internetzugangs durch staatliche Stellen oder Internetanbieter (Sperrlisten).
\item Keine Sperrungen von Internetanschlüssen.
\item Verbot der Installation von Filtern in die Infrastruktur des Internet.
\item Entfernung von Internet-Inhalten nur auf Anordnung unabhängiger und unparteiischer Richter.
\item Einführung eines uneingeschränkten Zitierrechts für Multimedia-Inhalte, das heute unverzichtbar für die öffentliche Debatte in Demokratien ist.
\item Schutz von Plattformen zur freien Meinungsäußerung im Internet (partizipatorische Websites, Foren, Kommentare in Blogs), die heute durch unzureichende Gesetze bedroht sind, welche Selbstzensur begünstigen (abschreckende Wirkung).
\end{itemize}

\wahlprogramm{Datenschutz}\label{datenschutz:datenschutz}
\antrag{KV Trier / Trier-Saarburg}\version{22:51 18. Jun. 2010}
 Datenschutz ist ein Grundrecht. Dies hat das Bundesverfassungsgericht schon 1983 festgestellt, als es das Recht auf informationelle Selbstbestimmung begründete.

Mit zunehmender Wandlung zu einer Wissens- und Informationsgesellschaft gewinnt der Datenschutz an Bedeutung. Immer mehr Informationen über unser tägliches Leben liegen heute in elektronischer Form vor und können automatisiert verarbeitet und zusammengeführt werden.

Deswegen gilt es, die Grundsätze des Datenschutzes (Datensparsamkeit, Datenvermeidung, Zweckbindung und Erforderlichkeit) noch konsequenter in den Vordergrund zu stellen, denn Datenschutz wird nicht allein durch technische Maßnahmen erreicht, sondern insbesondere durch organisatorische.

\subsection*{Änderungen des Landesdatenschutzgesetzes}
\wahlprogramm{Änderungen des Landesdatenschutzgesetzes}\label{datenschutz:lds}
\antrag{Unglow}\version{22:51 18. Jun. 2010}

\subsubsection{Modul 1}
\abstimmung
Das aus dem siebziger Jahren stammende Datenschutzrecht muss dringend an die Erfordernisse des Informations- und Kommunikationszeitalters angepasst werden. Die Piratenpartei strebt ein gut lesbares, allgemein verständliches und unbürokratisches Datenschutzrecht an. Die gesetzlichen Regelungen müssen unabhängig von der zukünftigen technischen Entwicklung Wirkung entfalten.

\subsubsection{Modul 2}
\abstimmung
Sinnvolle Regelungen aus der Novellierung des Bundesdatenschutzgesetzes sollen in Landesrecht übernommen werden, wie z.B. die Informationspflichten bei Datenpannen und die Fort- und Weiterbildungsmaßnahmen für Datenschutzbeauftragte.

\subsection*{Wirksame Kontrolle gewährleisten}
\wahlprogramm{Wirksame Kontrolle gewährleisten}\label{datenschutz:kontrolle}
\antrag{Unglow}\version{22:51 18. Jun. 2010}

\subsubsection{Modul 1}
\abstimmung
Wesentliche Probleme im Bereich Datenschutz sind oftmals nicht auf gesetzliche Lücken, sondern auf den mangelnden Vollzug der bestehenden Gesetze zurückzuführen. Der Landesdatenschutzbeauftragte, welcher für die Kontrolle des Datenschutzes zuständig ist, ist jedoch personell so schwach ausgestattet, dass eine wirksame Kontrolle unmöglich ist und Datenschutzverstöße oft nicht auffallen, geschweige denn geahndet werden können.

\subsubsection{Modul 2}
\abstimmung
Die Piratenpartei wird deshalb die Behörde des Landesdatenschutzbeauftragten organisatorisch, personell und finanziell so stärken, dass eine wirksame Kontrolle der bestehenden Datenschutzgesetze gewährleistet werden kann. Inbesondere müssen anlasslose Kontrollen ermöglicht werden. Zudem wollen wir die Sanktionsmöglichkeiten erhöhen, sodass Datenschutzverstöße nicht mehr aus der Protokasse bezahlt werden können und Strafen nicht zu einem betriebswirtschaftlichen Faktor verkommen.

\subsubsection{Modul 3}
\abstimmung
Bei staatlichen IT-Projekten wie der ELENA-Datenbank, der elektronischen Gesundheitskarte oder elektronischen Ausweisdokumenten wird der Datenschutz regelmäßig missachtet. Oft kommt erst nach Eingriff der Datenschutzbeauftragten und öffentlichem Protest durch Bürger und Nicht-Regierungsorganisationen das Thema Datenschutz überhaupt auf die Agenda. Die PIRATEN werden gewährleisten, dass die Datenschutzbeauftragten bei staatlichen Projekten unmittelbar mit einbezogen werden und der Datenschutz zu einer Kernanforderung bei diesen Projekten wird.


\wahlprogramm{Elektronische Gesundheitskarte}\label{datenschutz:egk}
\antrag{KV Trier/Trier-Saarburg}\version{22:51 18. Jun. 2010}
\abstimmung
Transparenz heißt für uns nicht die Schaffung eines „gläsernen Patienten“. Wir lehnen die elektronische Gesundheitskarte ab und werden uns für deren Stopp einsetzen.

\wahlprogramm{Stärkung des Landesdatenschutzbeauftragten}\label{datenschutz:staerkung}
\antrag{KV Trier/Trier-Saarburg}\version{22:51 18. Jun. 2010}
\abstimmung
Ein starker Datenschutz setzt handlungsfähige Datenschützer voraus. Aus diesem Grund soll das Amt des Landesdatenschutzbeauftragten nach dem Vorbild Schleswig-Holsteins zu einem unabhängigen Landeszentrum für Datenschutz umgebaut werden. Dieses soll in Zukunft auch für den nichtöffentlichen Bereich und für Auskünfte nach dem Informationsfreiheitsgesetz zuständig sein. Dazu muss diese Institution auch personell deutlich ausgebaut werden.

\subsection*{Digitale Selbstverteidigung}
\wahlprogramm{Digitale Selbstverteidigung}\label{datenschutz:verteidigung}
\antrag{Unglow}\version{22:51 18. Jun. 2010}
\subsubsection{Digitale Selbstverteidigung}
\abstimmung
Datenschutz und informationelle Selbstbestimmung gewährleisten die Kontrolle über die eigenen Daten. Durch die immer umfangreicher werdende Datenverarbeitung im Informationszeitalter ist Datenschutz wichtiger denn je. Trotzdem fehlt in weiten Teilen der Bevölkerung noch das Bewusstsein für den sorgfältigen Umgang mit eigenen und fremden Daten. Für vermeintliche Rabatte oder geringe Gewinnchancen sind viele bereit ihre persönlichen Daten preiszugeben, ohne sich über das Ausmaß dieser Entscheidung bewusst zu sein. Die Rechte, die der Staat seinen Bürgern einräumt, können nur Wirkung entfalten, wenn die Menschen sie bewusst ausüben können. Die Piratenpartei will deshalb die Voraussetzungen für eine wirksame digitale Selbstverteidung schaffen.
\subsubsection{Datenschutz als Bildungsauftrag}
\abstimmung
Wir betrachten Datenschutz als staatliche Bildungsaufgabe und wollen alle Bildungsträger in Rheinland-Pfalz in diese Aufgabe einbeziehen. Aufklärung über Datenschutz ist nicht nur Aufgabe der Schulen, sondern auch der politischen Bildungseinrichtungen, der Volkshochschulen, der Ausbildungseinrichtungen und anderer Bildugsstätten.

Die Menschen müssen in der Lage sein, die Bedeutung der Privatsphäre für eine freiheitliche Gesellschaft und ein selbstbestimmtes Leben zu erkennen und frühzeitig über die Gefahren aufgeklärt werden, die von Staat, Wirtschaft und von unachtsamer Datenpreisgabe ausgehen. Der verantwortungsvolle Umgang mit eigenen Daten und den Daten Dritter muss vermittelt werden.

Die Auskunfts-, Änderungs- und Löschansprüche, welche die Datenschutzgesetze einräumen, sind vielen Menschen nicht bekannt. Wir werden durch Informationskampagnen und Hilfsangebote dafür sorgen, dass diese Rechte wahrgenommen werden können.

\subsubsection{Selbstdatenschutz durch Information und Transparenz}
\abstimmung
Bürger müssen umfassend über Datenerhebungen und -verarbeitung informiert werden um ihre Rechte wahrnehmen zu können. Deshalb werden wir datenverarbeitende Unternehmen zu mehr Transparenz verpflichten: Kunden müssen klar und deutlich über das Ausmaß und den Zweck von Datensammlung und -verarbeitung aufgeklärt und über die Konsequenzen informiert werden. Nur so ist gewährleistet, dass die Betroffenen ihre Daten tatsächlich freiwillig und bewusst herausgeben.

\subsubsection{Informationelle Selbstbestimmung in sozialen Netzwerken}
\abstimmung
Insbesondere junge Menschen nutzen vermehrt Soziale Netzwerke im Internet um sich mit Freunden auszutauschen, neue Kontakte zu knüpfen und gemeinsamen Interessen nachzugehen. Der Datenschutz wird in vielen dieser Netzwerke jedoch sträflich vernachlässigt.

Wir werden die gesetzlichen Rahmenbedingungen schaffen, damit jeder unbeschwert und ohne Angst vor Datenmissbrauch oder Cyber-Mobbing an diesen Netzwerken teilhaben kann. Wir werden für eine wirksame Durchsetzung der informationellen Selbstbestimmung in diesen Netzwerken sorgen. Jeder Nutzer muss zu jeder Zeit die Kontrolle darüber behalten, wer welche Daten einsehen darf. Die Nutzung von personenbezogenen Daten durch die Betreiber, ohne explizite Einwilligung des Nutzers werden wir unterbinden.

\subsection*{Datenschutz auf Bundesebene} 
\wahlprogramm{Datenschutz auf Bundesebene}\label{datenschutz:bundesebene}
\antrag{Unglow}\version{22:51 18. Jun. 2010}
\subsubsection{Modul 1}
\abstimmung
Die Piratenpartei Rheinland-Pfalz wird sich im Bundesrat für eine Verbesserung des Datenschutzes auf Bundesebene stark machen. Insbesondere werden wir die Einführung eines Arbeitnehmerdatenschutzgesetzes voran treiben. Wir werden uns im Bundesrat gegen datenschutzfeindliche Gesetze wie z.B. eine neue Vorratsdatenspeicherung stellen und uns dafür einsetzen, Gesetze wie das BKA-Gesetz und ELENA zu entschärfen bzw. datenschutzgerecht umzugestalten.

\subsection*{Datenschutz auf Landesebene}
\wahlprogramm{Datenschutz auf Landesebene}\label{datenschutz:landesebene}
\antrag{Unglow}\version{22:51 18. Jun. 2010}
\subsubsection{Datenschutz auf Landesebene}
\abstimmung
Ein wesentlicher Grundsatz des Datenschutzes ist die Datensparsamkeit. Diese besagt, dass nur so wenige Daten wie notwendig gesammelt werden sollen. Um diesem Grundsatz gerecht zu werden, wird die Piratenpartei Rheinland-Pfalz alle vom Land erfassten Daten auf ihre Notwendigkeit und Zweckmäßigkeit hin überprüfen.

\subsubsection{Google Analytics in der Landesverwaltung}
\abstimmung
Die Piratenpartei Rheinland-Pfalz wird den illegalen Einsatz der Software "Google Analytics" in der Landesverwaltung stoppen. Laut dem Tätigkeitsbericht des Landesdatenschutzbeauftragten wird das Programm von einigen Stellen eingesetzt, obwohl dessen Betrieb gegen deutsches Datenschutzrecht verstößt. Wir werden sicherstellen, dass jeder Bürger die Webseiten des Landes nutzen kann, ohne dabei ausspioniert zu werden.

\subsubsection{Meldedaten}
\konkurrenz{sec:datenschutz:weitergabe}
\abstimmung
Einige Stadtverwaltungen in in Rheinland-Pfalz bessern ihr Budget auf, indem sie die persönliche Daten ihrer Bürger weiterverkaufen. Kirchen, Parteien, die GEZ, Banken und noch viele weitere Unternehmen kommen auf diese Weise an Datensätze. Bürger, die dies nicht wollen, können dem Datenverkauf widersprechen (sogenanntes Opt-Out), was allerdings mit Aufwand verbunden ist und vielen Menschen nicht bekannt ist. In der Regel erfährt man nur auf konkrete Nachfrage von diesem Datenhandel und der Möglichkeit des Widerspruchs. Die Piratenpartei Rheinland-Pfalz hält dieses Vorgehen für nicht akzeptabel. Wir werden deshalb ein Opt-In-Verfahren einführen, bei dem der Bürger der Herausgabe seiner Daten bewusst zustimmen muss und explizit angibt, welchen Gruppen der Zugang zu seinen Daten gestattet werden soll; der Verkauf an andere Gruppen wird unterbunden.

\subsubsection{Behördliche Datenschutzbeauftragte}
\abstimmung
Neben dem Landesbauftragten für den Datenschutz sind die behördlichen Datenschutzbeauftragten ein wichtiges Organ um den Datenschutz im Land zu gewährleisten. Leider haben sie für diese verantwortungsvolle Aufgabe oft zu wenig Zeit zur Verfügung. Die behördlichen Datenschutzbeauftragten sollen sich nach unserer Auffassung Vollzeit um ihre Aufgabe kümmern können und in alle datenschutzrelevanten Vorhaben einbezogen werden. Außerdem wollen wir die Vernetzung und den Austausch zwischen Landes- und behördlichen Datenschutzbeauftragten fördern.

\subsubsection{Auskunftsrecht gegenüber der Verwaltung}
\abstimmung
Jeder Bürger hat ein Recht auf Auskünft über die zu seiner Person gespeicherten Daten. Dieses Recht gilt auch gegenüber der Verwaltung. Trotzdem kommt es immer wieder vor, dass der Staat aus einem vermeintlichen Geheimhaltungsinteresse dieses Recht untergräbt. Die Piratenpartei wird durchsetzen, dass alle Bürger auch gegenüber der Landesverwaltung einen durchsetzbaren und unentgeltlichen Anspruch auf Selbstauskunft und gegebenenfalls auf Korrektur, Sperrung oder Löschung von unrichtigen oder unrechtmäßig gespeicherten Daten haben.

\subsubsection{Datenschutz bei der Gesetzgebung}
\abstimmung
Datenschutz ist mehr als ein politisches Thema. Die Verarbeitung persönlicher Daten durchdringt heute alle gesellschaftlichen Bereiche. Bei fast allen Gesetzen spielen persönliche Daten der Bürger eine Rolle. Die Piratenpartei wird deshalb sicherstellen, dass der Datenschutz in allen Bereichen der Gesetzgebung mit einbezogen und von vorne herein geachtet wird.
 
\newpage
\wahlprogramm{Datenherausgabe durch Bürgerämter nur nach Zustimmung}\label{sec:datenschutz:weitergabe}
\antrag{KV Trier/Trier-Saarburg}\konkurrenz{datenschutz:landesebene} - Meldedaten \version{22:51 18. Jun. 2010}
\abstimmung
Eine Weitergabe von Informationen über Bürger ohne deren Einwilligung lehnen wir ab.

Privatpersonen, Firmen, Kirchen, Parteien und andere Einrichtungen fordern von Bürgerämtern gegen geringe Gebühren Daten über Bürger ohne deren Einwilligung an, um diese zu privaten oder kommerziellen Zwecken zu verwenden. Diese Praxis widerspricht dem Grundrecht auf Informationelle Selbstbestimmung. Stattdessen muss in Zukunft sichergestellt sein, dass die Erlaubnis der Bürger eingeholt wurde, bevor Informationen über sie herausgegeben werden. Wurde diese Erlaubnis erteilt, soll der Bürger auf Anfrage Informationen über die getätigten Abfragen erhalten und seine Erlaubnis jederzeit widerrufen können.

\subsection*{Datenschutz in der Wirtschaft}
\wahlprogramm{Datenschutz in der Wirtschaft}
\antrag{Unglow}\version{22:51 18. Jun. 2010}
\abstimmung
\subsubsection{Modul 1}
Die bestehenden Datenschutzgesetze können den Datenschutz in der digitalen und vernetzten Welt des 21. Jahrhunderts nicht mehr gewährleisten. Datenskandale häufen sich, Firmen spionieren ihre Mitarbeiter aus, sensible Kundendaten gelangen in die Hände von Kriminellen. Der Datenhandel blüht: Permanent werden persönliche Daten von Millionen von Bundesbürgern gehandelt, ohne dass der Staat gesetzliche Rahmenbedingungen schafft, die jedem Bürger ermöglichen, an der Gesellschaft des 21. Jahrhunderts teilzuhaben, ohne zum gläsernen Bürger zu werden.
 
\wahlprogramm{Betriebliche Datenschutzbeauftragte}\label{datenschutz:beauftragte}
\antrag{Unglow}\version{22:51 18. Jun. 2010}
\subsubsection{Modul 1}
\abstimmung
Viele Unternehmen in Rheinland-Pfalz haben derzeit keinen betrieblichen Datenschutzbeauftragten eingesetzt, obwohl sie dazu verpflichtet sind. Obwohl der Landesdatenschutzbeauftragte dies bemängelt gibt es erhebliche Defizite. Die Sanktionen müssen in diesem Bereich verstärkt werden, sodass der Datenschutz als Sparmaßnahme für Unternehmen nicht mehr in Frage kommt.

\subsubsection{Modul 2}
\abstimmung
Darüber hinaus wollen wir den Kündigungsschutz der betrieblichen Datenschutzbeauftragten stärken und ihnen Fort- und Weiterbildungsmaßnahmen ermöglichen. Die Einsatzbereitschaft und die Durchsetzungsfähigkeit der betrieblichen Datenschutzbeauftragten trägt oftmals mehr zum Datenschutzniveau eines Unternehmens bei als gesetzliche Vorgaben. Die betrieblichen Datenschutzbeauftragten müssen deshalb in die Lage versetzt werden, das Datenschutzbewusstsein in der Wirtschaft zu stärken und wirksam durchzusetzen.
 
\wahlprogramm{Schutz von Unternehmensdatenbanken}\label{datenschutz:dbs}
\antrag{Unglow}\version{22:51 18. Jun. 2010}
\subsubsection{Modul 1}
\abstimmung
Der Landesdatenschutzbeauftragte bemängelt den Schutz von Unternehmensdaten als vielfach unzureichend. Die mangelhafte Absicherung und Zugriffskontrolle erleichtert es, Daten illegal auszulesen und weiterzugeben. Die vohandenen Strafbestimmungen reichen aus Sicht der PIRATEN offenbar nicht aus, um den wirksamen Schutz von Kundendaten zu gewährleisten. Die missbräuchliche Verwendung von Daten im Rahmen eingeräumter Zugriffsrechte muss ebenso unter Strafe gestellt werden wie der unbefugte Zugriff durch Dritte.

\subsubsection{Modul 2}
\abstimmung
Um Datenmissbrauch und aufdecken zu können muss protokolliert werden, wer wann welche personenbezogenendaten in welcher Weise verarbeitet hat. Diese Daten müssen beweissicher gespeichert und wiederum gegen unbefugten Zugriff und Manipulationen gesichert sein. Die Protokolldaten unterliegen ihrerseits einer strikten Zweckbindung und dürfen nicht für die Verhaltens- und Leistungskontrolle von Angestellten verwendet werden.
 
\wahlprogramm{Datenschutz der Arbeitnehmer}\label{datenschutz:arbeitnehmer}
\antrag{Unglow}\version{22:51 18. Jun. 2010}
\subsubsection{Modul 1}
\abstimmung
Im Arbeitsverhältnis werden zahlreiche persönliche und hochsensible Daten über Beschäftigte gesammelt und verarbeitet. Durch Informationstechnik können diese Daten fast ohne Aufwand zusammengeführt uns ausgewertet werden. Dies bietet Arbeitgebern neue, bedenkliche Kontroll- und Überwachungsmöglichkeiten, die den Persönlichkeitsrechten der Arbeitnehmer entgegenstehen. Der Gesetzgeber hat bislang versäumt, dieser Entwicklung Rechnung zu tragen. Die Piratenpartei will diesen Missstand beseitigen und ein Arbeitnehmerdatenschutzgesetz verabschieden.

\subsubsection{Modul 2}
\abstimmung
Konkret werden wir u.a.:
\begin{itemize}
\item klar festlegen, welche Daten Unternehmen und öffentliche Stellen im Rahmen des Einstellungsverfahrens und des Arbeitsverhältnisses sammeln und verarbeiten dürfen
\item anlass- und verdachtslose Abgleiche von Personaldaten unterbinden
\item die Verhaltens- und Leistungskontrolle von Arbeitnehmern strikt begrenzen
\item die Achtung der Persönlichkeitsrechte beim Einsatz von Videoüberwachung, Ortungssystemen und sonstigen Überwachungssystemen gewährleisten
\item den Beschäftigten umfassende Auskunfts-, Benachrichtigungs-, Widerrufs- und Löschrechte einräumen
\item den Schutz der in Deutschland tätigen Arbeitnehmer internationaler Unternehmen durchsetzen
\item die Kontrolle durch betriebliche Datenschutzbeauftragte sicherstellen
\item wirksame Sanktionen bei Verstößen gegen den Arbeitnehmerdatenschutz einführen
\end{itemize}
 
\wahlprogramm{Datenhandel unterbinden}
\antrag{Unglow}\version{22:51 18. Jun. 2010}
\subsubsection{Alternative A}
\abstimmung
Die Piratenpartei wird den Handel mit personenbezogenen Daten verbieten.

\abstimmung{Alternative B}
Die Piratenpartei wird den Handel mit personenbezogenen Daten einschränken. Wir werden den Handel mit Daten nur nach ausdrücklicher Zustimmung des Betroffenen erlauben. Diese Zustimmung muss unabhängig von sonstigen vertraglichen Vereinbarungen sein und darf auf letztere keinen Einfluss haben. Insbesondere darf ein Vertragsabschluss nicht von der Zustimmung zum Datenhandel abhängig gemacht werden.
 
\newpage
\wahlprogramm{Datensparsamkeit in Unternehmen}
\antrag{Unglow}\version{22:51 18. Jun. 2010}
\subsubsection{Modul 1}
\abstimmung
Wir wollen Unternehmen zur Datensparsamkeit verpflichten. Es darf nicht sein, dass die Angabe von nicht benötigten private Daten zur Voraussetzung werden, um Bestellungen zu tätigen, Software zu benutzen oder Vereinen und Gemeinschaften beizutreten. Zudem fordern wir Opt-In Lösungen, bei denen jeder der weiteren Verwendung seiner Daten oder auch zum Beispiel dem Erhalt von Werbung ausdrücklich zustimmen muss und nicht nur nachträglich widersprechen kann.
 
\wahlprogramm{Illegale Entsorgung von Daten}
\antrag{Unglow}\version{22:51 18. Jun. 2010}
\subsubsection{Modul 1}
\abstimmung
So wie die illegale Müllentsorgung unsere Umwelt schädigt, ist die illegale Entsorgung von Daten eine Bedrohung für die Privatsphäre von Kunden, Patienten, Arbeitnehmern und vielen weiteren Bürgern.

Um die ordnungsgemäße Vernichtung von privaten Daten bei der Entsorgung zu gewährleisten, werden wir dem Landesdatenschutzbeauftragten mehr Kontrollbefugnisse in diesem Bereich zusprechen und die Strafen für illegale Datenentsorgung erhöhen. Dies werden wir mit einer Aufklärungskampagne zur korrekten Entsorgung von Daten verbinden.
 
\subsection*{Sicherheitspolitik unter Achtung der Bürgerrechte}
\wahlprogramm{Sicherheitspolitik}
\antrag{Unglow}\version{22:51 18. Jun. 2010}
\subsubsection{Modul 1}
\abstimmung
Mit einer erschreckenden Geschwindigkeit wurde das Recht auf Privatsphäre in den letzten Jahren zu Gunsten einer unwirksamen Sicherheitspolitik eingeschränkt.

Systeme und Methoden, die der Staat gegen seine Bürger einsetzen kann, müssen der ständigen Bewertung und genauen Prüfung durch gewählte Mandatsträger unterliegen. Wenn die Regierung Bürger beobachtet, ohne dass sie eines Verbrechens verdächtig sind, ist dies eine fundamental inakzeptable Verletzung des Bürgerrechts auf Privatsphäre.

\subsubsection{Modul 2}
\abstimmung
Die pauschale Verdächtigung und anlasslose Überwachung aller Bürger hat generell zu unterbleiben. Eine als 'präventive Strafverfolgung' verschleierte Abschaffung der Unschuldsvermutung lehnen wir unbedingt ab.
 
\wahlprogramm{Vertrauliche Kommunikation}
\antrag{Unglow}\version{22:51 18. Jun. 2010}
\subsubsection{Modul 1}
\abstimmung
Das Briefgeheimnis soll erweitert werden zu einem generellen Kommunikationsgeheimnis, das die grundgesetzlich geschützte Privatheit und Integrität von Kommunikation auch in elektronischen Medien wie dem Internet garantiert. Zugriff auf die Kommunikationsmittel oder die Überwachung eines Bürgers darf Ermittlungsbehörden nur im Falle eines begründeten und konkreten Tatverdachtes erlaubt werden, dass dieser Bürger ein Verbrechen plant oder begangen hat. In jedem Fall ist ein richterlicher Beschluss erforderlich. In allen anderen Fällen muss der Staat annehmen, seine Bürger seien unschuldig. Diesem Kommunikationsgeheimnis muss ein starker gesetzlicher Schutz gegeben werden, da Regierungen wiederholt gezeigt haben, dass sie bei sensiblen Informationen nicht vertrauenswürdig sind.

\subsubsection{Modul 2}
\abstimmung
Speziell eine verdachtsunabhängige Vorratsdatenspeicherung von Kommunikationsdaten widerspricht nicht nur der Unschuldsvermutung, sondern auch allen Prinzipien einer freiheitlich demokratischen Gesellschaft. Der vorherrschende Kontrollwahn stellt eine weitaus ernsthaftere und langfristigere Bedrohung unserer Gesellschaft dar als der internationale Terrorismus und erzeugt ein Klima des Misstrauens und der Angst. Flächendeckende staatliche Überwachung, fragwürdige Rasterfahndungen und zentrale Datenbanken mit unbewiesenen Verdächtigungen (Anti-Terror-Datei) sind Mittel, deren Einsatz wir grundsätzlich ablehnen.
 
\wahlprogramm{Datensparsame Sicherheitspolitik}
\antrag{Unglow}\version{22:51 18. Jun. 2010}
\subsubsection{Modul 1}
\abstimmung
Der Staat sammelt selbst hemmungslos Daten über seine Bürger und vernetzt die gesammelten Daten zunehmend miteinander, was zu einem gläsernen Bürger führt. Mit der Steuer-ID, die jeder Bundesbürger mit seiner Geburt erhält und die erst 20 Jahre nach seinem Tod gelöst wird, sind die Daten der Bürger nun einfacher zu verarbeiten und können so besser vernetzt werden, was das Erstellen eines umfangreichen Profils zu jedem Bürger erleichtert. Mit Hilfe der Vorratsdatenspeicherung sollten die Kommunikationsdaten eines jeden Deutschen, der per Telefon oder Internet kommuniziert, überwacht und gespeichert werden, ohne dass ein Verdacht besteht. Diese Maßnahme wurde vorübergehend sogar vom Bundesverfassungsgericht gekippt. Mit der ELENA-Datenbank werden mittlerweile alle Arbeitnehmerdaten zentral erfasst. In den Bundesländern wird auch über neue Datenbanken, wie zum Beispiel die Einführung einer Schüler-ID, diskutiert. Gesetze wie die Vorratsdatenspeicherung, bei der Bürger verdachtsunabhängig überwacht werden, lehnt die Piratenpartei grundsätzlich ab. Auch andere Maßnahmen der verdachtsunabhängigen Massenüberwachung wie Kennzeichenscanner lehnen wir ab. Den Einsatz solcher Maßnahmen in Rheinland-Pfalz wird die Piratenpartei verhindern.
 
\wahlprogramm{Sicherheitspolitik}
\antrag{KV Trier/Trier-Saarburg}\version{22:51 18. Jun. 2010}
\subsubsection{Biometrische Daten}
\abstimmung
Wir lehnen die Erfassung biometrischer Daten ohne Anfangsverdacht sowie deren Speicherung ohne nachgewiesene Straftat kategorisch ab.

\subsubsection{Keine automatisierte Kennzeichenerfassung}
\abstimmung
Obwohl das Bundesverfassungsgericht eindeutig klargestellt hat, dass eine verdachtsunabhängige, flächendeckende, automatisierte Kennzeichenerfassung zum Abgleich mit Fahndungsdaten nicht mit dem Grundgesetz vereinbar ist, wird dieses erneut diskutiert. Einen solchen Eingriff in die Persönlichkeitsrechte lehnen wir entschieden ab.

Wir gehen sogar weiter als das Verfassungsgericht: Auch ein stichprobenartiger Abgleich ist für uns nicht akzeptabel.
 
\wahlprogramm{Echte Sicherheitspolitik auf Basis von Fakten}
\antrag{Unglow}\version{22:51 18. Jun. 2010}
\subsubsection{Modul 1}
\abstimmung
Die Bekämpfung der Kriminalität ist eine wichtige staatliche Aufgabe. Sie ist nach unserer Überzeugung nur durch eine intelligente, rationale und evidenzbasierte Sicherheitspolitik auf der Grundlage wissenschaftlicher Erkenntnisse zu gewährleisten. Um sinnvolle Sicherheitsmaßnahmen zu fördern und schädliche Maßnahmen beenden zu können, wollen wir alle bestehenden Befugnisse und Programme der Sicherheitsbehörden systematisch und nach wissenschaftlichen Kriterien überprüfen auf Wirksamkeit, Kosten, schädliche Nebenwirkungen, auf Alternativen und auf ihre Vereinbarkeit mit den Menschen- und Bürgerrechten.

\subsubsection{Modul 2}
\abstimmung
Wir wollen, dass künftig jeder Vorschlag für neue Sicherheitsmaßnahmen noch im Entwurfsstadium von der Europäischen Grundrechteagentur oder einer entsprechenden deutschen Einrichtung auf diese Kriterien hin begutachtet wird. Nur durch einen solchen "Gesetzes-TÜV" kann weiteren verfassungswidrigen Angriffen auf unsere Grundrechte frühzeitig entgegen gewirkt werden. Der Grundrechteagentur müssen dafür alle nötigen finanziellen und personellen Ressourcen zur Verfügung gestellt werden.

\subsubsection{Modul 3}
\abstimmung
Um den fortschreitenden Abbau der Bürgerrechte seit 2001 zu stoppen, fordern wir ein Moratorium für weitere Grundrechtseingriffe im Namen der inneren Sicherheit ein, solange nicht die systematische Überprüfung der bestehenden Befugnisse abgeschlossen ist.

\subsubsection{Modul 4}
\abstimmung
Zur Gewährleistung der Freiheitsrechte und zur Sicherung der Effektivität von Gefahrenabwehr und Strafverfolgung treten wir dafür ein, dass eine staatliche Informationssammlung, Kontrolle und Überwachung künftig nur noch gezielt bei Personen erfolgt, die einer Straftat konkret verdächtigt sind. Zum Schutz unserer offenen Gesellschaft und im Interesse einer effizienten Sicherheitspolitik wollen wir auf anlasslose, massenhafte, automatisierte Datenerhebungen, Datenabgleichungen und Datenspeicherungen verzichten. In einem freiheitlichen Land ist eine derart breite Erfassung beliebiger Personen ohne Anlass und Verdacht inakzeptabel.

\subsubsection{Modul 5}
\abstimmung
Die Sicherheitsforschung aus Steuergeldern wollen wir demokratisieren und an den Bedürfnissen und Rechten der Bürgerinnen und Bürger ausrichten. In beratenden Gremien sollen künftig neben Verwaltungs- und Industrievertretern in gleicher Zahl auch Volksvertreter sämtlicher Fraktionen, Kriminologen, Opferverbände und Nichtregierungsorganisationen zum Schutz der Freiheitsrechte und Privatsphäre vertreten sein. Eine Entscheidung über die Ausschreibung eines Projekts soll erst getroffen werden, wenn eine öffentliche Untersuchung über die Auswirkungen des jeweiligen Forschungsziels auf unsere Grundrechte (impact assessment) vorliegt.

\subsubsection{Modul 6}
\abstimmung
Die Entwicklung von Technologien zur verstärkten Überwachung, Erfassung und Kontrolle von Bürgerinnen und Bürgern lehnen wir ab. Stattdessen muss die Sicherheitsforschung auf sämtliche Optionen zur Kriminal- und Unglücksverhütung erstreckt werden und eine unabhängige Untersuchung von Wirksamkeit, Kosten, schädlichen Nebenwirkungen und Alternativen zu den einzelnen Vorschlägen zum Gegenstand haben.

\subsubsection{Modul 7}
\abstimmung
Weil auch die gefühlte Sicherheit eine wichtige Voraussetzung für unser Wohlbefinden ist, wollen wir zudem erforschen lassen, wie das öffentliche Sicherheitsbewusstsein gestärkt und wie verzerrten Einschätzungen und Darstellungen der Sicherheitslage entgegen gewirkt werden kann.
 
\wahlprogramm{Polizei- und Ordnungsbehördengesetz}
\antrag{Unglow}\version{22:51 18. Jun. 2010}
\subsubsection{Modul 1}
\abstimmung
Wir lehnen jegliche Versuche ab, durch eine Neufassung des Polizei- und Ordnungsbehördengesetzes Online-Durchsuchungen zu erlauben, Rasterfahndungen zu legitimieren oder Befugnisse zur Telekommunikationsüberwachung zu verschärfen. Stattdessen setzen wir uns für eine wissenschaftliche Evaluation aller bestehenden Sicherheitsbefugnisse ein.
 
\wahlprogramm{Eindeutige, gut lesbare Kennzeichnung von Polizisten}\label{datenschutz:polizei}
\antrag{Piraten aus RLP}\konkurrenz{datenschutz:eindeutig}\version{22:51 18. Jun. 2010}
\abstimmung
Polizisten sollen bei Einsätzen in Gruppen eine eindeutige, gut lesbare Identifikationsnummer tragen, um Übergriffe durch Polizisten nachvollziehen und aufklären zu können. Für den Fall unverhältnismäßiger Gewaltanwendung durch Polizisten oder anderer gesetzeswidriger Handlungen muss sichergestellt werden, dass eine spätere Identifikation von Sicherheitskräften möglich ist.
 
\wahlprogramm{Eindeutige Kennzeichnung von Polizisten}
\antrag{KV Trier/Trier-Saarburg}\konkurrenz{datenschutz:polizei}\version{22:51 18. Jun. 2010}
\abstimmung
Bei geplanten Veranstaltungen wie Demonstrationen oder Einsätzen bei Sportereignissen sollen Polizisten eine eindeutige Identifikationsnummer tragen, um Übergriffe durch Polizisten nachvollziehen zu können.

Für den Fall unverhältnismäßiger Gewaltanwendung durch Polizisten oder anderer gesetzeswidriger Handlungen muss sichergestellt werden, dass eine spätere Identifikation von Sicherheitskräften möglich ist. Dabei ist das Gleichgewicht zwischen dem Schutz der Persönlichkeitsrechte und der Identifizierbarkeit der Polizisten zu wahren. Im Fall einer Anzeige soll erst durch einen richterlichen Beschluss die Feststellung der Identität erfolgen. Hierfür ist ein geeignetes und praktikables Verfahren zur Verteilung der Identifikationsnummern und deren Gestaltung in Zusammenarbeit mit der Polizei zu entwickeln.
 
\wahlprogramm{POLIS Datenbank}
\antrag{Unglow}\version{22:51 18. Jun. 2010}
\subsubsection{Modul 1}
\abstimmung
In der Vergangenheit kam es zu rechtswidring Zugriffen auf das polizeiliche Informationssystem POLIS. In dem System befinden sich Daten über alle Personen, die als Tatverdächtige auffällig geworden, aber nicht zwangsläufig schuldig sind. Da diese Daten in einem Rechtsstaat besonders schutzwürdig sind, werden wir dieses System grundlegend überprüfen lassen und sicherstellen, dass alle nötigen Vorkehrungen getroffen werden, um Datenmissbrauch zu verhindern.

\subsubsection{Modul 2}
\abstimmung
Im Rahmen dieser Überprüfung werden wir auch alle gespeicherten Daten hinsichtlich ihrer Notwendigkeit und Zweckmäßigkeit überprüfen. Daten die nicht unbedingt benötigt, oder anlasslos gespeichert werden sind unzulässig.
 
\newpage
\wahlprogramm{Videoüberwachung}
\antrag{Unglow}\version{22:51 18. Jun. 2010}
\subsubsection{Modul 1}
\abstimmung
Die flächendeckende Überwachung des öffentlichen Raums durch Videokameras oder andere Maßnahmen darf nicht zugelassen werden. Wir fordern ein allgemeines Verbot der Überwachung des öffentlichen Raums, von dem nur einzelne, richterlich angeordnete Ausnahmen zulässig sind.

\subsubsection{Modul 2}
\abstimmung
Die anlasslose und pauschale Videoüberwachung im öffentlichen Raum dient lediglich der gefühlten Sicherheit und dringt unverhältnismäßig in die Privatsphäre der Menschen ein. Wir werden stattdessen wirksame Maßnahmen durchsetzen. Wir lehnen jegliche Pläne zum Ausbau der Videoüberwachung zum Beispiel an Bushaltestellen oder Schulen strikt ab. Kameras tragen nicht zum Abbau sondern höchstens zur Verlagerung von Kriminalität bei und bieten Opfern keinen Schutz. Die Kosten für die Installation und die Überwachung der Kameras stehen in keiner Relation zum Nutzen. Eine Neuorientierung hin zu effektiven Lösungen wie besserer Straßenbeleuchtung und mehr Polizeistreifen ist dringend erforderlich und wird von uns vorangetrieben.

\subsubsection{Modul 3}
\abstimmung
Wir lehnen insbesondere den allgemeinen, präventiven, behördlichen Einsatz von Überwachungstechnologie während Demonstrationen ab, da dieser die Versammlungsfreiheit und freie Meinungsäußerung massiv einschränkt.
 
\wahlprogramm{Versammlungsfreiheit}
\antrag{Unglow}\version{22:51 18. Jun. 2010}
\subsubsection{Modul 1}
\abstimmung
Die Möglichkeit zur Organisation von und Teilnahme an Versammlungen ist ein wichtiges Grundrecht. In anderen Bundesländern wurde dieses Recht durch Änderungen am Versammlungsgesetz erheblich eingeschränkt. Jeglichen Plänen die Versammlungsfreiheit in Rheinland-Pfalz ebenfalls einzuschränken stellen wir uns entschieden entgegen.

\subsubsection{Modul 2}
\abstimmung
Immer häufiger wird seitens der Polizei im Vorfeld von und während Demonstrationen Kontrolle und Überwachungsdruck ausgeübt. Wir lehnen anlassunabhägige Kontrollen und Durchsuchungen von Menschen und Fahrzeugen entschieden ab und setzen uns für entsprechende gesetzliche Änderungen ein, die dies verbieten. Wir fordern Freiheit statt Angst und den Schutz der Menschen vor Einschüchterung durch den Staat bei Wahrnehmung ihrer Rechte.

\subsubsection{Modul 3}
\abstimmung
Die Überwachung von Demonstrationen mit Foto- oder Videokameras oder ähnlichen Instrumenten lehnen wir ab. Überwachung auf Demonstrationen gefährdet die Meinungs- und Versammlungsfreiheit und damit unsere Demokratie. Wir wollen das Versammlungsrecht stärken und verdachts- und anlassunabhängige Überwachungsmaßnahmen stärker kontrollieren und sanktionieren. Jede polizeiliche Überwachungsmaßnahme muss vollständig dokumentiert und begründet werden und dem Landesdatenschutzbeauftragten zur Kontrolle übermittelt werden.

\subsubsection{Modul 4}
\abstimmung
In der Vergangenheit kam es zu Situationen, in denen Polizisten auf Demonstrationen die Rechte von Bürgern missachtet haben. Bei Beschwerden gestaltete sich die Aufklärung als schwierig. Polizisten sollten gegen Kollegen ermitteln oder aussagen. Um solche Fälle zukünftig besser aufklären zu können, fordern wir die Einrichtung einer von der Polizei unabhängigen Beschwerdestelle. Diese muss auch das anonyme Melden von Fehlverhalten durch Kollegen möglich machen, die sich aktuell nicht trauen, Beobachtungen rechtswidrigen Verhaltens zur Anzeige zu bringen.
