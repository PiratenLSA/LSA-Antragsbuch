\section{Umwelt}

\wahlprogramm{Mehr Transparenz und Bürgerbeteiligung}
\antrag{Silberpappel}\version{03:13, 20. Jun. 2010}

\subsubsection{Mehr Transparenz und Bürgerbeteiligung}
\abstimmung
Viele der heutigen Umweltprobleme – vom Schrumpfen der Artenvielfalt bis zum Versagen der Atommülldeponierung – sind auch das Resultat einer Ohnmacht der Bürger gegenüber den Interessen stark mit dem Staat verflochtener Wirtschaftskräfte.

Daher fordern wir beim Thema Umwelt mehr Transparenz im Handeln von Regierungen und Unternehmen und eine stärkere Beteiligung der Bürger an politischen Entscheidungsprozessen.

Der freie und nutzerfreundliche Zugang zu Umweltinformationen ist eine wichtige Voraussetzung hierfür und muss weiter verbessert werden.
 
\wahlprogramm{Natur als Gemeingut}
\antrag{Silberpappel}\version{03:13, 20. Jun. 2010}

\subsubsection{Natur als Gemeingut}
\abstimmung
Die Natur ist ein Gemeingut.

\subsubsection{Natur als Gemeingut Fortsetzung}
\abstimmung
Sie gehört allen Menschen gleichermaßen.

\subsubsection{Pflicht zum Umwelt- und Naturschutz}
\abstimmung
Daraus ergibt sich die Pflicht zum Schutz von Natur und Umwelt, damit diese nicht durch übermäßige Nutzung durch einzelne oder Gruppen zerstört und damit der Allgemeinheit entzogen werden. Dies gilt auch generationsübergreifend.

\subsubsection{Ablehnung von Naturraum-Monopolisierung}
\abstimmung
Übermäßige Monopolisierung von Naturräumen wie z.B. Privatstrände oder großräumig eingezäunte Gebiete lehnen wir ab.

\subsubsection{Nutzung Alternative I}
\abstimmung
Naturschutz darf nicht zu einer generellen Trennung des Menschen von der Natur führen.

\subsubsection{Nutzung Alternative II}
\abstimmung
Naturschutz darf nicht dazu führen, dass der Mensch generell aus der Natur ''ausgesperrt'' wird.

\subsubsection{Freizeit-Nutzung}
\abstimmung
Die Nutzung von Naturräumen für Sport und Freizeit muss grundsätzlich möglich sein. Generelle Verbote von z.B. Mountainbiken, Geocaching oder Baden in Seen lehnen wir ab. Solche Verbote sollen nur zielgerichtet für einzelne Gebiete ausgesprochen werden, die eines besonderen Schutzes bedürfen.
 
\wahlprogramm{Nachhaltigkeit}
\antrag{Thomas Heinen}\version{03:13, 20. Jun. 2010}

\subsubsection{Nachhaltigkeit}
\abstimmung
Wir stehen für das Prinzip der Nachhaltigkeit. Darunter verstehen wir die Entwicklung einer zukunftsfähigen Gesellschaft, die natürliche Ressourcen so nutzt und bewahrt, dass diese auch den nachfolgenden Generationen zur Verfügung stehen und der Artenreichtum unseres Planeten dauerhaft erhalten bleibt. Hierzu ist ein bewusster und verantwortungsvoller Umgang mit den Ressourcen und ihre faire Verteilung erforderlich. Bei nachwachsenden Ressourcen müssen Verbrauch und Regeneration im Gleichgewicht sein. Bei nicht nachwachsenden Ressourcen wie Bodenschätzen ist die Einführung einer Kreislaufwirtschaft oberstes Ziel.
 
\wahlprogramm{Feinstaubbelastung}
\antrag{Thomas Heinen}\version{03:13, 20. Jun. 2010}

\subsubsection{Feinstaubbelastung}
Wir setzen uns für die Förderung des ÖPNV, für die Förderung der Schiene im Gütertransport, für regionale Wirtschaftskreisläufe ohne lange Transportwege und für neue Konzepte für den Individualverkehr ein. Im Interesse der Gesundheit aller Einwohner setzen wir uns somit für eine Verminderung der Feinstaubbelastung ein. Weitgehend wirkungslose Alibimaßnahmen wie die sogenannten Feinstaubplaketten und Umweltzonen lehnen wir dagegen ab.