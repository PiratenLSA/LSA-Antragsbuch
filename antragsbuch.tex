\documentclass[halfparskip,a4paper,10pt]{scrartcl}
\usepackage[utf8]{inputenc}
\usepackage[ngerman]{babel}  % to make umlauts work
\usepackage{ulem} % only needed for \sout (strike out)
\usepackage[left=1cm,right=1cm,top=1cm,bottom=2cm]{geometry} % margins
\usepackage{amsfonts}
\usepackage{graphicx}
\usepackage{color}
\usepackage[pdftex,a4paper=true, colorlinks=true, pdftitle={Antragsbuch LPT RLP 2010.2}, pdfsubject={Antragsbuch}, pdfauthor={Jochen Schaefer (JoSch)}, pdfpagemode=UseNone, pdfstartview=FitH,pdfhighlight={/N}, pdfversion=1.7]{hyperref}
\usepackage{titletoc}
\usepackage{titlesec}

\newcounter{abstimmung}
\setcounter{abstimmung}{0}
\newcommand{\antrag}[1]{{\sffamily{Antragssteller: #1}}}
\newcommand{\abstimmung}{\begin{Form}%
\CheckBox[name=ja\arabic{abstimmung}]{Ja}\ \ %
\CheckBox[name=nein\arabic{abstimmung}]{Nein}\ \  %
\CheckBox[name=ent\arabic{abstimmung}]{Enthaltung}\\\medskip Notizen:\\%
\TextField[bordercolor={1 0 0},name=line\arabic{abstimmung},width=\linewidth,multiline=true,height=7em]{ }\smallskip%
\end{Form}%
\stepcounter{abstimmung}\\}

\newcommand{\konkurrenz}[1]{ konkurrierend zu \autoref{#1} }
\newcommand{\zusatz}[1]{ ergänzend zu \autoref{#1} }
\newcommand{\version}[1]{ Wiki-Version: {#1} }

%% Spezielle Zähler
\newcounter{satzung}[part]
\newcommand{\satzung}[1]{\refstepcounter{satzung}\addcontentsline{toc}{subsection}{SÄA \numberline{\arabic{satzung}} #1}\textbf{\Large SÄA \arabic{satzung} - #1}\medskip\\}
\newcommand{\satzungname}{SÄA}

\newcounter{wahlprogramm}[part]
\newcommand{\wahlprogramm}[1]{\refstepcounter{wahlprogramm}\setcounter{subsubsection}{0}%
\addcontentsline{toc}{subsection}{WPA \numberline{\arabic{section}.\arabic{wahlprogramm}} #1}%
\medskip\textbf{\large WPA \arabic{section}.\arabic{wahlprogramm} - #1}\smallskip\\}
\newcommand{\wahlprogrammname}{WPA}

\makeatletter
\@addtoreset{section}{part}
\@addtoreset{wahlprogramm}{section}
\makeatother
\renewcommand{\thesubsubsection}{\arabic{subsubsection}}

\title{Antragsbuch für die Landesmitgliederversammlung 2011.1}
\author{Piratenpartei Sachsen-Anhalt}
\date{Stand: \today}

\begin{document}
\titlespacing*{\section}{0pt}{0pt}{0pt}
\titlespacing*{\subsection}{0pt}{0pt}{0pt}
\titlespacing*{\subsubsection}{0pt}{0pt}{0pt}
\titlespacing*{\paragraph}{0pt}{0pt}{0pt}

\setcounter{tocdepth}{2}
\maketitle
\tableofcontents

\part{Satzungsänderungen}
\section{Eingereichte Satzungsänderunganträge}
\satzung{§ 2 (2) - Mitgliedschaft (Änderung)}
\antrag{Stephan Schurig}

\paragraph{Antragstext}:

Der Landesparteitag möge beschließen folgende Änderung des \href{http://wiki.piratenpartei.de/LSA:Satzung#.C2.A7_2_-_Mitgliedschaft}{§2 (2)} der Satzung des LV Sachsen-Anhalt zu vollziehen:

\einruecken{(2) Der Landesverband und jede \textbf{untergeordnete} Gliederung führt ein Piratenverzeichnis auf entsprechender Ebene.}

\paragraph{Alte Fassung}:

\einruecken{(2) Der Landesverband und jede \textbf{niedere} Gliederung führt ein Piratenverzeichnis auf entsprechender Ebene.}

\paragraph{Begründung}:

Wertneutralere Formulierung. 

% -----

\satzung{Ladungsmodalitäten - Anpassung §9b (2) (Der Landesparteitag) an Bundessatzung}
\antrag{René Emcke}

\paragraph{Antragstext}:

Der Landesparteitag möge beschließen, \href{http://wiki.piratenpartei.de/LSA:Satzung#.C2.A7_9b_-_Der_Landesparteitag}{§9b (2) (Der Landesparteitag)} der Landessatzung wie folgt zu ändern:

\einruecken{(2) Der Landesparteitag tagt mindestens einmal jährlich. Die Einberufung erfolgt aufgrund eines Vorstandsbeschlusses. Wenn ein Zehntel der Piraten, mindestens aber zehn Piraten es beim Vorstand beantragen, muss dieser binnen 2 Wochen einen Parteitag einberufen. Der Vorstand lädt jedes Mitglied schriftlich \textbf{per Brief oder Fax} mindestens 4 Wochen vorher ein. \textbf{Es gilt per Brief das Datum des Poststempels, per Fax der mit Datum und Unterschrift vom Versender bestätigte Sendebericht. Ist eine E-Mail-Adresse bekannt, so kann vorher per E-Mail eingeladen werden. Die reguläre Einladung kann entfallen, wenn das Mitglied den Empfang der E-Mail spätestens 4 Wochen vor dem Landesparteitag bestätigt hat.} Die Einladung hat Angaben zum Tagungsort, Tagungsbeginn, vorläufiger Tagesordnung und der Angabe, wo weitere, aktuelle Veröffentlichungen gemacht werden, zu enthalten. Spätestens 1 Wochen vor dem Parteitag sind die Tagesordnung in aktueller Fassung, die geplante Tagungsdauer und alle bis dahin dem Vorstand eingereichten Anträge im Wortlaut zu veröffentlichen.}

\paragraph{Alte Fassung}:

\einruecken{(2) Der Landesparteitag tagt mindestens einmal jährlich. Die Einberufung erfolgt aufgrund eines Vorstandsbeschlusses. Wenn ein Zehntel der Piraten, mindestens aber zehn Piraten es beim Vorstand beantragen, muss dieser binnen 2 Wochen einen Parteitag einberufen. Der Vorstand lädt jedes Mitglied schriftlich \textbf{(Brief, Email oder Fax)} mindestens 4 Wochen vorher ein. Die Einladung hat Angaben zum Tagungsort, Tagungsbeginn, vorläufiger Tagesordnung und der Angabe, wo weitere, aktuelle Veröffentlichungen gemacht werden, zu enthalten. Spätestens 1 Wochen vor dem Parteitag sind die Tagesordnung in aktueller Fassung, die geplante Tagungsdauer und alle bis dahin dem Vorstand eingereichten Anträge im Wortlaut zu veröffentlichen.}

\paragraph{Begründung}:

\begin{itemize}
\item Anpassung der Landessatzung an die \href{https://wiki.piratenpartei.de/Satzung#.C2.A7_9b_-_Der_Bundesparteitag}{Bundessatzung} in Bezug auf Ladungsmöglichkeiten
\item Verankerung der bereits praktizierten Einladungsmöglichkeit per Email in der Satzung (Rechtssicherheit)
\item Verankerung von Definitionen über fristgerecht erfolgte Zustellung (Rechtssicherheit)
\end{itemize}

% -----

\satzung{Landesfinanzordnung}
\antrag{Alexander Zinser}

\paragraph{Antragstext}:

Der Landesparteitag möge beschließen die \href{http://wiki.piratenpartei.de/LSA:Satzung#Abschnitt_B:_Finanzordnung}{derzeitge Landesfinanzordnung} durch folgenden Text zu ersetzen: 

\einruecken{§1 Allgemeines\\
(1) Es gilt im Wesentlichen die Bundesfinanzordnung.\\
(2) Der Vorstand ist dem Vier-Augen-Prinzip verpflichtet. Jede Transaktion muß von zwei Vorstandsmitgliedern unterzeichnet werden, wobei der übrige Vorstand unverzüglich in Kenntnis zu setzen ist, oder durch einen Vorstandsbeschluss gedeckt sein.\\
(3) Der Schatzmeister des Landesverbandes kann gegen Transaktionen sein Veto einlegen, wenn es die Finanzlage erfordert.\\
(4) Der Schatzmeister des Landesverbandes kann von untergeordneten Gliederungen alle für den Rechenschaftsbericht notwendigen Daten einfordern. Sollte dies nicht möglich sein, hat er zeitnah Ordnungsmaßnahmen zu beantragen.\\\\
§2 Mitgliedsbeitrag\\
(1) Der Mitgliedsbeitrag wird zum Jahresbeginn vollständig an die für das Mitglied zuständige Gliederung überwiesen. Der Mitgliedsbeitrag wird von der für das Mitglied zuständigen Gliederung quartalsweise über die Umlage an die höheren Gliederungen überwiesen.\\
(2) Der Mitgliedsbeitrag wird abzüglich des Bundesanteils wie folgt aufgeteilt: 50\% an den Landesverband, 25\% an den zuständigen Kreisverband und 25\% an den zuständigen Ortsverband. Sofern eine Gliederung nicht existiert, gehen die Gelder an die jeweils übergeordnete Gliederung.\\\\
§3 Virtuelle Kreisverbände\\
(1) Basierend auf den politischen Grenzen werden für alle Kreise ohne existierenden Kreisverband Konten in der Buchhaltung geschaffen (virtuelle Kreisverbände). Auf diese Konten werden alle Finanzen gebucht, die einem tatsächlich existierenden Kreisverband zustünden.}

\paragraph{Alte Fassung}:

\einruecken{1. Es gilt im Wesentlichen die Bundesfinanzordnung.\\
2. Der Vorstand ist dem Vier-Augen-Prinzip verpflichtet. Jede Transaktion muß von zwei Vorstandsmitgliedern unterzeichnet werden, wobei der übrige Vorstand unverzüglich in Kenntnis zu setzen ist, oder durch einen Vorstandsbeschluss gedeckt sein.\\
3. Der Schatzmeister des Landesverbandes kann gegen Transaktionen sein Veto einlegen, wenn es die Finanzlage erfordert.\\
4. Der Schatzmeister des Landesverbandes kann von untergeordneten Gliederungen alle für den Rechenschaftsbericht notwendigen Daten einfordern. Sollte dies nicht möglich sein, hat er zeitnah Ordnungsmaßnahmen zu beantragen.}

\paragraph{Begründung}:

Erweiterung der derzeitigen Landesfinanzordnung. §1 ist die alte Finanzordnung, §2 regelt Umlage der Mitgliedsbeiträge (50\% Land, 25\% Kreis, 25\% Ort oder Gesamtschlüssel mit Bund: 40\% Bund, 30\% Land, 15\% Kreis, 15\% Ort), §3 Unterkonten beim LV für Kreise ohne KV.

% -----

\satzung{Umlage PartFin}\label{satzung:partfin1}
\antrag{Alexander Zinser}
\begin{itemize}
\item \konkurrenz{satzung:partfin2}
\item \konkurrenz{satzung:partfin3}
\end{itemize}

\paragraph{Antragstext}:

Der Landesparteitag möge beschließen, folgenden Abschnitt in die \href{http://wiki.piratenpartei.de/LSA:Satzung#Abschnitt_B:_Finanzordnung}{Landesfinanzordnung} aufzunehmen: 

\einruecken{§XX - Umlage Parteienfinanzierung\\
Die Gelder aus der Parteienfinanzierung werden auf Landesebene nach folgendem Schlüssel umgelegt:\\
(1) 10\% der Parteienfinanzierung verbleibt bis zur nächsten Abschlagszahlung, mindestens jedoch für ein Jahr, als Rücklage beim Landesverband. Aufgelöste Rücklagen werden zur aktuellen Abschlagszahlung addiert und entsprechend diesem Schüssel umgelegt.\\
(2) Vom verbleibenden Betrag gehen 50\%, mindestens jedoch ein Sockelbetrag von 3600 EUR per anno, an den Landesverband. Der Restbetrag geht an die untergliederten Kreisverbände.\\
(3) Die Verteilung des Anteils der Kreisverbände erfolgt zu je einem Drittel nach Sockel, nach Einwohner und nach Fläche der Kreisverbände.\\
(3a) Der Sockelanteil eines Kreisverbandes berechnet sich aus dem Verhältnis Anzahl der politischen Kreise des Kreisverbandes zu Anzahl der politischen Kreise des Landes.\\
(3b) Der Anteil nach Einwohner berechnet sich aus dem Verhältnis Einwohnerzahl des Gebietes des Kreisverbandes zu Einwohnerzahl des Landes.\\
(3c) Der Anteil nach Fläche berechnet sich aus dem Verhältnis Fläche des Gebietes des Kreisverbandes zu Fläche des Landes.\\
(4) Sofern in einem politischen Kreis noch kein Kreisverband existiert, wird der entsprechende Betrag gegen ein virtuelles Unterkonto des Landesverbandes gebucht. Von diesem Unterkonto sollen primär Aktionen in dem jeweiligen Gebiet finanziert werden. Der Landesvorstand ist berechtigt diesen Betrag begründet anderweitig zu verwenden.\\
(5) Anspruch auf Auszahlung aus der Parteienfinanzierung besteht ab dem Monat der Gründung eines Kreisverbandes.}

% -----

\satzung{Umlage PartFin BaWü}\label{satzung:partfin2}
\antrag{Alexander Zinser}
\begin{itemize}
\item \konkurrenz{satzung:partfin1}
\item \konkurrenz{satzung:partfin3}
\end{itemize}

\paragraph{Antragstext}:

Der LPT möge beschließen, den folgenden Text an geeigneter Stelle in die \href{http://wiki.piratenpartei.de/LSA:Satzung#Abschnitt_B:_Finanzordnung}{Landesfinanzordnung} aufzunehmen:

\einruecken{§X Parteienfinanzierung\\
(1) Die Parteienfinanzierung für den Landesverband und all seine Untergliederungen werden nach folgendem Schlüssel unter den Gliederungen verteilt.\\
(2) Dem Landesverband stehen 50\%, den Kreisverbänden 25\% und den Ortsverbänden 25\% der Parteienfinanzierung zu.\\
(3) Unter den Gliederungen gleicher Ebene wird die Parteienfinanzierung durch die Anzahl der stimmberechtigten Mitglieder des Landesverbandes geteilt. Anschließend wird mit der Anzahl der stimmberechtigten Mitglieder der betroffenen Gliederung multipliziert. Die Anzahl der stimmberechtigten Mitglieder jeder Gliederung wird durch den Landesvorstand festgestellt. Stichtag ist jeweils der 31.12. des Vorjahres.\\
(4) Sofern eine Gliederung nicht existiert, gehen die Gelder an die jeweils übergeordnete Gliederung.\\
(5) Der Landesverband verteilt die Parteienfinanzierung quartalsweise auf seine Gliederungen.}

\paragraph{Begründung}:

Umlageschlüssel wesentlich einfacher als LQFB-Sieger (KISS-Prinzip, wa?)

% -----

\satzung{Umlage PartFin BaWü Sicher}\label{satzung:partfin3}
\antrag{Alexander Zinser}
\begin{itemize}
\item \konkurrenz{satzung:partfin1}
\item \konkurrenz{satzung:partfin2}
\end{itemize}

\paragraph{Antragstext}:

Der LPT möge beschließen, den folgenden Text an geeigneter Stelle in die \href{http://wiki.piratenpartei.de/LSA:Satzung#Abschnitt_B:_Finanzordnung}{Landesfinanzordnung} aufzunehmen:

\einruecken{§X Parteienfinanzierung\\
(1) Die Parteienfinanzierung für den Landesverband und all seine Untergliederungen werden nach folgendem Schlüssel unter den Gliederungen verteilt.\\ 
(2) Dem Landesverband stehen 50\%, den Kreisverbänden 25\% und den Ortsverbänden 25\% der Parteienfinanzierung zu.\\
(3) Unter den Gliederungen gleicher Ebene wird die Parteienfinanzierung durch die Anzahl der stimmberechtigten Mitglieder des Landesverbandes geteilt. Anschließend wird mit der Anzahl der stimmberechtigten Mitglieder der betroffenen Gliederung multipliziert. Die Anzahl der stimmberechtigten Mitglieder jeder Gliederung wird durch den Landesvorstand festgestellt. Stichtag ist jeweils der 31.12. des Vorjahres.\\
(4) Sofern eine Gliederung nicht existiert, gehen die Gelder an die jeweils übergeordnete Gliederung.\\
(5) Abschlagszahlungen werden zurückgelegt und am 01.01. des Folgejahres ausgeschüttet.}

\paragraph{Begründung}:

Einfacher Umlageschlüssel und im worst case vollständig resistent gegenüber Rückzahlungen an Landtag et al. da Abschlagszahlungen erst mal vollständig zurückgelegt werden.

% -----

\satzung{Finanzrat}
\antrag{Alexander Zinser}

\paragraph{Antragstext}:

Der Landesparteitag möge beschließen, folgenden Abschnitt in die \href{http://wiki.piratenpartei.de/LSA:Satzung#Abschnitt_B:_Finanzordnung}{Landesfinanzordnung} aufzunehmen:

\einruecken{§X - Finanzrat\\
(1) Der Landesparteitag wählt einmal jährlich zwei Piraten des Landesverbandes in den Finanzrat der Piratenpartei Deutschland.}

\paragraph{Begründung}:

Es existiert keine entsprechende Regelung auf Landesebene. Über die Frequenz der Wahl zum Finanzrat macht auch die Bundesfinanzordnung keine Aussage.

% -----

\satzung{Gliederungen}\label{satzung:gliederungen1}
\antrag{Alexander Zinser}
\begin{itemize}
\item \konkurrenz{satzung:gliederungen2}
\item \konkurrenz{satzung:gliederungen3}
\end{itemize}

\paragraph{Antragstext}:

Der Landesparteitag möge beschließen, \href{http://wiki.piratenpartei.de/LSA:Satzung#.C2.A7_7_-_Gliederung}{§7 (Gliederung)} der Landessatzung wie folgt zu ändern

\einruecken{(1) Der Landesverband Sachsen-Anhalt gliedert sich in Orts-, Kreis- und Regionalverbände.\\
(2) Regionalverbände sind Kreisverbände im Sinne der Bundessatzung, deren Gebiet sich über mehr als einen politischen Kreis erstreckt. Eine Koexistenz von Kreis- und Regionalverband auf dem selben Gebiet ist nicht zulässig.\\
(3) Gründet sich eine Untergliederung oder ändert ihre Satzung, so muss dem Landesvorstand die aktuelle Satzung vorgelegt werden.\\
(4) Die Geschäftsordnung des Vorstandes einer Untergliederung ist von allen Vorstandsmitgliedern zu unterschreiben und dem Landesvorstand in Kopie vorzulegen. Die Geschäftsordnung ist an geeigneter Stelle online zu stellen. Änderungen an der Geschäftsordnung sind dem Landesvorstand unverzüglich zu melden sowie in der Onlineversion zu aktualisieren.}

\paragraph{Alte Fassung}:

\einruecken{(1) Die Gliederung des Landesverbands regelt die Bundessatzung.}

\paragraph{Begründung}:

\begin{itemize}
\item Die Landessatzung verweist bisher nur auf die Bundessatzung.
\item Der Begriff Regionalverband ist bisher nicht definiert.
\end{itemize}

% -----

\satzung{Gliederungen (Alternative mit Gründungsklausel)}\label{satzung:gliederungen2}
\antrag{René Emcke}
\begin{itemize}
\item \konkurrenz{satzung:gliederungen1}
\item \konkurrenz{satzung:gliederungen3}
\end{itemize}

\paragraph{Antragstext}:

Der Landesparteitag möge beschließen, \href{http://wiki.piratenpartei.de/LSA:Satzung#.C2.A7_7_-_Gliederung}{§7 (Gliederung)} der Landessatzung wie folgt zu ändern

\einruecken{(1) Der Landesverband Sachsen-Anhalt gliedert sich in Orts-, Kreis- und Regionalverbände.\\
(2) Regionalverbände sind Kreisverbände im Sinne der Bundessatzung, deren Gebiet sich über mehr als einen politischen Kreis erstreckt. Eine Koexistenz von Kreis- und Regionalverband auf dem selben Gebiet ist nicht zulässig.\\
(3) Der Gründung eines Kreisverbandes müssen mindestens drei akkreditierte Piraten aus jedem politischen Kreis mehrheitlich zustimmen. Insgesamt müssen der Gründung mindestens zehn akkreditierte Piraten mehrheitlich zustimmen.\\
(4) Sofern der zuständige Kreisverband keine anderen Regelungen getroffen hat, gilt für die Gründung von Ortsverbänden Absatz (3) Satz 2.\\
(5) Gründet sich eine Untergliederung oder ändert ihre Satzung, so muss dem Landesvorstand die aktuelle Satzung vorgelegt werden.\\
(6) Die Geschäftsordnung des Vorstandes einer Untergliederung ist von allen Vorstandsmitgliedern zu unterschreiben und dem Landesvorstand in Kopie vorzulegen. Die Geschäftsordnung ist an geeigneter Stelle online zu stellen. Änderungen an der Geschäftsordnung sind dem Landesvorstand unverzüglich zu melden sowie in der Onlineversion zu aktualisieren.}

\paragraph{Alte Fassung}:

\einruecken{(1) Die Gliederung des Landesverbands regelt die Bundessatzung.}

\paragraph{Begründung}:

\begin{itemize}
\item Die Landessatzung verweist bisher nur auf die Bundessatzung.
\item Der Begriff Regionalverband ist bisher nicht definiert.
\item gemäß \href{http://lqfb.piraten-lsa.de/lsa/initiative/show/33.html}{erfolgreicher LF-Initiative} mit beschränkender Klausel für (Neu)Gründungen
\item Satzungsverankerung der notwendigen separaten Zustimmung der Mitglieder aus allen beteiligten politischen Kreisen bei Gründung von kreisübergreifenden Regionalverbänden
\item Sicherstellung der Arbeitsfähigkeit und Legitimation von Untergliederungen durch eine ausreichende Anzahl zustimmender Mitglieder
\item Verhinderung von Gründungen durch lediglich 3 (Mindestanzahl für einen Vorstand) Mitglieder, die sich bei Wahl des Vorstandes gegenseitig wählen (Legitimation) 
\end{itemize}

% -----

\satzung{§ 11 - Satzungs- und Programmänderung (3)}
\antrag{Roman Ladig}

\paragraph{Antragstext}:

Die Landesmitgliederversammlung möge beschließen, \href{http://wiki.piratenpartei.de/LSA:Satzung#.C2.A7_11_-_Satzungs-_und_Programm.C3.A4nderung}{§11 (3)} der Landessatzung wie folgt zu ändern:

\einruecken{(3) Vom Landesparteitag kann ein eigenes Grundsatzprogramm für den Landesverband sowie Wahlprogramme für Kommunal- und Landtagswahlen verabschiedet werden. Diese dürfen dem Grundsatzprogramm der Piratenpartei Deutschland nicht widersprechen.}

\paragraph{Alte Fassung}:

\einruecken{(3) Das Grundsatzprogramm der Piratenpartei Deutschland wird vom Landesverband übernommen. Ein eigenes Wahlprogramm basierend auf den Werten des Grundsatzprogrammes kann auf Landesebene für Kommunal- und Landtagswahlen bei Bedarf vom Landesparteitag verabschiedet werden. }

\paragraph{Begründung}:

Ein Landesgrundsatzprogramm hilft, die Position von Sachsen-Anhalt innerhalb des Bundes besser zu beschreiben, regionale Unterschiede aufzuzeigen und sich gegenüber anderen Landesverbänden falls nötig abzugrenzen. Darüberhinaus können von einem entwickelten Grundsatzprogramm leichter Wahlprogramme und Schlüsselpapiere abgeleitet und Anregungen für die Weiterentwicklung des Grundsatzprogramms der Piratenpartei Deutschland gefunden werden.

% -----

\satzung{§ 10 - Bewerberaufstellung für die Wahlen zu Volksvertretungen }
\antrag{Roman Ladig}

\paragraph{Antragstext}:

Die Landesmitgliederversammlung möge beschließen, \href{http://wiki.piratenpartei.de/LSA:Satzung#.C2.A7_10_-_Bewerberaufstellung_f.C3.BCr_die_Wahlen_zu_Volksvertretungen}{§10} der Landessatzung wie folgt zu ändern: 

\einruecken{(1) Die Bewerberaufstellung für die Wahlen zu Volksvertretungen erfolgt nach Maßgabe der Wahlgesetze und den Vorgaben der Bundessatzung. Soweit die Vorschriften der Wahlgesetze nicht vorgehen oder ein anderes vorschreiben, gilt im Übrigen das Prozedere in den nachfolgenden Absätzen.\\
(2) Landeslisten werden von der Mitgliederversammlung des Landesverbandes aufgestellt, sofern nicht eine gemeinsame Liste zusammen mit dem Bundesverband zur Europawahl aufgestellt wird.\\
(3) Die Mitglieder werden nach § 9b dieser Satzung zur Wahl geladen. Lassen die Wahlgesetze kürzere Ladungsfristen zu, so genügt deren Einhaltung. In der Einladung wird ausdrücklich auf die Bewerberaufstellung hingewiesen. Die Mitgliederversammlung ist beschlussfähig, wenn mindestens 10 \% der stimmberechtigten Mitglieder anwesend sind.\\
(4) Wahlkreisbewerber werden\\
1. In Wahlkreisen, deren Grenzen deckungsgleich mit denen eines oder mehrerer Regional- bzw. Kreisverbände sind, von den existierenden Gliederungen selbst aufgestellt,\\
2. in sonstigen Fällen beruft der Landesvorstand die Wahlkreisversammlung ein. In diesen Versammlungen wählen jeweils die in einem gemeinsamen Wahlkreis wohnhaften Piraten einen gemeinsamen Wahlkreisbewerber,\\
3. falls Punkt 2. nicht möglich ist zu Landtagswahlen auch in einer Landesversammlung der zur Wahl des Landtages wahlberechtigten Piraten gewählt.\\
(5) Die Bewerberaufstellung zu Kommunalwahlen nach dem Kommunalwahlgesetz regeln die Gliederungen unterhalb des Landesverbandes selbst.}

\paragraph{Alte Fassung}:

\einruecken{(1) Die Bewerberaufstellung für die Wahlen zu Volksvertretungen erfolgt nach den Regularien der einschlägigen Gesetze sowie den Vorgaben der Bundessatzung.\\
(2) Die Aufstellung kann sowohl als Mitgliederversammlung des zuständigen Stimm- bzw. Wahlkreises als auch im Rahmen einer anderen Mitgliederversammlung stattfinden, sofern gewährleistet wird, dass alle Stimmberechtigten in angemessener Zeit und Form eingeladen wurden und nur die Stimmberechtigten an der Wahl teilnehmen. Die Einladung muss dabei explizit auf die Bewerberaufstellung hinweisen.}

\paragraph{Begründung}:

In der aktuellen Satzung ist in §10 die Bewerberaufstellung für die Wahl zu Volksvertretung nicht genügend geregelt. So fehlte z.b. ein Passus über die Abgrenzung der Gliederungshoheiten (Land vs. Kommunal) beim Aufstellen der Wahlkreisbewerber.

% -----

\satzung{Liquid Democracy}
\antrag{Karl}

\paragraph{Antragstext}:

Der Landesparteitag möge beschließen den folgenden Abschnitt an passender Stelle in die Satzung des Landesverbandes Sachsen-Anhalt einzufügen.

\einruecken{Liquid Democracy\\\\
(1) Die Piratenpartei Deutschland Sachsen-Anhalt nutzt zur Willensbildung über das Internet eine geeignete Software. Diese muss die “Anforderungen für den Liquid Democracy Systembetrieb” erfüllen, welche vom Vorstand beschlossen werden.\\\\
Die Mindestanforderungen sind:\\\\
a) Jedes Mitglied muss die Möglichkeit haben, Anträge im System zu stellen. Zulassungsquoren und Antragskontingente sind zulässig, müssen jedoch für alle Mitglieder gleich sein.\\
b) Das System muss ohne Moderatoren auskommen.\\
c) In das System eingebrachte Anträge dürfen nicht gegen den Willen des Antragsstellers von anderen Mitgliedern verändert oder gelöscht werden können.\\
d) Jedem Mitglied muss es innerhalb eines bestimmten Zeitraums möglich sein, Alternativanträge einzubringen.\\
e) Das eingesetzte Abstimmungsverfahren darf Anträge, zu denen es ähnliche Alternativanträge gibt, nicht prinzipbedingt bevorzugen oder benachteiligen. Mitgliedern muss es möglich sein, mehreren konkurrierenden Anträgen gleichzeitig zuzustimmen. Der Einsatz eines Präferenzwahlverfahrens ist hierbei zulässig.\\
f) Es muss möglich sein, die eigene Stimme mindestens themenbereichsbezogen durch Delegation an ein anderes Mitglied zu übertragen. Diese Delegationen müssen jederzeit widerrufbar sein und übertragenes Stimmgewicht muss weiter übertragen werden können. Selbstgenutztes Stimmgewicht darf nicht weiter übertragen werden.\\
(2) Der Vorstand stellt den dauerhaften und ordnungsgemäßen Betrieb des Systems sicher.\\
(3) Jedem Mitglied ist Einsicht in den abstimmungsrelevanten Datenbestand des Systems zu gewähren. Während einer Abstimmung darf der Zugriff auf die jeweiligen Abstimmdaten anderer Mitglieder vorübergehend gesperrt werden.\\
(4) Die Organe sind gehalten, das Liquid Democracy System zur Einholung von Empfehlungen zur Grundlage ihrer Beschlüsse zu nutzen und vom diesen Empfehlungen abweichende Entscheidungen zu begründen. Das Schiedsgericht ist davon ausgenommen.\\
(5) Die Organe der Partei sind angehalten, die Anträge, die im Liquid Democracy System positiv beschieden wurden, vorrangig zu behandeln.\\
(6) Teilnahmeberechtigt ist jeder Pirat, der nach der Satzung Mitglied der Piratenpartei Sachsen-Anhalt ist. Jeder Pirat erhält genau einen persönlichen Zugang, der nur von ihm genutzt werden darf.\\
(7) Verstößt ein Nutzer wiederholt und in erheblichem Maße gegen die Nutzungsbedingungen des Systems, so kann der Vorstand als Ordnungsmaßnahme dem Nutzer auf Zeit das Recht entziehen, Anträge oder andere Texte in das System einzustellen. Im Falle technischer Angriffe auf das System, die von einem angemeldeten Benutzer ausgehen, kann dieses Benutzerkonto durch Administratoren vorübergehend gesperrt werden.}

\paragraph{Begründung}:

Das Konzept der Liqiud Democracy und deren Umsetzung in der Piratenpartei in Form von Liquid Feedback, bilden zusammen wohl eines der vielversprechendsten Projekte innerhalb der Partei und haben ein gewaltiges Potential die Art, wie Demokratie praktiziert wird, zu verändern. Daher ist es wichtig diese besondere Stellung innerhalb der Partei auch in der Satzung abzubilden.

Bis jetzt werden die Ergebnisse von Liquid Feedback meist als Meinungsbilder interpretiert, doch diese Aussage wird der tatsächlichen Relevanz der ausgearbeiteten Anträge nicht mehr gerecht. Damit diese nicht mehr übergangen oder ignoriert werden können, sollen Anträge als Empfehlungen an die Parteiorgane gelten.

Eine abweichende Entscheidung sollte von den Organen begründet werden, hierzu ist ausreichend, dass eine Begründung im Rahmen des Protokolls des jeweiligen Organs festgehalten wird. Die Begründung dient zur Nachvollziehbarkeit der getroffenen Entscheidung und somit zur innerparteilichen Transparenz.

% -----

\satzung{Landesvorstand Piraten LSA - Amtszeitbegrenzung/Wiederwahl}
\antrag{Markus Hünniger}

\paragraph{Antragstext}:

Ich beantrage der LPT möge beschließen den \href{http://wiki.piratenpartei.de/LSA:Satzung#.C2.A7_9a_-_Der_Vorstand}{Punkt (3) in §9a} zu ergänzen durch: 

\einruecken{Der Vorsitzende und der 2. Vorsitzende können nur 3x in Folge für diese Ämter kandidieren und dürfen dieses Amt maximal 6 Jahre übernehmen. Danach ist eine Kandidatur für ein Vorstandsamt, für die Dauer der geleisteten Amtszeit (in Jahren, aufgerundet) nicht zulässig.}

\paragraph{Begründung}:

Die Mitglieder, welche diese Ämter ausfüllen, sind sich so bewusst das es für Sie nur eine zeitliche begrenzte Aufgabe ist. Man richtet sich auf diesen Posten nicht ein. Es erzeugt kein Beharrungsvermögen und lässt Sie danach abkühlen und aus der Schusslinie nehmen, sowie begrenzt auf einfach Art und Weise das Amt. Es lässt Zeit um sich politisch abzukühlen, in sein normales Leben zurückzukehren, sich neue zu Besinnen und Kraft zu schöpfen!

% -----

\satzung{Kandidatur, Amtszeit, Wiederwahl, Landtag, Kreistage, Stadträte}
\antrag{Markus Hünniger}

\paragraph{Antragstext}:

Der Landesparteitag möge beschließen, \href{http://wiki.piratenpartei.de/LSA:Satzung#.C2.A7_10_-_Bewerberaufstellung_f.C3.BCr_die_Wahlen_zu_Volksvertretungen}{§10} der Landessatzung durch folgenden neuen Absatz zu ergänzen:

\einruecken{Piraten, die zwei Legislaturperioden (unabhängig von der Dauer) Mitglieder von Volksvertretungen waren, können für die Dauer der geleisteten Zeit als Mitglied dieser Vertretung nicht wieder für das gleiche Gremium kandidieren.}

\paragraph{Begründung}:

Die Mitglieder, welche diese Ämter ausfüllen, sind sich so bewusst, dass es für sie nur eine zeitliche begrenzte Aufgabe ist. Man richtet sich auf diesen Posten nicht ein. Es erzeugt kein Beharrungsvermögen und lässt sie danach abkühlen, aus der Schusslinie nehmen und begrenzt auf einfach Art und Weise das Amt. Es lässt Zeit um sich politisch abzukühlen, in sein normales Leben zurückzukehren, sich neue zu Besinnen und Kraft zu schöpfen! Nach Ablauf der Ruhezeit, kann man dann wieder kandidieren!

% -----

\satzung{ein Pirat - ein Mandat}
\antrag{Markus Hünniger}

\paragraph{Antragstext}:

Der Landesparteitag möge beschließen, \href{http://wiki.piratenpartei.de/LSA:Satzung#.C2.A7_10_-_Bewerberaufstellung_f.C3.BCr_die_Wahlen_zu_Volksvertretungen}{§10} der Landessatzung durch folgenden neuen Absatz zu ergänzen:

\einruecken{Ein Pirat darf nicht mehr als ein Mandat innehaben.}

\paragraph{Begründung}:

Eine Person, ein Mandat. Keine Häufung.

z.B. entweder Landtag oder Stadtrat, nicht beides, Entw. Bundestag oder LT nicht beides.

% -----

\satzung{Gliederungen\_3}\label{satzung:gliederungen3}
\antrag{Kevin Oelze}
\begin{itemize}
\item \konkurrenz{satzung:gliederungen1}
\item \konkurrenz{satzung:gliederungen2}
\end{itemize}

\paragraph{Antragstext}:

Der Landesparteitag möge beschließen, \href{http://wiki.piratenpartei.de/LSA:Satzung#.C2.A7_7_-_Gliederung}{§7 (Gliederung)} der Landessatzung wie folgt zu ändern

\einruecken{(1) Der Landesverband Sachsen-Anhalt gliedert sich in Orts-, Kreis- und Regionalverbände.\\
(2) Regionalverbände sind Kreisverbände im Sinne der Bundessatzung, deren Gebiet sich über mehr als einen politischen Kreis erstreckt. Eine Koexistenz von Kreis- und Regionalverband auf dem selben Gebiet ist nicht zulässig.\\
(3) Gründet sich eine Untergliederung oder ändert ihre Satzung, so muss dem Landesvorstand die aktuelle Satzung vorgelegt werden.}

\paragraph{Alte Fassung}:

\einruecken{(1) Die Gliederung des Landesverbands regelt die Bundessatzung.}

\paragraph{Begründung}:

\begin{itemize}
\item Die Landessatzung verweist bisher nur auf die Bundessatzung.
\item Der Begriff Regionalverband ist bisher nicht definiert.
\item Die Bekanntgabe der GO ist obligatorisch, auch das Verfahren ist selbstverständlich, eine Meldung an den LaVo bei jeder Änderung ist nicht zielführend.
\end{itemize}

% -----

\satzung{Maximale Spendenhöhe von 5000 Euro}
\antrag{Christian Kunze}

\paragraph{Antragstext}:

Der Landesparteitag möge beschließen, folgenden neuen Punkt der \href{http://wiki.piratenpartei.de/LSA:Satzung#Abschnitt_B:_Finanzordnung}{Finanzordnung} des Landesverbands hinzuzufügen:

\einruecken{Die maximale Spendenhöhe an den Landesverband von juristischen Personen (Firmen, Konzerne) und Privatpersonen darf 5000 Euro nicht überschreiten.}

\paragraph{Begründung}:

Es soll eine Maßnahme zur Prävention von zu starker Einflussnahme auf den Landesverband sein. Da wir bis jetzt noch keine so hohe Spende hatten, sollte es uns stutzig machen, wenn jemand z. B. vor der nächsten Wahl einen höheren Betrag spenden will und was er damit zu erreichen versucht. Weitere Begründungen folgen auf dem Parteitag.

%%\part{Grundsatzprogramm}
%%\part{Wahlprogramm}
\section{Präambel}
\wahlprogramm{Präambel}\label{praeambel:unglow}
\antrag{Unglow}\konkurrenz{praeambel:trier}\version{17:14, 18. Jun. 2010}
\subsubsection{Absatz 1}
\abstimmung
Im Zuge der Digitalen Revolution aller Lebensbereiche sind trotz aller Lippenbekenntnisse die Würde und die Freiheit des Menschen in bisher ungeahnter Art und Weise gefährdet. Dies geschieht zudem in einem Tempo, das die gesellschaftliche Meinungsbildung und die staatliche Gesetzgebung ebenso überfordert wie den Einzelnen selbst. Gleichzeitig schwinden die Möglichkeiten, diesen Prozess mit demokratisch gewonnenen Regeln auf der Ebene eines einzelnen Staates zu gestalten, dahin.
\subsubsection{Absatz 2}
\abstimmung
Die Globalisierung des Wissens und der Kultur der Menschheit durch Digitalisierung und Vernetzung stellt deren bisherige rechtliche, wirtschaftliche und soziale Rahmenbedingungen ausnahmslos auf den Prüfstand. Nicht zuletzt die falschen Antworten auf diese Herausforderung leisten einer entstehenden totalen und totalitären, globalen Überwachungsgesellschaft Vorschub. Die Angst vor internationalem Terrorismus lässt Sicherheit vor Freiheit als wichtigstes Gut erscheinen – und viele in der Verteidigung der Freiheit fälschlicherweise verstummen.
\subsubsection{Absatz 3}
\abstimmung
Die Globalisierung des Wissens und der Kultur der Menschheit durch Digitalisierung und Vernetzung stellt deren bisherige rechtliche, wirtschaftliche und soziale Rahmenbedingungen ausnahmslos auf den Prüfstand. Nicht zuletzt die falschen Antworten auf diese Herausforderung leisten einer entstehenden totalen und totalitären, globalen Überwachungsgesellschaft Vorschub. Die Angst vor internationalem Terrorismus lässt Sicherheit vor Freiheit als wichtigstes Gut erscheinen – und viele in der Verteidigung der Freiheit fälschlicherweise verstummen.
\subsubsection{Absatz 4}
\abstimmung
Informationelle Selbstbestimmung, freier Zugang zu Wissen und Kultur und die Wahrung der Privatsphäre sind die Grundpfeiler der zukünftigen Informationsgesellschaft. Nur auf ihrer Basis kann eine demokratische, sozial gerechte, freiheitlich selbstbestimmte, globale Ordnung entstehen. Die Piratenpartei versteht sich daher als Teil einer weltweiten Bewegung, die diese Ordnung zum Vorteil aller mitgestalten will.
\subsubsection{Absatz 5}
\abstimmung
Die Piratenpartei will sich auf die im Programm genannten Themen konzentrieren, da wir nur so die Möglichkeit sehen, diese wichtigen Forderungen in Zukunft durchzusetzen. Gleichzeitig glauben wir, dass diese Themen für Bürger aus dem gesamten traditionellen politischen Spektrum unterstützenswert sind, und dass eine Positionierung in diesem Spektrum uns in unserem gemeinsamen Streben nach Wahrung der Privatsphäre und Freiheit für Wissen und Kultur hinderlich sein würde.

\wahlprogramm{Präambel}\label{praeambel:trier}
\antrag{KV Trier/Trier-Saarburg}\konkurrenz{praeambel:unglow}\version{17:14, 18. Jun. 2010}
\subsubsection{Wer sind die Piraten}
\abstimmung
\textit{Freiheitsrechte und die Gestaltung der modernen Informations- und Wissensgesellschaft sind die Kernanliegen der Piratenparteien in ganz Europa und weltweit – und natürlich auch bei uns in Rheinland-Pfalz.}

Seit ihrer Gründung 2006 in Berlin wirkt die Piratenpartei Deutschland gemäß ihrer grundgesetzlichen Pflichtenan der "Willensbildung des Volkes" mit. Während des Wahlkampfs zur Europawahl und Bundestagswahl 2009 erlebte die Piratenpartei einen raschen Mitgliederzuwachs. Bei der Bundestagswahl konnte sie als neue Partei sofort 2\% der Stimmen erreichen. Für die schwedische Schwesterpartei sitzen zwei Abgeordnete im Europaparlament.

Der uralte Traum, alles Wissen und alle Kultur der Menschheit zusammenzutragen, zu speichern und heute und in der Zukunft verfügbar zu machen, ist durch die rasante technische Entwicklung der vergangenen Jahrzehnte in greifbare Nähe gerückt. Wie jede bahnbrechende Neuerung erfasst diese vielfältige Lebensbereiche und führt zu tiefgreifenden Veränderungen. Die Piratenpartei möchte die Chancen dieser Situation nutzen und vormöglichen Gefahren warnen.
Informationelle Selbstbestimmung, freier Zugang zu Wissen und Kultur und die Wahrung der Privatsphäre sind die Grundpfeiler der zukünftigen Informationsgesellschaft. Nur auf dieser Basis kann eine selbstbestimmte, sozial gerechte, freiheitlich-demokratische Grundordnung erhalten bleiben. Die Piratenpartei ist Teil einer weltweiten Bewegung, die diese Ordnung zum Vorteil aller mitgestalten will.

\subsubsection{Unsere Ziele}
\abstimmung
\textit{Grundrechte verteidigen}
Die Piratenpartei setzt sich für einen stärkeren Schutz und eine unbedingte Beachtung der Menschen- und Bürgerrechte ein. Die gesamte Politik muss sich an ihnen orientieren. 

\textit{Informationelle Selbstbestimmung}
Das Recht des Einzelnen, die Nutzung seiner persönlichen Daten zu kontrollieren, muss garantiert werden. Dies gilt dem Staat gegenüber ebenso wie im Wirtschaftsbereich. Wir wollen weder den gläsernen Bürger noch den gläsernen Konsumenten.

\textit{Transparenz}
Alles staatliche Handeln muss transparent und für jeden nachvollziehbar sein. Nach unserer Überzeugung ist dies unabdingbare Voraussetzung für eine moderne Wissensgesellschaft in einer freiheitlichen und demokratischen Ordnung.

\textit{Bildung ermöglichen}
Jeder Mensch hat das Recht auf freien Zugang zu Information und Bildung. Dies ist notwendig, um jedem Menschen unabhängig von seiner sozialen Herkunft ein größtmögliches Maß an gesellschaftlicher Teilhabe zu ermöglichen. Bildung ist eine der wichtigsten Ressourcen der Gesellschaft und der Wirtschaft, da nur durch den Erhalt, die Weitergabe und die Vermehrung von Wissen auf Dauer Fortschritt und gesellschaftlicher Wohlstand gesichert werden können.

\textit{Patente}
Wir lehnen Patente auf Lebewesen und Gene, auf Geschäftsideen und auch auf Software ab, weil sie unzumutbare und unverantwortliche Konsequenzen haben, weil sie die Entwicklung der Wissensgesellschaft behindern, weil sie allgemeine Güter ohne angemessene Gegenleistung und ohne Not privatisieren und weil sie kein Erfindungspotenzial im ursprünglichen Sinne besitzen.

\textit{Open Access}
Aus dem Staatshaushalt wird eine Vielzahl schöpferischer Tätigkeiten finanziert. Da diese Werke von der Allgemeinheit finanziert werden, sollten sie auch der Allgemeinheit kostenlos zur Verfügung stehen.

\textit{Urheberrecht fair gestalten}
Das Urheberrecht muss auf die Anforderungen der sich entwickelnden Informationsgesellschaft angepasst werden und muss die Bedürfnisse von Konsumenten und Produzenten gleichermaßen berücksichtigen, auch in Hinblick darauf, dass die Grenzen zwischen Konsument und Produzent immer mehr verschwi

\subsubsection{Die Piraten in Rheinland-Pfalz}
\abstimmung
Die Piratenpartei Deutschland hat Landesverbände in allen Bundesländern. In Rheinland-Pfalz wurde der Landesverband 2008 in Koblenz gegründet.

Die Forderungen des Piratenprogramms spielen auch auf Landesebene eine große Rolle. Wir setzen uns in unserem Bundesland deshalb für bessere Bildungschancen, mehr Transparenz in der Politik, mehr Mitbestimmung und Wahrung der Grundrechte ein.
Die folgenden Vorschläge für eine zukünftige Politik in Rheinland-Pfalz haben wir auf Basis der piratigen Grundsätze und des Parteiprogramms der Piratenpartei Deutschland erstellt.

%%\section{Privatsphäre, Datenschutz und Bürgerrechte - Grundpfeiler der freiheitlichen Informationsgesellschaft}

\subsection*{Bedeutung von Datenschutz und Privatsphäre}
\wahlprogramm{Bedeutung von Datenschutz und Privatsphäre}\label{datenschutz:bedeutung}
\antrag{Unglow}\version{22:51 18. Jun. 2010}
\abstimmung
Der Schutz der Privatsphäre und der Datenschutz gewährleisten Würde und Freiheit des Menschen, die freie Meinungsäußerung, demokratische Teilhabe und in der Folge unsere freiheitlich-demokratische Gesellschaftsform, die in der Vergangenheit auch unter Einsatz zahlloser Menschenleben erkämpft und verteidigt wurde.

Jeder einzelne Schritt auf dem Weg zum Überwachungsstaat mag noch so überzeugend begründet sein - doch als Deutsche und Europäer wissen wir aus Erfahrung, wohin dieser Weg führt. Diesen Entwicklungen stellen wir uns entschieden entgegen und sagen dem Überwachungsstaat den Kampf an.

Das Recht auf Wahrung der Privatsphäre ist ein unabdingbares Fundament einer demokratischen Gesellschaft. Die Meinungsfreiheit und das Recht auf persönliche Entfaltung sind ohne diese Voraussetzung nicht zu verwirklichen.

Die Piratenpartei hat das Ziel, die Privatsphäre der Bürger vor unberechtigten und unverhältnismäßigen Eingriffen durch Staat und Wirtschaft zu schützen. Die Überwachungspolitik der letzten Jahre wollen wir umkehren, um eine freiheitsfreundliche Sicherheitspolitik unter Achtung der Bürgerrechte zu gewährleisten.

\subsection*{Kein Überwachungsstaat - das Recht in Ruhe gelassen zu werden}
\wahlprogramm{Datenschutz}\label{datenschutz:datenschutz}
\antrag{Piraten aus RLP}\version{22:51 18. Jun. 2010}
\subsubsection{Kein Überwachungsstaat - das Recht in Ruhe gelassen zu werden}
\abstimmung
Systeme und Methoden, die der Staat gegen seine Bürger einsetzen kann, müssen der ständigen Bewertung und genauen Prüfung durch gewählte Mandatsträger unterliegen. Wenn die Regierung Bürger beobachtet, ohne dass sie eines Verbrechens verdächtig sind, ist dies eine fundamental inakzeptable Verletzung des Bürgerrechts auf Privatsphäre.

Die pauschale Verdächtigung und anlasslose Überwachung aller Bürger hat generell zu unterbleiben. Eine als 'präventive Strafverfolgung' verschleierte Abschaffung der Unschuldsvermutung lehnen wir unbedingt ab.

Die flächendeckende Überwachung des öffentlichen Raums durch Videokameras oder andere Maßnahmen darf nicht zugelassen werden. Wir fordern ein allgemeines Verbot der Überwachung des öffentlichen Raums, von dem nur einzelne, richterlich angeordnete Ausnahmen zulässig sind.
Jedem Bürger muss das Recht auf Anonymität garantiert werden, das unserer Verfassung innewohnt. Die Weitergabe personenbezogener Daten vom Staat an die Privatwirtschaft hat in jedem Falle zu unterbleiben.

\subsubsection{Informationelle Selbstbestimmung}
\abstimmung
Das Recht des Einzelnen, die Verwendung seiner persönlichen Daten zu kontrollieren, muss gestärkt werden. Jegliche kommerzielle Nutzung persönlicher Daten muss verboten sein, solange sie nicht ausdrücklich vom Betroffenen erlaubt wird. Dazu müssen insbesondere die Datenschutzbeauftragten völlig unabhängig agieren können. Neue Methoden wie das Scoring machen es erforderlich, nicht nur die persönlichen Daten kontrollieren zu können, sondern auch die Nutzung aller Daten, die zu einem Urteil über eine Person herangezogen werden können. Jeder Bürger muss gegenüber den Betreibern zentraler Datenbanken einen durchsetzbaren und wirklich unentgeltlichen Anspruch auf Selbstauskunft, Korrektur, Sperrung oder Löschung der Daten haben. Ausgenommen davon sind Fälle, in denen ein öffentliches Interesse zur Erfüllung der staatlichen Aufgaben vorliegt.

\subsubsection{Biometrische Daten und Gentests}
\abstimmung
Erhebung und Nutzung biometrischer Daten und Gentests erfordern aufgrund des hohen Missbrauchspotentials eine besonders kritische Bewertung und Kontrolle von unabhängiger Stelle. Der Aufbau zentraler Datenbanken mit solchen Daten muss unterbleiben. Die Verwendung biometrischer Merkmale in Passdokumenten hat zu unterbleiben oder auf Freiwilligkeit zu beruhen. Es ist gegenüber Drittstaaten durchzusetzen, dass diese Pässe unabhängig von biometrischen Merkmalen vollständig gültig sind. Massengentests für polizeiliche Zwecke, bei denen die Vorgeladenen nicht individuell verdächtigt werden, müssen als anlasslose Verdächtigungen gewertet und entsprechend untersagt werden.

\subsubsection{Besonderheiten digitaler Daten}
\abstimmung
Generell müssen die Bestimmungen zum Schutze personenbezogener Daten die Besonderheiten digitaler Daten, wie etwa mögliche Langlebigkeit und schwer kontrollierbare Verbreitung, stärker berücksichtigen. Gerade weil die Piratenpartei für eine stärkere Befreiung von Information, Kultur und Wissen eintritt, fordern wir Datensparsamkeit, Datenvermeidung und unabhängige Kontrolle von Stellen, die personenbezogene Daten verwenden. Wenn diese nämlich für wirtschaftliche oder Verwaltungszwecke genutzt werden, können sie die Freiheit und die informationelle Selbstbestimmung des Bürgers unnötig einschränken und den Überwachungsdruck verstärken. Zu einem effektivem Datenschutz gehört aus Sicht der Piratenpartei ausserdem das Recht des Bürgers, über ungewollte Datenabflüsse personenbezogener Daten aus Unternehmen und Behörden unverzüglich und lückenlos informiert zu werden.

\newpage
\subsubsection{Konkrete Maßnahmen}
\abstimmung
Konkrete Forderungen im Bereich der Privatsphäre und der Inneren Sicherheit sind:
\begin{itemize}
\item Durchsetzung des Folterverbots
\item Bessere, wirksame Kontrolle von Geheimdiensten und Polizei national und europaweit
\item Solange kein europaweiter einheitlicher Datenschutz auf hohem Niveau existiert, dürfen die Hürden für den Informationsaustausch zwischen der deutschen Polizei und der anderer Mitgliedsstaaten nicht weiter abgesenkt werden.
\item Kein Informationsaustausch mit Staaten ohne wirksamen Datenschutz
\item Einführung einer Informations- und Auskunftspflicht gegenüber den Betroffenen beim Datenaustausch zwischen Polizeien der EU-Länder
\item Rücknahme der EU-Richtlinie über die Vorratsdatenspeicherung und Stopp aller Planungen zur Wiedereinführung des Gesetzes
\item keine Vorratsspeicherung von Flug-, Schiff- und sonstigen Passagierdaten (PNR: Passenger Name Records)
\item Stopp der anlasslosen Übermittlung von Passagierdaten
\item keine Weitergabe von solchen Passagierdaten an Dritte
\item Abschaffung des ELENA-Systems zum elektronischen Einkommensnachweis
\item Stopp des Jugendmedienschutz-Staatsvertrags
\item Rücknahme des ZugangsErschwerungsgesetzes
\item Stopp des SWIFT-Abkommens mit den USA
\item Stopp der Volkszählung 2011 und Rücknahme des Zensus-Gesetzes zur
\item Rücknahme der Auslandskopfüberwachung
\item kein automatisiertes KFZ-Kennzeichen-Scanning
\item Abschaffung der biometrischen Daten in Pässen und Ausweisen. Verzicht auf RFID-Chips in Ausweisdokumenten.
\item Einrichtung einer unabhängigen deutschen Datenschutzbehörde mit Sanktions-Recht
\item keine 'präventive' Strafverfolgung (keine Aufhebung der Unschuldsvermutung)
\item keine Internierungslager (Gefängnis ohne Aburteilung) in Deutschland
\item Abbau von Echelon-Abhörzentralen auf deutschem Boden
\item Abschaffung der "Anti-Terror-Datei", der "Visa-Warndatei" und anderer unrechtmäßiger Datenbanken
\item Stärkung des allgemeinen Informantenschutzes
\item Abschaffung der Beugehaft für Zeugen
\item Wiederherstellung der Trennung von Polizei und Geheimdiensten. Rücknahme der geheimdienstlichen Befugnisse für das BKA.
\item Schutz von Ermittlungsdaten vor automatischem Austausch zwischen Polizeien verschiedener Staaten
\item Einführung eines eindeutigen, gut sichtbaren Identifikationsmerkmals (Nummer oder Name) für Polizisten bei Einsätzen zur Identifikation
\item Verzicht auf Videoüberwachung von öffentlichen Plätzen etc., Videoüberwachung generell verstärkt ersetzen durch unbewaffnete Polizeistreifen.
\item Keine automatische Gesichts- oder Verhaltenskontrolle
\item Ausweitung des Persönlichkeits-Kernbereichs auf elektronische Medien (z.B. Mail bei Webmailern, Laptop)
\item Keine geheimen Durchsuchungen - weder online noch offline!
\item Überprüfung und ggf. Aufhebung der unter dem Namen 'Anti-Terror-Maßnahme' eingeführten Regelungen, die seit dem 11.9.2001 installiert wurden
\item Einführung einer Meldepflicht von Unternehmen und Behörden bei Datenpannen
\end{itemize}

\subsection*{Überwachung abbauen, Befugnisse evaluieren}
\wahlprogramm{Überwachung abbauen, Befugnisse evaluieren}\label{datenschutz:ueberwachung}
\antrag{Piraten aus RLP}\version{22:51 18. Jun. 2010}
\subsubsection{Überwachung abbauen}
\abstimmung
Gemeinsam mit dem Bürgerrechtsbündnis 'Freiheit statt Angst' fordern wir:
\begin{itemize}
\item Abschaffung der flächendeckenden Protokollierung der Kommunikation und unserer Standorte (Vorratsdatenspeicherung)
\item Abschaffung der flächendeckenden Erhebung biometrischer Daten, sowie von RFID-Ausweisdokumenten
\item Schutz vor Bespitzelung am Arbeitsplatz durch ein Arbeitnehmerdatenschutzgesetz
\item Berücksichtigung des Datenschutzes für Bürger- und Arbeitnehmer/innen bereits in der Konzeptionsphase aller öffentlicher eGovernment-Projekte
\item Keine einheitliche Schülernummer (Berliner SchülerID)
\item Keine Weitergabe von Informationen über Menschen ohne triftigen Grund und konkreten Anlass
\item Keine europaweite Vereinheitlichung staatlicher Informationssammlungen (Stockholmer Programm)
\item Keine systematische Überwachung des Zahlungsverkehrs oder sonstige Massendatenanalyse (Stockholmer Programm)
\item Kein Informationsaustausch mit den USA und anderen Staaten ohne wirksamen Grundrechtsschutz
\item Abbau von Videoüberwachung und Verbot des Einsatzes von Verhaltenserkennungssystemen
\item Keine pauschale Registrierung aller Flug- und Schiffsreisenden (PNR-Daten)
\item Keine geheime Durchsuchung von Privatcomputern, weder online noch offline
\item Keine Einführung der Elektronischen Gesundheitskarte in der derzeit geplanten Form
\end{itemize}

\subsubsection{Evaluierung der bestehenden Überwachungsbefugnisse}
\abstimmung
Wir fordern eine unabhängige Überprüfung aller bestehenden Überwachungsbefugnisse im Hinblick auf ihre Wirksamkeit, Kosten, schädliche Nebenwirkungen und Alternativen.

\subsubsection{Moratorium für neue Überwachungsbefugnisse}
\abstimmung
Nach der inneren Aufrüstung der letzten Jahre fordern wir einen sofortigen Stopp neuer Gesetzesvorhaben auf dem Gebiet der inneren Sicherheit, die mit weiteren Grundrechtseingriffen verbunden sind.

\subsubsection{Gewährleistung der Meinungsfreiheit und des freien Meinungs- und Informationsaustauschs über das Internet}
\abstimmung
\begin{itemize}
\item Keine Beschränkung des Internetzugangs durch staatliche Stellen oder Internetanbieter (Sperrlisten).
\item Keine Sperrungen von Internetanschlüssen.
\item Verbot der Installation von Filtern in die Infrastruktur des Internet.
\item Entfernung von Internet-Inhalten nur auf Anordnung unabhängiger und unparteiischer Richter.
\item Einführung eines uneingeschränkten Zitierrechts für Multimedia-Inhalte, das heute unverzichtbar für die öffentliche Debatte in Demokratien ist.
\item Schutz von Plattformen zur freien Meinungsäußerung im Internet (partizipatorische Websites, Foren, Kommentare in Blogs), die heute durch unzureichende Gesetze bedroht sind, welche Selbstzensur begünstigen (abschreckende Wirkung).
\end{itemize}

\wahlprogramm{Datenschutz}\label{datenschutz:datenschutz}
\antrag{KV Trier / Trier-Saarburg}\version{22:51 18. Jun. 2010}
 Datenschutz ist ein Grundrecht. Dies hat das Bundesverfassungsgericht schon 1983 festgestellt, als es das Recht auf informationelle Selbstbestimmung begründete.

Mit zunehmender Wandlung zu einer Wissens- und Informationsgesellschaft gewinnt der Datenschutz an Bedeutung. Immer mehr Informationen über unser tägliches Leben liegen heute in elektronischer Form vor und können automatisiert verarbeitet und zusammengeführt werden.

Deswegen gilt es, die Grundsätze des Datenschutzes (Datensparsamkeit, Datenvermeidung, Zweckbindung und Erforderlichkeit) noch konsequenter in den Vordergrund zu stellen, denn Datenschutz wird nicht allein durch technische Maßnahmen erreicht, sondern insbesondere durch organisatorische.

\subsection*{Änderungen des Landesdatenschutzgesetzes}
\wahlprogramm{Änderungen des Landesdatenschutzgesetzes}\label{datenschutz:lds}
\antrag{Unglow}\version{22:51 18. Jun. 2010}

\subsubsection{Modul 1}
\abstimmung
Das aus dem siebziger Jahren stammende Datenschutzrecht muss dringend an die Erfordernisse des Informations- und Kommunikationszeitalters angepasst werden. Die Piratenpartei strebt ein gut lesbares, allgemein verständliches und unbürokratisches Datenschutzrecht an. Die gesetzlichen Regelungen müssen unabhängig von der zukünftigen technischen Entwicklung Wirkung entfalten.

\subsubsection{Modul 2}
\abstimmung
Sinnvolle Regelungen aus der Novellierung des Bundesdatenschutzgesetzes sollen in Landesrecht übernommen werden, wie z.B. die Informationspflichten bei Datenpannen und die Fort- und Weiterbildungsmaßnahmen für Datenschutzbeauftragte.

\subsection*{Wirksame Kontrolle gewährleisten}
\wahlprogramm{Wirksame Kontrolle gewährleisten}\label{datenschutz:kontrolle}
\antrag{Unglow}\version{22:51 18. Jun. 2010}

\subsubsection{Modul 1}
\abstimmung
Wesentliche Probleme im Bereich Datenschutz sind oftmals nicht auf gesetzliche Lücken, sondern auf den mangelnden Vollzug der bestehenden Gesetze zurückzuführen. Der Landesdatenschutzbeauftragte, welcher für die Kontrolle des Datenschutzes zuständig ist, ist jedoch personell so schwach ausgestattet, dass eine wirksame Kontrolle unmöglich ist und Datenschutzverstöße oft nicht auffallen, geschweige denn geahndet werden können.

\subsubsection{Modul 2}
\abstimmung
Die Piratenpartei wird deshalb die Behörde des Landesdatenschutzbeauftragten organisatorisch, personell und finanziell so stärken, dass eine wirksame Kontrolle der bestehenden Datenschutzgesetze gewährleistet werden kann. Inbesondere müssen anlasslose Kontrollen ermöglicht werden. Zudem wollen wir die Sanktionsmöglichkeiten erhöhen, sodass Datenschutzverstöße nicht mehr aus der Protokasse bezahlt werden können und Strafen nicht zu einem betriebswirtschaftlichen Faktor verkommen.

\subsubsection{Modul 3}
\abstimmung
Bei staatlichen IT-Projekten wie der ELENA-Datenbank, der elektronischen Gesundheitskarte oder elektronischen Ausweisdokumenten wird der Datenschutz regelmäßig missachtet. Oft kommt erst nach Eingriff der Datenschutzbeauftragten und öffentlichem Protest durch Bürger und Nicht-Regierungsorganisationen das Thema Datenschutz überhaupt auf die Agenda. Die PIRATEN werden gewährleisten, dass die Datenschutzbeauftragten bei staatlichen Projekten unmittelbar mit einbezogen werden und der Datenschutz zu einer Kernanforderung bei diesen Projekten wird.


\wahlprogramm{Elektronische Gesundheitskarte}\label{datenschutz:egk}
\antrag{KV Trier/Trier-Saarburg}\version{22:51 18. Jun. 2010}
\abstimmung
Transparenz heißt für uns nicht die Schaffung eines „gläsernen Patienten“. Wir lehnen die elektronische Gesundheitskarte ab und werden uns für deren Stopp einsetzen.

\wahlprogramm{Stärkung des Landesdatenschutzbeauftragten}\label{datenschutz:staerkung}
\antrag{KV Trier/Trier-Saarburg}\version{22:51 18. Jun. 2010}
\abstimmung
Ein starker Datenschutz setzt handlungsfähige Datenschützer voraus. Aus diesem Grund soll das Amt des Landesdatenschutzbeauftragten nach dem Vorbild Schleswig-Holsteins zu einem unabhängigen Landeszentrum für Datenschutz umgebaut werden. Dieses soll in Zukunft auch für den nichtöffentlichen Bereich und für Auskünfte nach dem Informationsfreiheitsgesetz zuständig sein. Dazu muss diese Institution auch personell deutlich ausgebaut werden.

\subsection*{Digitale Selbstverteidigung}
\wahlprogramm{Digitale Selbstverteidigung}\label{datenschutz:verteidigung}
\antrag{Unglow}\version{22:51 18. Jun. 2010}
\subsubsection{Digitale Selbstverteidigung}
\abstimmung
Datenschutz und informationelle Selbstbestimmung gewährleisten die Kontrolle über die eigenen Daten. Durch die immer umfangreicher werdende Datenverarbeitung im Informationszeitalter ist Datenschutz wichtiger denn je. Trotzdem fehlt in weiten Teilen der Bevölkerung noch das Bewusstsein für den sorgfältigen Umgang mit eigenen und fremden Daten. Für vermeintliche Rabatte oder geringe Gewinnchancen sind viele bereit ihre persönlichen Daten preiszugeben, ohne sich über das Ausmaß dieser Entscheidung bewusst zu sein. Die Rechte, die der Staat seinen Bürgern einräumt, können nur Wirkung entfalten, wenn die Menschen sie bewusst ausüben können. Die Piratenpartei will deshalb die Voraussetzungen für eine wirksame digitale Selbstverteidung schaffen.
\subsubsection{Datenschutz als Bildungsauftrag}
\abstimmung
Wir betrachten Datenschutz als staatliche Bildungsaufgabe und wollen alle Bildungsträger in Rheinland-Pfalz in diese Aufgabe einbeziehen. Aufklärung über Datenschutz ist nicht nur Aufgabe der Schulen, sondern auch der politischen Bildungseinrichtungen, der Volkshochschulen, der Ausbildungseinrichtungen und anderer Bildugsstätten.

Die Menschen müssen in der Lage sein, die Bedeutung der Privatsphäre für eine freiheitliche Gesellschaft und ein selbstbestimmtes Leben zu erkennen und frühzeitig über die Gefahren aufgeklärt werden, die von Staat, Wirtschaft und von unachtsamer Datenpreisgabe ausgehen. Der verantwortungsvolle Umgang mit eigenen Daten und den Daten Dritter muss vermittelt werden.

Die Auskunfts-, Änderungs- und Löschansprüche, welche die Datenschutzgesetze einräumen, sind vielen Menschen nicht bekannt. Wir werden durch Informationskampagnen und Hilfsangebote dafür sorgen, dass diese Rechte wahrgenommen werden können.

\subsubsection{Selbstdatenschutz durch Information und Transparenz}
\abstimmung
Bürger müssen umfassend über Datenerhebungen und -verarbeitung informiert werden um ihre Rechte wahrnehmen zu können. Deshalb werden wir datenverarbeitende Unternehmen zu mehr Transparenz verpflichten: Kunden müssen klar und deutlich über das Ausmaß und den Zweck von Datensammlung und -verarbeitung aufgeklärt und über die Konsequenzen informiert werden. Nur so ist gewährleistet, dass die Betroffenen ihre Daten tatsächlich freiwillig und bewusst herausgeben.

\subsubsection{Informationelle Selbstbestimmung in sozialen Netzwerken}
\abstimmung
Insbesondere junge Menschen nutzen vermehrt Soziale Netzwerke im Internet um sich mit Freunden auszutauschen, neue Kontakte zu knüpfen und gemeinsamen Interessen nachzugehen. Der Datenschutz wird in vielen dieser Netzwerke jedoch sträflich vernachlässigt.

Wir werden die gesetzlichen Rahmenbedingungen schaffen, damit jeder unbeschwert und ohne Angst vor Datenmissbrauch oder Cyber-Mobbing an diesen Netzwerken teilhaben kann. Wir werden für eine wirksame Durchsetzung der informationellen Selbstbestimmung in diesen Netzwerken sorgen. Jeder Nutzer muss zu jeder Zeit die Kontrolle darüber behalten, wer welche Daten einsehen darf. Die Nutzung von personenbezogenen Daten durch die Betreiber, ohne explizite Einwilligung des Nutzers werden wir unterbinden.

\subsection*{Datenschutz auf Bundesebene} 
\wahlprogramm{Datenschutz auf Bundesebene}\label{datenschutz:bundesebene}
\antrag{Unglow}\version{22:51 18. Jun. 2010}
\subsubsection{Modul 1}
\abstimmung
Die Piratenpartei Rheinland-Pfalz wird sich im Bundesrat für eine Verbesserung des Datenschutzes auf Bundesebene stark machen. Insbesondere werden wir die Einführung eines Arbeitnehmerdatenschutzgesetzes voran treiben. Wir werden uns im Bundesrat gegen datenschutzfeindliche Gesetze wie z.B. eine neue Vorratsdatenspeicherung stellen und uns dafür einsetzen, Gesetze wie das BKA-Gesetz und ELENA zu entschärfen bzw. datenschutzgerecht umzugestalten.

\subsection*{Datenschutz auf Landesebene}
\wahlprogramm{Datenschutz auf Landesebene}\label{datenschutz:landesebene}
\antrag{Unglow}\version{22:51 18. Jun. 2010}
\subsubsection{Datenschutz auf Landesebene}
\abstimmung
Ein wesentlicher Grundsatz des Datenschutzes ist die Datensparsamkeit. Diese besagt, dass nur so wenige Daten wie notwendig gesammelt werden sollen. Um diesem Grundsatz gerecht zu werden, wird die Piratenpartei Rheinland-Pfalz alle vom Land erfassten Daten auf ihre Notwendigkeit und Zweckmäßigkeit hin überprüfen.

\subsubsection{Google Analytics in der Landesverwaltung}
\abstimmung
Die Piratenpartei Rheinland-Pfalz wird den illegalen Einsatz der Software "Google Analytics" in der Landesverwaltung stoppen. Laut dem Tätigkeitsbericht des Landesdatenschutzbeauftragten wird das Programm von einigen Stellen eingesetzt, obwohl dessen Betrieb gegen deutsches Datenschutzrecht verstößt. Wir werden sicherstellen, dass jeder Bürger die Webseiten des Landes nutzen kann, ohne dabei ausspioniert zu werden.

\subsubsection{Meldedaten}
\konkurrenz{sec:datenschutz:weitergabe}
\abstimmung
Einige Stadtverwaltungen in in Rheinland-Pfalz bessern ihr Budget auf, indem sie die persönliche Daten ihrer Bürger weiterverkaufen. Kirchen, Parteien, die GEZ, Banken und noch viele weitere Unternehmen kommen auf diese Weise an Datensätze. Bürger, die dies nicht wollen, können dem Datenverkauf widersprechen (sogenanntes Opt-Out), was allerdings mit Aufwand verbunden ist und vielen Menschen nicht bekannt ist. In der Regel erfährt man nur auf konkrete Nachfrage von diesem Datenhandel und der Möglichkeit des Widerspruchs. Die Piratenpartei Rheinland-Pfalz hält dieses Vorgehen für nicht akzeptabel. Wir werden deshalb ein Opt-In-Verfahren einführen, bei dem der Bürger der Herausgabe seiner Daten bewusst zustimmen muss und explizit angibt, welchen Gruppen der Zugang zu seinen Daten gestattet werden soll; der Verkauf an andere Gruppen wird unterbunden.

\subsubsection{Behördliche Datenschutzbeauftragte}
\abstimmung
Neben dem Landesbauftragten für den Datenschutz sind die behördlichen Datenschutzbeauftragten ein wichtiges Organ um den Datenschutz im Land zu gewährleisten. Leider haben sie für diese verantwortungsvolle Aufgabe oft zu wenig Zeit zur Verfügung. Die behördlichen Datenschutzbeauftragten sollen sich nach unserer Auffassung Vollzeit um ihre Aufgabe kümmern können und in alle datenschutzrelevanten Vorhaben einbezogen werden. Außerdem wollen wir die Vernetzung und den Austausch zwischen Landes- und behördlichen Datenschutzbeauftragten fördern.

\subsubsection{Auskunftsrecht gegenüber der Verwaltung}
\abstimmung
Jeder Bürger hat ein Recht auf Auskünft über die zu seiner Person gespeicherten Daten. Dieses Recht gilt auch gegenüber der Verwaltung. Trotzdem kommt es immer wieder vor, dass der Staat aus einem vermeintlichen Geheimhaltungsinteresse dieses Recht untergräbt. Die Piratenpartei wird durchsetzen, dass alle Bürger auch gegenüber der Landesverwaltung einen durchsetzbaren und unentgeltlichen Anspruch auf Selbstauskunft und gegebenenfalls auf Korrektur, Sperrung oder Löschung von unrichtigen oder unrechtmäßig gespeicherten Daten haben.

\subsubsection{Datenschutz bei der Gesetzgebung}
\abstimmung
Datenschutz ist mehr als ein politisches Thema. Die Verarbeitung persönlicher Daten durchdringt heute alle gesellschaftlichen Bereiche. Bei fast allen Gesetzen spielen persönliche Daten der Bürger eine Rolle. Die Piratenpartei wird deshalb sicherstellen, dass der Datenschutz in allen Bereichen der Gesetzgebung mit einbezogen und von vorne herein geachtet wird.
 
\newpage
\wahlprogramm{Datenherausgabe durch Bürgerämter nur nach Zustimmung}\label{sec:datenschutz:weitergabe}
\antrag{KV Trier/Trier-Saarburg}\konkurrenz{datenschutz:landesebene} - Meldedaten \version{22:51 18. Jun. 2010}
\abstimmung
Eine Weitergabe von Informationen über Bürger ohne deren Einwilligung lehnen wir ab.

Privatpersonen, Firmen, Kirchen, Parteien und andere Einrichtungen fordern von Bürgerämtern gegen geringe Gebühren Daten über Bürger ohne deren Einwilligung an, um diese zu privaten oder kommerziellen Zwecken zu verwenden. Diese Praxis widerspricht dem Grundrecht auf Informationelle Selbstbestimmung. Stattdessen muss in Zukunft sichergestellt sein, dass die Erlaubnis der Bürger eingeholt wurde, bevor Informationen über sie herausgegeben werden. Wurde diese Erlaubnis erteilt, soll der Bürger auf Anfrage Informationen über die getätigten Abfragen erhalten und seine Erlaubnis jederzeit widerrufen können.

\subsection*{Datenschutz in der Wirtschaft}
\wahlprogramm{Datenschutz in der Wirtschaft}
\antrag{Unglow}\version{22:51 18. Jun. 2010}
\abstimmung
\subsubsection{Modul 1}
Die bestehenden Datenschutzgesetze können den Datenschutz in der digitalen und vernetzten Welt des 21. Jahrhunderts nicht mehr gewährleisten. Datenskandale häufen sich, Firmen spionieren ihre Mitarbeiter aus, sensible Kundendaten gelangen in die Hände von Kriminellen. Der Datenhandel blüht: Permanent werden persönliche Daten von Millionen von Bundesbürgern gehandelt, ohne dass der Staat gesetzliche Rahmenbedingungen schafft, die jedem Bürger ermöglichen, an der Gesellschaft des 21. Jahrhunderts teilzuhaben, ohne zum gläsernen Bürger zu werden.
 
\wahlprogramm{Betriebliche Datenschutzbeauftragte}\label{datenschutz:beauftragte}
\antrag{Unglow}\version{22:51 18. Jun. 2010}
\subsubsection{Modul 1}
\abstimmung
Viele Unternehmen in Rheinland-Pfalz haben derzeit keinen betrieblichen Datenschutzbeauftragten eingesetzt, obwohl sie dazu verpflichtet sind. Obwohl der Landesdatenschutzbeauftragte dies bemängelt gibt es erhebliche Defizite. Die Sanktionen müssen in diesem Bereich verstärkt werden, sodass der Datenschutz als Sparmaßnahme für Unternehmen nicht mehr in Frage kommt.

\subsubsection{Modul 2}
\abstimmung
Darüber hinaus wollen wir den Kündigungsschutz der betrieblichen Datenschutzbeauftragten stärken und ihnen Fort- und Weiterbildungsmaßnahmen ermöglichen. Die Einsatzbereitschaft und die Durchsetzungsfähigkeit der betrieblichen Datenschutzbeauftragten trägt oftmals mehr zum Datenschutzniveau eines Unternehmens bei als gesetzliche Vorgaben. Die betrieblichen Datenschutzbeauftragten müssen deshalb in die Lage versetzt werden, das Datenschutzbewusstsein in der Wirtschaft zu stärken und wirksam durchzusetzen.
 
\wahlprogramm{Schutz von Unternehmensdatenbanken}\label{datenschutz:dbs}
\antrag{Unglow}\version{22:51 18. Jun. 2010}
\subsubsection{Modul 1}
\abstimmung
Der Landesdatenschutzbeauftragte bemängelt den Schutz von Unternehmensdaten als vielfach unzureichend. Die mangelhafte Absicherung und Zugriffskontrolle erleichtert es, Daten illegal auszulesen und weiterzugeben. Die vohandenen Strafbestimmungen reichen aus Sicht der PIRATEN offenbar nicht aus, um den wirksamen Schutz von Kundendaten zu gewährleisten. Die missbräuchliche Verwendung von Daten im Rahmen eingeräumter Zugriffsrechte muss ebenso unter Strafe gestellt werden wie der unbefugte Zugriff durch Dritte.

\subsubsection{Modul 2}
\abstimmung
Um Datenmissbrauch und aufdecken zu können muss protokolliert werden, wer wann welche personenbezogenendaten in welcher Weise verarbeitet hat. Diese Daten müssen beweissicher gespeichert und wiederum gegen unbefugten Zugriff und Manipulationen gesichert sein. Die Protokolldaten unterliegen ihrerseits einer strikten Zweckbindung und dürfen nicht für die Verhaltens- und Leistungskontrolle von Angestellten verwendet werden.
 
\wahlprogramm{Datenschutz der Arbeitnehmer}\label{datenschutz:arbeitnehmer}
\antrag{Unglow}\version{22:51 18. Jun. 2010}
\subsubsection{Modul 1}
\abstimmung
Im Arbeitsverhältnis werden zahlreiche persönliche und hochsensible Daten über Beschäftigte gesammelt und verarbeitet. Durch Informationstechnik können diese Daten fast ohne Aufwand zusammengeführt uns ausgewertet werden. Dies bietet Arbeitgebern neue, bedenkliche Kontroll- und Überwachungsmöglichkeiten, die den Persönlichkeitsrechten der Arbeitnehmer entgegenstehen. Der Gesetzgeber hat bislang versäumt, dieser Entwicklung Rechnung zu tragen. Die Piratenpartei will diesen Missstand beseitigen und ein Arbeitnehmerdatenschutzgesetz verabschieden.

\subsubsection{Modul 2}
\abstimmung
Konkret werden wir u.a.:
\begin{itemize}
\item klar festlegen, welche Daten Unternehmen und öffentliche Stellen im Rahmen des Einstellungsverfahrens und des Arbeitsverhältnisses sammeln und verarbeiten dürfen
\item anlass- und verdachtslose Abgleiche von Personaldaten unterbinden
\item die Verhaltens- und Leistungskontrolle von Arbeitnehmern strikt begrenzen
\item die Achtung der Persönlichkeitsrechte beim Einsatz von Videoüberwachung, Ortungssystemen und sonstigen Überwachungssystemen gewährleisten
\item den Beschäftigten umfassende Auskunfts-, Benachrichtigungs-, Widerrufs- und Löschrechte einräumen
\item den Schutz der in Deutschland tätigen Arbeitnehmer internationaler Unternehmen durchsetzen
\item die Kontrolle durch betriebliche Datenschutzbeauftragte sicherstellen
\item wirksame Sanktionen bei Verstößen gegen den Arbeitnehmerdatenschutz einführen
\end{itemize}
 
\wahlprogramm{Datenhandel unterbinden}
\antrag{Unglow}\version{22:51 18. Jun. 2010}
\subsubsection{Alternative A}
\abstimmung
Die Piratenpartei wird den Handel mit personenbezogenen Daten verbieten.

\abstimmung{Alternative B}
Die Piratenpartei wird den Handel mit personenbezogenen Daten einschränken. Wir werden den Handel mit Daten nur nach ausdrücklicher Zustimmung des Betroffenen erlauben. Diese Zustimmung muss unabhängig von sonstigen vertraglichen Vereinbarungen sein und darf auf letztere keinen Einfluss haben. Insbesondere darf ein Vertragsabschluss nicht von der Zustimmung zum Datenhandel abhängig gemacht werden.
 
\newpage
\wahlprogramm{Datensparsamkeit in Unternehmen}
\antrag{Unglow}\version{22:51 18. Jun. 2010}
\subsubsection{Modul 1}
\abstimmung
Wir wollen Unternehmen zur Datensparsamkeit verpflichten. Es darf nicht sein, dass die Angabe von nicht benötigten private Daten zur Voraussetzung werden, um Bestellungen zu tätigen, Software zu benutzen oder Vereinen und Gemeinschaften beizutreten. Zudem fordern wir Opt-In Lösungen, bei denen jeder der weiteren Verwendung seiner Daten oder auch zum Beispiel dem Erhalt von Werbung ausdrücklich zustimmen muss und nicht nur nachträglich widersprechen kann.
 
\wahlprogramm{Illegale Entsorgung von Daten}
\antrag{Unglow}\version{22:51 18. Jun. 2010}
\subsubsection{Modul 1}
\abstimmung
So wie die illegale Müllentsorgung unsere Umwelt schädigt, ist die illegale Entsorgung von Daten eine Bedrohung für die Privatsphäre von Kunden, Patienten, Arbeitnehmern und vielen weiteren Bürgern.

Um die ordnungsgemäße Vernichtung von privaten Daten bei der Entsorgung zu gewährleisten, werden wir dem Landesdatenschutzbeauftragten mehr Kontrollbefugnisse in diesem Bereich zusprechen und die Strafen für illegale Datenentsorgung erhöhen. Dies werden wir mit einer Aufklärungskampagne zur korrekten Entsorgung von Daten verbinden.
 
\subsection*{Sicherheitspolitik unter Achtung der Bürgerrechte}
\wahlprogramm{Sicherheitspolitik}
\antrag{Unglow}\version{22:51 18. Jun. 2010}
\subsubsection{Modul 1}
\abstimmung
Mit einer erschreckenden Geschwindigkeit wurde das Recht auf Privatsphäre in den letzten Jahren zu Gunsten einer unwirksamen Sicherheitspolitik eingeschränkt.

Systeme und Methoden, die der Staat gegen seine Bürger einsetzen kann, müssen der ständigen Bewertung und genauen Prüfung durch gewählte Mandatsträger unterliegen. Wenn die Regierung Bürger beobachtet, ohne dass sie eines Verbrechens verdächtig sind, ist dies eine fundamental inakzeptable Verletzung des Bürgerrechts auf Privatsphäre.

\subsubsection{Modul 2}
\abstimmung
Die pauschale Verdächtigung und anlasslose Überwachung aller Bürger hat generell zu unterbleiben. Eine als 'präventive Strafverfolgung' verschleierte Abschaffung der Unschuldsvermutung lehnen wir unbedingt ab.
 
\wahlprogramm{Vertrauliche Kommunikation}
\antrag{Unglow}\version{22:51 18. Jun. 2010}
\subsubsection{Modul 1}
\abstimmung
Das Briefgeheimnis soll erweitert werden zu einem generellen Kommunikationsgeheimnis, das die grundgesetzlich geschützte Privatheit und Integrität von Kommunikation auch in elektronischen Medien wie dem Internet garantiert. Zugriff auf die Kommunikationsmittel oder die Überwachung eines Bürgers darf Ermittlungsbehörden nur im Falle eines begründeten und konkreten Tatverdachtes erlaubt werden, dass dieser Bürger ein Verbrechen plant oder begangen hat. In jedem Fall ist ein richterlicher Beschluss erforderlich. In allen anderen Fällen muss der Staat annehmen, seine Bürger seien unschuldig. Diesem Kommunikationsgeheimnis muss ein starker gesetzlicher Schutz gegeben werden, da Regierungen wiederholt gezeigt haben, dass sie bei sensiblen Informationen nicht vertrauenswürdig sind.

\subsubsection{Modul 2}
\abstimmung
Speziell eine verdachtsunabhängige Vorratsdatenspeicherung von Kommunikationsdaten widerspricht nicht nur der Unschuldsvermutung, sondern auch allen Prinzipien einer freiheitlich demokratischen Gesellschaft. Der vorherrschende Kontrollwahn stellt eine weitaus ernsthaftere und langfristigere Bedrohung unserer Gesellschaft dar als der internationale Terrorismus und erzeugt ein Klima des Misstrauens und der Angst. Flächendeckende staatliche Überwachung, fragwürdige Rasterfahndungen und zentrale Datenbanken mit unbewiesenen Verdächtigungen (Anti-Terror-Datei) sind Mittel, deren Einsatz wir grundsätzlich ablehnen.
 
\wahlprogramm{Datensparsame Sicherheitspolitik}
\antrag{Unglow}\version{22:51 18. Jun. 2010}
\subsubsection{Modul 1}
\abstimmung
Der Staat sammelt selbst hemmungslos Daten über seine Bürger und vernetzt die gesammelten Daten zunehmend miteinander, was zu einem gläsernen Bürger führt. Mit der Steuer-ID, die jeder Bundesbürger mit seiner Geburt erhält und die erst 20 Jahre nach seinem Tod gelöst wird, sind die Daten der Bürger nun einfacher zu verarbeiten und können so besser vernetzt werden, was das Erstellen eines umfangreichen Profils zu jedem Bürger erleichtert. Mit Hilfe der Vorratsdatenspeicherung sollten die Kommunikationsdaten eines jeden Deutschen, der per Telefon oder Internet kommuniziert, überwacht und gespeichert werden, ohne dass ein Verdacht besteht. Diese Maßnahme wurde vorübergehend sogar vom Bundesverfassungsgericht gekippt. Mit der ELENA-Datenbank werden mittlerweile alle Arbeitnehmerdaten zentral erfasst. In den Bundesländern wird auch über neue Datenbanken, wie zum Beispiel die Einführung einer Schüler-ID, diskutiert. Gesetze wie die Vorratsdatenspeicherung, bei der Bürger verdachtsunabhängig überwacht werden, lehnt die Piratenpartei grundsätzlich ab. Auch andere Maßnahmen der verdachtsunabhängigen Massenüberwachung wie Kennzeichenscanner lehnen wir ab. Den Einsatz solcher Maßnahmen in Rheinland-Pfalz wird die Piratenpartei verhindern.
 
\wahlprogramm{Sicherheitspolitik}
\antrag{KV Trier/Trier-Saarburg}\version{22:51 18. Jun. 2010}
\subsubsection{Biometrische Daten}
\abstimmung
Wir lehnen die Erfassung biometrischer Daten ohne Anfangsverdacht sowie deren Speicherung ohne nachgewiesene Straftat kategorisch ab.

\subsubsection{Keine automatisierte Kennzeichenerfassung}
\abstimmung
Obwohl das Bundesverfassungsgericht eindeutig klargestellt hat, dass eine verdachtsunabhängige, flächendeckende, automatisierte Kennzeichenerfassung zum Abgleich mit Fahndungsdaten nicht mit dem Grundgesetz vereinbar ist, wird dieses erneut diskutiert. Einen solchen Eingriff in die Persönlichkeitsrechte lehnen wir entschieden ab.

Wir gehen sogar weiter als das Verfassungsgericht: Auch ein stichprobenartiger Abgleich ist für uns nicht akzeptabel.
 
\wahlprogramm{Echte Sicherheitspolitik auf Basis von Fakten}
\antrag{Unglow}\version{22:51 18. Jun. 2010}
\subsubsection{Modul 1}
\abstimmung
Die Bekämpfung der Kriminalität ist eine wichtige staatliche Aufgabe. Sie ist nach unserer Überzeugung nur durch eine intelligente, rationale und evidenzbasierte Sicherheitspolitik auf der Grundlage wissenschaftlicher Erkenntnisse zu gewährleisten. Um sinnvolle Sicherheitsmaßnahmen zu fördern und schädliche Maßnahmen beenden zu können, wollen wir alle bestehenden Befugnisse und Programme der Sicherheitsbehörden systematisch und nach wissenschaftlichen Kriterien überprüfen auf Wirksamkeit, Kosten, schädliche Nebenwirkungen, auf Alternativen und auf ihre Vereinbarkeit mit den Menschen- und Bürgerrechten.

\subsubsection{Modul 2}
\abstimmung
Wir wollen, dass künftig jeder Vorschlag für neue Sicherheitsmaßnahmen noch im Entwurfsstadium von der Europäischen Grundrechteagentur oder einer entsprechenden deutschen Einrichtung auf diese Kriterien hin begutachtet wird. Nur durch einen solchen "Gesetzes-TÜV" kann weiteren verfassungswidrigen Angriffen auf unsere Grundrechte frühzeitig entgegen gewirkt werden. Der Grundrechteagentur müssen dafür alle nötigen finanziellen und personellen Ressourcen zur Verfügung gestellt werden.

\subsubsection{Modul 3}
\abstimmung
Um den fortschreitenden Abbau der Bürgerrechte seit 2001 zu stoppen, fordern wir ein Moratorium für weitere Grundrechtseingriffe im Namen der inneren Sicherheit ein, solange nicht die systematische Überprüfung der bestehenden Befugnisse abgeschlossen ist.

\subsubsection{Modul 4}
\abstimmung
Zur Gewährleistung der Freiheitsrechte und zur Sicherung der Effektivität von Gefahrenabwehr und Strafverfolgung treten wir dafür ein, dass eine staatliche Informationssammlung, Kontrolle und Überwachung künftig nur noch gezielt bei Personen erfolgt, die einer Straftat konkret verdächtigt sind. Zum Schutz unserer offenen Gesellschaft und im Interesse einer effizienten Sicherheitspolitik wollen wir auf anlasslose, massenhafte, automatisierte Datenerhebungen, Datenabgleichungen und Datenspeicherungen verzichten. In einem freiheitlichen Land ist eine derart breite Erfassung beliebiger Personen ohne Anlass und Verdacht inakzeptabel.

\subsubsection{Modul 5}
\abstimmung
Die Sicherheitsforschung aus Steuergeldern wollen wir demokratisieren und an den Bedürfnissen und Rechten der Bürgerinnen und Bürger ausrichten. In beratenden Gremien sollen künftig neben Verwaltungs- und Industrievertretern in gleicher Zahl auch Volksvertreter sämtlicher Fraktionen, Kriminologen, Opferverbände und Nichtregierungsorganisationen zum Schutz der Freiheitsrechte und Privatsphäre vertreten sein. Eine Entscheidung über die Ausschreibung eines Projekts soll erst getroffen werden, wenn eine öffentliche Untersuchung über die Auswirkungen des jeweiligen Forschungsziels auf unsere Grundrechte (impact assessment) vorliegt.

\subsubsection{Modul 6}
\abstimmung
Die Entwicklung von Technologien zur verstärkten Überwachung, Erfassung und Kontrolle von Bürgerinnen und Bürgern lehnen wir ab. Stattdessen muss die Sicherheitsforschung auf sämtliche Optionen zur Kriminal- und Unglücksverhütung erstreckt werden und eine unabhängige Untersuchung von Wirksamkeit, Kosten, schädlichen Nebenwirkungen und Alternativen zu den einzelnen Vorschlägen zum Gegenstand haben.

\subsubsection{Modul 7}
\abstimmung
Weil auch die gefühlte Sicherheit eine wichtige Voraussetzung für unser Wohlbefinden ist, wollen wir zudem erforschen lassen, wie das öffentliche Sicherheitsbewusstsein gestärkt und wie verzerrten Einschätzungen und Darstellungen der Sicherheitslage entgegen gewirkt werden kann.
 
\wahlprogramm{Polizei- und Ordnungsbehördengesetz}
\antrag{Unglow}\version{22:51 18. Jun. 2010}
\subsubsection{Modul 1}
\abstimmung
Wir lehnen jegliche Versuche ab, durch eine Neufassung des Polizei- und Ordnungsbehördengesetzes Online-Durchsuchungen zu erlauben, Rasterfahndungen zu legitimieren oder Befugnisse zur Telekommunikationsüberwachung zu verschärfen. Stattdessen setzen wir uns für eine wissenschaftliche Evaluation aller bestehenden Sicherheitsbefugnisse ein.
 
\wahlprogramm{Eindeutige, gut lesbare Kennzeichnung von Polizisten}\label{datenschutz:polizei}
\antrag{Piraten aus RLP}\konkurrenz{datenschutz:eindeutig}\version{22:51 18. Jun. 2010}
\abstimmung
Polizisten sollen bei Einsätzen in Gruppen eine eindeutige, gut lesbare Identifikationsnummer tragen, um Übergriffe durch Polizisten nachvollziehen und aufklären zu können. Für den Fall unverhältnismäßiger Gewaltanwendung durch Polizisten oder anderer gesetzeswidriger Handlungen muss sichergestellt werden, dass eine spätere Identifikation von Sicherheitskräften möglich ist.
 
\wahlprogramm{Eindeutige Kennzeichnung von Polizisten}
\antrag{KV Trier/Trier-Saarburg}\konkurrenz{datenschutz:polizei}\version{22:51 18. Jun. 2010}
\abstimmung
Bei geplanten Veranstaltungen wie Demonstrationen oder Einsätzen bei Sportereignissen sollen Polizisten eine eindeutige Identifikationsnummer tragen, um Übergriffe durch Polizisten nachvollziehen zu können.

Für den Fall unverhältnismäßiger Gewaltanwendung durch Polizisten oder anderer gesetzeswidriger Handlungen muss sichergestellt werden, dass eine spätere Identifikation von Sicherheitskräften möglich ist. Dabei ist das Gleichgewicht zwischen dem Schutz der Persönlichkeitsrechte und der Identifizierbarkeit der Polizisten zu wahren. Im Fall einer Anzeige soll erst durch einen richterlichen Beschluss die Feststellung der Identität erfolgen. Hierfür ist ein geeignetes und praktikables Verfahren zur Verteilung der Identifikationsnummern und deren Gestaltung in Zusammenarbeit mit der Polizei zu entwickeln.
 
\wahlprogramm{POLIS Datenbank}
\antrag{Unglow}\version{22:51 18. Jun. 2010}
\subsubsection{Modul 1}
\abstimmung
In der Vergangenheit kam es zu rechtswidring Zugriffen auf das polizeiliche Informationssystem POLIS. In dem System befinden sich Daten über alle Personen, die als Tatverdächtige auffällig geworden, aber nicht zwangsläufig schuldig sind. Da diese Daten in einem Rechtsstaat besonders schutzwürdig sind, werden wir dieses System grundlegend überprüfen lassen und sicherstellen, dass alle nötigen Vorkehrungen getroffen werden, um Datenmissbrauch zu verhindern.

\subsubsection{Modul 2}
\abstimmung
Im Rahmen dieser Überprüfung werden wir auch alle gespeicherten Daten hinsichtlich ihrer Notwendigkeit und Zweckmäßigkeit überprüfen. Daten die nicht unbedingt benötigt, oder anlasslos gespeichert werden sind unzulässig.
 
\newpage
\wahlprogramm{Videoüberwachung}
\antrag{Unglow}\version{22:51 18. Jun. 2010}
\subsubsection{Modul 1}
\abstimmung
Die flächendeckende Überwachung des öffentlichen Raums durch Videokameras oder andere Maßnahmen darf nicht zugelassen werden. Wir fordern ein allgemeines Verbot der Überwachung des öffentlichen Raums, von dem nur einzelne, richterlich angeordnete Ausnahmen zulässig sind.

\subsubsection{Modul 2}
\abstimmung
Die anlasslose und pauschale Videoüberwachung im öffentlichen Raum dient lediglich der gefühlten Sicherheit und dringt unverhältnismäßig in die Privatsphäre der Menschen ein. Wir werden stattdessen wirksame Maßnahmen durchsetzen. Wir lehnen jegliche Pläne zum Ausbau der Videoüberwachung zum Beispiel an Bushaltestellen oder Schulen strikt ab. Kameras tragen nicht zum Abbau sondern höchstens zur Verlagerung von Kriminalität bei und bieten Opfern keinen Schutz. Die Kosten für die Installation und die Überwachung der Kameras stehen in keiner Relation zum Nutzen. Eine Neuorientierung hin zu effektiven Lösungen wie besserer Straßenbeleuchtung und mehr Polizeistreifen ist dringend erforderlich und wird von uns vorangetrieben.

\subsubsection{Modul 3}
\abstimmung
Wir lehnen insbesondere den allgemeinen, präventiven, behördlichen Einsatz von Überwachungstechnologie während Demonstrationen ab, da dieser die Versammlungsfreiheit und freie Meinungsäußerung massiv einschränkt.
 
\wahlprogramm{Versammlungsfreiheit}
\antrag{Unglow}\version{22:51 18. Jun. 2010}
\subsubsection{Modul 1}
\abstimmung
Die Möglichkeit zur Organisation von und Teilnahme an Versammlungen ist ein wichtiges Grundrecht. In anderen Bundesländern wurde dieses Recht durch Änderungen am Versammlungsgesetz erheblich eingeschränkt. Jeglichen Plänen die Versammlungsfreiheit in Rheinland-Pfalz ebenfalls einzuschränken stellen wir uns entschieden entgegen.

\subsubsection{Modul 2}
\abstimmung
Immer häufiger wird seitens der Polizei im Vorfeld von und während Demonstrationen Kontrolle und Überwachungsdruck ausgeübt. Wir lehnen anlassunabhägige Kontrollen und Durchsuchungen von Menschen und Fahrzeugen entschieden ab und setzen uns für entsprechende gesetzliche Änderungen ein, die dies verbieten. Wir fordern Freiheit statt Angst und den Schutz der Menschen vor Einschüchterung durch den Staat bei Wahrnehmung ihrer Rechte.

\subsubsection{Modul 3}
\abstimmung
Die Überwachung von Demonstrationen mit Foto- oder Videokameras oder ähnlichen Instrumenten lehnen wir ab. Überwachung auf Demonstrationen gefährdet die Meinungs- und Versammlungsfreiheit und damit unsere Demokratie. Wir wollen das Versammlungsrecht stärken und verdachts- und anlassunabhängige Überwachungsmaßnahmen stärker kontrollieren und sanktionieren. Jede polizeiliche Überwachungsmaßnahme muss vollständig dokumentiert und begründet werden und dem Landesdatenschutzbeauftragten zur Kontrolle übermittelt werden.

\subsubsection{Modul 4}
\abstimmung
In der Vergangenheit kam es zu Situationen, in denen Polizisten auf Demonstrationen die Rechte von Bürgern missachtet haben. Bei Beschwerden gestaltete sich die Aufklärung als schwierig. Polizisten sollten gegen Kollegen ermitteln oder aussagen. Um solche Fälle zukünftig besser aufklären zu können, fordern wir die Einrichtung einer von der Polizei unabhängigen Beschwerdestelle. Diese muss auch das anonyme Melden von Fehlverhalten durch Kollegen möglich machen, die sich aktuell nicht trauen, Beobachtungen rechtswidrigen Verhaltens zur Anzeige zu bringen.

%%\section{Transparenz}
\subsection*{Informationsfreiheit im 21. Jahrhundert - Offene Daten für mündige Bürger!}

\wahlprogramm{Staatliche Daten veröffentlichen}\label{transparenz:daten}
\antrag{Unglow}\version{04:02, 20. Jun. 2010}
\subsubsection{Modul 1}
\abstimmung
Der Zugang zu Wissen und Information ist die Grundlage für unsere freiheitlich-demokratische Informations- und Wissensgesellschaft. Wir PIRATEN setzen uns daher für eine Stärkung der Informationsfreiheit und einen freien und offenen Zugang zu allen staatlichen und staatlich geförderten Informationsbeständen ein.

\subsubsection{Modul 2}
\abstimmung
Sämtliche staatlichen Daten müssen grundsätzlich der Öffentlichkeit und damit jedermann frei zugänglich gemacht werden. Unter staatlichen Daten verstehen wir alle staatlichen und staatlich finanzierten Informationen, ausgenommen personenbezogene Daten und ggf. wenige klar zu definierende und begründende Ausnahmefälle. Diese Ausnahmeregelungen sind möglichst eng und eindeutig zu formulieren und dürfen nicht pauschal ganze Behörden oder Themengebiete ausgrenzen.
 
\wahlprogramm{Demokratie durch Transparenz}
\antrag{Piraten aus RLP}\version{04:02, 20. Jun. 2010}
\subsubsection{Transparenz des Staatswesens und Lobbyismus}
\abstimmung
Die politische Arbeit wird in Deutschland stark von Lobbyinteressen gesteuert. Unternehmensvertreter nehmen unbemerkt Einfluss auf Politiker und arbeiten sogar an Gesetzen mit. Abhängigkeiten zwischen Unternehmen und Politikern müssen aufgedeckt werden. Abgeordnete sollen ihre Nebentätigkeiten und die gegebenenfalls daraus resultierenden Einkünfte veröffentlichen. Abgeordnete der Piratenpartei werden mit gutem Beispiel vorangehen und dies mit dem Einzug ins Parlament offenlegen. Dem Bürger muss klar ersichtlich sein, welche Interessen hinter Gesetzesinitiativen stecken und wer, wie und wann auf den Gesetzgebungsprozess Einfluss genommen hat.

Zu einem transparenten Staat gehören neben den Regelungen zu Lobby- und Nebentätigkeiten von Parlamentariern und Amtsträgern auch die gelebte Verpflichtung, Entscheidungsfindungsprozesse für den Bürger wahrnehmbar und nachvollziehbar öffentlich zu machen, wie auch Verordnungen, Diskussionspapiere und Vertragswerke so zu gestalten, dass diese so kurz wie nötig, so sprechend wie möglich und für den Bürger verständlich gehalten sind. Wir lehnen geheime Ausschüsse ab.

Gleichzeitig müssen die Interessen der Bürger besser vertreten werden. Zudem sollen Nichtregierungsorganisationen gefördert werden, die für die Rechte und Interessen der Bürger eintreten.

\subsubsection{Höchste demokratische Standards für Deutschland}
\abstimmung
Die PIRATEN treten ein, für eine nachvollziehbare und transparente Politik und Verwaltung in Deutschland. Deutschland sollte sich an die höchsten demokratischen Standards halten und innerhalb Europas eine vorbildliche Rolle diesbezüglich anstreben. Deshalb sollten solche Prinzipien wie transparente Staatsführung, schnelle und gerechte Gerichtsverfahren und die Redefreiheit stets beachtet werden. In diesen Tagen und in dieser Zeit ist es wesentlich, den gesetzlichen Schutz der Bürger vor willkürlichen Staatszugriffen weiterhin durchzusetzen.

\subsubsection{Gläsener Staat statt Gläserner Bürger}
\abstimmung
Eine demokratische Gesellschaft braucht einen transparenten Staat und keine gläsernen Bürger. Die Bürger müssen die Möglichkeit haben, sich frei und unbeobachtet zu versammeln, und ihre Meinung ohne Furcht vor staatlicher Überwachung ausdrücken zu können. Um dies in die Informationsgesellschaft zu übertragen, muss das Recht auf anonyme Kommunikation ausgebaut werden. Deswegen muss das Korrespondenzgeheimnis auf digitale Kommunikation ausgeweitet werden.

\subsubsection{Offenlegung von Nebeneinkünften und Nebentätigkeiten von Amts- und Mandatsträgern}
\abstimmung
Mandatsträger und Ausübende politischer Ämter müssen zur Offenlegung sämtlicher Nebeneinkünfte und Nebentätigkeiten verpflichtet sein. Nur wenn der Bürger weiß von wem die genannten Personen bezahlt werden und für wen sie arbeiten kann er sich ein vollständiges Bild über deren Unabhängigkeit oder ggf. deren Abhängigkeit machen. Die Offenlegungspflicht soll auch für unentgeltliche [nicht private) Nebentätigkeiten, wie Ehrenämter in Vereinen und Verbänden gelten.

\subsubsection{Begrenzung von Neben- und Folgetätigkeiten}
\abstimmung
Amtsträger dürfen neben ihrem Amt und wenigstens 2 Jahre nach Beendigung ihrer Amtstätigkeit nicht in Unternehmen, Vereinen oder Verbänden tätig sein, die direkt durch die Amtstätigkeit betroffen sind. Auf Antrag eines Betroffenen kann die jeweilige Situation von einem Ausschuss gesondert untersucht werden, um unverhältnismäßige Eingriffe in die Berufsfreiheit zu vermeiden.
 
\subsection*{Gläserner Staat}
\wahlprogramm{Gläserner Staat}
\antrag{KV Trier/Trier-Saarburg}\zusatz{transparenz:daten}\version{04:02, 20. Jun. 2010}
\subsubsection{Gläserner Staat}
\abstimmung
Der Anspruch der Gesellschaft auf Wissen endet dort, wo die Privatsphäre beginnt. Persönlichkeitsrechte wie die informationelle Selbstbestimmung sind Grundpfeiler für die freiheitlich demokratische Grundordnung unseres Staates. Damit der Bürger seiner Kontrollpflicht dem Staat gegenüber nachkommen kann, muss dieser offen und transparent aufgestellt sein.

Die Demokratie soll gestärkt werden, indem mehr Mitwirkungsmöglichkeiten und Einblicke in die Abläufe gewährt werden. Durch Einsicht in die Staatsgeschäfte können Korruption, Bürokratie und Lobbyismus erkannt werden. Inkompetenzen und Versäumnisse werden schneller aufgedeckt.
 
\subsection*{Informationsfreiheit ist Bürgerrecht!}
\wahlprogramm{Informationsfreiheit / Prinzip der Öffentlichkeit}\label{transparenz:infofreiheit}
\antrag{Unglow}\version{04:02, 20. Jun. 2010}
\subsubsection{Modul 1}
\abstimmung
Die alte Weisheit „Wissen ist Macht“ gilt in der Informationsgesellschaft mehr denn je. Nur wer umfänglich informiert ist, kann fundierte Entscheidungen fällen. Eine umfassende Information von Bürgern und Bürgerinnen ist auch Voraussetzung für politisches Engagement und demokratische Kontrolle der vom Volk legitimierten Macht. Jeder Bürger kann staatliche Angaben selbst überprüfen, aus neuen Blickwinkeln betrachten und neue, vorher unbekannte Zusammenhänge entdecken. Dies führt zu einer Demokratisierung der Informationskanäle und erhöht die Kontrollmöglichkeiten der Zivilgesellschaft gegenüber dem Staat. Gemäß dem Mehr-Augen-Prinzip können Angaben gemeinschaftlich besser überprüft, Entscheidungen hinterfragt und kritisiert werden. Verbesserungsvorschläge können von Allen erarbeitet werden und die besten Lösungen können umgesetzt werden. Dem Missbrauch und der Willkür Einzelner wird vorgebeugt.

\subsubsection{Modul 2}
\abstimmung
Wir PIRATEN wollen daher Parlamente und Behörden und die rechtlichen Grundlagen so umgestalten, dass sie diesem gesamtgesellschaftlichen Anspruch der Informationsfreiheit für alle Bürger Rechnung tragen. Wir setzen uns dafür ein, dass sich der Staat vom Prinzip der Geheimhaltung abkehrt und ein Prinzip der Öffentlichkeit einführt, welches den mündigen Bürger in den Mittelpunkt staatlichen Handelns und Gestaltens stellt. Dies schafft nach der festen Überzeugung der Piratenpartei die unabdingbaren Voraussetzungen für eine moderne Wissensgesellschaft in einer freiheitlichen und demokratischen Ordnung.
 
\wahlprogramm{Informationsfreiheit / Keine Zensur!}
\antrag{Piraten aus RLP}\version{04:02, 20. Jun. 2010}
\subsubsection{Keine Zensur!}
\abstimmung
Die Bestrebungen etablierter Parteien, eine Inhaltsfilterung im Internet zu etablieren, lehnen wir kategorisch ab. Staatliche Kontrolle des Informationsflusses, also Zensur, ist ein Instrument von totalitären Regimen und hat in einer Demokratie nichts verloren. Der Kampf gegen rechtswidrige Angebote im Internet muss jederzeit mit rechtsstaatlichen und transparenten Mitteln geführt werden. Bereits die Etablierung einer Zensurinfrastruktur ist inakzeptabel. Die Beurteilung der Rechtswidrigkeit muss gemäß der in Deutschland geltenden Gewaltenteilung und Zuständigkeit getroffen werden.

\subsubsection{ZugangsErschwerungsGesetz aufheben!}
\abstimmung
Die PIRATEN werden sich dafür stark machen, den Irrweg des ZugangsErschwerungsGesetzes zu beenden und dieses Zensur-Gesetz aufzuheben.

\subsubsection{ZugangsErschwerungsGesetz aufheben!}\footnote{Richtiger Titel?}
\abstimmung
Auch den Jugendmedienschutzstaatsvertrag (JMStV), den die rheinlandpfälzische Landesregierung vorangetrieben hat, lehnen wir kategorisch ab, da er in unseren Augen einen völlig falschen Weg im Jugendschutz beschreitet. Wir fordern Aufklärung und die Vermittlung von Medienkompetenz an Kinder, Jugendliche und Eltern statt einer Zensur von Inhalten im Rundfunk oder Internet.
 
\wahlprogramm{Transparenz}
\antrag{KV Trier/Trier-Saarburg}\zusatz{transparenz:infofreiheit}\version{04:02, 20. Jun. 2010}
\subsubsection{Wissen ist Macht}
\abstimmung
"Wissen ist Macht" wird bislang eher als Legitimation dafür verwendet, Wissen für sich zu behalten, abzuschotten und zu monopolisieren. Eine erfolgreiche Gesellschaft des 21. Jahrhunderts muss den Satz erweitern zu "Wissen ist Macht – wenn es allen gehört". Denn eingesperrtes Wissen ist gesellschaftlich totes Wissen, nutzt zunächst nur dem, der daraus "Kapital" schlägt, wenn überhaupt. Denn noch viel häufiger liegt das Wissen verschlossen in Tresoren, weil es vergessen oder falsch verstanden wird.

Geteiltes Wissen wächst schneller als isoliertes Wissen. Die Wissenschaftsgemeinschaft weiß das schon langeund bewertet den Rang eines Forschers deshalb nach seinen Publikationen und der Häufigkeit, mit der er zitiert wird. Verbraucherschützer, Umweltschutz-Organisationen, Bündnisse für Verkehrsprojekte und viele andere Organisationen und Initiativen, die die Interessen der Menschen vertreten, warten darauf, dass die öffentliche Verwaltung ihre Informationsschätze teilt und nicht versteckt. Die Piratenpartei versteht sich als Vertreter der Wissensgesellschaft.
 
\wahlprogramm{Transparenz}
\antrag{KV Trier/Trier-Saarburg}\version{04:02, 20. Jun. 2010}
\subsubsection{Transparente Information über Großprojekte}
\abstimmung
Bei der Planung und Umsetzung von Großprojekten wie Nürburgring oder Hochmoselübergang sollen frühzeitig alle relevanten Informationen veröffentlicht werden. Daneben sollen die betroffenen Bürger angemessen und frühzeitig beteiligt werden. Bei einer Verlegung in private Rechtsformen muss diese Veröffentlichungspflicht weiterhin gewährleistet sein. Wir wollen eine offenere Kommunikation bei der Planung und Umsetzung von Großprojekten anstoßen.
 
\subsection*{Moderne Verwaltung mit offenen Daten!}
\wahlprogramm{Offene Daten}
\antrag{Unglow}\version{04:02, 20. Jun. 2010}
\subsubsection{Modul 1}
\abstimmung
Staatliche Daten, wie Wetter- und Geodaten, Verkehrs- und Einwohnerstatistiken, müssen allen Bürgern zur Verfügung stehen und dürfen nicht länger großen Teilen der Gesellschaft vorenthalten werden. Die heutige Informationspolitik schließt wertvolle Daten in Aktenschränken oder nicht allgemein verarbeitbaren Dateiformaten ein. Bürger bekommen wichtige Informationen nur auf Nachfrage. Wir wollen das Potential der weltweiten Vernetzung ausschöpfen und werden deshalb offene Schnittstellen zum Abruf dieser Daten für jedermann einführen.

\subsubsection{Modul 2}
\abstimmung
Die modernen Informationstechnologien machen eine proaktive, zeitnahe Veröffentlichung und Verbreitung von staatlichen Informationen in offenen und strukturierten Datenformaten kostengünstig und schnell möglich. Die Piratenpartei tritt dafür ein, dass alle staatlichen Stellen von diesen Möglichkeiten Gebrauch machen, statt der Verbreitung dieser Informationen Steine in den Weg zu legen. Wir wollen durchsetzen, dass Rohdaten in maschinenlesbaren Formaten bereitgestellt werden, die eine schrankenlose Weiterverarbeitung durch Nicht-Regierungsorganisationen, Forschungseinrichtungen und interessierte Bürger zulassen.

\subsubsection{Modul 3}
\abstimmung
Eine Veröffentlichung von Daten in Rohform und der Zugriff über offene Schnittstellen ermöglicht vielfältige Anwendungen. Die Piratenpartei betrachtet daher die Veröffentlichung von staatlichen Informationen in offenen, strukturierten Formaten als ein wesentliches Merkmal eines demokratischen Informationszeitalters. Open-Data- und Semantic-Web-Initiativen, welche für die Veröffentlichung von strukturierten Daten eintreten, wollen wir deshalb explizit fördern. Ebenso wollen wir den Einsatz freier Software in allen Einrichtungen des Landes forcieren. Langfristige Verträge mit Monopolisten lehnen wir ab.
 
\wahlprogramm{Transparenz}
\antrag{KV Trier/Trier-Saarburg}\version{04:02, 20. Jun. 2010}
\subsubsection{Freie und plattformunabhängige Dateiformate für staatliche Veröffentlichungen}
\abstimmung
Offene Formate garantieren, dass Informationen auch langfristig lesbar sind. Diese müssen möglichst in durchsuchbarer Form zur Verfügung gestellt werden.

Der Zugang zu veröffentlichten Informationen darf nicht davon abhängen, welches Computersystem jemand benutzt, ob spezielle Software installiert oder gekauft wurde. Deshalb ist es erforderlich, Veröffentlichungen in einer Form vorzunehmen, die auf offenen standardisierten Formaten basiert.

\subsubsection{Offene Dateiformate in der Verwaltung}
\abstimmung
Wir werden dafür sorgen, dass die Verwaltungen des Landes und der Kommunen vollständig auf offene Dateiformate umsteigen. Dies vereinfacht den Datenaustausch zwischen den Behörden untereinander und mit den Bürgern.

\subsubsection{Freie Software in Behörden und staatlichen Einrichtungen}
\abstimmung
Sicherheit und langfristige Kosteneinsparungen durch Einsatz von freier Software.

Durch die Offenheit des Quellcodes bei freier Software gibt es keine Abhängigkeit von einem bestimmten Softwarehersteller. Dies verbessert die Möglichkeiten für spätere Anpassungen, wenn sich beispielsweise rechtliche Rahmenbedingungen für Behörden ändern. Bei freier Software entfallen außerdem auf lange Sicht große Summen für Lizenzgebühren. Den kurzfristig höheren Kosten für Einarbeitungsaufwand stehen so mittel- und langfristige Einsparungen gegenüber. Wartungsverträge können mit Firmen vor Ort geschlossen werden, was die regionale Wirtschaft fördert.
 
\wahlprogramm{Transparenz}
\antrag{Piraten aus RLP}\version{04:02, 20. Jun. 2010}
\subsubsection{Kooperation mit Microsoft aufkündigen}
Die Verträge der Landesregierung mit dem Software-Monopolisten Microsoft zum Einsatz von Software in Schulen, Hochschulen und Verwaltung sowie Bereich des Jugendmedienschutzes und der "Medienkompetenzförderung" lehnen wir ab und werden wir aufkündigen.
 
\subsection*{Auskunftsanspruch verbessern!}
\wahlprogramm{Auskunftsanspruch}
\antrag{Unglow}\version{04:02, 20. Jun. 2010}
\subsubsection{Modul 1}
\abstimmung
Wir wollen gewährleisten, dass jeder Bürger unabhängig von der Betroffenheit und ohne den Zwang zur Begründung sein Recht durchsetzen kann, auf allen Ebenen der staatlichen Ordnung Einsicht in die Aktenvorgänge und die den jeweiligen Stellen zur Verfügung stehenden Informationen zu nehmen. Dies gilt für schriftliches Aktenmaterial ebenso wie für digitale oder andere Medien.

\subsubsection{Modul 2}
\abstimmung
Ausnahmeregelungen zum Auskunftsanspruch sind eng und eindeutig zu formulieren und dürfen nicht pauschal ganze Behörden oder Verwaltungsgebiete ausnehmen. Für eine breite und effiziente Nutzung der Daten ist die Auskunftsstelle verpflichtet, Zugang in Form einer Akteneinsicht oder einer Materialkopie zu gewähren. Der Zugang soll zeitnah und mit einer klaren und fairen Kostenregelung erfolgen. Verweigerung des Zugangs muss schriftlich begründet werden und kann vom Antragsteller sowie von betroffenen Dritten gerichtlich überprüft werden lassen, wobei dem Gericht zu diesem Zweck voller Zugang durch die öffentliche Stelle gewährt werden muss.

\subsubsection{Modul 3}
\abstimmung
Wir werden alle öffentlichen Stellen verpflichten, regelmäßig sowohl Organisations- und Aufgabenbeschreibungen zu veröffentlichen, einschließlich Übersichten der Arten von Unterlagen, auf die zugegriffen werden kann, als auch einen jährlichen öffentlichen Bericht über die Handhabung des Auskunftsrechts.
 
\subsection*{Korruption erschweren!}
\wahlprogramm{Korruption}
\antrag{Unglow}\version{04:02, 20. Jun. 2010}
\subsubsection{Lobbyismus aufdecken}
\abstimmung
Damit für die Rheinland-Pfälzischen Bürgerinnen und Bürger klar ersichtlich ist, wer die Politik im Land beeinflusst, werden wir ein vollständiges Lobbyistenregister auf Landesebene einführen, in dem alle Verbände und Vertreter aufgeführt werden, die Einfluss auf Gesetzgebungsprozesse oder deren Ausgestaltung durch Verordnungen haben. In den Ministerien dürfen keine Mitarbeiter von Unternehmen dauerhaft ihre Arbeit verrichten. Lediglich in transparenten Anhörungen dürfen diese als Sachverständige angehört werden. Anhörungen zu Gesetzesinitiativen oder anderen Vorhaben der Landesregierung müssen stets öffentlich angekündigt werden und für jeden zugänglich sein. Insbesondere Verbraucherverbände, Bürgerrechts- und Menschenrechtsorganisationen müssen von Anfang an in Gesetzgebungsprozesse eingeweiht werden und Gelegenheit zur Stellungnahme bekommen. Alle Stellungnahmen von Interessenverbänden müssen öffentlich z.B. über das Internet zugänglich gemacht werden.

\subsubsection{Vergaberegister zur Korruptionsbekämpfung}
\abstimmung
Wir wollen ein Vergaberegister schaffen, mit dessen Hilfe bereits auffällig gewordene Firmen künftig von der Vergabe öffentlicher Aufträge ausgeschlossen werden. Diese Informationen sollen nicht nur Behörden zur Verfügung stehen, sondern auch der interessierten Öffentlichkeit. Das Korruptionsbekämpfungsgesetz von Nordrhein-Westfalen kann hier als Vorlage dienen.

\subsubsection{Offenlegung der Nebeneinkünfte von Landtagsabgeordneten}
\abstimmung
Die Höhe und Herkunft aller Einnahmen aus Nebentätigkeiten müssen einzeln und in vollem Umfang veröffentlicht werden. Dazu werden wir ein Modell erarbeiten, das über die Regelungen auf Bundesebene hinausgeht. Das dreistufige System reicht nicht aus, da die höchste Stufe von 7000 Euro nichts darüber aussagt, wie hoch die Nebeneinkünfte tatsächlich ausfallen. Um mögliche Interessenkonflikte erkennen zu können, müssen die zusätzlichen Einkünfte transparent offengelegt werden.
 
\subsection*{Transparenter Haushalt}
\wahlprogramm{Transparenter Haushalt}
\antrag{KV Trier/Trier-Saarburg}\version{04:02, 20. Jun. 2010}
\subsubsection{Transparenter Haushalt}
\abstimmung
Die Transparenz im Haushalt des Landes und bei der Verwendung von sonstigen Landesmitteln muss dringend verbessert werden. Haushaltswahrheit und Haushaltsklarheit sind nicht im erforderlichen Maß gewährleistet.

Wir werden uns dafür einsetzen, dass die Haushalte der überwiegend aus öffentlichen Mitteln finanzierten Stiftungen unter verstärkte parlamentarische Kontrolle gestellt werden.
 
\subsection*{Veröffentlichungsdienst 2.0 - freier Zugang zum Landesrecht!}
\wahlprogramm{Veröffentlichungsdienst 2.0 - freier Zugang zum Landesrecht!}
\antrag{Unglow}\version{04:02, 20. Jun. 2010}
\subsubsection{Modul 1}
\abstimmung
Unwissenheit schützt vor Strafe nicht. Aber sich über geltendes Recht - also Vorschriften, Erlasse, Verordnungen oder Entscheidungen - zu informieren, könnte heute wesentlich einfacher sein.

Wir wollen deshalb eine zentrale Anlaufstelle im Internet umsetzen, die neben Rechtsprechung und Gesetzgebung auch Verordnungen, Umsetzungsrichtlinien, Berichte, Empfehlungen, Analysen, amtliche Bekanntmachungen, Gesetzesentwürfe und sonstige Drucksachen von Land und Kommunen enthält, komplett mit Suchfunktion, Änderungsverfolgung, Querverweisen und Kommentarmöglichkeit.

\subsubsection{Modul 2}
\abstimmung
Das Material wird, sofern nicht ohnehin gemeinfrei, unter eine liberale Lizenz gestellt, die eine (auch kommerzielle) Weiterverwendung der Texte zulässt. Dabei sollen offene, einheitliche Schnittstellen für die automatische Abfrage und frei zugängliche Datenformate genutzt werden.

\subsubsection{Modul 3}
\abstimmung
Von diesem einfachen Zugriff profitieren alle Bürger und Unternehmen. Auch die Arbeit der staatlichen Stellen (Verwaltung, Gerichte, Landtag) wird durch eine einheitliche Plattform für die Veröffentlichung von Dokumenten und Daten erleichtert.
 
\subsection*{Weitere Maßnahmen für Rheinland-Pfalz}
\wahlprogramm{Weitere Maßnahmen}\label{transparenz:weiter}
\antrag{Unglow}\version{04:02, 20. Jun. 2010}
\subsubsection{Modul 1}
\abstimmung
Um die Informationsfreiheit im obigen Rahmen vollumfänglich zu gewährleisten, wollen die Rheinland-Pfälzischen PIRATEN folgende Maßnahmen ergreifen:
\begin{itemize}
\item die Digitalisierung aller staatlichen Unterlagen, die neu erstellt werden
\item Forschungsprojekte zur Digitalisierung alter Unterlagen sowie die Erforschung von Langzeitarchivierungsstrategien
\item den freien Zugang zu allen Gesetzen und Gesetzesentwürfen, bereits in der Entstehungsphase
\item den freien Zugang zu allen Beschlüssen des Landtages und anderer politischer Gremien
\item die komplette Offenlegung des Abstimmungsverhaltens im Rheinland-Pfälzischen Landtag und seinen Ausschüssen
\item die komplette Offenlegung des Abstimmungsverhaltens der Landesregierung im Bundesrat
\item die komplette Offenlegung der Nebeneinkünfte der Landtagsabgeordneten und Minister
\item den freien Zugang zu allen finanziellen Ausgaben der Landesregierung, der Ministerien, des Landtags und seiner Fraktionen
\item den freien Zugang zu allen Messdaten, die staatlichen Institutionen vorliegen (Wetterdaten, Flugverkehrsdaten, Gewässerdaten, Katasterdaten, Luftbilder, u.v.m)
\item den freien Zugang zu allen statistischen Erhebungen, die durch die Verwaltung oder in deren Auftrag vorgenommen werden
\item das Angebot von offenen Schnittstellen zur automatischen Abfrage der bereitgehaltenen Dokumente, Daten und Informationen in standardisierten, offenen Formaten
\item die Einrichtung einer kostenlosen Beratungsstelle, die den Bürgern und Bürgerinnen offene Fragen und komplexe Sachverhalte erläutert
\item die finanzielle Förderung von Open-Data- und Semantic-Web-Initiativen und Forschung in diesem Bereich
\item die Zusammenarbeit mit Rheinland-Pfälzischen Hochschulen zur Digitalisierung, Aufbereitung und Zurverfügungstellung aller Daten in offenen Formaten
\item die Unabhängigkeit des Landesbeauftragten für Informationsfreiheit und eine bessere finanzielle und personelle Ausstattung der Behörde
\item das Angebot aller Ausschreibungen in einem standardisierten, maschinenlesbaren Datenformat
\item die Einführung einer Meldepflicht für alle Behörden bei Datenpannen und ein standardisiertes Verfahren zur Benachrichtigung der Betroffenen
\item die Veröffentlichung aller Verträge der Landesregierung und der Ministerien mit Unternehmen
\item die Einführung eines vollständigen Lobbyistenregisters auf Landesebene
\item eine klare Kennzeichnung, welche Passagen in Gesetzesentwürfen von wem hinzugefügt wurden
\item die umgehende Bekanntmachung von Art und Umfang aller Abhörmaßnahmen, Observationen oder Datenabfragen inklusive der Information von welcher Polizeibehörde oder welchem Geheimdienst diese auf welcher rechtlichen Grundlage durchgeführt werden, sowie die umfassende Information der Betroffenen sofort nach Ende der Maßnahme
\item die ausschließliche Verwendung quelloffener Software durch die Verwaltung
\end{itemize}
 
\wahlprogramm{Ergänzung zu 'Weitere Maßnahmen'}
\antrag{Silberpappel}\zusatz{transparenz:weiter}\version{04:02, 20. Jun. 2010}
\subsubsection{Protokolle von öffentlichen Gemeinderatssitzungen}
\abstimmung
Verpflichtung der Gemeindeverwaltungen zur Veröffentlichung der Protokolle von öffentlichen Gemeinderatssitzungen im Internet.

\subsubsection {Gemeindesatzungen}
\abstimmung
Verpflichtung der Gemeindeverwaltungen zur Veröffentlichung der Satzungen der Gemeinde im Internet

%%\section{Bildung}

\subsection*{Präambel: Wert von Bildung und finanzielle Mittel}
\wahlprogramm{Präambel: Wert von Bildung und finanzielle Mittel}
\antrag{Piraten aus RLP}\version{03:56, 20. Jun. 2010}\\
\abstimmung

\subsubsection{Präambel: Wert von Bildung und finanzielle Mittel}
Wir sehen Bildung als unabdingbares Menschenrecht und fordern Chancengleichheit und den freien Zugang zu Informationen und Bildung für alle Menschen sowie eine demokratische Organisation der Lehr- und Lerneinrichtungen. Wir fordern einen massiven Ausbau der Investitionen ins Bildungssystem und die Gewährleistung freien, selbstbestimmten Lernens im gesamten Bildungsweg. In einer global vernetzten Wissensgesellschaft ist Bildung die wichtigste Ressource eines jeden Menschen und Voraussetzung für die freie Entfaltung seiner Persönlichkeit. Sie garantiert seine Entwicklung zum freien und mündigen Bürger. Die kulturellen und persönlichen Entfaltungsmöglichkeiten der Menschen basieren auf dem allgemeinen Bildungsniveau sowie der persönlichen Qualifizierung jedes Bürgers. Um ein hohes Bildungsniveau erreichen und halten zu können, müssen die finanziellen Mittel des Bildungssystems erhöht werden und Priorität vor allen anderen Ausgaben erhalten. Alle anderen Aufgaben haben hinter der Bildung zurückzustehen.

\wahlprogramm{Präambel: Wert von Bildung und finanzielle Mittel}
\antrag{Niemand13}\version{03:56, 20. Jun. 2010}

\subsubsection{Investitionen in Bildung aufstocken}
\abstimmung
Der prozentuale Anteil der Ausgaben für den Bereich Bildung am gesamten Bruttoinlandsprodukt sinkt jährlich. Wir fordern drastische Investitionssteigerungen, um gute Bildung für jedermann zu ermöglichen. Wir fordern die Einstellung neuen Lehrpersonals an Hochschulen in ausreichender Zahl, um sowohl allen Studieninteressierten einen Platz in dem von Ihnen gewünschten Fach und Abschluss zur Verfügung stellen zu können, als auch allen Studierenden eine individuelle Betreuung durch die Leiter der jeweiligen Lehrveranstaltungen zu gewährleisten. Die an Forschung und Lehre Beteiligten müssen besser entlohnt werden. Die Entwicklung rückläufiger Investitionen in Universitäten wollen wir stoppen. Schlechte Lernbedingungen und prekäre Beschäftigung werden wir nicht dulden. Im Bereich der Schulen fordern wir die Einstellung von mehr Lehrern und kleinere Klassen von maximal 20 Schülern pro Klasse. Wir fordern die Abschaffung des Studienkontenmodells, das finanziell Schwächere in der Durchführung und am erfolgreichen Abschluss eines Studiums effektiv benachteiligt. Die verfassungswidrige Landeskinderregelung muss ersatzlos aus dem Landeshochschulgesetz gestrichen werden. Durch ausreichende Möglichkeiten für Teilzeit- und Abendstudien wollen wir auch Berufstätigen und anderweitig zeitlich Belasteten ein Studium ermöglichen.

\newpage
\subsubsection{Gebührenfreiheit sichern}
\abstimmung
Gebühren jeglicher Art sowie finanzielle und personelle Engpässe – gerade an den Hochschulen – schränken den Zugang zu Bildung ein und werden deshalb von uns kategorisch abgelehnt. Ein Studium ohne Abhängigkeit von Krediten und ohne Schuldenberg nach Studienabschluss muss gewährleistet sein. Wir fordern daher die gesetzlich verankerte Gebührenfreiheit und einen drastischen Ausbau der Investitionen in Schule und Hochschule: Das Bildungsangebot darf sich nicht weiter den knappen Ausgaben anpassen, sondern wir wollen die Ausgaben im Bildungsbereich an die Notwendigkeiten angleichen! Ein Studium ohne Abhängigkeit von Krediten muss gewährleistet sein. Eine private Finanzierung öffentlicher Bildungseinrichtungen muss stets kritisch hinterfragt werden. Ein Einfluss auf Lehrinhalte muss ausgeschlossen sein. Einer Kommerzialisierung von Schulen und Hochschulen stellen wir uns entschieden entgegen. Exzellenzinitiativen wollen wir kritisch überprüfen, damit sich nicht in Konkurrenz um Fördergelder nur noch wenige Hochschulen gute Lehre und Forschung leisten können.
 
\subsection*{Lehr- und Lernmittelfreiheit und Open Access für Rheinland-Pfalz!}
\wahlprogramm{Lehr- und Lernmittelfreiheit und Open Access für Rheinland-Pfalz!}\label{wp:bildung:freiheit1}
\antrag{Niemand13}\zusatz{wp:bildung:freiheit2}\version{03:56, 20. Jun. 2010}

\subsubsection{Lehr- und Lernmittelfreiheit und Open Access für Rheinland-Pfalz!}
\abstimmung
Zum ersten Mal in der Geschichte der Menschheit besteht die Möglichkeit unser komplettes Wissen zu sammeln, zu speichern und für die Allgemeinheit zugänglich zu machen. Gerade im Bereich der Forschung und Lehre bieten sich hier ungeahnte Möglichkeiten. Leider werden diese stark beschnitten. Wir fordern eine vollständige Lern- und Lehrmittelfreiheit für Rheinland-Pfalz. Die Verwendung und das Schaffen von freien Werken zur Vermittlung von Wissen müssen vom Land unterstützt und ausgebaut werden. Freie Werke sind nicht nur kostenfrei im Unterricht einsetzbar, sondern ermöglichen dazu dem Lehrenden ohne rechtliche Hürden die Lernmittel auf seinen Unterricht anzupassen. Alle in den Bibliotheken bereitstehenden Bücher und Zeitschriften sollen, auch in digitaler Form, für die Studierenden und Mitarbeiter frei zugänglich und verfügbar sein. Das Problem nicht bereitstehender oder auch nicht auffindbarer Bücher würde damit gelöst. Aufwendige Fernleihen müssen der Vergangenheit angehören. Die Publikationen aus staatlich finanzierter oder geförderter Forschung und Lehre werden oft in kommerziellen Verlagen publiziert, deren Qualitätssicherung von ebenfalls meist staatlich bezahlten Wissenschaftlern im Peer-Review-Prozess übernommen wird. Die Publikationen werden jedoch nicht einmal Bibliotheken der Forschungseinrichtungen kostenlos zur Verfügung gestellt. Wir dulden nicht, dass der Steuerzahler für Produktion, Qualitätssicherung und Nutzung insgesamt dreifach für die Kosten der Publikationen im Milliardenbereich aufkommt. Wir fordern, dass alle wisschenschaftlichen Publikationen, die aus öffentlich geförderter Forschung hervorgehen, auch allen Bürgern kostenfrei zur Verfügung stehen.

\subsubsection{Open Access-Offensive für Rheinland-Pfalz!}
\abstimmung
Wir unterstützen die Berliner Erklärung der Open-Access-Bewegung und fordern die Zugänglichmachung des wissenschaftlichen und kulturellen Erbes der Menschheit über das Internet nach dem Prinzip des Open Access. Wir sehen es als Aufgabe des Staates an, dieses Prinzip an den von ihm finanzierten und geförderten Einrichtungen durchzusetzen. Wir fordern den Einsatz offener Software in Forschung und Lehre. Software ist Wissen und wir wollen nicht länger Millionen an Steuergeldern für geschlossene und intransparente Systeme ausgeben. Mit der Förderung und dem Einsatz von offener Software wollen wir für Transparenz an den Hochschulen, für Erweiterbarkeit der System durch Interessierte und für die Förderung von kleinen und mittelständischen Unternehmen im Land sorgen. Software muss Studierenden und Mitarbeitern an jedem Hochschulrechner, zumindest als Alternative, angeboten werden. Auch bei der Neuanschaffung von Programmen oder dem Neuaufbau von Systemen und Datenbanken wollen wir, dass Open-Source-Lösungen eingesetzt werden. Wir lehnen die Anschaffung proprietärer Software bei existierenden Open-Source- Alternativen grundsätzlich ab. Studierende dürfen im Rahmen ihres Studiums nicht zur Nutzung oder gar zur Anschaffung bestimmter proprietärer Software genötigt werden, genauso wenig wie Mitarbeiter. Umfassende Kooperationsverträge mit Software- Monopolisten lehnen wir ab. Im Rahmen des „Open Date“ sollen Hochschulen all ihre Daten über offene, standardisierte Schnittstellen allen Interessierten kostenlos zur Verfügung stellen.

\wahlprogramm{Lernmittelfreiheit - Für eine kostenlose Schulbildung}\label{wp:bildung:freiheit2}
\antrag{Piraten aus RLP}\zusatz{wp:bildung:freiheit1}\version{03:56, 20. Jun. 2010}

\subsubsection{Lernmittelfreiheit - Für eine kostenlose Schulbildung}
\abstimmung
Damit auch sozial schwache Kinder nicht benachteiligt werden und da nach unsere Überzeugung Schule kostenlos sein muss, fordern wir eine komplette Lernmittelfreiheit für Rheinland-Pfalz. Des weiten wollen wir die Erstellung von kostenlosen Lernmaterialien als Alternative zu den kommerziellen Lernmaterialien fördern.
 
\wahlprogramm{Lehrmittel}\label{wp:bildung:lehrmittel1}
\antrag{KV Trier/Trier-Saarburg}\konkurrenz{wp:bildung:lehrmittel2}\version{03:56, 20. Jun. 2010}

\subsubsection{Einsatz von Lehrmitteln unter freien Lizenzen}
\abstimmung
Wir wollen, dass an Bildungseinrichtungen Lehrmittel mit freien Lizenzen verwendet werden. Dies trägt zur Kostensenkung bei.

\subsubsection{Mehr Nutzung von freier Software}
\abstimmung
Freie Software ist kostengünstiger für Schulen und Eltern. Der Zugang ist damit in jedem Haushalt mit Computer gesichert.
 
\newpage
\wahlprogramm{Lehrmittel}\label{wp:bildung:lehrmittel2}
\antrag{Piraten aus RLP}\konkurrenz{wp:bildung:lehrmittel1}

\subsubsection{Lizenzfreies Unterrichtsmaterial}
\abstimmung
Die Veröffentlichung von Unterrichtsmaterialien und Unterrichtsentwürfen unter freien Lizenzen und via Internet soll gefördert werden. Dies vereinfacht den Lehrkräften die Verwendung bestehender und die Erarbeitung neuer Unterrichtsmaterialien. Auf einer staatlich finanzierten Plattform soll den Lehrern der leichte Austausch und die gegenseitige Qualitätssicherung (beispielsweise durch eine Begutachtung seitens mehrerer Kollegen (peer-review)) ermöglicht werden.

\subsubsection{Schulbücher unter offner Lizenz}
\abstimmung
\begin{enumerate}
\item Die Erstellung von Schulbüchern unter freier Lizenz (z.b. GPL) soll staatlich gefördert werden.
\item Die Autorenleistungen, für die jeweilige Erstellung und Aktualisierung, werden hierbei jeweils einmalig durch das Land finanziert, sodass eine jeweilige dauerhafte Vergütung pro Medium entfällt.
\item Interessierte haben die Möglichkeit an den freien Produkten mitzuarbeiten und sie nach Belieben zu verändern und zu verbessern.
\item Die Qualität der Einsendungen wird durch eine Begutachtung seitens mehrerer Kollegen (peer-review) sichergestellt. Auf Qualität geprüfte Versionen werden für alle Nutzer erkennbar zertifiziert.
\item Eine Veröffentlichung soll immer sowohl in Digital-, als auch als Papierform erfolgen. Druckversionen der Medien werden zum Selbstkostenpreis angeboten. Sofern das Schulbuch von einer Klasse verwendet wird, muss dieses den jeweiligen Schülern als kostenfreies Printexemplar zur Verfügung gestellt werden.
\end{enumerate} 

\subsection*{Bildungseinrichtungen demokratisieren!}
\wahlprogramm{Bildungseinrichtungen demokratisieren!}\label{wp:bildung:demokratie1}
\antrag{Niemand13}\version{03:56, 20. Jun. 2010}
\begin{itemize}
\item \konkurrenz{wp:bildung:demokratie2}
\item \konkurrenz{wp:bildung:demokratie3}
\end{itemize}

\subsubsection{Bildungseinrichtungen demokratisieren!}
\abstimmung
Bildungseinrichtungen sind für SchülerInnen und StudentInnen ein prägender und umfassender Teil des Lebens. Sie sind deswegen als Lebensraum der Lernenden zu begreifen, der durch sie mitbestimmt werden muss. In Schulen müssen SchülerInnen ein Mitspracherecht bei der Gestaltung ihres Schulalltags haben. Demokratische Werte müssen vermittelt und vor gelebt werden, um die Akzeptanz der Entscheidungen zu erhöhen, das Gemeinschaftsgefühl zu stärken und selbstbestimmtes Lernen im ausreichenden Maße zu ermöglichen. Wir fordern eine grundlegende demokratische Organisation von Schule und Hochschule.

\subsubsection{Hochschulrat abschaffen - Hochschulen demokratisch gestalten!}
\abstimmung
Bei den Universitäten stellen sowohl das bestehende Ungleichgewicht zugunsten des Hochschulrats, als auch die geplante Novelle des Landeshochschulgesetzes eine Entmündigung der breiten Mehrheit zugunsten nicht gewählter Gremienvertreter und des Präsidialamts dar. Was als „Autonomie der Hochschule“ angepriesen wurde, verkehrt sich in ihr Gegenteil: Hochschulen verlieren die Unabhängigkeit, welche für die Erfüllung ihrer Aufgaben, die eine freiheitlich-demokratische Gesellschaft ihnen übertragen hat, unentbehrlich ist. Demokratische Entscheidungsstrukturen dürfen nicht weiter durch wirtschaftliche Einflüsse oder die Etablierung autoritärer Strukturen beeinträchtigt und unterwandert werden. Wir fordern die Abschaffung des Hochschulrates und die Übertragung aller Kompetenzen auf den Senat. Die unabhängige Mitwirkung aller Interessengruppen in den demokratischen Willensbildungsprozessen der Hochschulen muss gesichert werden und sich im Hochschulgesetz widerspiegeln. Studentischen VertreterInnen sollen aufgrund der Größe der Studierendenschaft mit einer Drittelparität in allen entscheidungsbefugten Gremien vertreten sein.

\subsubsection{Beabsichtigtes Landeshochschulgesetz stoppen!}
\abstimmung
Verschärft wird die oben aufgezeigte Entwicklung durch das neue Landeshochschulgesetz (LHG), das unter dem Deckmantel der Autonomie der Hochschule demokratische Grundstrukturen unterminiert: Die Entmachtung demokratischer Gremien und der Ausbau präsidialer Entscheidungskompetenzen, die Begünstigung der Trennung von Forschung und Lehre sowohl durch die Einrichtung von Forschungskollegs, als auch durch die Möglichkeit der Freistellung von ProfessorInnen von der Lehre für bis zu 10 Jahre, und die Schaffung von Einfallstoren für Unternehmen durch die Gründung von Hochschulverbünden und außeruniversitären Betrieben, die auch Privatunternehmen offen stehen. Wir dagegen fordern, dass VertreterInnen der Studierendenschaft in den entscheidungsbefugten, universitären Gremien nicht länger untervertreten sind und lehnen die beabsichtigte Novelle des LHG in der derzeitigen Form ab.
 
\subsection*{Bildungseinrichtungen demokratisieren!}
\wahlprogramm{Bildungseinrichtungen demokratisieren!}\label{wp:bildung:demokratie2}
\antrag{Piraten aus RLP}\version{03:56, 20. Jun. 2010}
\begin{itemize}
\item \konkurrenz{wp:bildung:demokratie1}
\item \konkurrenz{wp:bildung:demokratie3}
\end{itemize}

\subsubsection{Mehr Demokratie an Schulen wagen, Schülervertretungen stärken (Variante 1)}
\abstimmung
Um als demokratiekomptenter Bürger aufzuwachsen ist es wichtig, dass Schüler schon frühzeitig Demokratie-Lernen und erfahren. Die Schule sollte sie dabei unterstützen. Wir fordern die Ausweitung der Demokratie an Schulen und die verstärkte Mitbestimmung der Schülerschaft (bspw. durch die flächendeckende Einführung von Schülerparlamenten). Schülervertretungen müssen besser über ihre Rechte informiert und geschult werden.

\subsubsection{Mehr Demokratie an Schulen wagen, Schülervertretungen stärken (Variante 2 [aus BW-Programm])}
\abstimmung
Die gelebte Vermittlung der Grundprinzipien unserer demokratischen Staats- und Gesellschaftsform ist eine der Aufgaben staatlicher Bildungseinrichtungen. An allen rheinland-pfälzischen Schulen sollen deshalb schrittweise Klassenräte und Schülerparlamente eingeführt werden. Durch die frühe Möglichkeit, sich an (schul-)politischen Entscheidungen zu beteiligen und Themen zu erarbeiten, soll unter anderem der Politikverdrossenheit unter Jugendlichen vorgebeugt werden. Außerdem können Kinder und Jugendliche demokratische Prinzipien und Werte auf diese Art und Weise kennen und schätzen lernen, wodurch sie kritischer mit extremistischem Gedankengut umgehen können.
 
\subsection*{Bildungseinrichtungen demokratisieren!}
\wahlprogramm{Demokratisierung der Bildung}\label{wp:bildung:demokratie3}
\antrag{KV Trier/Trier-Saarburg}\version{03:56, 20. Jun. 2010}
\begin{itemize}
\item \konkurrenz{wp:bildung:demokratie1}
\item \konkurrenz{wp:bildung:demokratie2}
\end{itemize}

\subsubsection{Demokratisierung der Bildung}
\abstimmung
Wir setzen uns für eine Demokratisierung der Schul- und Bildungslandschaft ein. Wir wollen die Demokratisierung des Bildungsbereichs unter anderem durch weitergehende Rechte für die Schülermitverwaltungen und die Studentenschaften erreichen.
 
\subsection*{Freies, individuelles Lernen ermöglichen!}
\wahlprogramm{Freies, individuelles Lernen ermöglichen!}\label{wp:bildung:frei1}
\antrag{Niemand13}\konkurrenz{wp:bildung:frei2}\version{03:56, 20. Jun. 2010}

\subsubsection{Schutz vor Überwachung}
\abstimmung
Jeder Mensch ist ein Individuum mit persönlichen Neigungen, Stärken und Schwächen. Institutionelle Bildung soll daher den Einzelnen unterstützen seine Begabungen zu entfalten, Schwächen abzubauen und neue Interessen und Fähigkeiten zu entdecken. Neben starren Lehr- und Stundenplänen, werden vor allem einige Formen der Leistungsbewertung diesen Forderungen nicht gerecht. Insbesondere die Bewertung von Verhalten nach einem vorgegebenen Normenraster z.B. durch Kopfnoten lehnen wir ab. Für ein freies Lernen und Lehren ist der Schutz vor Überwachung und Zensur unabdingbare Voraussetzung. Wer sich beobachtet fühlt oder nicht mehr sicher weiß, wer was über ihn weiß, der wird sein Verhalten anpassen und sich in seinem Lehr- und Lernprozess nicht frei entfalten.

\subsubsection{Schutz vor Zensur und Informationskontrolle}
\abstimmung
Eine Zensur behindert den Zugang zu Information, zu Wissen und zu Demokratie und wird von uns daher aufs Schärfste bekämpft. Wir fordern den uneingeschränkten Zugang zu allen Informationen.

\subsubsection{Kontrolle von Datenverarbeitung der Lernenden}
Ü\abstimmung
berwachung - auch in Form von Data-Warehousing-Systemen, in denen massenhaft Studierendendaten gespeichert, gesammelt und ausgewertet werden - lehnen wir ab. Für alle Systeme, die personenbezogene Daten von Lernenden oder Lehrenden verarbeiten, fordern wir maximale Transparenz, Nachvollziehbarkeit bzgl. der Datenabfragen und wirksame organisatorische und technische Maßnahmen zum Schutz vor Missbrauch. Verwaltungssysteme müssen auch stets die Lehre unterstützen und dürfen keinesfalls von sich aus Auswirkungen auf die Gestaltung des Lehrbetriebs nehmen.

\subsubsection{Barrierefreiheit}
\abstimmung
Eine Barrierefreiheit setzen wir für alle Systeme
 
\wahlprogramm{Freies, individuelles Lernen ermöglichen!}\label{wp:bildung:frei2}
\antrag{KV Trier/Trier-Saarburg}\konkurrenz{wp:bildung:frei1}\version{03:56, 20. Jun. 2010}

\subsubsection{Bildung als Teil der individuellen Entwicklung}
- zurückgezogen, da gleich \ref{wp:bildung:frei1} -
 
\wahlprogramm{Persönlichkeitsrechte von Schülern und Lehrern achten}
\antrag{KV Trier/Trier-Saarburg}\version{03:56, 20. Jun. 2010}

\subsubsection{Persönlichkeitsrechte von Schülern und Lehrern achten}
\abstimmung
Die Privat- und Intimsphäre sowie das Recht auf informationelle Selbstbestimmung von Schülern und Lehrern müssen gewahrt bleiben. Videoüberwachung und private Sicherheitsdienste haben keinen Platz in Schulen. Präventive Durchsuchungen und Kontrollen oder Urinuntersuchungen sind zu unterlassen. Die Unschuldsvermutung gilt auch für Schüler. Diese unter Generalverdacht zu stellen, zerstört das Vertrauen zu Schule und Lehrern, ohne welches Unterricht und Erziehung aber nicht möglich sind.
 
\wahlprogramm{Freies, individuelles Lernen ermöglichen!}
\antrag{Piraten aus RLP}\version{03:56, 20. Jun. 2010}

\subsubsection{Individuelle Bildung}
\abstimmung
Derzeit ist das Bildungsangebot in vielen Hinsichten stark eingeschränkt und umfasst wenig Spielraum für die optimale Entfaltung der eigenen Fähigkeiten. Daher sollen Maßnahmen gefördert werden, die die Auswahl von Bildungsangeboten erhöht.

\subsubsection{Lebenslanges Recht auf Bildung}
\abstimmung
Das Recht auf Bildung soll sich auf das gesamte Lebensalter erstrecken um die Möglichkeiten der Bürger für freie Selbstentfaltung und Lebensgestaltung zu ermöglichen. Bisher beschränkt sich die Ausbildung fast ausschließlich auf die jüngeren Generationen, älteren Menschen wird die Möglichkeit der Aus- und Weiterbildung derzeit nicht in dem selbem Maße zugestanden wie den Jüngeren.

\subsubsection{Individuelle Förderung}
\abstimmung
Jeder Schüler hat seine Individuellen Stärken, Schwächen und Bedürfnisse, werden diese nicht erkannt und gefördert verschlechtert sich das allgemeine Schulklima und die individuelle Leistungsfähigkeit wird nicht voll ausgeschöpft. Wir möchten eine bessere Förderung einzelner Schüler und deren Interessen. Dies kann durch Angebote wie Arbeitsgruppen Wahlpflichtfächer und Förderuntericht erreicht werden.
 
\newpage
\subsection*{Mehr Geld für Bildung}
\wahlprogramm{Mehr Geld für Bildung}\label{wp:bildung:geld1}
\antrag{Piraten aus RLP}\konkurrenz{wp:bildung:geld2}\version{03:56, 20. Jun. 2010}

\subsubsection{Mehr Geld für Bildung}
\abstimmung
Im Vergleich der Bundesländer ist das Land Rheinland-Pfalz eines der Bundesländer mit den niedrigsten Ausgaben für den Bereich Bildung. Wir fordern, dass der Bildung im Land Rheinland-Pfalz ein höherer Stellenwert zukommt. Darum wollen die finanzielle Ausstattung von Schulen und Universitäten verbessern.
 
\wahlprogramm{Finanzierung von Bildung und Forschung}\label{wp:bildung:geld2}
\antrag{KV Trier/Trier-Saarburg}\konkurrenz{wp:bildung:geld1}\version{03:56, 20. Jun. 2010}

\subsection*{Finanzierung von Bildung und Forschung}
\abstimmung
Für eine reiche Industrienation wie Deutschland ist es unverständlich, dass hier nur ein im internationalen Vergleich verschwindend geringer Teil der öffentlichen Mittel in Bildung und Forschung investiert wird. Bildung und Forschung sind eine Investition in die Zukunft unserer Gesellschaft und in jeden Menschen. Wir fordern daher eine bessere finanzielle Ausstattung von Bildungsstätten aller Art und gleichermaßen der Forschung mit staatlichen Mitteln. Schönrechnereien – wie die Einbeziehung von Lehrerpensionen – lehnen wir dabei ab.
 
\wahlprogramm{Wirtschaftlicher Bildungssoli}
\antrag{Piraten aus RLP}\version{03:56, 20. Jun. 2010}

\subsubsection{Wirtschaftlicher Bildungssoli}
\abstimmung
Bei der Finanzierung des Bildungssystems ist unsere komplette Gesellschaft gefordert. Vor allem unsere Wirtschaft profitiert maßgeblich von gut ausgebildeten Kräften. Wir möchten deshalb, dass sich die wirtschaftlichen Unternehmen durch eine zweckgebundene Bildungsabgabe an der Finanzierung unseres Bildungssystems noch stärker beteiligen und so ihren positiven Beitrag zur Zukunft Deutschlands leisten. Für Unternehmen wird sich diese Investition lohnen, da sie von den besser ausgebildeten, leistungsfähigeren Kräften später stark profitieren werden.
 
\newpage
\subsection*{Vereinbarkeit von Familie und Beruf}
\wahlprogramm{Vereinbarkeit von Familie und Beruf}
\antrag{Piraten aus RLP}\version{03:56, 20. Jun. 2010}

\subsubsection{Veränderungen im Umfeld der Kinder}
\abstimmung
Das Umfeld, in dem Kinder aufwachsen, verändert sich. Da in immer mehr Familien beide Elternteile berufstätig sind oder die Eltern der Kinder getrennt leben, muss sich das Bildungssystem an diese Verhältnisse anpassen. Es müssen zusätzliche Angebote geschaffen werden, welche die Eltern unterstützen und entlasten.

\subsubsection{Vereinbarkeit von Familie und Beruf}
\abstimmung
Darum muss das staatliche Betreuungsangebot für Kinder und Jugendliche ausgebaut werden, so dass Familie und Beruf vereinbar werden. Dabei darf es nicht vorkommen, dass die Betreuung lediglich eine beaufsichtigte Verwahrung der Kinder und Jugendlichen ist. Vielmehr müssen Möglichkeiten geschaffen werden, wie sich die Kinder und Jugendlichen entwickeln und entfalten können.
 
\wahlprogramm{Vereinbarkeit von Familie und Beruf}
\antrag{KV Trier/Trier-Saarburg}\version{03:56, 20. Jun. 2010}

\subsubsection{Familienfreundliche Ganztagesbetreuung an Schulen}
\abstimmung
Staatliche Bildungseinrichtungen sollen den Familien dabei helfen, die notwendige Flexibilität zu erreichen, den Anforderungen des Familien- und Berufslebens gerecht zu werden. Dafür soll an allen Schulen ein Angebot zur Ganztagesbetreuung geschaffen werden. Das Betreuungsangebot ergänzt den Unterricht um zusätzliche Bildungsmöglichkeiten und außerschulische Aktivitäten. Neben Wahlfächern, Hausaufgabenbetreuung und Nachhilfe soll ein möglichst breites Angebot an kulturellen oder sportlichen Tätigkeiten ermöglicht werden. Dabei ist die Zusammenarbeit mit Vereinen ausdrücklich erwünscht und zu beiderseitigem Vorteil.

\subsubsection{Schulspeisung}
\abstimmung
Eine gesunde Ernährung ist aus Gründen der körperlichen und geistigen Entwicklung und der Konzentrationsfähigkeit der Kinder wichtig.

Schulspeisungen können dabei helfen, dass sich Kinder ausgewogen und gesund ernähren. Wir fordern daher die Einführung gesunder und ausgewogener Schulspeisungen an allen Schulen und Kindertagesstätten.

Die Finanzierung dieser Schulspeisungen ist dabei so zu gestalten, dass alle Schüler unabhängig von der sozialen oder finanziellen Lage der Familie daran teilnehmen können. Zur Vermeidung sozialer Ausgrenzung sollen finanzielle Erleichterungen so gestaltet sein, dass andere Schüler nicht erfahren, wer gefördert wird.

Bei der Planung sollte auch berücksichtigt werden, ob die Verwaltungskosten für die Essensgebühren die Einnahmen übersteigen oder eine vollständig kostenlose Schulspeisung günstiger wäre.
 
\subsection*{Organisationsstruktur / Ausstattung}
\wahlprogramm{Gleichbehandlung der Träger}
\antrag{Piraten aus RLP}\version{03:56, 20. Jun. 2010}

\subsubsection{Gleichbehandlung der Träger}
\abstimmung
Konfessionelle, soziale, kulturelle oder sonstige Zugangsbeschränkungen sind in Einrichtungen, die (auch zu Teilen) öffentlich finanziert werden, nicht zulässig. Bei der öffentlichen Finanzierung von Einrichtungen sind alle Träger gleich zu stellen.
 
\wahlprogramm{Kein Schulsponsoring}
\antrag{Piraten aus RLP}\version{03:56, 20. Jun. 2010}

\subsubsection{Kein Schulsponsoring}
\abstimmung
Die Schule sollte ein neutraler Raum sein, indem sich die Schüler frei entwickeln können und der dementsprechend von der Gemeinschaft finanziert wird. Wir lehnen deshalb Schulsponsoring durch die Privatwirtschaft ab.
 
\wahlprogramm{Bessere Unterstützung der Kommunen beim Ausbau und Erhalt der Schulinfrastruktur}
\antrag{Piraten aus RLP}\version{03:56, 20. Jun. 2010}

\subsubsection{Bessere Unterstützung der Kommunen beim Ausbau und Erhalt der Schulinfrastruktur}
\abstimmung
Schulen sind ein Platz, an dem unsere Heranwachsenden viele Stunden ihres Lebens verbringen. Die Leistung der Heranwachsenden und ihr Verhältnis zur Institution Schule hängt nicht unmaßgeblich davon ab, ob sie sich dort wohlfühlen. Ein guter Zustand des Gebäudes und die bestmögliche Ausstattung können dazu maßgeblich beitragen. Wir fordern, dass das Land die Kommunen beim Ausbau und Erhalt der Schulinfrastruktur optimal unterstützt.
 
\wahlprogramm{Ein Laptop für jeden Schüler / Internet in jedem Klassensaal}
\antrag{Pirat aus RLP}\version{03:56, 20. Jun. 2010}

\subsubsection{Ein Laptop für jeden Schüler / Internet in jedem Klassensaal}
\abstimmung
Die Ausstattung mit digitalen Arbeitsmitteln und ein Internetzugang für alle Lernenden ist eine Grundvoraussetzung für den Zugang zur Informations- und Wissensgesellschaft und einer aktiven Teilhabe an dieser. Die Schulen müssen deshalb an die technischen Gegebenheiten unserer Zeit angepasst werden. Schüler und Lehrer sollten die Möglichkeit haben spontan und flexibel das Internet als ergänzendes Mittel des Unterrichts (bspw. bei Gruppenarbeiten) zu nutzen. Computerräume sind in ihrer Verfügbarkeit und Flexibilität stark begrenzt. Wir fordern möchten deshalb jeden Schüler bei seiner Einschulung mit einem schulischen Leih-Laptop bzw. Netbook ausstatten, welcher bei der Einschulung in die weiterführende Schule dann durch ein angepasstes Modell ersetzt wird.

\subsubsection{Gemeinsame Ausarbeitung der Laptopausstattung mit Schülern, Lehrern und Eltern}
\abstimmung
Die Laptop bzw. Netbookausstattung, insbesondere die Wahl der passenden Softwarepakete, findet gemeinsam mit den Schülern, Lehrern und Eltern statt, um sie optimal an die Wünsche und Gegegebenheiten anzupassen

\subsubsection{Internet in jedem Klassensaal}
\abstimmung
Um die Laptops umfassend nutzen zu können fordern wir die Ausstattung der kompletten Schulgebäude mit drahtlosem Internet.

\subsubsection{ }
\abstimmung
Für Lehrkräfte sollen Schulungen angeboten werden, mit denen sie sich bei Bedarf in diesem Bereich methodisch weiterbilden können.
 
\subsection*{Bessere Ausstattung von öffentlichen Bibliotheken}
\wahlprogramm{Bessere Ausstattung von öffentlichen Bibliotheken}
\antrag{Piraten aus RLP}\version{03:56, 20. Jun. 2010}

\subsubsection{Bessere Ausstattung von öffentlichen Bibliotheken}
\abstimmung
Obwohl zahlreiche Bibliotheken bereits erste Schritte auf dem Weg zu umfassenden Medien- und Informationszentren unternommen haben, sollten insbesondere Computerarbeitsplätze, Internetzugänge, Zugänge zu Datenbanken und umfangreiche Bestände mit neuen Informations-, Bildungs- und Unterhaltungsträgern weiter ausgebaut und effektiv finanziert werden, vor allem im ländlichen Raum.
 
\subsection*{Reduzierung des Unterrichtsausfalls}
\wahlprogramm{Reduzierung des Unterrichtsausfalls}
\antrag{Piraten aus RLP}\version{03:56, 20. Jun. 2010}

\subsubsection{Reduzierung des Unterrichtsausfalls}
\abstimmung
Eine besondere Aufgabe stellt die Reduzierung des Unterrichtsausfalls dar. In der Zukunft ist aufgrund des demographischen Wandels mit einem Sinken der Schülerzahlen zu rechnen. Es ist jedoch nicht im Sinne unserer Kinder, dass wir die Zeit bis dahin mit Unterrichtsausfall und Aushilfslehrern einfach aussitzen. Der Unterrichtsausfall kurzfristig reduziert werden. Jeder Schüler hat ein Anrecht darauf, dass der für ihn vorgesehene Unterricht stattfindet.

\subsubsection{Einstellen neuer Lehrkräfte und bessere Optimierung des Systems (Vorschlag 1 / ACHTUNG: Abhängig von beschlossener Klassengröße)}
\abstimmung
Wir möchten das System analysieren und organisatorische Optimierungsmöglichkeiten zur Reduzierung des Unterrichtsaufalls nutzen. Darüber hinaus soll der Mangel an Lehrern durch Neueinstellungen bei Bedarf ausgeglichen werden. Durch unser zukünftiges Ziel die Klassengröße auf maximal 20 Schüler zu begrenzen, werden wir auch in Zukunft eine hohe Anzahl an Lehrkräften benötigen.

\subsubsection{Einstellen neuer Lehrkräfte und bessere Optimierung des Systems (Vorschlag 2 / ACHTUNG: Abhängig von beschlossener Klassengröße)}
\abstimmung
Wir möchten das System analysieren und organisatorische Optimierungsmöglichkeiten zur Reduzierung des Unterrichtsaufalls nutzen. Darüber hinaus soll der Mangel an Lehrern durch Neueinstellungen bei Bedarf ausgeglichen werden. Durch unser Ziel die Klassengröße kurzfristig auf maximal 20 Schüler und längerfristig auf maximal 15 Schüler zu begrenzen, werden wir auch in Zukunft eine hohe Anzahl an Lehrkräften benötigen.

\subsection*{Säkularisierung der Bildung}
\wahlprogramm{Säkularisierung der Bildung}
\antrag{KV Trier/Trier-Saarburg}\version{03:56, 20. Jun. 2010}

\subsubsection{Säkularisierung der Bildung}
\abstimmung
Wo Menschen unterschiedlichen Glaubens zusammenleben, müssen staatliche Bildungseinrichtungen weltanschaulich neutral sein. Der bisher in Landesverfassung und Schulgesetz vorhandene Religions- und Gottesbezug sollte deswegen gestrichen werden.
 
\wahlprogramm{Ethik-Unterricht}
\antrag{KV Trier/Trier-Saarburg}\version{03:56, 20. Jun. 2010}

\subsubsection{Ethik-Unterricht}
\abstimmung
Wir möchten für alle Schüler Ethikunterricht flächendeckend bereits ab der ersten Klasse anbieten.
 
\newpage
\subsubsection{Bildungsstandards}
\wahlprogramm{Bildungsstandards}
\antrag{KV Trier/Trier-Saarburg}\version{03:56, 20. Jun. 2010}

\subsubsection{Bildungsstandards}
\abstimmung
Auf Basis bildungspolitischer Erkenntnisse und der Diskrepanz zu derzeit herrschenden Bildungs-Missständen in Deutschland fordern wir die zügige Umsetzung der Bildungsempfehlungen (vom Institut zur Qualitätsentwicklung im Bildungswesen, HU Berlin und der Kultusministerkonferenz der Länder) nach festgesetzten Bildungsstandards auf Bundes- und Länderebene. Zur Gewährleistung bundeseinheitlicher Bildungsstandards in allen Bundesländern übernimmt das ausführende Organ der Bundesregierung die qualitätsführende Kontrolle und Evaluation.

\subsubsection{Vergleichbarkeit und bundesweiter Rahmen}
\abstimmung
Um die Vorteile des föderativen Schulsystems mit den Vorteilen eines zentral geregelten Bildungssystems zu verbinden, fordern wir mehr Richtlinienkompetenzen für den Bund. Dies betrifft insbesondere die Bereiche Vergleichbarkeit von Abschlüssen, Strukturausgleich und Freizügigkeit.
 
\subsubsection{Kindergärten/Kindertagesstätten u. Vorschulen}
\wahlprogramm{Kindergärten}
\antrag{Piraten aus RLP}\version{03:56, 20. Jun. 2010}

\subsubsection{Freier Zugang zu Kindergärten und Kindertagesstätten}
\abstimmung
Eltern müssen Kindergärten bzw. Kindertagesstätten für ihre Kinder frei wählen können. Jedem Kind muss dazu bis zum Schuleintritt ein kostenloser Kindergartenplatz in einem staatlichen Kindergarten in der Nähe zur Verfügung stehen.

\subsubsection{Ganztagsbetreuung / private Kinderbetreuung}
\abstimmung
Außerdem muss eine staatliche Ganztagsbetreung unter gleichen Bedingungen z.B. in Kindertagesstätten gewährleistet sein. Auch alternative Betreuungsangebote wie private Kinderbetreuung in Kleingruppen müssen staatlich finanziert werden. Eltern müssen über das Angebot ausreichend informiert werden.
 
\wahlprogramm{Bessere Ausbildung und Bezahlung von Erziehern}
\antrag{Piraten aus RLP}\version{03:56, 20. Jun. 2010}

\subsubsection{Bessere Ausbildung und Bezahlung von Erziehern}
\abstimmung
Von Erziehern und Betreuern im vorschulischen Bereich wird immer mehr gefordert. Wir planen daher, Bezahlung sowie Aus- und Fortbildung der Arbeitenden den neuen Anforderungen anzupassen, um die stärkere Belastung zu berücksichtigen.
 
\wahlprogramm{Vorschulische Förderung}
\antrag{Piraten aus RLP}\version{03:56, 20. Jun. 2010}

\subsubsection{Vorschulische Förderung}
\abstimmung
Der vorschulischen Förderung von Kindern kommt eine zentrale Bedeutung zu. Sie muss gewährleisten, dass alle Kinder unabhängig von ihrer sozialen, finanziellen und kulturellen Herkunft mit guten Grundvoraussetzungen ihre Schullaufbahn beginnen können. Alle Fördermöglichkeiten müssen für Kinder und Eltern kostenlos und frei zugänglich angeboten werden.
 
\subsection*{Grundschulen}
\wahlprogramm{Grundschulen}
\antrag{Piraten aus RLP}\version{03:56, 20. Jun. 2010}

\subsubsection{Verbesserung der Zusammenarbeit zwischen Grundschulen und weiterführenden Schulen}
\abstimmung
Wir wollen die Zusammenarbeit zwischen Grundschulen und weiterführenden Schulen verbessern, um Schülern gerade den Übergang von der 4. zur 5. Klasse zu erleichtern. Den Grundschulen wird es so ermöglicht die Schüler noch besser auf die Anforderungen der weiterführenden Schulen vorzubereiten. Auch die weiterführenden Schulen profitieren davon, wenn sie sehen welche Fähigkeiten und Arbeitstechniken Schüler in der Grundschule schon lernen und somit in der 5. Klasse mitbringen.
 
\newpage
\subsection*{Förderschulen}
\wahlprogramm{Förderschulen}
\antrag{unbekannt}\version{03:56, 20. Jun. 2010}

\subsubsection{Die Zukunft der Förderschulen (Variante 1)}
\abstimmung
In diesem Zusammenhang muss sich gleichermaßen über die Zukunft der Förderschulen und zudem verstärkt über bessere Integrationsmodelle Gedanken gemacht werden. Auch dieses Thema möchten wir nach pragmatischen Gesichtspunkten angehen und zusammen mit wissenschaftlichen Experten ein zukunftsfähiges Modell für diesen Bereich entwickeln.

\subsubsection{Die Zukunft der Förderschulen (Variante 2)}
\abstimmung
In Rheinland-Pfalz ist für lernbehinderte, körperbehinderte oder sonstige Kinder mit Förderbedarf das Risiko einer Förderschuleinstufung und der daraus folgenden Ausgrenzung aus dem Regelschulbetrieb im internationalen Vergleich besonders hoch. Der gemeinsame Unterricht von Kindern mit und ohne Behinderung wirkt sich, wie internationale Studien beweisen, auf den Lernerfolg beider Gruppen positiv aus. Deshalb wollen wir das hierzulande betriebene Modell der Förderschule soweit möglich verlassen und eine Schule für alle ermöglichen. Dies erfordert unter anderem bauliche Maßnahmen zum barrierefreien Zugang an Schulen.
 
\subsection*{Weiterführende Schulen}
\wahlprogramm{Die zukünftige Struktur unseres Bildungssystems}\label{wp:bildung:struktur}
\antrag{Piraten aus RLP}\konkurrenz{wp:bildung:eingliedrig}\version{03:56, 20. Jun. 2010}

\subsubsection{Die zukünftige Struktur unseres Bildungssystems}
\abstimmung
Die zukünftige Struktur unseres Bildungssystems soll sich an dem Wohl unserer Kinder ausrichten und nicht an einer ideologischen Diskussion. Wir möchten deshalb die verstärkte Erforschung von verschiedenen Schultypen und Schulformen fördern und die Diskussion ganz pragmatisch im Sinne zukünftiger Generationen gestalten. Es soll sich letztendlich für ein Modell entschieden werden, welches aus wissenschaftlich fundierter Sicht für unsere Kinder die besten Bildungs- und Zukunftschancen bringt.
 
\newpage
\wahlprogramm{Die zukünftige Struktur unseres Bildungssystems}\label{wp:bildung:eingliedrig}
\antrag{Piraten aus RLP}\konkurrenz{wp:bildung:struktur}\version{03:56, 20. Jun. 2010}

\subsubsection{Eingliedriges Schulsystem (Vorschlag 1)}
\abstimmung
In den letzten Jahren wurden zahlreiche Schwächen im deutschen Bildungssystem aufgedeckt. Die PISA-Studien haben gezeigt, dass es im deutschen Bildungssystem deutliche Schwächen gibt. In Deutschland hängt Bildung immer noch stark von dem sozialen Status der Eltern ab.

Die Chancen auf einen guten Schulabschluss, dürfen nicht von dem sozialen Status der Eltern abhängen. Dies stellt nicht nur eine Benachteiligung sozial schwacher Kinder da, Deutschland kann es sich auch nicht leisten nur einen Teil seiner Kinder gut auszubilden. Das Schulsystem muss dementsprechend umgebaut werden, dass die Schüler individuell gefördert werden und dass ihnen während ihrer gesamten Schullaufbahn alle Wege offen stehen.

Wir wollen die Reform, die mit der Realschule-Plus begonnen wurde, zu ihrem logischen Ende bringen. Die Piratenpartei befürwortet deshalb den Ausbau von Gesamtschulen in Rheinland-Pfalz. Langfristig soll an jeder Schule in Rheinland-Pfalz jeder Schulabschluss erreicht werden können. Dadurch soll vermieden werden, dass Schüler schon in der fünften Klasse eine Schullaufbahn einschlagen, die auf einen bestimmten Abschluss hinausläuft. Wie sich die Schulen intern organisieren, wollen wir den Schulen überlassen.

\subsubsection{Eingliedriges Schulsystem (Vorschlag 2)}
\abstimmung
Nicht nur die PISA-Studien haben gezeigt, dass es im deutschen Bildungssystem deutliche Schwächen gibt. In Deutschland hängt Bildung immer noch stark von dem sozialen Status der Eltern ab.

Die Chancen auf einen guten Schulabschluss dürfen nicht von dem sozialen Status der Eltern abhängen. Dies stellt nicht nur eine Benachteiligung sozial schwacher Kinder da, Deutschland kann es sich auch nicht leisten nur einen Teil seiner Kinder gut auszubilden. Das Schulsystem muss dementsprechend umgebaut werden, dass die Schüler individuell gefördert werden und ihnen während ihrer gesamten Schullaufbahn alle Wege offen stehen. Wir wollen die Reform, die mit der Realschule-Plus begonnen wurde, zu ihrem logischen Ende bringen.

Die Piratenpartei befürwortet deshalb den Ausbau von Gesamtschulen in Rheinland-Pfalz. Langfristig soll an jeder weiterführenden Schule in Rheinland-Pfalz jeder Schulabschluss erreicht werden können. So kann vermieden werden, dass Schüler schon in der fünften Klasse eine Schullaufbahn einschlagen, die auf einen bestimmten Abschluss hinausläuft. Die interne Organisation soll den jeweiligen Schulen überlassen werden.

\wahlprogramm{Die zukünftige Struktur unseres Bildungssystems}
\antrag{Piraten aus RLP}\version{03:56, 20. Jun. 2010}

\subsubsection{Unterstützung der Schulen durch nicht lehrendes Personal}
\abstimmung
Schulen müssen nach eigenem Ermessen auch mit ausreichend nicht-lehrendem Personal und finanziellen Ressourcen ausgestattet werden. Dazu zählen beispielsweise technische Assistenten, Sozialarbeiter und Personal für administrative Angelegenheiten.
 
\wahlprogramm{Die zukünftige Struktur unseres Bildungssystems}
\antrag{Piraten aus RLP}\version{03:56, 20. Jun. 2010}

\subsubsection{Kleinere Klassen (Vorschlag 1)}
\abstimmung
Um eine individuelle Förderung zu ermöglichen wollen wir schnellstmöglich den Klassenteiler an Rheinland-Pfälzischen Schulen auf 20 absenken und die Schulen finanziell besser ausstatten.

\subsubsection{Kleinere Klassen (Vorschlag 2)}
\abstimmung
Um eine individuelle Förderung zu ermöglichen wollen wir schnellstmöglich den Klassenteiler an Rheinland-Pfälzischen Schulen auf 15 absenken und die Schulen finanziell besser ausstatten.

\subsubsection{Kleinere Klassen (Vorschlag 3)}
\abstimmung
Um eine individuelle Förderung zu ermöglichen wollen wir schnellstmöglich den Klassenteiler an Rheinland-Pfälzischen Schulen kurzfristig auf 20 absenken und die Schulen finanziell besser ausstatten. Langfristig soll die größe eines Klassenverbandes maximal 15 Schüler betragen.
 
\wahlprogramm{Benotung und Bewertungskriterien}
\antrag{Piraten aus RLP}\version{03:56, 20. Jun. 2010}

\subsubsection{Benotung und Bewertungskriterien}
\abstimmung 
Die Aussagekraft einer Note außerhalb der Rahmenbedingungen, in der sie erhoben wurde, ist sehr gering. Eine Bewertung der Leistung kann nur als Orientierungshilfe für Schüler, Eltern und Lehrer innerhalb der Schullaufbahn dienen. Um diesen Zweck zu erfüllen, sollte die Bewertung von Schülern differenzierter als durch Noten erfolgen. Dazu gibt es zahlreiche Ansätze, die in der täglichen Praxis stärker umgesetzt werden müssen. Insbesondere sind detailliert aufgeschlüsselte fachliche Bewertungen wünschenswert. Kopfnoten lehnen die Piraten grundsätzlich ab.
 
\wahlprogramm{Zusätzliche Förderung und Betreuung in den Klassen 5 und 6 durch Doppelbesetzung}
\antrag{Piraten aus RLP}\version{03:56, 20. Jun. 2010}

\subsubsection{Zusätzliche Förderung und Betreuung in den Klassen 5 und 6 durch Doppelbesetzung}
\abstimmung 
Inbesondere die Schüle der Orientierungsstufe benötigen zusätzliche Förderung und Betreuung, um sich schnellstmöglich auf das neue Umfeld, die Arbeitsmethoden und das Leistungsniveau der weiterführenden Schule einzustellen. Gerade in den Bereichen Deutsch und Mathematik weisen die Schüler oft einen sehr großen Unterschied in ihrer Entwicklung und bei ihrem bisherigen Lernfortschritt auf. Ein einzelner Lehrer kann die große Anzahl der Schüler nicht gleichermaßen fördern. Wir fordern deshalb in den Klassen 5 und 6 jeweils eine Doppelbesetzung der Unterrichtsstunden.

\subsubsection{Ergänzung}
\abstimmung 
Die zweite Lehrkraft könnte hierbei auch ein Student in Ausbildung sein.
 
\wahlprogramm{Einbeziehung von Fachleuten in den Schulunterricht}
\antrag{Piraten aus RLP}\version{03:56, 20. Jun. 2010}

\subsubsection{Einbeziehung von Fachleuten in den Schulunterricht}
\abstimmung 
In stärkerem Maße als bisher sollen Fachleute auch in anderen Schularten als in den Berufsschulen in den Schulunterricht einbezogen werden – nicht nur für Gastvorträge, sondern als quereinsteigende Fachleute mit pädagogischer Eignung und entsprechender Zusatzausbildung. Bei Auswahl und Fortbildung dieser Experten ist darauf zu achten, dass der Unterricht in der Schule weltanschaulich neutral gehalten werden muss.

\newpage
\wahlprogramm{Reduzierung von Leistungsdruck und Stress}\label{wp:bildung:stress1}
\antrag{Piraten aus RLP}\konkurrenz{wp:bildung:stress2}\version{03:56, 20. Jun. 2010}

\subsubsection{Reduzierung von Leistungsdruck und Stress}
\abstimmung 
Gerade junge Menschen benötigen Zeit zur Entwicklung. Sie sollten ihre Kindheit und Jugend aktiv erleben, mit Freunden etwas unternehmen, Vereinsaktivitäten betreiben können und vieles mehr. Es ist nicht zweckmäßig, dass die Schüler dabei einem forlaufenden Leistungsdruck und Stress durch die Schule ausgesetzt sind. Wir fordern deshalb die Struktur und Inhalte der Bildungseinrichtungen in der Art zu überarbeiten, dass der forlaufende Leistungsdruck und Stress durch die Schule reduziert wird.

\subsubsection{Eine Schulzeit bis zum Abitur von mind. 12 1/2 Jahren}
\abstimmung 
Die Verkürzung der Schulzeit bei fast unverändertem Lehrplaninhalt ist unzweckmäßig. Menschen durchlaufen ihre Schulzeit dadurch möglicherweise schneller, doch persönliche Reife und das Erwerben von Lebenserfahrung benötigen dennoch ihre Zeit. Wir fordern deshalb eine Schulzeit bis zum Abitur von mind. 12 1/2 Jahren.

\subsubsection{Kein G8}
\abstimmung 
Das sogenannte G8-Gymnasium lehnen wir ab.
 
\wahlprogramm{Reduzierung von Leistungsdruck und Stress}\label{wp:bildung:stress2}
\antrag{KV Trier/Trier-Saarburg}\konkurrenz{wp:bildung:stress1}\version{03:56, 20. Jun. 2010}

\subsubsection{Leistungsdruck und Schulstress verringern}
\abstimmung 
Überfüllte Lehrpläne und Lernstandserhebungen sind hohe Stressfaktoren und setzen die Schüler unnötig unter Druck. Die Bildungspläne müssen angepasst werden, besonders der Bildungsplan des Gymnasiums an die zwölfjährige Schullaufbahn. Statt Lernstandserhebungen wie PISA oder VERA, die nur den Wissensstand messen, sollen langfristige Evaluationsverfahren eingesetzt werden, die auch die Selbstreflexion der Schüler einbeziehen und somit die Lernprozesse unterstützen.
 
\wahlprogramm{Politische Bildung}
\antrag{Piraten aus RLP}\version{03:56, 20. Jun. 2010}

\subsubsection{Ausweitung der Politischen Bildung (Vorschlag 1)}
\abstimmung 
Demokratische und politische Bildung ist ein elementarer Bestandteil, um heranwachsende Menschen auf das Leben und die Beteiligung in unserer freiheitlichen, demokratischen Gesellschaft vorzubereiten. Nur wenn sie die Demokratie umfassend begreifen, können sie sich an ihr aktiv beteiligen und sie dadurch erhalten. Wir fordern deshalb den landesweiten Ausbau der politischen Bildung

\subsubsection{Ausweitung der politischen Bildung (Vorschlag 2)}
\abstimmung 
Um eine Demokratie umfassend mitzugestalten und vor allem kontrollieren zu können, benötigen Menschen umfassende demokratische Handlungskompetenzen. Wir fordern deshalb eine Ausweitung der demokratischen und gleichermaßen der politischen Bildung und dementsprechend auch mehr Zeitkontingente für den Sozialkundeunterricht im schulischen Alltag.

\subsubsection{Ausweitung des Sozialkundeunterrichts (Vorschlag 1)}
\abstimmung 
Dem Sozialkundeunterricht steht an rheinland-pfälzischen Schulen ein sehr geringes Stundenkontingent zu (an Gymnasien bspw. zwei Unterrichtsstunden pro Woche in Klasse 9 und eine Unterrichtsstunde pro Woche in Klasse 10). Dies ist eindeutig zu wenig, um den Schülern angemessene soziale, politische und wirtschaftliche Kompetenzen und Fähigkeiten zu vermitteln. Wir fordern, unabhängig von der Schulform, einen Sozialkundeunterricht ab Klasse 7 mit einem Stundenkontigent von mindestens drei Stunden pro Woche.

\subsubsection{Ausweitung des Sozialkundeunterrichts (Vorschlag 2)}
\abstimmung 
Um als mündiger Bürger an der demokratischen Willensbildung mitzuwirken wird ein Grundverständnis unseres politischen Systems benötigt. Aus diesem Grund fordern wir eine verbesserte politische Bildung und einen Ausbau des Sozialkundeunterrichts.
 
\newpage
\wahlprogramm{Mehr Zeit für Projekte}
\antrag{Piraten aus RLP}\version{03:56, 20. Jun. 2010}

\subsubsection{Mehr Zeit für Projekte}
\abstimmung 
Wir alle wissen, am besten lernt man, indem man viel selbst ausprobiert und übt. Handlungsorientierung sollte dementsprechend in der Schule im Vordergrund stehen. Wir fordern mehr Zeit für Projekte an Schulen und vor allem auch mehr Zeit für die Zusammenarbeit zwischen mehreren Schulen. Die Schüler erhalten so die Möglichkeit sich mit Themen intensiver auseinanderzusetzen und selbst aktiv zu werden.
 
\wahlprogramm{Mehr Zeit für Projekte}
\antrag{Piraten aus RLP}\version{03:56, 20. Jun. 2010}

\subsubsection{Flächendeckende Einführung von Informatik im Leistungskursangebot der Oberstufe}
\abstimmung
Wir fordern die flächendeckende Einführung von Informatik im Leistungskursangebot an allen rheinland-pfälzischen Schulen mit Oberstufe.
 
\wahlprogramm{Mehr Zeit für Projekte}
\antrag{Piraten aus RLP}\version{03:56, 20. Jun. 2010}

\subsubsection{Lernen fürs Leben}
\abstimmung
Viele Aspekte des alltäglichen Lebens werden in der Schule nicht aufgegriffen. Wir fordern, dass die Aspekte Ernährung, Gesundheit, Medienkompetenz, Verbraucherkompetenz und Kritisches Denken in Projektgruppen mit praktischer Ausrichtung die Schüler auf ein mündiges, selbstbestimmtes und informiertes Leben vorbereiten. Wichtig hierbei ist, dass Schüler aktiv involviert werden und praktisch arbeiten.

\subsubsection{Ernährung, Bewegung, Gesundheit}
\abstimmung
Wir setzen uns dafür ein, dass die Themen Gesundheit, Ernährung und Bewegung unter aktuellen wissenschaftlichen Erkenntnissen und in ausreichendem Maß an Schulen gelehrt werden. Erklärtes Ziel ist es, Schülern eine ausgewogene Lebensweise zu vermitteln. Dies kann gefördert werden, indem theoretische Überlegungen praktisch angewandt werden.

\subsubsection{Ernährung}
\abstimmung
Durch gemeinsames Kochen und Essen, bei gleichzeitiger Erläuterung der theoretischen Hintergründe, werden die Schüler zu einer ausgewogenen Ernähung angeregt.

\subsubsection{Bewegung}
\abstimmung
Der Spass an der Bewegung sollte gefördert werden. Statt des üblichen Rahmenlehrplans, sollten Sportarten einzeln angeboten werden. Ob sich ein Schüler letztendlich für Leichtathletik, Teamsport oder Kraftsport entscheidet soll seine persönliche Entscheidung sein.

\subsubsection{Gesundheit}
\abstimmung
Die Schüler sollen über die Bereiche Sexualität, Gewalt und Suchtprävention ausgiebig aufgeklärt werden.
 
\wahlprogramm{Kritisches Denken}
\antrag{Piraten aus RLP}\version{03:56, 20. Jun. 2010}

\subsubsection{Kritisches Denken, Medienkompetenz und Umgang mit Verbraucherrechten}
\abstimmung
Der Umgang mit Medien und das kritische Hinterfragen von aktuellen Begebenheiten ist eine wichtige Kernkompetenz des Lebens. Die Komplexität des heutigen Informations- Dienstleistungs-, Medien- und Produktangebots erfordert oft die kritische Auseinandersetzung mit sozialen, wirtschaftlichen, gesellschaftlichen und rechtlichen Aspekten.

\subsubsection{(Ergänzung) Projektgruppen}
\abstimmung
1. In Projektgruppen sollen daher praktische Erfahrungen zu folgenden Bereichen gesammelt werden:

2. Informationsbeschaffung,-Selektion und -Diskussion 3. Mediengestaltung, Medienkompetenz 4. Datenschutz und verantwortlicher Umgang mit Daten 5. Auseinandersetzung mit Verträgen und Verbindlichkeiten 6. Haushaltsplanung, Finanzierung, Umgang mit Geld

7. Diskussion von Nachrichten, Religion und politischem Tagesgeschehen

\subsubsection{(Ergänzung) Einrichtung eines neuen Unterrichtsfachs}
\abstimmung
Wir fordern deshalb, dass für diese Erfordernisse ein neues Unterrichtsfach eingerichtet wird. Das Vermitteln dieser Fähigkeiten kann in einem Frontalunterricht jedoch nicht funktionieren. Deshalb müssen sie im Rahmen eines offenen Unterrichtes vermittelt werden, der aktive Teilhabe und praktisches Arbeiten fordert und fördert.

\wahlprogramm{Erweitertes Angebot an Fremdsprachen}
\antrag{Piraten aus RLP}\version{03:56, 20. Jun. 2010}

\subsubsection{Erweitertes Angebot an Fremdsprachen}
\abstimmung
Derzeit werden Synergieeffekte, die sich beim Lernen bestimmter Sprachkombinationen ergeben nicht sinnvoll genutzt. Dies liegt vorallem an dem stark eingeschränkten Angebot an Sprachen, das in der Regel derzeit nur Latein/Französich/Englisch umfasst. Als zweite Fremdsprache würde sich zum Beispiel Spanisch eignen da es hier große Überschneidungen mit dem Französichen und auch dem Latein gibt. Wir kämpfen daher für ein größeres Angebot von Sprachkursen an Schulen.Unterrichtsgeschehen integriert.
 
\wahlprogramm{Bundeswehr}
\antrag{Thomas Heinen}\zusatz{wp:bildung:planspiele}\version{03:56, 20. Jun. 2010}

\subsubsection{Keine Bundeswehr an Schulen}
\abstimmung
Wir sehen die Entsendung von Jugendoffizieren der Bundeswehr für Lehrzwecke in Klassenzimmer und zur Aus- bzw. Weiterbildung von Lehrkräften sehr kritisch. Klassenzimmer sollen nicht zu Rekrutierungsbüros werden.

\subsubsection{ }
\abstimmung
Von der Bundeswehr ausgebildete Referendare, einseitiges Unterrichtsmaterial, Bundeswehrbesuche und von Soldaten gestaltete Unterrichtseinheiten mit Abiturprüfungsinhalten dienen der Manipulation und Rekrutierung, nicht der Erziehung zur eigenständigen Auseinandersetzung mit der Problematik.

\subsubsection{Kooperation des Landes mit Bundeswehr auflösen}
\abstimmung
Die Kooperationsvereinbarung des Landes RLP mit der Bundeswehr zum Einsatz von Jugendoffizieren im Unterricht an rheinland-pfälzischen Schulen lehnen wir ab und fordern deren Aufkündigung. Einseitige Information und Bundeswehrplanspiele haben im Unterricht nichts verloren. Wir fordern einen ausgewogenen Unterricht und die kontroverse Darstellung und Diskussion von Themen, die in der Öffentlichkeit umstritten erscheinen. Die Bundeswehr darf an Schulen nur informieren, wenn gleichzeitig auch Kritiker zu Wort kommen.
 
\wahlprogramm{Planspiele}\label{wp:bildung:planspiele}
\antrag{unbekannt}\zusatz{wp:bildung:bundeswehr}\version{03:56, 20. Jun. 2010}

\subsubsection{Mehr staatliche geförderte Planspiele im Unterricht}
\abstimmung
Planspiele sind, gerade im politischen Unterricht, ein hervoragend geeignetes Mittel, um Schülern selbst komplexe Sachverhalte spielend und handlungsorientiert näher zu bringen und sie für Demokratie und Politik zu begeistern. Wir fordern deshalb mehr Zeit und mehr Angebote für Planspiele im Unterricht. Die Weiterentwicklung und Durchführung von Planspielen ist allerdings oft auch mit starken finanziellen Kosten verbunden, diese sollen vom Staat getragen werden, damit alle Schüler davon partizipieren können.
 
\newpage
\subsection*{Hochschulen}
\wahlprogramm{Zugang zu Bildung}\label{wp:bildung:zugang}
\antrag{Niemand13}\konkurrenz{wp:bildung:beschraenkung}\version{03:56, 20. Jun. 2010}

\subsubsection{Zugang zu Bildung verbessern!}
\abstimmung
Der freie Zugang zu Information, Bildung, Ausbildung und Weiterbildung ist für die Gesellschaft und eine starke Demokratie dringend notwendig und eine der wichtigsten Ressourcen und Investitionen in die Zukunft. Er ist daher im Interesse aller und es ist vordergründige staatliche Aufgabe eine gute und moderne Bildungsinfrastruktur zu finanzieren und jederman frei und kostenlos zur Verfügung zu stellen. Der Zugang zur Hochschule ist aktuell entgegen aller Lippenbekenntnisse stark eingerschränkt und für viele Menschen unmöglich. Viele Menschen können ihr Recht auf ein Studium nicht wahrnehmen, viele müssen ihr Studium vorzeitig abbrechen. Bewerber werden durch hohe NC-Hürden daran gehindert, überhaupt erst ein Studium zu beginnen. Gründe verschiedenster Art, wie Kindererziehung, soziales Engagement, Studiengangwechsel, Selbstfinanzierung und/oder familiäre, bzw. persönliche Schwierigkeiten erschweren die Durchführung des Studiums in Regelstudienzeit. Unserer Auffassung nach ist viel zu wenig Lehrpersonal vorhanden, um allen Studieninteressierten die Möglichkeit zur Aufnahme eines von ihnen gewünschten Studiums zu geben, oder auch nur den schon Studierenden gute Lernbedingung zu bieten und deren individuelle Betreuung zu ermöglichen. Die Konsequenz ist aktuell der Ausschluss Interessierter von einem Studium ihrer Wahl und überfüllte Veranstaltungen.

\subsubsection{Bildungsbarrieren abbauen!}
\abstimmung
Arbeitsaufwand und Anzahl Studierender pro Veranstaltung sind zu hoch, so dass Dozierende sich zwischen Vernachlässigung der Lehre und damit der Verpflichtung gegenüber den Studierenden oder Vernachlässigung der Forschung und damit der eigenen wisschenschaftlichen Karriere entscheiden müssen. Wir fordern die Gewährleistung des in der Verfassung verbrieften Rechts auf Bildung für alle Menschen und wollen die Hochschulen so ausstatten, dass dies uneingeschränkt wahrgenommen werden kann. Körperliche, soziale und finanzielle Beeinträchtigungen dürfen kein Hindernis für die Zulassung zu einem Studium und dessen erfolgreicher Durchführung und Beendigung sein. Eine ausreichende Finanzierung und Ausstattung der Hochschulen wollen wir sicherstellen. Die deutliche Erhöhung des BAFöG-Satzes sehen wir als dringend notwendig an und messen ihre hohe Priotität bei. Wir fordern die Abschaffung von Zulassungsbeschränkungen für alle Studiengänge. Mit einem an der Anzahl der Studieninteressierten orientierten Ausbau von Studienplätzen wollen wir jegliche Zulassungsbeschränkung obsolet machen. Die Wahl des Studienganges muss auf Grund des Interesses und nicht auf Grund von hohen NC-Hürden getroffen werden.
 
\wahlprogramm{Gegen Zulassungsbeschränkungen}\label{wp:bildung:beschraenkung}
\antrag{Piraten aus RLP}\konkurrenz{wp:bildung:zugang}\version{03:56, 20. Jun. 2010}

\subsubsection{Gegen Zulassungsbeschränkungen}
\abstimmung
Angehende Studenten werden durch Zulassungsbeschränkungen in ihrer Freiheit eingeschränkt. Wir fordern deshalb langfristig einen weiteren Ausbau der Universitäten, so dass Zulassungsbeschränkungen abgeschafft werden können.
 
\wahlprogramm{Keine Studiengebühren}\label{wp:bildung:gebuehren1}
\antrag{Piraten aus RLP}\konkurrenz{wp:bildung:gebuehren2}\version{03:56, 20. Jun. 2010}

\subsubsection{Keine Studiengebühren}
\abstimmung
Die Steigerung der Qualität und den Ausbau der Universitäten wollen wir dabei nicht durch Studiengebühren finanzieren. Die Piratenpartei lehnt Studiengebühren für das Erststudium generell ab.
 

\wahlprogramm{Abschaffung Studiengebühren}\label{wp:bildung:gebuehren2}
\antrag{Thomas Heinen}\konkurrenz{wp:bildung:gebuehren1}\version{03:56, 20. Jun. 2010}

\subsubsection{Abschaffung Studiengebühren}
\abstimmung
Jeder Mensch hat das Recht auf die Teilhabe an der Gesellschaft, auf Bildung und kulturelle Betätigung. Studiengebühren und andere finanzielle Zusatzbelastungen im Studium halten Menschen aber vom Studieren ab. Wir fordern daher die Abschaffung der Studiengebühren und weiterer finanzieller Zusatzbelastungen für Studierende wie Verwaltungsgebühren, um barriere- und kostenfreie Bildung für alle zu realisieren. Das Land muss dafür Sorge tragen, dass den Universitäten und studentischen Organisationen die finanziellen Ausfälle ersetzt werden.
 
\wahlprogramm{Bologna-Reform}
\antrag{Niemand13}\version{03:56, 20. Jun. 2010}

\subsubsection{Bologna-Reform reformieren!}
\abstimmung
Wir fordern ein freies und selbstbestimmtes Studium ohne bürokratische Hürden, ohne stetigen Leistungsdruck und starren vorgegebenen Stundenplan, wie sie heute Studierenden- Alltag sind. Durch hohen Leistungsdruck, Dauerüberprüfung und eine rigorose Modularisierung bleibt kein Freiraum mehr für individuelle Schwerpunktsetzung. Wir wollen die Regelstudienzeit der Bachelorgänge prüfen und die Prüfungslast mit dem Ziel der Reduzierung evaluieren. Den permanten Prüfungsdruck sowie den Einfluss von Einzelleistungen auf die Gesamtnote wollen wir herabsetzen. Wir wollen eine Ausweitung der Kombinationsmöglichkeiten der Fächer untereinander, so dass eine breit gefächerte, freie Bildung möglich wird. Dabei müssen auch die Fächer gleichwertig berücksichtigt werden, die abseits des jeweilig üblichen Fächerkanons liegen oder aus fachbezogenen Studiengängen stammen. Um die durch den Bachelor zu erzielende Erleichterung von Auslandsaufenthalten zu erreichen, müssen zukünftig auch sämtliche, bei Auslandsaufenthalten in den eigenen Fächern erbrachten Leistungen, anerkannt werden.

\subsubsection{Vielfalt bewahren!}
\abstimmung
Kleine und ohnehin schon untervertretene Studienfächer wollen wir am Leben erhalten: Lehre und Forschung in solchen Fächern darf nicht aus mangelnder Popularität eingestellt werden! Wir fordern die Umsetzung der eigentlichen Ziele, die die Bologna-Reform mit ihrem aktuellen Konzept für Bachelor und Master verfehlt hat: Die Schaffung einfach verständlicher und gut vergleichbarer Abschlüsse, die Erhöhung der internationalen Mobilität und die Reduzierung der Zahl der StudienabbrecherInnen durch ein verkürztes, überschaubares Studium. Wir fordern einen massiven Ausbau der Master-Studienplätze! Derzeit ist nur einem Bruchteil der BachelorabsolventInnen ein Platz sicher. Dies führt zu neuen Bildungshürden und die Abschlussnote wird den persönlichen Fähigkeiten vorangestellt. Jedem Interessenten muss ein Masterstudium ermöglicht werden! Zulassungsquoten lehnen wir ab.
 
\subsubsection{Familienfreundlichere Hochschulen}
\antrag{Piraten aus RLP}\version{03:56, 20. Jun. 2010}

\subsubsection{Familienfreundlichere Hochschulen}
\abstimmung
Hochschulen sollen familienfreundlicher gestaltet werden. Dies betrifft sowohl die Arbeit in Forschung, Lehre und Verwaltung als auch das Studium. Eine akademische Karriere muss parallel zur Kindererziehung möglich sein. Hierzu sollen (gerade auch für Professoren, Doktoranden und den wissenschaftlichen Nachwuchs) verstärkt Teilzeitstellen angeboten werden. Gleichzeitig muss die Kinderbetreuung an Hochschulen ausgebaut werden, so dass für alle Kinder von Studenten oder Angestellten der Universität Betreuungsplätze zur Verfügung stehen.
 
\subsubsection{Höhere Qualität der Bildung}
\antrag{Piraten aus RLP}\version{03:56, 20. Jun. 2010}

\subsubsection{Höhere Qualität der Bildung an Universitäten}
\abstimmung
Aufgrund von Überfüllung und starker Unterfinanzierung sind an den Universitäten erhebliche qualitative Mängel in der Ausbildung der Studenten entstanden. Die Qualität des Studiums sinkt durch Überfüllung und eine schlechte Betreuung. Aus den gleichen Gründen wird es zunehmend schwerer, das Studium schnell zu beenden. Wir fordern eine bessere finanzielle Ausstattung der Universitäten, die Einstellung zusätzlicher Dozenten und den Ausbau der Universitäten, so dass genügend Platz für die Studierenden zur Verfügung steht.

\newpage
\wahlprogramm{Freies und selbstbestimmtes Studieren}
\antrag{Piraten aus RLP}\version{03:56, 20. Jun. 2010}

\subsubsection{Freies und selbstbestimmtes Studieren}
\abstimmung
Auch an den Universitäten in Rheinland-Pfalz gibt es massive Probleme, dies haben die zahlreichen Studentenproteste gezeigt. Das Studium wurde durch die Bachelor- und Masterstudiengänge zunehmend verschult, die Studenten werden immer mehr eingeengt. Die Piratenpartei fordert deshalb eine schnelle Reform der Bologna-Reformen, damit wieder ein individuelles und selbstbestimmtes Studieren ermöglicht wird. Als erste Maßnahme sollen dabei auch die entsprechenden Studienordnungen für die Umsetzung der Bologna-Refom im Land Rheinland-Pfalz überprüft und mit Blick auf ein freies und selbstbestimmtes Studieren überarbeitet werden.
 
\subsection*{Lehrer / Lehrerausbildung}
\wahlprogramm{Praxiserfahrung}
\antrag{Piraten aus RLP}\version{03:56, 20. Jun. 2010}

\subsubsection{Praxiserfahrung für Lehramtsstudenten ab dem 1. Semester}
\abstimmung
Wir setzen uns für eine Erhöhung der frühzeitigen praktischen Tätigkeiten der Lehramtsstudierenden ein.
 
\wahlprogramm{Duales Studium für angehende Lehrkräfte}
\antrag{Piraten aus RLP}\version{03:56, 20. Jun. 2010}

\subsubsection{Duales Studium für angehende Lehrkräfte}
\abstimmung
Wir möchten, dass das Lehramtsstudium in ein duales Studium, ähnlich einer Berufsakademie, umgewandelt wird. Die Studierenden erhalten während ihrer Ausbildung schon ein Gehalt und werden regelmäßig ins Unterrichtsgeschehen integriert.

\subsubsection{Ergänzung}
\abstimmung
Hierdurch lassen sich Engpässe im Personal hervorragend ausgleichen.
 
\wahlprogramm{Neutralität, Unabhängigkeit und Gerechtigkeit der Lehre}\label{wp:bildung:neutralitaet}
\antrag{Pirat aus RLP}\konkurrenz{wp:bildung:beamten}\version{03:56, 20. Jun. 2010}

\subsubsection{Neutralität, Unabhängigkeit und Gerechtigkeit der Lehre sichern, Beamtenstatus bewahren}
\abstimmung
Neutralität, Unabhängig und Gerechtigkeit sind für ein demokratisches Bildungssystem von elementarer Bedeutung. Nur ein Lehrer, der nicht politischem, gesellschaftlichem und wirtschaftlichem Druck ausgesetzt ist, kann Inhalte objektiv und kritisch vermitteln und so bei Schülern die kritischen Denkprozesse anstoßen, welche in einer Demokratie so dringend benötigt werden.

\subsubsection{Modul2}
\abstimmung
Sollte der Beamtenstatus abgeschafft werden, währe im Bildungsbereich befristeten Arbeitsverträgen und einer Entlassung, bspw. über die Ferien, Tür und Tor geöffnet. Die dringend benötigte Motivation der Lehrkräfte für ihre gesellschaftlich elementare Aufgabe wird durch solche prekären Arbeitsverhältnisse stark reduziert. In den Ferien soll ein Lehrer die Möglichkeit haben sich auf das neue Unterrichtsjahr vorzubereiten. Eine drohende Arbeitslosigkeit mit bürokratischen Maßnahmen des Arbeitsamtes ist in dieser Zeit absolut unakzeptabel.

\subsubsection{Modul3}
\abstimmung
Wir fordern deshalb den Erhalt und die Fortführung des Beamtenstatus.

\wahlprogramm{Anreize für Lehrer}
\antrag{unbekannt}\version{03:56, 20. Jun. 2010}

\subsubsection{Anreize für Lehrer schaffen, Demotivation und Burn-Out rechtzeitig verhindern}
\abstimmung
Die stark überwiegende Mehrheit der Lehrer im Land setzt sich für einen guten Unterricht und ihre Schüler ein. Gerade aber diese stark engagierten Lehrkräfte sind früher oder später durch Frustration von Demotivation und Burn-Out-Syndrom betroffen. Zudem sind die Aufstiegschancen in der Beamtenlaufbahn als Lehrer sehr begrenzt. Nicht der Beamtenstatus, sondern fehlende Anreize für engagierte Lehrkräfte und vor allem die fehlende Ausbildung im Umgang mit Frustration und der zunehmenden Belastung im Schulalltag sind hier das Problem. Dieses löst man nicht, indem man den Beamtenstatus abschafft und engagierten Lehrkräften zusätzlich die berufliche Sicherheit entzieht. Wir fordern stattdessen die Beamtenlaufbahn mit mehr Anreizen und Aufstiegschancen zu gestalten und vor allem mehr Hilfen für engagierte Lehrer anzubieten, um Symptome eines Born-Outs rechtzeitiger zu erkennen und zu vermeiden.

\wahlprogramm{Beamtenstatus abschaffen}\label{wp:bildung:beamten}
\antrag{KV Trier/Trier-Saarburg}\version{03:56, 20. Jun. 2010}
\begin{itemize}
\item \konkurrenz{wp:bildung:neutralitaet}
\item \konkurrenz{wp:bildung:prekaer}
\end{itemize} 

\subsubsection{Beamtenstatus abschaffen}
\abstimmung
Wir setzen uns dafür ein, den Beamtenstatus im Bildungsbereich abzuschaffen und auf gleichberechtigte und faire Arbeitsbedingungen für alle Lehrenden im Schul- und Hochschulbereich hinzuwirken.
 

\wahlprogramm{Prekäre Beschäftigungssituationen}\label{wp:bildung:prekaer}
\antrag{unbekannt}\konkurrenz{wp:bildung:beamten}\version{03:56, 20. Jun. 2010}

\subsubsection{Prekäre Beschäftigungssituationen im Bildungssektor verhindern!}
\abstimmung
Durch die Abschaffung des Beamtenstatus darf im Bildungssektor prekären Beschäftigungsverhältnissen nicht Tür und Tor geöffnet werden. Prekäre Beschäftigungsverhältnisse senken die Motivation und sorgen für Verunsicherung bei den Angestellten. Sie sind für die Erfüllung der hohen Erwartungen, welche an Bildungseinrichtungen gestellt werden, nicht förderlich. Engagierten Lehrern muss eine langfristige Perspektive geboten werden. Entlassungen über die Ferien aus Kostengründen und stark befristete Verträge (Ausnahme: Vertretungslehrkräfte) schließen wird definitiv aus. Was für Lehrer gilt, muss auch für Erzieher gelten. Auch ihnen soll eine langfristige Perspektive geboten werden.
 
\wahlprogramm{Zweiklassengesellschaft}
\antrag{KV Trier/Trier-Saarburg}\version{03:56, 20. Jun. 2010}

\subsubsection{Gegen eine Zweiklassengesellschaft im Lehrer- und Dozentenbereich}
\abstimmung
Von der Öffentlichkeit weitgehend unbemerkt schleicht sich eine Zweiteilung im Bereich der Bildungsvermittler ein: Auf der einen Seite stehen gut abgesicherte Beamte auf Lebenszeit, auf der anderen Seite billige Honorarkräfte, die in den Schulen große Teile des Nachmittagsunterrichts und der Betreuung übernehmen beziehungsweise die an den Hochschulen als Lehrbeauftragte in vielen Bereichen dafür sorgen, dass überhaupt noch ein ausreichendes Lehr- und Betreuungsangebot vorhanden ist. Die Piratenpartei Rheinland-Pfalz wird sich dafür einsetzen, neue unbefristete Hochschulstellen vor allem im Bereich der wissenschaftlichen Mitarbeiter einzurichten. Bestehende Lehraufträge an Schulen und Hochschulen wollen wir angemessener als bisher vergüten und befristete in unbefristete Arbeitsverträge umwandeln.
 
\subsection*{Erwachsenenbildung}
\wahlprogramm{Erwachsenenbildung}
\antrag{Pirat aus RLP}\version{03:56, 20. Jun. 2010}

\subsubsection{Erwachsenenbildung}
\abstimmung
Die Erwachsenenbildung ist ein weites Feld. Sie reicht von Alphabetisierungskursen und Sprachkursen im Rahmen der Integration von Zuwanderern über die betriebliche Fortbildung und privatwirtschaftliche Qualifizierung bis hin zu einem Zweit- oder Drittstudium an einer Hochschule. Die Landesregierung sollte den Aufbau von frei zugänglichem Lehr- und Unterrichtsmaterialien in diesen Bereichen finanziell fördern, um den Zugang zu Bildung auch für Erwachsene zu vereinfachen.

\subsubsection{Ausbau des Volkshochschulangebots}
\abstimmung
Die Piratenpartei regt daher an das System der Volkshochschulen durch den Ausbau zertifizierter Fortbildungsmöglichkeiten zu stärken. Dies kann parallel zu den existierenden privatwirtschaftlich geführten Bildungsunternehmen und -initiativen vonstatten gehen. Dazu sollen die Volkshochschulen durch die Einführung von Summerschools, Kursen und Curricula in Kooperation mit den Berufsakademien, Fachhochschulen und Universitäten noch effizienter als bisher in unsere Bildungslandschaft integriert werden. Hierzu ist die Bereitstellung von Online-Werkzeugen, die ein orts- und zeitunabhängiges Lernen fördern und ermöglichen, unerlässlich. Angeregt wird daher die staatlich finanzierte beziehungsweise staatlich geförderte Bereitstellung von Lernplattformen zum integrierten Lernen als flankierende Maßnahme.

\subsubsection{ }
\abstimmung
Wir wollen ein integratives Konzept „Lebenslanges Lernen“ aufbauen, das Volkshochschulen mit Schulen, Fachhochschulen, Berufsschulen, Universitäten und andere Bildungseinrichtungen zu einem Verbund der Erwachsenenbildung effektiv zusammenführt.

\newpage
\subsection*{Pluralismus vs. Extremismusprävention}
\wahlprogramm{Förderung von Projekten zur Meinungsvielfalt}\label{wp:bildung:projekte}
\antrag{Pirat aus RLP}\konkurrenz{wp:bildung:demokratie}\version{03:56, 20. Jun. 2010}

\subsubsection{Förderung von Projekten zur Meinungsvielfalt}
\abstimmung
Meinungsvielfalt stellt eine Grundlage unserer Demokratie dar. Mit so genannten "Anti-Extremismus-Projekten" und "Anti-Extremismus-Prävention" wird versucht, Meinungsvielfalt zu unterdrücken und gesellschaftlich radikale Positionen zu diskreditieren. Wir setzen uns daher gegen pauschale "Extremismusprävention" ein an deren Ende zwangsläufig eine uniforme Gesellschaft stehen muss. Wir PIRATEN stehen für Meinungsvielfalt und demokratische Diskussion auch radikaler und innovativer Positionen.
 
\wahlprogramm{Demokratie schützen}\label{wp:bildung:demokratie}
\antrag{Pirat aus RLP}\konkurrenz{wp:bildung:projekte}\version{03:56, 20. Jun. 2010}

\subsubsection{Demokratie schützen - Förderung der Extremismusprävention}
\abstimmung
Die verschiedenen Formen des Extremismus stellen eine fortlaufende Gefahr für unsere freiheitlich demokratische Grundordnung dar. Die Menschenrechte, allen voran die unantastbare Würde des Menschen, sind die essentielle Grundlage für unsere demokratische Gesellschaft. Offene Denkmuster wie Rassismus, Fremdenfeindlichkeit oder auch Antisemitismus können und dürfen wir nicht dulden. Wo die würde des Menschen (bspw. durch rassitischen Hass) angegriffen wird, sind der Meinungsfreiheit deutliche Grenzen gesetzt. Wir setzen uns deshalb für die verstärkte Förderung der Extremismusprävention (insbesondere der Rechtsextremismusprävention) ein. Gerade ehrenamtlichen Projekten soll hierbei die bestmögliche, insbesondere finanzielle, Unterstützung zu Teil werden.
 
\subsection*{Nachwort}
\wahlprogramm{Nachwort}
\antrag{Niemand13}\version{03:56, 20. Jun. 2010}
\subsubsection{Nachwort - freie Bildung ist Sauerstoff für die Demokratie}
\abstimmung
Die Piratenpartei Rheinland-Pfalz will die obigen Forderungen auf allen Ebenen konsequent vertreten und umsetzen und so ein freies Lernen sowie Bildungsgerechtigkeit und Chancengleichheit etablieren und soziale Ungleichheit beseitigen. Wir sehen Bildung als Schlüsselfaktor zur gesellschaftlichen Teilhabe in der Informationsgesellschaft und als Grundlage für Frieden und Demokratie.

%%\section{Demokratie und Teilhabe}

\subsection*{Einleitung zum Thema: Modernisierung der Demokratie}
\wahlprogramm{Modernisierung der Demokratie}
\antrag{Unbekannt}\version{03:43, 20. Jun. 2010}

\subsubsection{Einleitung zum Thema: Modernisierung der Demokratie}
\abstimmung
Die Art und Weise wie sich Bürger in unserer Demokratie engagieren hat sich über die letzten Jahrzehnte zunehmend verändert. Statt sich in Parteien zu organisieren und am Ende jeder Legislaturperiode einmal zur Wahl zu gehen, bringen sich die Bürger zunehmend mit Hilfe von Organisationen und Bürgerinitiativen direkt in den demokratischen Prozess ein. Es reicht also nicht mehr, nur alle vier oder fünf Jahre eine Wahl zu veranstalten, um dem Verlangen der Bürger nach politischer Teilhabe gerecht zu werden. Um dieser Veränderung gerecht zu werden, müssen mehr Möglichkeiten geschaffen werden, wie sich die Bürger auch auf Landesebene direkt einbringen können.
 
\subsection*{Transparenz in der Demokratie}
\wahlprogramm{Transparenz in der Demokratie}
\antrag{Unbekannt}\version{03:43, 20. Jun. 2010}

\subsubsection{Transparenz in der Demokratie}
\abstimmung
Um diese Möglichkeiten zu schaffen will die Piratenpartei die Transparenz in der Erarbeitungsphase von Gesetzen auf Landesebene verbessern. Informationen, die die Bürger benötigen, um sich in die politischen Prozesse einzubringen, sollen schnell, übersichtlich und einfach zugänglich gemacht werden.
 
\subsection*{Deliberative Demokratie}
\wahlprogramm{Deliberative Demokratie}
\antrag{Unbekannt}\version{03:43, 20. Jun. 2010}

\subsubsection{Möglichkeiten der deliberativen Demokratie}
\abstimmung
Die Piratenpartei will die Möglichkeiten der deliberativen Demokratie, also Bürgerbeteiligung, in der Erarbeitungsphase von Gesetzen in Rheinland-Pfalz verstärkt nutzen. Das Veranstalten von Bürgerkongressen, Planungszellen oder auch das Nutzen von elektronischen Beteiligungsmöglichkeiten über das Internet darf keine Ausnahme sein, sondern muss bei wichtigen Gesetzen zur Regel werden.
 
\subsection*{Niedrigere Quoren für die direkte Demokratie und obligatorische Volksentscheide}
\wahlprogramm{Niedrigere Quoren für die direkte Demokratie und obligatorische Volksentscheide}\label{wp:demokratie:niedrig}
\antrag{Unbekannt}\konkurrenz{wp:demokratie:mehr}\version{03:43, 20. Jun. 2010}

\subsubsection{Möglichkeiten von Volksbegehren und Volksentscheiden auf Landesebene ausbauen}
\abstimmung
Die Piratenpartei will die Möglichkeiten von Volksbegehren und Volksentscheiden auf Landesebene ausbauen. Wir fordern die Absenkung der Quoren für die direkt demokratischen Instrumente in Rheinland-Pfalz. Bei großen landespoltischen Themen wie der Kommunalreform wird die Piratenpartei immer die Bürger selbst in einen Volksentscheid die endgültige Entscheidung treffen lassen. Zudem wollen wir einen obligatorischen Volksentscheid bei Änderungen der Landesverfassung.
 
\wahlprogramm{Mehr Bürgerbeteiligung}\label{wp:demokratie:mehr}
\antrag{Thomas Heinen}\konkurrenz{wp:demokratie:niedrig}\version{03:43, 20. Jun. 2010}

\subsubsection{Mehr Bürgerbeteiligung - weniger Hürden bei Volksbegehren}
\abstimmung
Die Piratenpartei steht für mehr direkte Beteiligung an öffentlichen Entscheidungen. Neben weiterreichenden Konzepten für die direkte Demokratie setzt sich die Piratenpartei auch ganz konkret für eine Förderung von Volksabstimmungen und eine Vereinfachung von Volksbegehren ein.

Um die bislang nahezu unüberwindbaren Hürden für direktdemokratische Mitbestimmung in Rheinland-Pfalz herabzusetzen fordert der Verein Mehr Demokratie e.V. die Senkung des Unterschriftenquorums und die Abschaffung des Zustimmungsquorums. Zudem wollen sie, dass Bürger über mehr Themen begehren können, beispielsweise auch über Bebauungspläne.

Wir schließen uns den Forderungen des Vereins an und setzen uns für folgende Neuregelungen ein: Die Sammelfrist soll von zwei auf sechs Monate ausgedehnt und die Anzahl der benötigten Unterschriften von ca. 10\% auf 5\% gesenkt werden. Neben dem Auslegen in Amtsräumen soll auch ein freies Sammeln gestattet sein. Wir setzen uns dafür ein, jedes zugelassene Volksbegehren grundsätzlich öffentlich im Landtag zu behandeln.

Weiterhin wollen wir bei Volksabstimmungen die Abschaffung oder zumindest die Senkung der Mindestzahl an Ja-Stimmen (Zustimmungsquoren).
 
\wahlprogramm{Förderung von Volksabstimmungen}
\antrag{Silberpappel}\zusatz{wp:demokratie:mehr}\version{03:43, 20. Jun. 2010}

\subsubsection{Ergänzung 'Mehr Bürgerbeteiligung - weniger Hürden bei Volksbegehren'}
\abstimmung
\textit{Im Abschnitt 'Mehr Bürgerbeteiligung - weniger Hürden bei Volksbegehren' soll 'Förderung von Volksabstimmungen und eine Vereinfachung von Volksbegehren ein.' so ergänzt werden:}

Förderung von Volksabstimmungen / Bürgerentscheiden und eine Vereinfachung von Volksbegehren / Bürgerbegehren ein. 

\textit{(Volksabstimmungen und Volksbegehren sind Landesebene, Bürgerentscheide und Bürgerbegehren sind Stadt- / Gemeindeebene)}

\subsubsection{Ergänzung 'Mehr Bürgerbeteiligung - weniger Hürden bei Volksbegehren'}
\abstimmung
\textit{Im Abschnitt 'Mehr Bürgerbeteiligung - weniger Hürden bei Volksbegehren" soll zwischen "Sammeln gestattet sein.' und 'Wir setzen uns' ergänzt werden:}

Die Themenbegrenzung soll auf ein Mindestmaß reduziert werden.

 
\subsection*{Strikte Gewaltenteilung}
\wahlprogramm{Strikte Gewaltenteilung}
\antrag{Piraten aus RLP}\version{03:43, 20. Jun. 2010}

\subsubsection{Strikte Gewaltenteilung}
\abstimmung
Die strikte Gewaltenteilung soll gesetzlich verankert werden. Insbesondere soll die gleichzeitige Ausübung von Amt und Mandat verboten werden.
 
\subsection*{Förderalismus stärken}
\wahlprogramm{Förderalismus stärken}
\antrag{Unbekannt}\version{03:43, 20. Jun. 2010}

\subsubsection{Förderalismus stärken}
\abstimmung
Die Piratenpartei bekennt sich zum Föderalismus und setzt sich für eine Stärkung des Föderalismus ein. Der Föderalismus gibt den Bürgern in Deutschland wesentlich mehr Einflussmöglichkeiten als in zentralistischen Systemen. Durch den Föderalismus ist es für Verbände, Bürgerinitiativen aber auch für einzelne Bürger in vielen Fällen wesentlich einfacher Politik zu beeinflussen. Nach Ansicht der Piraten sollten Entscheidungen immer auf der niedrigst möglichen Ebene getroffen werden.

\subsubsection{Für einen transparenten Förderalismus}
\abstimmung
Gleichzeitig will die Piratenpartei den Föderalismus aber klarer und transparenter machen. Es muss für die Bürger klar erkennbar sein, welche Ebene eine Entscheidung getroffen hat. Zudem setzen wir uns für eine Entflechtung der Finanzbeziehungen zwischen Bund und Ländern ein.
 
\subsection*{Kein Religionsbezug in der Landesverfassung}
\wahlprogramm{Kein Religionsbezug in der Landesverfassung}
\antrag{KV Trier/Trier-Saarburg}\version{03:43, 20. Jun. 2010}

\subsubsection{Kein Religionsbezug in der Landesverfassung}
\abstimmung
Ein weltlicher und demokratischer Staat steht für die Achtung von Menschen unabhängig ihrer religiösen Ansichten. Statt spezifischem Religionsbezug fordern wir ein Bekenntnis zu allgemein gültigen Werten, aufdenen die Gesellschaft aufbaut. Deutschland garantiert als weltlicher Staat Religionsfreiheit. Religiöse und religionsfreie Weltanschauungen sind Privatsache und die Freiheit der Wahl sowie Gleichbehandlung ist durch eine Verfassung ohne Bezüge zu einem Gott oder einer bestimmten Religion zu garantieren.
 
\subsection*{Öffentliche Petitionen nach Bundesvorbild}
\wahlprogramm{Öffentliche Petitionen nach Bundesvorbild}
\antrag{KV Trier/Trier-Saarburg}\version{03:43, 20. Jun. 2010}

\subsubsection{Öffentliche Petitionen nach Bundesvorbild}
\abstimmung
\textit{Stärkere Bürgerbeteiligung im Gesetzgebungsverfahren durch öffentliche Petitionen unter Einsatz von neuen Kommunikationsverfahren, dadurch Förderung des gesellschaftlichen Diskurses.}

Jedermann hat das Recht, sich mit Bitten und Beschwerden an die Volksvertretung zu wenden. Der Petitionsausschuss des Landtags vermittelt jedes Jahr bei über tausend Petitionen. Diese werden von Betroffenen vorwiegend gegen Behörden- und Gerichtsentscheidungen eingereicht.

Zusätzlich möchten wir den Bürgern Wege ermöglichen, an der Gesetzgebung mitzuwirken. Dazu gehören auch öffentliche Petitionen, die über ein ePetitions-Portal (nach Vorbild des Bundestages) zum gesellschaftlichen Diskurs einladen. Petitionen und Mitzeichnerunterschriften sollen online und offline gesammelt werden können.Petenten mit einer nicht unerheblichen Anzahl von Mitzeichnern sollen dabei ein Anhörungsrecht im Landtag erhalten.
 
\subsection*{Wahlalter für Landtags und Kommunalwahlen}
\wahlprogramm{Wahlalter für Landtags und Kommunalwahlen}\label{wp:demokratie:wahlalter1}
\antrag{Piraten aus RLP}\konkurrenz{wp:demokratie:wahlalter2}\version{03:43, 20. Jun. 2010}

\subsubsection{Wahlalter abschaffen - Mitbestimmungsrecht für alle}
\abstimmung
Die Piratenpartei kämpft für ein Menschenbild, indem der Mensch nicht erst ab 18 Jahren als politisch interessiert und mündig deklariert wird. Wahlreife definiert sich darüber, einen politischen Willen zu haben und diesen artikulieren zu können. Menschen können nur selbst entscheiden, wann sie ihrem politischen Willen Ausdruck verleihen können - unabhängig ihres Alters. Die Piratenpartei verlangt, dass dieses Menschenbild sich auch im Wahlsystem widerspiegelt und fordert daher die Abschaffung des Wahlalters. Wir erachten jegliche Altersgrenzen beim Wahlrecht als willkürlich. Um eine konkret spürbare Verbesserung schnell zu realisieren, soll als Übergangslösung kurzfristig das Wahlalter auf allen Ebenen auf 14 Jahre gesenkt werden.

\subsubsection{PIRATEN lehnen Familienwahlrecht ab}
\abstimmung
Die Piratenpartei lehnt ein Familienwahlrecht ab, da die Unmündigkeit der Kinder und Jugendlichen damit nicht abgeschafft, sondern noch verstärkt wird. Der von uns angestrebten Selbstbestimmung und Emanzipation steht ein Familienwahlrecht im Wege. Jeder Mensch soll selbst frei wählen und mitbestimmen können ohne Bevormundung durch Eltern oder andere Authoritäten.

\subsubsection{Positive Impulse durch Mitbestimmungsrecht für alle}
\abstimmung
Die Abschaffung des Wahlalters stellt einen immensen demokratischen und gesellschaftlichen Fortschritt dar und wird positive Veränderungen auf unsere Gesellschaft haben. Politik wird aus neuen Perspektiven gesehen werden und demokratische Entscheidungen werden sich stärker an einer politischen Nachhaltigkeit für die nachfolgenden Generationen ausrichten. Gleichsam wird das politische Interesse schon früh gefördert und demokratisches Miteinander erlernt.

\subsubsection{Politische Bildung ausbauen}
\abstimmung
Die Piratenpartei fordert begleitend zur Abschaffung des Wahlalters eine Reform der politischen Bildung. Kinder und Jugendliche müssen zusätzlich zum Politikunterricht frühestmöglich an demokratische Entscheidungsverfahren herangeführt werden und selbst mitbestimmen können. Schulen müssen in demokratische Bildungseinrichtungen verwandelt werden, in denen Schüler und Schülerinnen gleichberechtigt mit Eltern und Lehrern entscheiden. Nur so können Kinder und Jugendliche Demokratie erfahren und politisches Interesse und Gespür für politische Teilhabe entwickeln.
 
\wahlprogramm{Wahlalter für Landtags und Kommunalwahlen}\label{wp:demokratie:wahlalter2}
\antrag{Unbekannt}\konkurrenz{wp:demokratie:wahlalter1}\version{03:43, 20. Jun. 2010}

\subsubsection{Wahlalter absenken}
\abstimmung

Gerade die Themen auf Landesebene und Kommunaleben sind Themen, die Jugendliche in hohem Maße betreffen. So wird auf diesen Ebenen zum Beispiel über die Themen Bildung und den öffentlichen Nahverkehr diskutiert. Deshalb fordert die Piratenpartei eine Herabsetzung des Wahlalters für Landtags und Kommunalwahlen auf 16 Jahre, damit auch die Betroffenen selbst die Möglichkeit der demokratischen Teilhabe haben.

\subsubsection{Wahlalter absenken (Variante 2)}
\abstimmung
Gerade die Themen auf Landesebene und Kommunaleben sind Themen, die Jugendliche in hohem Maße betreffen. So wird auf diesen Ebenen zum Beispiel über die Themen Bildung und den öffentlichen Nahverkehr diskutiert. Deshalb fordert die Piratenpartei kurzfristig eine Herabsetzung des Wahlalters für Landtagswahlen auf 16 Jahre und für Kommunalwahlen auf 14 Jahre (langfristig für Landtagswahlen auf 14 Jahre und für Kommunalwahlen auf 12 Jahre), damit auch die Betroffenen selbst die Möglichkeit der demokratischen Teilhabe haben.

%%\section{Immaterialgüterrechte}

\subsection*{Urheberrecht und Nutzungsrechte}
\wahlprogramm{Urheberrecht und Nutzungsrechte}
\antrag{Unglow}\version{03:40, 20. Jun. 2010}
\subsubsection{Modul 1}
\abstimmung
Das Nutzungsrecht entfernt sich immer weiter vom Urheber und entwickelt sich hin zum Verwerterrecht. Musik- und Filmindustrie profitieren, während Nutzer kriminalisiert werden. Wir PIRATEN fordern für Privatleute ohne kommerzielle Interessen das Recht, Werke frei verwenden und kopieren zu dürfen. Der Einsatz von Maßnahmen, wie die DRM-Technologie oder ähnliche Kopierschutzmechanismen, die diese und andere rechtmäßige Nutzungen einseitig verhindern, soll untersagt werden. Abgeleitete Werke sind neue künstlerische Schöpfungen und müssen dem Kreativen grundsätzlich erlaubt sein. Dies wird durch eine Anpassung des Urheberrechts gewährleistet, für die wir uns im Bundesrat einsetzen werden.

\subsubsection{Modul 2}
\abstimmung
Der Künstler soll für jedes einzelne Werk die Lizenz frei wählen können.

\subsubsection{Modul 3}
\abstimmung
Die für eine internationale Neuausrichtung des Urheberrechts zu verhandelnden Themen müssen der öffentlichen Debatte gestellt werden und dürfen nicht einseitig durch die Lobbyinteressen der Rechteverwerter geprägt sein.

\subsubsection{Modul 4}
\abstimmung
Wir PIRATEN setzen uns für die Veröffentlichung von Lehrmaterialien unter freien Lizenzen und die bevorzugte Nutzung von freien Lehrmaterialien in der Bildung ein. Dies beinhaltet die Erstellung von Lehrmaterialien durch Lehrkräfte oder beauftragte Personen unter freien Lizenzen.

\subsubsection{Modul 5}
\abstimmung
Wir müssen zumindest folgendes am Urheberrecht ändern:
 
\wahlprogramm{Medien- oder Hardwareabgaben}
\antrag{Unglow}\version{03:40, 20. Jun. 2010}

\subsubsection{Modul 1}
\abstimmung
Eine Neubewertung der Pauschalabgaben ist nötig. Bis zu dieser Neubewertung wird im Sinne des Transparenzgebotes angestrebt, sowohl das resultierende Aufkommen nach Medien/Geräteart als auch seine Verteilung nach Empfänger öffentlich zu machen.
 
\wahlprogramm{Parlamente schreiben die Urheberrecht-Gesetze, nicht die Lobby}
\antrag{Unglow}\version{03:40, 20. Jun. 2010}

\subsubsection{Modul 1}
\abstimmung
Technische Maßnahmen, die verhindern, dass Kunden Kultur im Rahmen des Gesetzes nutzen, wie die sog. DRM-Technologie, werden wir verbieten.
 
\wahlprogramm{Neue Geschäftsmodelle fördern}
\antrag{Unglow}\version{03:40, 20. Jun. 2010}

\subsubsection{Modul 1}
\abstimmung
Für viele Künstler, Schriftsteller, Journalisten, Programmierer und andere Kulturarbeiter stellt heutzutage das Urheberrecht eine wesentliche Grundlage ihrer Geschäftsmodelle und Verdienstmöglichkeiten dar. Die Möglichkeiten der digitalen Vernetzung und Kommunikation und die in oft digitaler Form vorliegenden Werke verändern die Grundlagen für diese Geschäftsmodelle zum Teil radikal.

\subsubsection{Modul 2}
\abstimmung
Anstatt den alten Geschäftsmodellen nachzutrauern und sie mit unzumutbaren Eingriffen in die Privatsphäre der Bürger künstlich am Leben erhalten zu wollen, fordern die PIRATEN dazu auf, neue Geschäftsmodelle zu entwickeln. Diese Geschäftsmodelle sollen den Urhebern der digitalen Kulturgesellschaft ermöglichen, auf marktwirtschaftliche Art und Weise Erlöse aus der Verwertung ihrer Werke oder deren Umfeld zu erzielen, wenn sie dies anstreben.

\subsubsection{Modul 3}
\abstimmung
Überholte Vermittlerfunktionen von Rechteverwertern, die in der Vergangenheit z.B. in der Unterhaltungsmusikindustrie zu hohen Renditen geführt haben, sind größtenteils nicht mehr zeitgemäß und werden in diesem Umfang keinen Bestand haben. Die Ausschaltung von Zwischenhändlern ermöglicht es, dass den Künstlern vom Erlös ihrer Werke ein größerer Teil verbleibt und direkter zufließt. Außerdem wird damit das Spektrum der Kulturszene deutlich erweitert.

\subsubsection{Modul 4}
\abstimmung
Insbesondere die Verwendung von CreativeCommons-Lizenzen erlaubt heutzutage bereits die erfolgreiche wirtschaftliche Verwertung von Werken ohne jegliche Einschränkung bei der digitalen Privatkopie und deren Verbreitung.
 
\wahlprogramm{Keine Kulturflatrate}
\antrag{Piraten aus RLP}\version{03:40, 20. Jun. 2010}

\subsubsection{Keine Kulturflatrate!}
\abstimmung
Pauschalabgabesysteme unter staatlicher Aufsicht wie z.B. die so genannte ''Kulturflatrate'' lehnen wir ab. Wir sind davon überzeugt, dass solche Subventionen technischen Fortschritt und Innovation behindern. Es ist in unseren Augen nicht Aufgabe des Staates, bestimmte Geschäftsmodelle zu sichern oder gar zu subventionieren. Wir sehen in der freien Kopierbarkeit und Verfügbarkeit von immateriellen Kulturgütern einen Gewinn für unsere Gesellschaft.
 
\subsubsection{Patentrecht}
\wahlprogramm{Patentrecht}\label{wp:ip:patent1}
\antrag{Unglow}\konkurrenz{wp:ip:patent2}\version{03:40, 20. Jun. 2010}

\subsubsection{Modul 1}
\abstimmung
Das heutige Patentsystem erfüllt in vielerlei Hinsicht nicht mehr seinen ursprünglichen Zweck, Innovationen zu fördern. Im Gegenteil: Es erweist sich immer öfter als Innovationshemmnis und behindert den technischen und ökonomischen Fortschritt in vielen Bereichen.

\subsubsection{Modul 2}
\abstimmung
Wirtschaftlicher Erfolg ist in der Informationsgesellschaft zunehmend nicht mehr von technischen Erfindungen, sondern von Wissen und Information und deren Erschließung abhängig. Das Bestreben, diese Faktoren nun ebenso mittels des Patentsystems zu regulieren, steht unserer Forderung nach Freiheit des Wissens und Kultur der Menschheit diametral entgegen.

\subsubsection{Modul 3}
\abstimmung
Wir PIRATEN lehnen Patente auf Software und Geschäftsideen ab, weil sie die Entwicklung der Wissensgesellschaft behindern, weil sie gemeine Güter ohne Gegenleistung und ohne Not privatisieren und weil sie kein Erfindungspotential im ursprünglichen Sinne enthalten. Die gute Entwicklung klein- und mittelständischer IT-Unternehmen in Deutschland und ganz Europa hat beispielsweise gezeigt, dass auf dem Softwaresektor Patente völlig unnötig sind.

\subsubsection{Modul 4}
\abstimmung
Aus den gleichen Gründen dürfen Patente auf das Leben, inklusive der Patente auf Saatgut und Gene, nicht erteilt werden. Der Privatisierung der Biodiversität oder der Grundlage menschlichen, tierischen und pflanzlichen Lebens ist mit aller Entschiedenheit entgegenzutreten.

\subsubsection{Modul 5}
\abstimmung
Bei Saatgut und Tieren fordern wir hilfsweise kurzfristig die Formulierung eines uneingeschränkten 'Nachbaurechtes', damit die bisherige Patentierungspraxis nicht weiterhin die natürlichen Verhältnisse auf den Kopf stellt und Bauern ab sofort von solchen Klagen verschont werden. Vertragsbestimmungen, die dem widersprechen, sind für nichtig zu befinden.

\subsubsection{Modul 6}
\abstimmung
Pharmazeutische Patente erzeugen viele ethische Bedenken, nicht zuletzt in Verbindung mit Menschen aus Entwicklungsländern. Sie sind auch eine treibende Kraft für die steigenden Kosten im öffentlich finanzierten Gesundheitssystem. Wir verlangen die Initiierung einer Studie über den ökonomischen Einfluß pharmazeutischer Patente, verglichen mit andern Systemen zur Finanzierung medizinischer Forschung und Alternativen zum gegenwärtigen System.
 
\subsubsection{Patentrecht}
\wahlprogramm{Patentrecht}\label{wp:ip:patent2}
\antrag{KV Trier/Trier-Saarburg}\konkurrenz{wp:ip:patent1}\version{03:40, 20. Jun. 2010}

\subsubsection{Ablehnung von Patenten auf Pflanzen und Tiere}
\abstimmung
Naturressourcen gehören allen. Patente auf Pflanzen und Tiere blockieren die Entwicklung der Wirtschaft, die Einheit des Wissens und den allgemeinen Fortschritt der Menschheit zugunsten von Einzelinteressen und übermäßiger Ansammlung von Macht und Kapital. Wir setzen uns für die Sammlung, Pflege und Weiterentwicklung tradierter Genbestände im Einklang mit den Prinzipien fortschrittlicher Ressourcenentwickung in der Landwirtschaft ein.

\subsubsection{Stellungnahme zur Gentechnik}
\abstimmung
Wir fordern eine gentechnikfreie Landwirtschaft in Rheinland-Pfalz und das Verbot von Freisetzungsversuchen.

%%\section{Open Access}

\subsection*{Open Access - Zugang zu Wissen schaffen}
\wahlprogramm{Open Access}\label{wp:oa:zugang}
\antrag{Unglow}\version{03:36, 20. Jun. 2010}

\subsubsection{Modul 1}
\abstimmung 	Wissenschaft und Forschung sind zentrale Bausteine für ein zukunftsfähiges Deutschland und Rheinland-Pfalz. Wissenschaftliche Großprojekte und Grundlagenforschung lassen sich oft nur noch staatlich oder sogar im Verbund von mehreren Staaten durchführen.

\subsubsection{Modul 2}
\abstimmung
Mit öffentlichen Geldern geförderte Arbeit muss aber auch der Öffentlichkeit zugute kommen. Noch immer sind viele wissenschaftliche Erkenntnisse nur gegen Bezahlung erhältlich, und das, obwohl dank moderner Technik die Reproduktion der Werke praktisch kostenlos erfolgen kann. Dieses Problem ist auch vielen Wissenschaftlern bewusst, die daher zunehmend dazu übergehen, Arbeiten als Open-Access-Publikationen zu veröffentlichen und damit einen dauerhaften kostenfreien Zugang zu den Ergebnissen ihrer Forschung sicherzustellen. Diesen Trend möchten wir PIRATEN unterstützen, da wir glauben, dass ein leichterer Zugang zu Wissen zu erfolgreicherer Forschung und mehr Innovation führen wird und darüberhinaus sogar weltweit eine wohlstandsfördernde Wirkung entfaltet.

\subsubsection{Modul 3}
\abstimmung
Open Access heißt daher für uns, dass mit öffentlichen Geldern geförderte wissenschaftliche Arbeit und daraus resultierende Publikationen für jeden Menschen kostenfrei zugänglich sein müssen.

\subsubsection{Modul 4}
\abstimmung
Gleichzeitig muss eine Infrastruktur geschaffen werden, die digitale Archivierung und den dauerhaften einfachen Zugang zu Publikationen ermöglicht. Diese Aufgabe wird heute vorrangig von den etablierten Verlagen übernommen. Für Open-Access-Publikationen entwickeln sich entsprechende Mechanismen erst, oft in loser Kooperation von Bibliotheken und Universitäten. Derartige Initiativen wollen die PIRATEN auch finanziell fördern.

\subsubsection{Modul 5}
\abstimmung
In Rheinland-Pfalz soll jede Universität ein eigenes Open-Access-Repository führen in dem alle ihre Fachbereiche unterkommen. Dies vermeidet eine Zersplitterung in zu kleine Einheiten. Die Repositories sollen zwischen den Universitäten vernetzt werden, um die Durchsuchbarkeit und Verfügbarkeit von Wissen zu erhöhen. Es braucht einheitliche APIs (Zugangs- und Nutzungsschnittstellen der Software) auf der Serverseite der Repositories, um die Anschluss- und Verwendungsmöglichkeiten der Repositories zu erhöhen.

\subsubsection{Modul 6}
\abstimmung
Zur allgemeinen Förderung von Open Access sollten bei der Beurteilung von Anträgen auf Forschungsgelder nur noch Publikationen herangezogen werden, die auch öffentlich verfügbar sind.
 
\wahlprogramm{Open Access}
\antrag{KV Trier/Trier-Saarburg}\zusatz{wp:oa:zugang}\version{03:36, 20. Jun. 2010}

\subsubsection{Open Access}
\abstimmung
Die Publikationen aus staatlich finanzierter oder geförderter Forschung und Lehre werden oft in kommerziellen Verlagen publiziert, deren Qualitätssicherung von ebenfalls meist staatlich bezahlten Wissenschaftlern im Peer-Review-Prozess übernommen wird. Die Publikationen werden jedoch nicht einmal den Bibliotheken der Forschungseinrichtungen kostenlos zur Verfügung gestellt. Der Steuerzahler kommt also mehrfach für die Kosten der Publikationen auf. Wir unterstützen die Berliner Erklärung der Open-Access-Bewegung und verlangen die Zugänglichmachung des wissenschaftlichen und kulturellen Erbes der Menschheit nach dem Prinzip des Open Access. Wir sehen es als Aufgabe des Staates an, dieses Prinzip an den von ihm finanzierten und geförderten Einrichtungen durchzusetzen.
 
\paragraph{Anmerkung}: In \ref{wp:oa:zugang}.1 ''aber auch'' dann streichen.

\newpage
\subsection*{Open Access in der öffentlichen Verwaltung}
\wahlprogramm{Open Access in der Verwaltung}
\antrag{Unglow}\version{03:36, 20. Jun. 2010}

\subsubsection{Modul 1}
\abstimmung
Wir fordern die Einbeziehung von Software und anderen digitalen Gütern, die mit öffentlichen Mitteln produziert werden, in das Open-Access-Konzept. Werke, die von oder im Auftrag von staatlichen Stellen erstellt werden, sollen der Öffentlichkeit zur freien Verwendung zur Verfügung gestellt werden. Der Quelltext von Software muss dabei Teil der Veröffentlichung sein.

\subsubsection{Modul 2}
\abstimmung
Dies ist nicht nur zum direkten Nutzen der Öffentlichkeit, sondern die staatlichen Stellen können auch im Gegenzug von Verbesserungen durch die Öffentlichkeit profitieren (Open-Source-Prinzip/Freie Software). Weiterhin wird die Nachhaltigkeit der öffentlich eingesetzten IT-Infrastruktur verbessert und die Abhängigkeit von Softwareanbietern verringert.
 
\subsection*{Digitalisierung von Büchern}
\wahlprogramm{Digitalisierung von Büchern}
\antrag{KV Trier/Trier-Saarburg}\version{03:36, 20. Jun. 2010}

\subsubsection{Digitalisierung von Büchern}
\abstimmung
Wir planen die konsequente Digitalisierung der Werke, die in den Landesbibliotheken vorhanden sind und nicht mehr durch Verwertungsrechte geschützt sind. Die Werke sollen unter einer freien Lizenz veröffentlicht und im Internet der Öffentlichkeit frei zugänglich gemacht werden.
 
\newpage
\subsection*{Dauerhafte Verfügbarkeit öffentlich-rechtlicher Berichterstattung}
\wahlprogramm{Dauerhafte Verfügbarkeit öffentlich-rechtlicher Berichterstattung}\label{wp:oa:dauerhaft}
\antrag{Pirat aus RLP}\version{03:36, 20. Jun. 2010}

\subsubsection{Dauerhafte Verfügbarkeit öffentlich-rechtlicher Berichterstattung}
\abstimmung
Eine der Aufgaben des gebührenfinanzierten öffentlich-rechtlichen Rundfunks besteht in der Versorgung der Bevölkerung mit unabhängiger Berichterstattung. Die dabei erstellten Inhalte sind seit Umsetzung des 12. Rundfunkänderungsstaatsvertrags nur kurze Zeit in den Mediatheken der Rundfunkanstalten abrufbar, obwohl sie auch dauerhaft von öffentlichem Interesse sind, da sie beispielsweise als Quelle für die politische Diskussion dienen. Sie sollten deshalb zeitlich unbegrenzt zur Verfügung gestellt werden.
 
\wahlprogramm{Dauerhafte Verfügbarkeit öffentlich-rechtlicher Berichterstattung}
\antrag{Thomas Heinen}\zusatz{wp:oa:dauerhaft}\version{03:36, 20. Jun. 2010}

\subsubsection{ }
\abstimmung
Wir fordern die sofortige Überarbeitung des Staatsvertrages mit dem Ziel, die Inhalte, die durch die Bürger finanziert werden, langfristig für jeden Menschen frei verfügbar zu machen. Jeder Bürger hat einen Anspruch auf diese Inhalte. Die gesetzlichen Verweildauerregelungen müssen daher genauso wie der Drei-Stufen-Test umgehend auf den Prüfstand.
 
\subsection*{Freie Lizenzen für Inhalte der öffentlich-rechtlichen Rundfunkanstalten}
\wahlprogramm{Freie Lizenzen für Inhalte der öffentlich-rechtlichen Rundfunkanstalten}
\antrag{Pirat aus RLP}\version{03:36, 20. Jun. 2010}

\subsubsection{Freie Lizenzen für Inhalte der öffentlich-rechtlichen Rundfunkanstalten}
\abstimmung
Wenn die Allgemeinheit Fernseh- und Rundfunkprogramme bezahlt, soll sie diese auch uneingeschränkt nutzen können. Überwiegend aus deutschen Rundfunkgebühren finanzierte Inhalte sollen deshalb unter freie Lizenzen gestellt werden.

%%\section{Infrastrukturmonopole}

\subsection*{Infrastrukturmonopole}
\wahlprogramm{Infrastruktur}
\antrag{Unglow}\version{03:32, 20. Jun. 2010}

\subsubsection{Modul 1}
\abstimmung
Eine gute Infrastruktur ist eine grundlegende Voraussetzung, um Wirtschaftswachstum zu ermöglichen und Rheinland-Pfalz als Standort für Unternehmen attraktiv zu halten. Zudem wird eine zuverlässige und neutrale Infrastruktur benötigt, um freien Informationszugang und die Teilhabe am gesellschaftlichen Leben zu ermöglichen.

\subsubsection{Modul 2}
\abstimmung
Die Piratenpartei möchte verhindern, dass durch privatwirtschaftliche Interessen Infrastrukturen wettbewerbsverzerrend und auf Kosten der Gesellschaft beeinflusst werden.

\subsubsection{Modul 3}
\abstimmung
Die Infrastrukturen sind nicht nur die Basis für die Marktwirtschaft, sondern für das generelle Miteinander der Menschen. Durch dieses zentrale Element des Zusammenlebens entscheidet sich, wer aktiv an der Wirtschaft und dem kulturellen Leben teilhaben kann.

\subsubsection{Modul 4}
\abstimmung
Die Struktur und die Funktionsweise von Infrastrukturen muss transparent sein, um eine Nachvollziehbarkeit von außen zu ermöglichen. Der Zugang zu Infrastrukturen muss allen Teilen der Gesellschaft offen stehen.

\subsubsection{Modul 5}
\abstimmung
Der Staat ist für Verfügbarkeit und Zuverlässigkeit verantwortlich, um hohe Versorgungssicherheit, Effizienz und Nachhaltigkeit zu garantieren. Die Zugänge zu jeglicher Infrastruktur müssen sowohl für Produzenten und Anbieter als auch für Nutzer und Konsumenten möglichst unlimitiert und barrierefrei sein. Durch gleiche Zugangsmöglichkeiten wird der freie Wettbewerb zwischen den verschiedenen privaten Anbietern gefördert.

\subsubsection{Modul 6}
\abstimmung
Wir werden durch geeignete, öffentlich kontrollierbare und transparente Kontrollinstanzen dafür sorgen, dass die für Infrastruktur geltenden Regeln eingehalten werden. In Fällen, in denen diese Kontrollinstanzen versagen und Abhilfe auch nicht durch Auflagen, Verordnungen und Gesetze mit einem verhältnismäßigen und endlichen Aufwand erreicht werden kann, werden wir diese Infrastruktur verstaatlichen.
 
\wahlprogramm{Verkehrs- und Stromnetze}\label{wp:monopol:netz1}
\antrag{Unglow}\konkurrenz{wp:monopol:netz2}\version{03:32, 20. Jun. 2010}

\subsubsection{Modul 1}
\abstimmung
Straßen-, Schienen- und Stromnetze sowie Wasserwege gelten als natürliche Infrastrukturmonopole. Der Zugang zu diesen Teilen der Infrastruktur ist für unsere Gesellschaft überlebenswichtig. Gleichzeitig sind sie extrem anfällig für Wettbewerbsverzerrung. Nur wenn der Staat, als einzig öffentlich kontrollierbare Instanz, der Betreiber solcher Netze ist, kann sichergestellt werden, dass die von uns geforderten Ansprüche erfüllt werden.
 
\wahlprogramm{Verkehrs- und Stromnetze}\label{wp:monopol:netz2}
\antrag{marcus}\konkurrenz{wp:monopol:netz1}\version{03:32, 20. Jun. 2010}

\subsubsection{Alternativantrag}
\abstimmung
Straßen-, Schienen- und Stromnetze sowie Wasserwege gelten als natürliche Infrastrukturmonopole. Der Zugang zu diesen Teilen der Infrastruktur ist für unsere Gesellschaft überlebenswichtig. Gleichzeitig sind sie extrem anfällig für Wettbewerbsverzerrung. Nur wenn der Staat, als einzig öffentlich konrollierbare Instanz der \textbf{Besitzer} (= Eigentümer) solcher Netze ist, kann sichergestellt werden, dass die von uns geforderten Ansprüche erfüllt werden. Da er als Besitzer jederzeit geltende Pachtverträge bei Nichteinhaltung durch den Betreiber widerrufen kann. Darüber hinaus profitiert der Staat bei Wertsteigerungen der Infrastruktur durch die Möglichkeit der Pachtpreiserhöhung (Modell Public private Partnership) bei Begrenzung des Unternehmerischen Risikos.
 
\subsection*{Infrastruktur Internet}
\wahlprogramm{Infrastruktur Internet}\label{wp:monopol:inet1}
\antrag{Unglow}\konkurrenz{wp_monopol:inet2}\version{03:32, 20. Jun. 2010}

\subsubsection{Modul 1}
\abstimmung
Im Informationszeitalter ist das Internet als Infrastruktur von besonderer Bedeutung. Es ist Grundlage für den freien Meinungsaustausch, die Teilhabe am kulturellen und sozialen Leben, für Wissenschaft und politische Partizipation. Aufgrund dieser Relevanz muss die Verfügbarkeit des Netzes im Rahmen einer unpfändbaren Grundversorgung wie bei Radio und TV gewährleistet werden. Der gleichberechtigte Zugang jedes einzelnen Bürger muss besonders geschützt werden. Das Netz muss sich neutral gegenüber den transportierten Inhalten verhalten. Die Netzbetreiber tragen keine Verantwortung für die übertragenen Daten.

\subsubsection{Modul 2}
\abstimmung
Die Installation von Filtern in die Infrastruktur des Internets lehnen wir ab. Der Kampf gegen rechtswidrige Angebote im Internet muss jederzeit mit rechtsstaatlichen Mitteln geführt werden. Allein die Etablierung einer Zensurinfrastruktur ist bereits inakzeptabel. Die Beurteilung der Rechtswidrigkeit muss gemäß der in Deutschland geltenden Gewaltenteilung und Zuständigkeit getroffen werden.

\subsubsection{Modul 3}
\abstimmung
Der Ausschluss von Bürgern aus dem Internet ist nach Ansicht der Piratenpartei eine eklatante Bürgerrechtsverletzung. Eine Three-Strikes-Regelung nach französischem Vorbild oder ähnliche Maßnahmen lehnen wir deshalb strikt ab.

\subsubsection{Modul 4}
\abstimmung
Volks- und betriebswirtschaftlich sind Regionen mit schneller Internetanbindung stark aufgewertet. Daher werden wir den Ausbau schneller Internetverbindungen fördern und erleichtern.

\wahlprogramm{Infrastruktur Internet}\label{wp:monopol:inet2}
\antrag{KV Trier/Trier-Saarburg}\konkurrenz{wp_monopol:inet1}\version{03:32, 20. Jun. 2010}

\subsubsection{Breitbandausbau - Einleitung}
\abstimmung
Regionen ohne Breitbandtechnologie sind nicht nur wirtschaftlich benachteiligt und haben einen Standortnachteil, sie drohen auch von der kulturellen, politischen und technischen Entwicklung abgehängt zu werden.

\subsubsection{Breitbandausbau - Verfügbarkeit}
\abstimmung
Breitband-Internetverbindungen sollen wie Strom, Straßen, Telefon und andere Infrastruktur flächendeckend verfügbar sein.

\subsubsection{Neue Definition von Breitband}
\abstimmung
Die zur Zeit vom Bundeswirtschaftsministerium genannte untere Grenze der Breitbandgeschwindigkeit von 128 KBit/s ist dabei nicht ausreichend. Die Definition von Breitbandgeschwindigkeit soll in Zukunft der aktuellen technischen Entwicklung angepasst werden.

\subsubsection{Breitbandausbau - vorausschauender Ausbau}
\abstimmung
Beim Bau und der Sanierung von Straßen müssen vorausschauend Leerrohre gelegt werden, um einen kostengünstigen Breitbandausbau zu ermöglichen.
\subsubsection{Breitbandausbau - Ausbauförderung}
\abstimmung
Wir wollen unterversorgte Gebiete finanziell fördern, um den Ausbau voranzutreiben. Das Land soll einen Beauftragten einsetzen, dessen Aufgabe es ist, in den Kommunen gezielt über die Fördermittel zu informieren und für den Breitbandausbau zu werben.
%%\section{Für ein selbstbestimmtes Leben}

\wahlprogramm{Präambel}
\antrag{Thomas Heinen}\version{03:31, 20. Jun. 2010}

\subsubsection{Grundlagen}
\abstimmung
Jeder Mensch hat das Grundrecht auf freien Zugang zu Information und Bildung. Dies ist in einer freiheitlich-demokratischen Gesellschaft essentiell, um jedem Menschen, unabhängig von seiner sozialen Herkunft, ein größtmögliches Maß an Selbstbestimmung zu ermöglichen. Eine freiheitliche Gesellschaft lebt von der Teilhabe ihrer Bürger. Voraussetzung dafür ist die selbständige Entscheidung über die eigene Lebensgestaltung und über die Art der Teilhabe an der gesellschaftlichen Entwicklung. Die Grundlagen dafür sind Bildung und der Zugang zu Kultur.
 
\wahlprogramm{Selbstbestimmung}
\antrag{Unglow}\version{03:31, 20. Jun. 2010}

\subsubsection{Modul 1}
\abstimmung
Immer mehr Bereiche des täglichen Lebens werden vom Staat durch Vorschriften und Gesetze reguliert und reglementiert. Dadurch werden die Bürger vom Staat bevormundet und daran gehindert, ihren individuellen Lebensstil zu führen und sich frei zu entfalten. Die Piratenpartei setzt sich für die Reforum und ggf. Abschaffung von Gesetzen ein, die den Bürger unverhältnismäßig bevormunden.

\subsubsection{Modul 2}
\abstimmung
In den letzten Jahren werden zunehmend Gesetze diskutiert und teilweise auch verabschiedet, welche die Bürger bevormunden und sie in ihren Gewohnheiten oder ihrer Freizeitgestaltung einschränken. Die Freiräume der Bürger zur individuellen Gestaltung und Entfaltung ihres Lebens werden immer stärker beeinträchtigt. Die Piratenpartei stellt sich ungerechtfertigten Bevormundungen der Bürger durch den Staat entgegen.
 
\wahlprogramm{Für Paintball, Computer- und Videospiele}
\antrag{Unglow}\version{03:31, 20. Jun. 2010}

\subsubsection{Modul 1}
\abstimmung
Es wird diskutiert Paintball und so genannte ''Killerspiele'' zu verbieten, um Amokläufe zu verhindern. Dabei ist weder definiert, was genau unter ''Killerspielen'' zu verstehen ist, noch gibt es gesicherte Erkenntnisse, dass diese Spiele Amokläufe verursachen oder fördern. Computer- und Videospiele sind ein wesentlicher Teil der Jugendkultur. Sie sind künstlerisches und kulturelles Gut, vergleichbar mit Filmen oder Büchern. Die Piraten halten es für falsch, Spieler zu kriminalisieren, statt die eigentlichen gesellschaftlichen Probleme zu lösen. Wir setzen uns deshalb dafür ein, dass Computer- und Videospiele als Kulturgüter anerkannt werden und wenden uns gegen Verbote dieser Form von Kultur.

\subsubsection{Modul 2}
\abstimmung
Paintball ist ein Mannschaftssport und eine legitime Freizeitbeschäftigung Erwachsener. Es gibt keine Anzeichen, dass dieser Sport die Gewahlbereitschaft erhöht oder sogar Amokläufe verursacht. Die Forderungen nach einem Verbot von Paintball betrachtet die Piratenpartei als populistisch und stellt sich ihnen entgegen.
 
\wahlprogramm{Computerspiele}
\antrag{KV Trier/Trier-Saarburg}\version{03:31, 20. Jun. 2010}


\subsubsection{Gegen Stigmatisierung von eSport und Action-Computerspielen als ''Killerspiele''} 
\abstimmung
Die Bezeichnung ''Killerspieler'' diskreditiert in völlig inakzeptabler Weise Spieler, eSportler und sogar Jugendliche, die gerne ihrem Hobby nachgehen. Wir lehnen das von der Innenministerkonferenz geforderte generalisierende Verbot zur Herstellung und Verbreitung von Computerspielen strikt ab. Populistische Verbotsforderungen lösen vor allem dort keine Probleme, wo sie völlig andere Ursachen, als die vordergründig unterstellten, haben. Spieler sind keine Mörder und Gewalttäter. Nicht Verbote, sondern präventive Maßnahmen und die Stärkung medienkompetenten Handelns stärken den verantwortungsvollen Umgang mit elektronischen Medien.
 
\wahlprogramm{Computerspiele}
\antrag{KV Trier/Trier-Saarburg}\version{03:31, 20. Jun. 2010}
\subsubsection{eSport-Vereine anerkennen}
\abstimmung
Immer mehr Spieler organisieren sich in eSport-Vereinen, um gemeinsam ihrem Hobby nachzugehen. Ebenso wie traditionelle Sportvereine leisten sie dabei einen wichtigen Beitrag zum gesellschaftlichen Zusammenhalt. Viele bemühen sich beispielsweise um die Vermittlung von Medienkompetenz gegenüber Jugendlichen und Eltern. Wir möchten diese Arbeit honorieren und dafür sorgen, dass eSport-Vereine genauso wie andere Sportvereine als gemeinnützig anerkannt werden.

\wahlprogramm{Poker}
\antrag{Unglow}\version{03:31, 20. Jun. 2010}

\subsubsection{Modul 1}
\abstimmung
Das Pokerspielen hat in den letzten Jahren an Popularität gewonnen. In Rheinland-Pfalz hat dies leider dazu geführt, dass ein weitreichendes Pokerverbot geschaffen wurde. Öffentliche Pokerveranstaltungen und Internet-Poker wurden weitgehend verboten. Das Pokerverbot in Rheinland-Pfalz halten wir für ungerechtfertigt und werden uns deshalb für die Abschaffung dieses Gesetzes einsetzen.
 
\subsection*{Mehr Freiheit für Raucher und ein sinnvoller Nichtraucherschutz}
\wahlprogramm{Mehr Freiheit für Raucher und ein sinnvoller Nichtraucherschutz}\label{wp:selbst:raucher1}
\antrag{Unglow}\version{03:31, 20. Jun. 2010}
\begin{itemize}
\item \konkurrenz{wp:selbst:raucher2}
\item \konkurrenz{wp:selbst:raucher3}
\item \konkurrenz{wp:selbst:raucher4}
\end{itemize}

\subsubsection{Modul 1}
\abstimmung
Ein großer Teil der Deutschen raucht. Da Rauchen schädlich ist und Krebs auslösen kann, wurde versucht die Raucher durch übergroße Warnhinweise und höhere Tabaksteuern zwangszubelehren, ohne großen Erfolg. Da Raucher durch das Passivrauchen auch die Gesundheit von Nichtrauchern gefährden, wurden von den Bundesländern Nichtraucherschutzgesetze verabschiedet. An Arbeitsplätzen und in öffentlichen Gebäuden, bei denen keine Entscheidungsmöglichkeit besteht, sich dem Rauch zu entziehen, sind solche Maßnahmen nachvollziehbar und sinnvoll.

\subsubsection{Modul 2}
\abstimmung
Jedoch wurde auch das Rauchen in Gaststätten größtenteils verboten. In Nebenräumen von Gaststätten darf noch geraucht werden. Ob in Einraumgaststätten geraucht werden darf hängt in Rheinland-Pfalz von der Größe der Gaststätte ab, oder davon in welchem Umfang in der Gaststätte Speisen serviert werden. Das Risiko an Krebs zu erkranken sinkt jedoch nicht dadurch, dass man nur eine Kleinigkeit isst, statt einer richtigen Mahlzeit. Auch kleine Räume senken dass Risiko für Nichtraucher nicht. In der Regel dürfte die Luft in kleinen Gaststätten sogar schlechter sein, als in großen Gaststätten, womit das Risiko für Nichtraucher ansteigt.

\subsubsection{Modul 3}
\abstimmung
Die Regelungen zum Nichtraucherschutz in Rheinland-Pfalz sind für uns nicht nachvollziehbar. Die Freiheit der Bürger und der Gaststättenbetreiber auf nicht nachvollziehbare Weise beschnitten. Gleichzeitig findet aber kein effektiver Schutz der Angestellten im Gastronomiegewerbe statt. Wir setzen uns für eine nachvollziehbare und effektive Regelung zum Nichtraucherschutz ein, in der überall in Gaststätten geraucht werden darf, außer in Bereichen in denen Angestellte arbeiten.
 
\wahlprogramm{Mehr Freiheit für Raucher und ein sinnvoller Nichtraucherschutz}\label{wp:selbst:raucher2}
\antrag{Unglow}\version{03:31, 20. Jun. 2010}
\begin{itemize}
\item \konkurrenz{wp:selbst:raucher1}
\item \konkurrenz{wp:selbst:raucher3}
\item \konkurrenz{wp:selbst:raucher4}
\end{itemize}

\subsubsection{Modul 1}
\abstimmung
Wir werden uns dafür einsetzen, dass bei bestehenden Gastronomiebetrieben der Betreiber selbst entscheiden kann ob er eine Raucher- oder Nichtraucher- Lokalität betreiben möchte, soweit eine Trennung in einen abgeschlossenen Raucher- und Nichtraucherbereich nicht möglich ist. Hierbei ist gleichgültig ob Speisen angeboten werden oder nicht. Gleichzeitig muss bei neu erteilten Betriebserlaubnissen eine klare Regelung eine Trennung in einen separaten abgeschlossenen Bereich für beide Gruppen beinhalten. Ansonsten kann keine Betriebserlaubnis erteilt werden.

\subsubsection{Modul 2}
\abstimmung
Der Betreiber muss im Rahmen des Mitarbeiterschutzes sicherstellen, dass kein nichtrauchender Mitarbeiter in Bereichen in denen geraucht werden darf bedienen muss. Im Umkehrschluss bedeutet dies dass ein Mitarbeiter der Raucher ist zwar im Nichtraucherbereich bedienen kann, dies aber sicherlich den nichtrauchenden Gast stört, wenn die Bedienung nach Nikotin riecht.

\subsubsection{Modul 3}
\abstimmung
Für kleine Gastronomiebetriebe ohne Speiseangebot entscheidet der Betreiber ob er eine Raucher- oder Nichtraucherkneipe betreiben möchte. Genauso wie der Nichtraucher sich entscheiden kann dort zu verweilen oder auch das Lokal zu meiden.
 
\wahlprogramm{Mehr Freiheit für Raucher und ein sinnvoller Nichtraucherschutz}\label{wp:selbst:raucher3}
\antrag{MatthiasK}\version{03:31, 20. Jun. 2010}
\begin{itemize}
\item \konkurrenz{wp:selbst:raucher1}
\item \konkurrenz{wp:selbst:raucher2}
\item \konkurrenz{wp:selbst:raucher4}
\end{itemize}

\subsubsection{Modul 1}
\abstimmung
Die Regelungen zum Nichtraucherschutz in Rheinland-Pfalz sind für uns nicht nachvollziehbar, die Freiheit der Bürger und der Gaststättenbetreiber auf nicht nachvollziehbare Weise beschnitten. Gleichzeitig findet aber kein effektiver Schutz der Angestellten im Gastronomiegewerbe statt. Die staatliche Bevormundung von Bürgern und Gastronomiebetrieben muss ein Ende haben.

\subsubsection{Modul 2}
\abstimmung
Gastronomiebetreiber müssen selbst entscheiden können, ob sie eine Raucher- oder Nichtraucher- Lokalität betreiben möchten. Hierbei darf ein etwaiges Speisenangebot keine Rolle spielen.

\subsubsection{Modul 3}
\abstimmung
Ferner fordern wir eine einheitliche Kenntlichmachung an der Außenseite aller Betriebe, die auf die jeweiligen Verhältnisse hinweist. So ist jedem Bürger die Freiheit gegeben ein Lokal ohne Raucher- bzw. Nichtraucherbereich zu meiden.

\subsubsection{Modul 4}
\abstimmung
Der Betreiber muss im Rahmen des Mitarbeiterschutzes sicherstellen, dass kein nichtrauchender Mitarbeiter in Raucherbereichen bedienen muss.
 
\wahlprogramm{Mehr Freiheit für Raucher und ein sinnvoller Nichtraucherschutz}\label{wp:selbst:raucher4}
\antrag{Silberpappel}\version{03:31, 20. Jun. 2010}
\begin{itemize}
\item \konkurrenz{wp:selbst:raucher1}
\item \konkurrenz{wp:selbst:raucher2}
\item \konkurrenz{wp:selbst:raucher3}
\end{itemize}

\subsubsection{Einleitung}
\abstimmung
Die Regelungen zum Nichtraucherschutz in Rheinland-Pfalz sind für uns nicht nachvollziehbar. Die Freiheit der Bürger und der Gaststättenbetreiber wird auf nicht nachvollziehbare Weise beschnitten. Gleichzeitig findet aber kein effektiver Schutz der Angestellten im Gastronomiegewerbe statt.

\subsubsection{Rein inhaberbetriebene Lokalitäten}
\abstimmung
In Betrieben, in denen nur der / die Inhaber arbeiten, sollen selbst entscheiden können, ob sie eine Raucher- oder Nichtraucher- Lokalität betreiben möchten. Hierbei darf ein etwaiges Speisenangebot keine Rolle spielen.

\subsubsection{Angestellte}
\abstimmung
In Betrieben, in denen auch Angestellte arbeiten, gilt Rauchverbot.

\subsubsection{Raucherbereich}
\abstimmung
Betriebe, in denen Rauchverbot gilt, können einen abgeschlossenen Raucherbereich einrichten, in dem die Angestellten dann aber nicht arbeiten.

\subsubsection{Kennzeichnung}
\abstimmung
Ferner fordern wir eine einheitliche Kennzeichnung an der Außenseite aller Betriebe, die auf die jeweiligen Verhältnisse hinweist. So ist jedem Bürger die Freiheit gegeben, ein Lokal ohne Raucher- bzw. Nichtraucherbereich zu meiden.
 
\subsection*{Waffenkontrollen}
\wahlprogramm{Waffenkontrollen einschränken}
\antrag{KV Trier/Trier-Saarburg}\version{03:31, 20. Jun. 2010}

\subsubsection{ }
\abstimmung
Verdachtsunabhängige, unangekündigte Waffenkontrollen in privaten Wohnräumen verletzen das Grundrecht der Unverletzlichkeit der Wohnung. Deswegen möchten wir diese einschränken.
 
\wahlprogramm{Zuständigkeit für Waffenkontrolle}
\antrag{KV Trier/Trier-Saarburg}\version{03:31, 20. Jun. 2010}

\subsubsection{ }
\abstimmung
Soweit Kontrollen in Wohnungen unumgänglich sind, sollten diese nur von Polizisten durchgeführt werden, da diese im Gegensatz zu anderen Personen über die hierfür nötige Ausbildung und Routine im Umgang mit Schusswaffen verfügen.
 
\newpage
\wahlprogramm{Keine Hausdurchsuchungen zur Waffenkontrolle}
\antrag{Piraten aus RLP}\version{03:31, 20. Jun. 2010}

\subsubsection{Keine Hausdurchsuchungen zur Waffenkontrolle}
\abstimmung
Gegen den Willen von Bürgerinnen und Bürgern dürfen ohne richterlichen Durchsuchungsbeschluss keine Kontrollen von Wohnungen oder sonstigen Liegenschaften erfolgen.
 
\subsection*{Rechtliche Gleichstellung}
\wahlprogramm{Rechtliche Gleichstellung}
\antrag{KV Trier/Trier-Saarburg}\version{03:31, 20. Jun. 2010}

\subsubsection{ }
\abstimmung
Wir werden uns dafür einsetzen, dass das Land Rheinland-Pfalz sich im Bundesrat dafür stark macht, die rechtliche Gleichstellung aller Menschen unabhängig von ihrer sexuellen Identität voranzutreiben.

\subsubsection{ }
\abstimmung
Insbesondere sind Familien für uns all jene Lebenskonstellationen, in denen Verantwortung für Kinder und Eltern übernommen wird. Dabei ist die Anzahl der Verantwortlichen, deren Beziehung zueinander und deren Geschlecht unerheblich.

%%\section{Gesundheit}

\wahlprogramm{Elektronische Gesundheitskarte}\label{wp:gesundheit:egk1}
\antrag{Niemand13}\konkurrenz{wp:gesundheit:egk1}\version{03:29, 20. Jun. 2010}

\subsubsection{Elektronische Gesundheitskarte stoppen!}
\abstimmung
Gerade im Gesundheitswesen ist der Schutz der Privatsphäre von besonderer Wichtigkeit. Deshalb fordert die Piratenpartei einen effektiven Schutz der Patientendaten und wirksame Kontrollmechanismen. Mit der geplanten Elektronischen Gesundheitskarte laufen wir Gefahr, ein System einzuführen in dem umfangreich und unkontrolliert Patientendaten zentral gespeichert werden. Patientinnen und Patientinnen können ihr Recht auf informationelle Selbstbestimmung nicht mehr wahrnehmen. Die vielen Datenskandale der letzten Zeit zeigen eindrücklich die Risiken solcher Systeme. Gerade die Gesundheitsdaten der Bürgerinnen und Bürger werden große Begehrlichkeiten wecken. Die Piratenpartei fordert deshalb den Stopp der elektronischen Gesundheitskarte.
 

\wahlprogramm{Elektronische Gesundheitskarte}\label{wp:gesundheit:egk2}
\antrag{Unbekannt}\konkurrenz{wp:gesundheit:egk1}\version{03:29, 20. Jun. 2010}

\subsubsection{Elektronische Gesundheitskarte umgestalten}
\abstimmung
Gerade im Gesundheitswesen ist der Schutz der Privatsphäre von besondere Wichtigkeit. Gleichzeitig muss aber auch gerade das Gesundheitssystem für den Patienten möglichst transparent sein, damit er selbst die für seine Gesundheit besten Entscheidungen treffen kann. Dazu muss der Patienten auch mit möglichst vielen Rechten ausgestattet sein. Deshalb fordert die Piratenpartei sowohl einen effektiven Schutz der Patientendaten als auch transparentes Gesundheitssystem und ausreichend Recht für die Patienten. Mit der geplanten Elektronischen Gesundheitskarte laufen wir Gefahr, ein System einzuführen in dem umfangreich Patientendaten gespeichert werden. Die vielen Datenskandale der letzten Zeit zeigen die Risiken eines solchen System. Gerade die Gesundheitsdaten der Bürgerinnen und Bürger werden große Begehrlichkeiten Wecken. Die elektronische Gesundheitskarte bietet aber viele Vorteile, um die Effizienz des Gesundheitssystems zu steigern und die Versorgung der Patienten zu verbessern. Während der Patient durch die Gesundheitskarte transparenter werden würde ist die Arbeit der Ärzte relativ intransparent. Die Piratenpartei fordert deshalb, dass auf der elektronischen Gesundheitskarte nicht mehr Daten gespeichert werden sollen als auf der bisherigen Krankenkarte. Die Karte soll aber die Möglichkeit bieten, mehr Daten zu erfassen. Jeder Patient soll, wenn er will, selbst die Speicherung zusätzlicher auf der Gesundheitskarte zulassen können.
 
\subsubsection{Bewertung von Ärzten}
\antrag{Unbekannt}\version{03:29, 20. Jun. 2010}

\subsubsection{Bewertung von Ärzten}
\abstimmung
Aufgrund des Schutzes des Patienten ist die Erhebung von Daten zur Kontrolle der Leistung der Ärzte nur schwer möglich. Aber gerade solche Informationen währen auch für die Patienten wichtig. Es muss also auch ein Weg gefunden um dem Patienten möglichst viel Informationen zur Verfügung zu stellen. Um eine besser Kontrolle über die Leistung der Ärzte zu haben und um den Patienten mehr Informationen zur Verfügung zu stellen fordern wird, dass auch jeder Kassenpatient immer eine Rechnung bekommt auf der alle Leistungen die der Arzt erbracht hat aufgelistet sind. Zu jeder Rechnung ist dem Patient auch ein anonymer Fragebogen auszuhändigen, mit dem er bestätigen kann, dass er alle auf der Rechnung aufgeführten Leistungen auch wirklich erhalten hat. Zudem soll der Patient auf dem Fragebogen auch eine Bewertung über seine Zufriedenheit mit der Behandlung und der Beratung des Arztes abgeben können. Die Abgabe der Fragebögen soll möglichst einfach gestaltet werden. Die Daten aus solchen Erhebungen sollen allen Bürgern frei zugänglich gemacht werden.
 
\wahlprogramm{Pflege}
\antrag{Silberpappel}\version{03:29, 20. Jun. 2010}

\subsubsection{Menschenwürde}
\abstimmung
Kostendruck und Gewinnstreben haben in vielen Pflegeeinrichtungen dazu geführt, dass die Pflegebedürftigen unter Umständen leben müssen, die ihre Menschenwürde verletzen.

Wir wollen dafür sorgen, dass ruhigstellende Medikamente nur verabreicht werden, wenn dies dem Wohl des Pflegebedürftigen dient, oder zum Schutz der Pflegenden absolut notwendig ist.

Auch das Fesseln ans Bett (''Fixierung'') soll nur zulässig sein, wenn es zum Schutz des Pflegebedürftigen oder der Pflegenden unumgänglich ist.

\subsubsection{Wege}
\abstimmung
Um dies zu erreichen, setzen wir uns für eine ausreichende Personalausstattung in der Pflege ein, für effektivere Kontrollen und dafür, dass dabei nicht nur Zahlen geprüft, sondern auch Bewohner der Pflegeeinrichtung befragt werden.

\subsubsection{Demokratie}
\abstimmung
Angehörigenbeiräte sehen wir als weiteres sinnvolles Mittel, Qualität und Menschlichkeit in der Pflege zu fördern.

\subsubsection{Nachsatz}
\abstimmung
Die Würde des Menschen ist das höchste Gut in unseren Grundgesetz und muss auch in der Pflege oberstes Gebot sein.

\subsubsection{Transparenz}
\abstimmung
Der Medizinische Dienst der Krankenkassen (MDK) prüft Pflegeeinrichtungen und erstellt die sogenannten ''Einrichtungsbezogenen Pflegeberichte''. Diese dürfen nach derzeitiger Gesetzeslage nicht veröffentlicht werden.

\subsubsection{Transparenz herstellen Alternative I}
\abstimmung
Wir wollen dagegen eine Pflicht zur Veröffentlichung einführen.

\subsubsection{Transparenz herstellen Alternative II}
\abstimmung
Wir wollen die Veröffentlichung erlauben.

\subsubsection{Auswirkungen Transparenz}
\abstimmung
Durch solche Informationen können sich die Verbaucher ein Bild von der Qualität einzelner Pflegeeinrichtungen machen. So entsteht Druck auf die Pflegeeinrichtungen, Missstände zu beseitigen und Qualität zu erhöhen.
 
\newpage
\wahlprogramm{Gesundheit und Freiheit}
\antrag{Unbekannt}\version{03:29, 20. Jun. 2010}

\subsubsection{ }
\abstimmung
Die Gesundheit ist unser höchstes Gut.

Bewusst wird dies vielen Menschen erst dann, wenn die Abwesenheit der Gesundheit, also eine Erkrankung ins Spiel kommt. Eine Erkrankung kann uns mehr oder minder, zeitweise oder auch für immer in unseren geliebten Freiheiten einschränken. Die Erkrankung kann uns in eine Pflegebedürftigkeit bringen, bei der wir womöglich für den Rest unseres Lebens, vielleicht auch nur für eine kürzere Zeitspanne, auf die Hilfe anderer angewiesen sind. Erst in dieser Zeit, der Abwesenheit der Gesundheit merkt der Betroffene die Wertschätzung dieses Gutes "Gesundheit".

Die Gesundheit und die Freiheit eines jeden Menschen ist ein hohes zu schützende Gut. Erhalten Sie sich Ihre Freiheit und Ihre Gesundheit.

Zur Erhaltung unserer Gesundheit bedarf es auch einer großen Eigenverantwortung und Selbstdisziplin. Angefangen bei einer gesunden Ernährung, regelmäßigem aber nicht übertriebenem Sport, dem vermeiden von Gesundheitsrisiken durch z.B. den Konsumverzicht von tolerierten Drogen wie Nikotin und Alkohol oder nicht tolerierten Drogen wie LSD, Marihuana, Kokain und vielen anderen natürlichen und chemisch hergestellten Drogen.

Wie wichtig ist Ihnen Ihre Gesundheit oder auch die Gesundheit Ihrer Kinder? Wie frei sind Sie in Ihrer Entscheidung sich gesund zu ernähren? Wie viel Eigenverantwortung tragen Sie selbst zur Erhaltung Ihrer Gesundheit? Wie viel ist Ihnen /uns unsere Gesundheit Wert? Wie kann oder sollte nach Ihrer Meinung die Politik Einfluss nehmen auf eine gesunde Lebensführung ohne dabei die Freiheit des einzelnen einzuschränken? Wie weit muss die Freiheit einer Marktwirtschaft gesetzlich eingeschränkt werden um die Gesundheit der Menschen nicht zu gefährden, um langfristig das Überleben der Menschheit zu gewährleisten?

Viele Fragen, die weitere Fragen hervorbringen. Wer Fragen stellt ist oft unbequem. Das kann mehrere Ursachen haben? 1. Der Befragte kann nicht antworten weil er keine Antwort auf die Frage hat, sich mit dem Thema noch nicht gedanklich auseinandergesetzt hat. 2. Der Befragte ist an einer ehrlichen Beantwortung nicht interessiert weil er den Fragenden und auch andere in Unkenntnis lassen möchte, um Risiken oder Nebenwirkungen zu vertuschen um Kapital aus einem vermeintlich sicheren oder gesunden Produkt zu schlagen.

Fragen können scharf wie ein Messer, - für den Befragten, manchmal gefährlich sein. Fragen können neue Fragen aufbringen oder Antworten liefern die zumindest kurzfristig akzeptabel sind, bis die Antwort durch neue Erkenntnisse erneut in Frage gestellt werden muss.

Ehrliche Fragen und ehrliche Antworten führen zu einer TRANSPARENZ zu einer Klarheit und Weitsichtigkeit. Wahrheit bringt Klarheit.

Neben der Eigenverantwortung zur Erhaltung der Gesundheit besteht jedoch auch eine gesellschaftliche Verpflichtung für eine humane soziale Grundordnung durch die Wirtschaft. Die Menschen dürfen nicht die Sklaven der Wirtschaft sein sondern die Wirtschaft muss zum Wohle der Menschen da sein. Die Wirtschaft soll das Leben der gesamten Menschheit erleichtern.

Um bei Fragen der gesunden Ernährung zu bleiben wäre zu klären: Wie gesund sind die auf dem Markt angebotenen Lebensmittel? Wie viel Hormone wie viel Antibiotika wird in der Tierzucht eingesetzt? Wie viele Pestizide und Insektizide belasten unser Gemüse tatsächlich?

Andere Fragen im Gesundheitsbereich sind: Was ist mit der elektronischen Gesundheitskarte (EGK)? Wer will sie einführen, verdient damit Geld? Ist die EGK wirklich ein Vorteil für die Kranken? Stecken Versicherungen mit in der Entwicklung? Wie soll es mit den Krankenkassen und den Beiträgen weitergehen?
Ist die Kopfpauschale ein Vorteil oder entlastet Sie nur die Vermögenden?
 
\newpage
\wahlprogramm{Für Aufklärung – gegen Diskriminierung}\label{wp:gesundheit:auf1}
\antrag{Thomas Heinen}\konkurrenz{wp:gesundheit:auf2}\version{03:29, 20. Jun. 2010}

\subsubsection{Für Aufklärung – gegen Diskriminierung}
\abstimmung
Menschen mit Krankheiten, z.B. HIV/AIDS, sind Menschen wie du und ich. Es gibt keinen Grund sie zu diskriminieren. Die Piratenpartei Rheinland-Pfalz will deswegen dabei mithelfen, der Ausgrenzung der Betroffenen in der Gesellschaft ein Ende zu setzen. Dazu werden wir uns dafür einsetzen, dass Projekte in Rheinland-Pfalz gestartet und gefördert werden, welche einen positiven Beitrag zur Aufklärung und gegen Diskriminierung leisten.
 
\wahlprogramm{Für Aufklärung – gegen Diskriminierung}\label{wp:gesundheit:auf2}
\antrag{Piraten aus RLP}\konkurrenz{wp:gesundheit:auf1}\version{03:29, 20. Jun. 2010}

\subsubsection{Für Aufklärung – gegen Diskriminierung}
\abstimmung
Menschen mit Krankheiten sind Menschen wie du und ich. Es gibt keinen Grund sie zu diskriminieren. Die Piratenpartei Rheinland-Pfalz will deswegen dabei mithelfen, der Ausgrenzung der Betroffenen in der Gesellschaft ein Ende zu setzen. Dazu werden wir uns dafür einsetzen, dass Projekte in Rheinland-Pfalz gestartet und gefördert werden, welche einen positiven Beitrag zur Aufklärung und gegen Diskriminierung leisten.
%%\section{Wirtschaft}

\wahlprogramm{Präambel}
\antrag{KV Trier/Trier-Saarburg}\version{03:27, 20. Jun. 2010}

\subsubsection{Präambel Wirtschaft Teil 1}
\abstimmung
Die Piratenpartei Rheinland-Pfalz steht für eine nachhaltige und soziale Wirtschaftspolitik.

\subsubsection{Präambel Wirtschaft Teil 2}
\abstimmung
Wir setzen uns für fairen Wettbewerb, für die Förderung von Innovationen sowie gegen privatwirtschaftliche Monopole und übermäßige staatliche Regulierung der Unternehmen ein.
 
\wahlprogramm{Privatisierung öffentlicher Einrichtungen}\label{wp:wirt:privat1}
\antrag{Acamir}\version{03:27, 20. Jun. 2010}
\begin{itemize}
\item \konkurrenz{wp:wirt:privat2}
\item \konkurrenz{wp:wirt:privat3}
\end{itemize}

\subsubsection{Privatisierung öffentlicher Einrichtungen}
\abstimmung
Die Privatisierung von öffentlichen Einrichtungen, die für die Grundversorgung der Bevölkerung notwendig sind (z.B. ÖPNV, Müllabfuhr, Wasserversorgung, Krankenhäuser), ist zu stoppen. Der Verkauf solcher Einrichtungen an Investoren bei anschließendem Zurückmieten ( Sale and Lease Back Verträge) ist zu verbieten. Öffentliche Einrichtungen, die nicht für die Grundversorgung der Bevölkerung notwendig sind (z.B. Schwimmbäder oder andere Freizeiteinrichtungen), sollten nur privatisiert werden dürfen, wenn dadurch keine Monopolstellungen entstehen.
 
\wahlprogramm{Privatisierung öffentlicher Einrichtungen}\label{wp:wirt:privat2}
\antrag{Unbekannt}\version{03:27, 20. Jun. 2010}
\begin{itemize}
\item \konkurrenz{wp:wirt:privat1}
\item \konkurrenz{wp:wirt:privat3}
\end{itemize}

\subsubsection{Privatisierungen öffentlicher Einrichtungen}
\abstimmung
Über die Privatisierung öffentlicher Einrichtungen der kommunalen Grundversorgung sollen die Bürger vor Ort und damit die Kunden der Einrichtung direkt entscheiden dürfen.

\paragraph{Begründung}: Privatisierungen sind nicht grundsätzlich gut oder schlecht, egal was Parteien mit neo-liberaler oder sozialistischer Ideologie weismachen wollen. Privatisierungen sind meist schlecht durchgeführt, da die betreffenden Unternehmen jeweils eine kleine Gruppe von Politikern überzeugen müssen, was viel Raum für Korruption lässt. Andererseits sind auch viele Einrichtungen in staatlichem Besitz schlecht geführt und bieten nur suboptimalen Service bzw. überhöhte Preise.

Die Anbieter öffentlicher Dienstleistungen sollen direkt um ihre Kunden, also die Bürger des betreffenden Gebietes werben, nicht um die Gunst Einzelner in hohen Ämtern. Eine Vertragsklausel zur Rückführung in die öffentliche Hand bei ungenügenden Leistungen, wieder festgestellt durch den Bürger, ist wirksamer, als einem privaten oder staatlichen Dienstleister ein dauerhaftes Monopol zu geben.
 
\wahlprogramm{Privatisierung öffentlicher Einrichtungen}\label{wp:wirt:privat3}
\antrag{Marcus}\version{03:27, 20. Jun. 2010}
\begin{itemize}
\item \konkurrenz{wp:wirt:privat1}
\item \konkurrenz{wp:wirt:privat2}
\end{itemize}

\subsubsection{Privatisierung öffentlicher Einrichtungen}
\abstimmung
Sale and Lease Back Verträge (bei Immobilien+Sachanlagen) sind generell zu verbieten, weil 1. diese Finanzierungsform über die Gesamtlaufzeit grundsätzlich teurer ist, wie eine direkte Kreditaufnahme: Es muß der Unternehmergewinn über das Leasing mitbezahlt werden, dazu fallen Transaktionskosten für die Kommune an. 2. Sofern nicht ein Restkaufpreis gemäß dem steuerlichen Restwert (nach Abschreibungen) vereinbart wird, fallen Wertsteigerungen dem Privatinvestor zu. 3. Die Kommune verliert die Handlungshoheit über das Objekt und ist meistens langfristig gebunden.

\subsubsection{ }
\abstimmung
Cross Border Leasing ist grundsätzlich zu verbieten, weil 1.Ethisch verwerflich: es wird auf Steuervergünstigungen des Investors in den USA spekuliert. 2. CBL-Verträge werden grundsätzlich nach US Recht New York City abgeschlossen, dadurch völlige Intransparenz(hochkomplizierte Verträge, die nur von spezialisierten Rechtsanwälten geprüft werden können.3. Dadurch fallen Transaktionskosten von bis zu 10\% an, die von der Kommune zu tragen sind, ebenso wie das Risiko, dass die Steuervorteile von einem US-Finanzamt und Finanzgericht nicht anerkannt werden. Letzteres ist häufig der Fall.

\subsubsection{ }
\abstimmung
Die Privatisierung, also Auslagerung und Verkauf von öffentlichen Einrichtungen, die für die Grundversorgung der Bürger notwendig sind (Wasserwerke, städtische Energieversorger Gas,Strom, städtische Schienennetze,Stromleitungen, Krankenhäuser, Schulen, Verwaltungsgebäude, sonstige Anlagen, sind zu stoppen, da 1.die Kommune die Planungshoheit und Verfügungsgewalt verliert,2.langfristig finanzielle Nachteile zu erwarten sind.

\subsubsection{ }
\abstimmung
Die Autragsvergabe für die Übernahme der Aufgaben der Müllabfuhr, des ÖPNV, für den Betrieb von Krankenhäusern und anderen Dienstleistungen ist gemäß EU-Richtlinie unter den Bedingungen (Auschreibung) zu gestatten, daß keine Monopole/Oligopole entstehen, bzw. diese durch zeitliche Befristungen (im Falle Müllabfuhr) begrenzt werden.
 
\wahlprogramm{Wirtschaftsförderung überprüfen}\label{wp:wirt:wirt1}
\antrag{Acamir}\konkurrenz{wp:wirt:wirt2}\version{03:27, 20. Jun. 2010}

\subsubsection{Subventionen}
Subventionen sollen nur gewährt werden, wenn sie eine positive Wirkung für die Allgemeinheit entfalten, die durch andere Maßnahmen nicht zu erreichen wäre. Die Wirkung einer Subvention ist durch eine regelmäßige Erfolgskontrolle zu überprüfen. Subventionen für Unternehmen sollen nur befristet als Anschubfinanzierung gewährt werden. Aus Gründen der Transparenz sind direkte Subventionen indirekten (wie z.B. Steuererleichterungen) vorzuziehen.
 
\wahlprogramm{Wirtschaftsförderung überprüfen}\label{wp:wirt:wirt2}
\antrag{KV Trier/Trier-Saarburg}\konkurrenz{wp:wirt:wirt1}\version{03:27, 20. Jun. 2010}

\subsubsection{Wirtschaftsförderung überprüfen}
\abstimmung
Ausgaben, im besonderen Maße Subventionen, werden auf den Prüfstand gestellt. Subventionen sollen nur dort eingesetzt werden, wo wichtige wirtschafts- und forschungspolitische Ziele anders nicht erreicht werden können. Darüber hinaus müssen alle Subventionen degressiv oder zeitlich befristet sein, so dass man nach einem festen Zeitraum den Sinn dieser Subvention wieder prüfen muss.
 
\newpage
\wahlprogramm{Zwangsmitgliedschaft in Kammern und Verbänden abschaffen}\label{wp:wirt:zwang1}
\antrag{Salorta}\konkurrenz{wp:wirt:zwang2}\version{03:27, 20. Jun. 2010}

\subsubsection{Kammerzwang}
\abstimmung
Die Zwangsmitgliedschaft in Kammern und Verbänden in Deutschland wie in der Industrie- und Handelskammer (IHK) oder den Handwerkskammern ist ein Beispiel für unnötige Bürokratie. Viele Unternehmer und Selbständige haben kein Interesse an deren Leistungen und kennen diese oft nicht einmal. Trotzdem ist jeder Gewerbetreibende und jeder Gründer einer Firma ab dem ersten Tag zur Beitragszahlung an die IHK verpflichtet. Zwar kostet die Zwangsmitgliedschaft in der IHK nicht viel, dieser Beitrag ist jedoch nach Ansicht vieler Unternehmer der sinnloseste Beitrag für die Verwaltung. Diese Zwangsregelung trifft besonders kleine Gewerbetreibende oder Handwerker hart, die keine Leistungen in Anspruch nehmen. Selbst inaktive Firmen oder Betriebe, die sich in Auflösung befinden, sind zu dieser Abgabe verpflichtet. Für Selbständige kommt erschwerend hinzu, dass deren private Einkünfte an die IHK beziehungsweise die Handwerkskammer übermittelt werden, da sich nach deren Höhe die Abgabenhöhe an die Kammern bemisst. Dies stellt nach Auffassung der Piraten eine eklatante Verletzung der Privatsphäre von Selbständigen dar. Die vielfach praktizierte Zwangsmitgliedschaft in Kammern und Verbänden in Deutschland schränkt Unternehmer und Betriebe in ihrer Freiheit ein und bieten nicht durchgängig für den Zwangsbeitrag äquivalente Leistungen.

Wir treten daher für die Abschaffung der Zwangsmitgliedschaft mit Zwangsbeiträgen in Kammern und Verbänden ein, um diese durch eine freiwillige Beitrittsmöglichkeit zu ersetzen. Damit wird auch die Übermittlung der privaten Einkünfte von Selbständigen an die IHK beziehungsweise die Handwerkskammern beendet.
 
\wahlprogramm{Zwangsmitgliedschaft in Kammern und Verbänden abschaffen}\label{wp:wirt:zwang2}
\antrag{KV Trier/Trier-Saarburg}\konkurrenz{wp:wirt:zwang1}\version{03:27, 20. Jun. 2010}

\subsubsection{Kammerzwang}
\abstimmung
Wir planen, die Zwangsmitgliedschaft mit Zwangsbeiträgen in Kammern und Verbänden abzuschaffen und durch eine freiwillige Beitrittsmöglichkeit zu ersetzen. Hierzu wollen wir eine Bundesratsinitiative anregen.
 
\wahlprogramm{Begrenzung der Leiharbeit}
\antrag{KV Trier/Trier-Saarburg}\version{03:27, 20. Jun. 2010}

\subsubsection{Begrenzung der Leiharbeit}
\abstimmung
Nach französischem Vorbild sollen Leiharbeiter nicht eine billige Verfügungsmasse sein, mit der reguläre Beschäftigte unter Druck gesetzt werden können, sondern für die ihnen abverlangte Flexibilität mit einem Lohnzuschlag entschädigt werden. Wir wollen, dass das Land Rheinland-Pfalz dazu eine entsprechende Initiative im Bundesrat startet.
 
\wahlprogramm{Missbrauch von Praktika verhindern}
\antrag{KV Trier/Trier-Saarburg}\version{03:27, 20. Jun. 2010}

\subsubsection{Missbrauch von Praktika verhindern}
\abstimmung
Arbeitgeber, die Praktikanten als billige Arbeitskräfte ausbeuten, verhalten sich nicht nur unfair gegenüber den Praktikanten sondern auch gegenüber ihren Mitbewerbern und den sozialen Sicherungssystemen.

Darum wollen wir die Regelungen für Praktika verschärfen.

Probezeit, Werkstudententätigkeit und befristete Arbeitsverträge sind ausreichende Werkzeuge des Arbeitsmarkts, um Berufsanfängern den Start in das Berufsleben zu erleichtern.

\subsubsection{Praktika nur während Ausbildung}
\abstimmung
Ein Praktikumsvertrag soll nur im Rahmen von Schule, Studium oder Berufsausbildung geschlossen werden können.

\subsubsection{Praktika nur während Ausbildung}
\abstimmung
Praktikumsstellen müssen öffentlich ausgeschrieben werden, verpflichtend ist dabei die Angabe einer Mindestvergütung.

%%\section{Sozialpolitik}

\wahlprogramm{Präambel}
\antrag{KV Trier/Trier-Saarburg}\version{03:11, 20. Jun. 2010}

\subsubsection{Präambel}
\abstimmung
Freiheit hat auch mit dem Recht jedes Menschen zu tun, ein möglichst selbstbestimmtes Leben bis ins hohe Alter zu führen. Dazu ist man aber oft auch auf die Solidarität anderer angewiesen. Deshalb wollen wir, dass auch künftig Gesunde für die Kranken, Arbeitende für Arbeitslose, Jung für Alt und Alt für Jung eintreten. So kann eine gerechte Gesellschaft bestehen, die Freiheit für jeden verheißt.
 
\wahlprogramm{Sozialpolitik im Bundesrat}
\antrag{KV Trier/Trier-Saarburg}\version{03:11, 20. Jun. 2010}
\subsubsection{Sozialpolitik im Bundesrat}
\abstimmung
Wir wollen, dass sich das Land auch bei seiner Mitwirkung an der sozial- und gesundheitspolitischen Gesetzgebung im Bundesrat am Ideal einer gerechten Gesellschaft orientiert.
 
\wahlprogramm{Sozialpolitik als Grundrecht und Grundpflicht}
\antrag{Limbo}\version{03:11, 20. Jun. 2010}

\subsubsection{Sozialpolitik als Grundrecht und Grundpflicht}
\abstimmung
Die Piratenpartei sollte sehen, dass im Zuge des Demografischenwandels jeder Mensch zu sozialen Handeln herangeführt und erzogen werden sollte. Dies kann sich in Pflichtpraktika während der Schulzeit aber auch in Zwangsleistunge durch Arbeitslose ausdrücken. Jeder Mensch in Rheinland-Pfalz soll lernen gerne und mit Freude anderen zu Helfen. So wird die Piratenpartei das Land ein Stück gerechter machen. Wer Anreize (finanzielle oder materielle) schafft, wird auch den Menschen überzeugen können, seine Freizeit für mehr "Ehrenamt" zu nutzen.
 
\newpage
\wahlprogramm{Handlungsfreiheit und Würde von finanzschwachen Bürgern sicherstellen}
\antrag{Thomas Heinen}\version{03:11, 20. Jun. 2010}

\subsubsection{Handlungsfreiheit und Würde von finanzschwachen Bürgern sicherstellen}
\abstimmung
Gerade in der aktuellen Situation, in der Regierende die Bürger- und Menschenrechte nach und nach zu erodieren versuchen, brauchen wir eine wachsame und politisch aktive Zivilgesellschaft. In einem modernen Sozialstaat muss die Möglichkeit der Teilnahme am politischen und kulturellen Leben für alle Menschen sichergestellt werden. Diese Freiheit darf nicht durch staatliche Kürzungen, die eine mangelnde soziale Sicherung oder gar Existenzängste nach sich ziehen, eingeschränkt werden.

Aus finanzieller Notlage und Zukunftsängsten heraus kann keine Freiheit für politisches Handeln erwachsen. Das Schaffen von Zwangslagen führt bei den Betroffenen zu einer Radikalisierung der politischen Forderungen. Dies gefährdet die Demokratie in unserer Gesellschaft.

Daher wird sich die Piratenpartei Rheinland-Pfalz dafür einsetzen, dass die Handlungsfreiheit auch und gerade von finanzschwachen Bürgern sichergestellt und deren Würde nicht als Folge von bestimmten Kürzungen oder Änderungen im Sozialbereich verletzt wird.
%%\section{Kultur}

\subsection*{Freiräume für Kultur}
Kultur und Kulturgüter schaffen Unterhaltung, Wissen und Identität. Durch Kultur wird auch eine individuelle Entfaltung der Menschen ermöglicht. Darum sollte sich die Kultur möglichst frei entwickeln können, sowie möglichst vielen Menschen zugänglich sein.

Die Piratenpartei möchte deshalb in Rheinland-Pfalz Freiräume für Kultur schaffen. Wir wollen eine Infrastruktur zur Verfügung stellen, die von möglichst vielen Menschen zur Entfaltung ihrer Kreativität genutzt werden kann. Die gezielte Förderung einzelner Kunstprojekte lehnen wir allerdings ab, da dies eine staatliche Beeinflussung der Entwicklung von Kunst und Kultur darstellen würde.

\subsection*{Kostenloser Zugang zu Kulturgütern}
Der Zugang zu Kultur und Kulturgütern soll möglichst vielen Menschen ermöglicht werden. Dazu gehört, dass die Eintrittspreise zu staatlichen Kulturgütern möglichst niedrig sein sollen. Die Piratenpartei will auch den Zugang zu möglichst vielen Kulturgütern kostenlos gestalten.

\subsection*{Rundfunk}
\textbf{(ACHTUNG: vgl. auch Anträge unter OpenAccess!)}

\wahlprogramm{Rundfunk}\label{wp:kultur:rundfunk1}
\antrag{KV Trier/Trier-Saarburg}\version{03:24, 20. Jun. 2010}

\subsubsection{Unabhängigkeit der öffentlichen Rundfunkanstalten}
\abstimmung
Artikel 5 des Grundgesetzes schützt die Pressefreiheit. Derzeit sind die öffentlich-rechtlichen Medienanstalten aber alles andere als unpolitisch. Etwa die Hälfte der Mitglieder des Verwaltungsrates und ein Großteil der Mitglieder des Rundfunkrates des SWR werden direkt von den Landesregierungen von Baden-Württemberg und Rheinland-Pfalz bestimmt. Auch im ZDF sind knapp die Hälfte der Verwaltungs- und Fernsehräte von der Politik bestimmt.

\subsubsection{Rundfunkgebühreneinzug reformieren (GEZ abschaffen)}
\abstimmung
Wir setzen uns im Land und über den Bundesrat dafür ein, die GEZ mittelfristig überflüssig zu machen und abzuschaffen. Der öffentlich-rechtliche Rundfunk soll stattdessen über eine Abgabe für alle steuerpflichtigen Privatpersonen und Unternehmen finanziert werden, die von den Finanzämtern eingezogen wird. Auch hat sich der Umgang der GEZ mit persönlichen Daten als problematisch erwiesen. Deshalb soll sie, solange sie noch besteht, durch die Datenschutzbeauftragten des Landes und Bundes überwacht werden.
 
\newpage
\wahlprogramm{Rundfunk}
\antrag{Silberpappel}\zusatz{wp:kultur:rundfunk1}\version{03:24, 20. Jun. 2010}

\subsubsection{Unabhängigkeit der öffentlichen Rundfunkanstalten}
\abstimmung
Wir wollen die öffentlich-rechtlichen Medienanstalten unabhängiger von der Politik machen.

Soll an das Ende des Moduls ''Unabhängigkeit der öffentlichen Rundfunkanstalten'' gehängt werden.

 
\wahlprogramm{Rundfunk}\label{wp:kultur:rundfunk4}
\antrag{KV Trier/Trier-Saarburg}\version{03:24, 20. Jun. 2010}

\subsubsection{Dauerhafte Verfügbarkeit von durch öffentlich-rechtliche Rundfunkanstalten erstellten Inhalten}
\abstimmung
Vom gebührenfinanzierten öffentlich-rechtlichen Rundfunk erstellte Inhalte sind seit Umsetzung des 12. Rundfunkänderungsstaatsvertrags nur kurze Zeit in den Mediatheken der Rundfunkanstalten abrufbar, obwohl sie auch dauerhaft von öffentlichem Interesse sind, da sie beispielsweise als Quelle für die politische Diskussion dienen. Sie sollten deshalb zeitlich unbegrenzt zur Verfügung gestellt werden. Dafür werden wir uns einsetzen.
 
\wahlprogramm{Rundfunk}
\antrag{Thomas Heinen}\zusatz{wp:kultur:rundfunk4}\version{03:24, 20. Jun. 2010}

\subsubsection{ }
\abstimmung
Wir fordern die sofortige Überarbeitung des Staatsvertrages mit dem Ziel, die Inhalte, die durch die Bürger finanziert werden, langfristig für jeden Menschen frei verfügbar zu machen. Jeder Bürger hat einen Anspruch auf diese Inhalte. Die gesetzlichen Verweildauerregelungen müssen daher genauso wie der Drei-Stufen-Test umgehend auf den Prüfstand.
 
\newpage
\wahlprogramm{Rundfunk}
\antrag{KV Trier/Trier-Saarburg}\version{03:24, 20. Jun. 2010}

\subsubsection{Freie Lizenzen für Inhalte der öffentlich-rechtlichen Rundfunkanstalten}
\abstimmung
Wenn die Allgemeinheit Fernseh- und Rundfunkprogramme bezahlt, soll sie diese auch uneingeschränkt nutzen können. Aus deutschen Rundfunkgebühren finanzierte Inhalte sollen deshalb unter freie Lizenzen gestellt werden.
 
\wahlprogramm{Rundfunk}
\antrag{Pirat aus RLP}\version{03:24, 20. Jun. 2010}

\subsubsection{Öffentlich-rechtliche Programme werbefrei gestalten} 	\abstimmung
Um die Neutralität der öffentlich-rechtlichen Programme sicherzustellen, müssen diese komplett werbefrei sein. Die bisherige Regelung führt dazu, dass gerade vor 20 Uhr versucht wird quotenträchtige Formate einzusetzen, um Werbeeinnahmen zu generieren. Dies ist nicht im Sinne des öffentlich-rechtlichen Informationsauftrags. Durch ein komplett werbefreies Angebot kann zudem die Akzeptanz einer Abgabe für die öffentlich-rechtlichen Programme erhöht werden.

\subsubsection{Finanzierung (Vorschlag 1)}
\abstimmung
Die Ausfälle der Werbeeinnahmen sollen durch strukturelle Einsparungen erreicht werden. Dabei ist sicherzustellen, dass diese nicht auf Kosten der Informationsqualität erreicht werden.

\subsubsection{Finanzierung (Vorschlag 2)}
\abstimmung
Die Ausfälle der Werbeeinnahmen sollen durch eine geringfügige Erhöhung der Gebühren finanziert werden.

\subsubsection{Finanzierung (Vorschlag 3)}
\abstimmung
Die Ausfälle der Werbeeinnahmen sollen sowohl durch strukturelle Einsparungen bei den öffentlich-rechtlichen Sendern selbst, als auch durch eine geringfügige Erhöhung der Gebühren finanziert werden. Die Programm- und Informationqualität muss trotz Einsparungen jederzeit sicher gestellt sein.
 
\subsection*{Reform des Landesgesetzes über den Schutz der Sonn- und Feiertage}
\wahlprogramm{Abschaffung des Tanzverbots}
\antrag{Silberpappel}\version{03:24, 20. Jun. 2010}

\subsubsection{Abschaffung des Tanzverbots}
Das Tanzverbot in Rheinland-Pfalz wird durch das „Landesgesetz über den Schutz der Sonn- und Feiertage“ geregelt. Wir wollen die Paragrafen 6, 7 und 8 streichen (Verbot von Versammlungen und Veranstaltungen, Verbot von Sportveranstaltungen, Verbot von Tanzveranstaltungen).

\subsubsection{ }
\abstimmung
Der Staat soll hier nicht in die Freiheit des Einzelnen eingreifen.

\subsubsection{ }
\abstimmung
Wir setzen uns dafür ein, das Tanzverbot aufzuheben.
 
\wahlprogramm{Teilnahme am kulturellen Leben für alle}
\antrag{KV Trier/Trier-Saarburg}\version{03:24, 20. Jun. 2010}

\subsubsection{Teilnahme am kulturellen Leben für alle}
\abstimmung
Wir wollen, dass alle Menschen am kulturellen Leben teilhaben können. Bei der Förderung kultureller Einrichtungen soll darauf geachtet werden, dass diese auf Barrierearmut achten und Angebote für sozial schwache Besucher bieten, zum Beispiel deutlich reduzierte Eintrittspreise.
 
\wahlprogramm{Öffentlicher Raum für alle}
\antrag{Silberpappel}\version{03:24, 20. Jun. 2010}

\subsubsection{Öffentlicher Raum für alle}
\abstimmung
Die Nutzungsmöglichkeiten des öffentlichen Raums für alle müssen verbessert werden. Die Innenstädte gehören auch spielenden Kindern und skatenden Jugendlichen. Wir möchten den Gebrauch öffentlicher Gebäude durch Bürgervereinigungen, Vereine und Kulturgruppen fördern und setzen uns für entsprechende Verbesserungen in Nutzungs- und Haftungsregelungen ein.

%%\section{Landesfinanzen}

\subsection*{Landesfinanzen}
\wahlprogramm{Landesfinanzen}\label{wp:finanzen:land1}
\antrag{unbekannt}\konkurrenz{wp:finanzen:land2}\version{03:22, 20. Jun. 2010}
\subsubsection{Modul 1}
\abstimmung
Die Piratenpartei strebt einen ausgeglichenen Landeshaushalt an. Um dies zu erreichen wollen wir in erster Linie in dem Bereich der Kulturförderung und der Straßeninfrastruktur sparen.

\subsubsection{Modul 2}
\abstimmung
Zudem wollen wir auch im Bereich der Subventionen Geld einsparen.

\subsubsection{Modul 3}
\abstimmung
Auch durch Datensparsamkeit und den damit verbunden Bürokratieabbau wollen wir Geld einsparen.

\subsubsection{Modul 4}
Für die Bildungspolitik wollen wir hingegen mehr Geld ausgeben.
 
\wahlprogramm{Landesfinanzen}\label{wp:finanzen:land2}
\antrag{Acamir}\konkurrenz{wp:finanzen:land1}\version{03:22, 20. Jun. 2010}

\subsubsection{Ausgaben}
Die Piratenpartei strebt einen ausgeglichenen Landeshaushalt an. Durch den Abbau von Subventionen und den Verzicht auf teure Prestigeprojekte wie den Nürburgring-Freizeitpark wollen wir Geld einsparen. Für die Bildungspolitik wollen wir hingegen mehr Geld ausgeben.

\subsubsection{Einnahmen}
Wir wollen dass sich Rheinland-Pfalz im Bundesrat für eine Erhöhung der Erbschaftssteuer einsetzt um zusätzliche Einnahmen zu generieren.
 
\wahlprogramm{Landesfinanzen}
\antrag{Salorta}\version{03:22, 20. Jun. 2010}

\subsubsection{Modul 1: Ausgeglichener Haushalt}
\abstimmung
Die Piratenpartei strebt einen ausgeglichenen Landeshaushalt an.

\subsubsection{Modul 2: keine schuldenfinanzierten Ausgaben}
Die auf Dauer unverantwortliche Finanzierung von Landesausgaben über Schulden muss gestoppt werden.

\subsubsection{Modul 3: Datensparsamkeit/Bürokratieabbau}
\abstimmung
Durch Datensparsamkeit und den damit verbundenen Bürokratieabbau kann ein Beitrag zu den nötigen Einsparungen geleistet werden.

\subsubsection{Modul 4: Subventionsabbau/keine Prestigeprojekte}
\abstimmung
Große Sparpotentiale sehen wir im Abbau von Subventionen und dem Verzicht auf teure Prestigeprojekte wie den Nürburgring-Freizeitpark.

\subsubsection{Modul 5a: Neubau von Straßen 1} 
\abstimmung
Beim Neubau von Straßen sehen wir ebenfalls Möglichkeiten zur Kürzung von Ausgaben.

\subsubsection{Modul 5b: Neubau von Straßen 2}
\abstimmung
Beim Neubau von Straßen sehen wir Möglichkeiten zur Kürzung von Ausgaben.

\subsubsection{Modul 6a: Kulturförderung 1 (Ergänzung 5a)}
\abstimmung
Beim Neubau von Straßen und der Kulturförderung sehen wir ebenfalls Möglichkeiten zur Kürzung von Ausgaben.

\subsubsection{Modul 6b: Kulturförderung 2 (Ergänzung 5b)}
\abstimmung
Beim Neubau von Straßen und der Kulturförderung sehen wir Möglichkeiten zur Kürzung von Ausgaben.

\subsubsection{Modul 6c: Kulturförderung 3}
\abstimmung
Bei der Kulturförderung sehen wir ebenfalls Möglichkeiten zur Kürzung von Ausgaben

\subsubsection{Modul 6d: Kulturförderung 4}
\abstimmung
Bei der Kulturförderung sehen wir Möglichkeiten zur Kürzung von Ausgaben

\subsubsection{Modul 7: mehr Geld für Bildung}
\abstimmung
Bedarf für Mehrausgaben erkennen wir dagegen im Bereich der Bildungspolitik.

\subsubsection{Modul 8: Transparenz}
\abstimmung
Durch mehr Transparenz bei der staatlichen Auftragsvergabe bietet sich die Chance, Mauscheleien zu lasten der Steuerzahler zu verhindern.

\subsubsection{Modul 9: Ausgabenkontrolle durch direkte Demokratie}
\abstimmung
Mit Hilfe direktdemokratischer Elemente wie Bürgerbegehren und Volksentscheiden wollen wir eine größere Ausgabenkontrolle durch die Bürger erreichen.
 
\paragraph{Ergänzende Erklärung}: Von den Varianten der Module 5 und 6 sollte jeweils nur eine abgestimmt werden, abhängig davon welche Module vorher bereits angenommen wurden.

\subsection*{Einnahmen}
\wahlprogramm{Einnahmenseite}\label{wp:finanzen:einnahme1}
\antrag{Silberpappel}\konkurrenz{wp:finanzen:einnahme2}\version{03:22, 20. Jun. 2010}

\subsubsection{Erbschaftssteuer}
\abstimmung
Wir wollen, dass sich Rheinland-Pfalz im Bundesrat für eine Erhöhung der Erbschaftssteuer einsetzt, um zusätzliche Einnahmen zu generieren.

\subsubsection{Vermögenssteuer}
\abstimmung
Wir wollen, dass sich Rheinland-Pfalz im Bundesrat für die Wiedereinführung einer Vermögenssteuer einsetzt, um zusätzliche Einnahmen zu generieren.

\subsubsection{Erbschaftssteuer + Vermögenssteuer}
\abstimmung
Wir wollen, dass sich Rheinland-Pfalz im Bundesrat für eine Erhöhung der Erbschaftssteuer und die Wiedereinführung einer Vermögenssteuer einsetzt, um zusätzliche Einnahmen zu generieren.

\subsubsection{Familienunternehmen}
\abstimmung
Bei der Gestaltung ist darauf zu achten, dass Familienunternehmen, die Arbeitsplätze schaffen und erhalten, nicht zusätzlich belastet werden.
 
\wahlprogramm{Einnahmenseite}\label{wp:finanzen:einnahme2}
\antrag{marcus}\konkurrenz{wp:finanzen:einnahme1}\version{03:22, 20. Jun. 2010}

\subsubsection{Vermögenssteuer}
\abstimmung
Wir wollen, dass sich Rheinland-Pfalz im Bundesrat für die Wiedereinführung einer Vermögenssteuer unter Beachtung der v.BGH bedungenen Gleichbehandlung von immobilem und mobilem Vermögen, einsetzt. Begründung: 1996 brachte die letztmalig erhobene Vermögensteuer umgerechnet ca 4,6 Mrd Euro Einnahmen ein. Die Vermögenssteuer ist auf alle Vermögen über Euro 500.000 zu erheben, mit Ausnahme von Land- + Forstwirtschaft, da dort eine schleichende Enteignung drohen könnte.

\subsubsection{Erbschaftssteuer}
\abstimmung
Die Erbschaftssteuer ist auf alle Erbfälle, auch Betriebe auszudehnen. Bei betrieblichen Erbschaften ist den Erben, sofern der Betrieb weitergeführt wird, eine Ratenzahlung der Erbschaftssteuer auf begründeten Antrag (um die Existenz des Betriebes nicht zu gefährden) bis zu 20 Jahren zu ermöglichen. Ausnahmen: Land- und Forstwirtschaft.

\subsubsection{Börsenumsatzsteuer}
\abstimmung
Das Land Rheinland-Pfalz möge sich im Bundesrat für eine Wiedereinführung der Börsenumsatzsteuer v. 0,25\% auf Börsenumsätze einsetzen. Begründung: Diese Steuer bringt mehrere 100 Millionen Euro Einnahmen im Jahr, und wird in gleicher Höhe wieder vom größten Finanzplatz Europas, London erhoben.
 
\subsection*{Vereinfachung des Steuersystems}
\wahlprogramm{Vereinfachung des Steuersystems}
\antrag{Silberpappel}\version{03:22, 20. Jun. 2010}

\subsubsection{Erbschaftssteuer}
\abstimmung
Die Piratenpartei Rheinland-Pfalz setzt sich für eine deutliche Vereinfachung des Steuersystems ein. Nur ein einfaches, transparentes Steuersystem kann für jeden Bürger verständlich und damit gerecht sein.

\subsubsection{Ausnahmen verringern}
\abstimmung
Ausnahmen im Steuersystem müssen deutlich reduziert werden.

\subsubsection{Steuersparmodelle}
\abstimmung
Paragraf 15b des Einkommensteuergesetzes verbietet "Steuerstundungsmodelle" nach einem vorgefertigten Konzept. Davon sind hauptsächlich standardisierte Finanzprodukte für Kleinanleger betroffen, nicht aber maßgeschneiderte Steuersparmodelle für außergewöhnlich vermögende Bürger. Wir wollen uns über den Bundesrat dafür einsetzen, das Verbot auch auf maßgeschneiderte Steuersparmodelle zu erweitern.

\subsubsection{Umleiten von Gewinnen}
\abstimmung
Wir wollen erreichen, dass Tricks zur Steuerersparnis, wie das Umleiten von Unternehmensgewinnen in Steueroasen, verboten oder durch geeignete Maßnahmen uninteressant gemacht werden.
 
\textbf{Infos auf der Diskussionsseite}

\subsection*{Verbesserte Steuerprüfung}
\wahlprogramm{Verbesserte Steuerprüfung}\label{wp:finanzen:steuerpruefung}
\antrag{Silberpappel}\version{03:22, 20. Jun. 2010}

\subsubsection{Einleitung}
\abstimmung
Den öffentlichen Haushalten gehen durch Steuerbetrug Milliarden an Einnahmen verloren. Neben dem Personalmangel bei der Bekämpfung von Steuerhinterziehung sind beispielsweise Betriebsprüfer zu sehr kurzen Prüfzeiten bei den Betrieben angehalten, mit der Folge, dass Steuerhinterziehung häufig nicht aufgedeckt und somit geahndet werden kann.

\subsubsection{mehr Steuerprüfer einstellen}
\abstimmung
Jeder Steuerprüfer bringt ein Vielfaches an Einnahmen, verglichen mit dem, was er kostet. Deshalb wollen wir die Zahl der Steuerprüfer in Rheinland-Pfalz erhöhen.

\subsubsection{Steuerprüfer für Steuergerechtigkeit}
\abstimmung
Dies dient auch der Steuergerechtigkeit.

\subsubsection{Prüfzeiten in Großbetrieben}
\abstimmung
Die Prüfzeiten sollen in Großbetrieben ausgeweitet werden, um eine ausreichende Prüfung zu gewährleisten.

\subsubsection{Umsatzsteuerprüfungen}
\abstimmung
Die Umsatzsteuerprüfungen sollen durch Bereitstellung von Steuerprüfern des Landes gestärkt werden. Bereits existierende Zusagen und Vereinbarungen mit dem Bund sollen konsequent umgesetzt werden.

\subsubsection{Unabhängigkeit der Steuerprüfer}
\abstimmung
Wir setzen uns dafür ein, dass Steuerprüfer wirklich unabhängig arbeiten können.
 
\wahlprogramm{Verbesserte Steuerprüfung}
\antrag{marcus}\zusatz{wp:finanzen:steuerpruefung}\version{03:22, 20. Jun. 2010}

\subsubsection{Prüfungszeiträume für Einkommensmillionäre}
\abstimmung
Wir fordern, dass Einkommensmillionäre (Einkommen > 500.000 Euro) regelmäßig und vollständig geprüft werden. Durchschnittlich muss jeder geprüfte Einkommensmillionär Euro 135.000 Steuer nachzahlen (Quelle \href{http://www.welt.de/wirtschaft/article128891/Milliarden_Verlust_durch_fehlende_Steuerpruefung.html}{[1]}) Das gibt bei ca 15.000 Einkommensmillionären ca Euro 2 Mrd Mehreinnahmen für den Zeitraum v. 3 Jahren = ca 733 Millionen pro Jahr.
 
\subsection*{Staatsleistungen an Kirchen beenden}
\wahlprogramm{Staatsleistungen an Kirchen beenden}
\antrag{KV Trier/Trier-Saarburg}\version{03:22, 20. Jun. 2010}

\subsubsection{Staatsleistungen an Kirchen beenden}
\abstimmung
Die Länder zahlen jährlich ca. 400-500 Millionen Euro an die Kirchen, hauptsächlich für die Gehälter von Bischöfen und anderen Geistlichen. In Rheinland-Pfalz wurden dafür im aktuellen Landeshaushalt etwa 50 Millionen Euro veranschlagt. Viele Kommunen in Rheinland-Pfalz müssen darüber hinaus aufgrund jahrhundertealter Verträge eigene Zahlungen an Kirchengemeinden leisten. Wir möchten diese Zahlungsverpflichtungen von Land und Kommunen gesetzlich beenden und die Mittel in anderen Bereichen einsetzen.

%%\section{Umwelt}

\wahlprogramm{Mehr Transparenz und Bürgerbeteiligung}
\antrag{Silberpappel}\version{03:13, 20. Jun. 2010}

\subsubsection{Mehr Transparenz und Bürgerbeteiligung}
\abstimmung
Viele der heutigen Umweltprobleme – vom Schrumpfen der Artenvielfalt bis zum Versagen der Atommülldeponierung – sind auch das Resultat einer Ohnmacht der Bürger gegenüber den Interessen stark mit dem Staat verflochtener Wirtschaftskräfte.

Daher fordern wir beim Thema Umwelt mehr Transparenz im Handeln von Regierungen und Unternehmen und eine stärkere Beteiligung der Bürger an politischen Entscheidungsprozessen.

Der freie und nutzerfreundliche Zugang zu Umweltinformationen ist eine wichtige Voraussetzung hierfür und muss weiter verbessert werden.
 
\wahlprogramm{Natur als Gemeingut}
\antrag{Silberpappel}\version{03:13, 20. Jun. 2010}

\subsubsection{Natur als Gemeingut}
\abstimmung
Die Natur ist ein Gemeingut.

\subsubsection{Natur als Gemeingut Fortsetzung}
\abstimmung
Sie gehört allen Menschen gleichermaßen.

\subsubsection{Pflicht zum Umwelt- und Naturschutz}
\abstimmung
Daraus ergibt sich die Pflicht zum Schutz von Natur und Umwelt, damit diese nicht durch übermäßige Nutzung durch einzelne oder Gruppen zerstört und damit der Allgemeinheit entzogen werden. Dies gilt auch generationsübergreifend.

\subsubsection{Ablehnung von Naturraum-Monopolisierung}
\abstimmung
Übermäßige Monopolisierung von Naturräumen wie z.B. Privatstrände oder großräumig eingezäunte Gebiete lehnen wir ab.

\subsubsection{Nutzung Alternative I}
\abstimmung
Naturschutz darf nicht zu einer generellen Trennung des Menschen von der Natur führen.

\subsubsection{Nutzung Alternative II}
\abstimmung
Naturschutz darf nicht dazu führen, dass der Mensch generell aus der Natur ''ausgesperrt'' wird.

\subsubsection{Freizeit-Nutzung}
\abstimmung
Die Nutzung von Naturräumen für Sport und Freizeit muss grundsätzlich möglich sein. Generelle Verbote von z.B. Mountainbiken, Geocaching oder Baden in Seen lehnen wir ab. Solche Verbote sollen nur zielgerichtet für einzelne Gebiete ausgesprochen werden, die eines besonderen Schutzes bedürfen.
 
\wahlprogramm{Nachhaltigkeit}
\antrag{Thomas Heinen}\version{03:13, 20. Jun. 2010}

\subsubsection{Nachhaltigkeit}
\abstimmung
Wir stehen für das Prinzip der Nachhaltigkeit. Darunter verstehen wir die Entwicklung einer zukunftsfähigen Gesellschaft, die natürliche Ressourcen so nutzt und bewahrt, dass diese auch den nachfolgenden Generationen zur Verfügung stehen und der Artenreichtum unseres Planeten dauerhaft erhalten bleibt. Hierzu ist ein bewusster und verantwortungsvoller Umgang mit den Ressourcen und ihre faire Verteilung erforderlich. Bei nachwachsenden Ressourcen müssen Verbrauch und Regeneration im Gleichgewicht sein. Bei nicht nachwachsenden Ressourcen wie Bodenschätzen ist die Einführung einer Kreislaufwirtschaft oberstes Ziel.
 
\wahlprogramm{Feinstaubbelastung}
\antrag{Thomas Heinen}\version{03:13, 20. Jun. 2010}

\subsubsection{Feinstaubbelastung}
Wir setzen uns für die Förderung des ÖPNV, für die Förderung der Schiene im Gütertransport, für regionale Wirtschaftskreisläufe ohne lange Transportwege und für neue Konzepte für den Individualverkehr ein. Im Interesse der Gesundheit aller Einwohner setzen wir uns somit für eine Verminderung der Feinstaubbelastung ein. Weitgehend wirkungslose Alibimaßnahmen wie die sogenannten Feinstaubplaketten und Umweltzonen lehnen wir dagegen ab.
%%\section{Energieversorgung}

\subsection*{Umbau der Energieversorgung}

\wahlprogramm{Ökologische Energieversorgung}
\antrag{unbekannt}\version{03:14, 20. Jun. 2010}

\subsubsection{Ökologische Energieversorgung} 
\abstimmung
Die Piratenpartei strebt in Rheinland-Pfalz eine möglichst dezentrale, nachhaltige und ökologische Energieversorgung an.

\subsubsection{Abschaltung Großkraftwerke}
Langfristig streben wir die Abschaltung aller Großkraftwerke in Rheinland-Pfalz an.

\subsubsection{Erneuerbare Energien}
Dazu sollen alle Arten der erneuerbaren Energien, die in Rheinland-Pfalz sinnvoll eingesetzt werden können, auch eingesetzt und ausgebaut werden.
 
\subsection*{Priorität für eine ökologische Energieversorgung}
\wahlprogramm{Umbau der Rheinland-Pfälzischen Energieversorgung}
\antrag{unbekannt}\version{03:14, 20. Jun. 2010}

\subsubsection{Umbau der Rheinland-Pfälzischen Energieversorgung}
\abstimmung
Die Piratenpartei strebt diesen Umbau der Rheinland-Pfälzischen Energieversorgung an, auch wenn dabei lokal Nachteile entstehen, wie zum Beispiel optische Beeinträchtigungen durch Windräder, oder die Gefahr leichter Erdbeben durch Geothermiekraftwerke.
 
\wahlprogramm{Priorität für eine ökologische Energieversorgung}
\antrag{Limbo}\version{03:14, 20. Jun. 2010}

\subsubsection{Priorität im Staat}
\abstimmung
Um dem Bürger und jedem Menschen in Rheinland-Pfalz und der Bundesrepublik mit bestem Beisspiel vorran zu gehen, sollte das Land jedes öffentliche Gebäude und wo immer möglich Strom aus ausschließlich regenerativen Energien beziehen. Dies wird natürlich die Landeskasse belasten, sollte aber als Investition in die Zukunft gesehen werden und könnte versucht werden durch höhere Parkgebühren in Innenstädten teilweise gegenzufinanzieren. Nebeneffekt wäre evtl. ein Wechsel der Autofahrer auf den ÖPNV.
 
\subsection*{Klein- statt Großkraftwerke}
\wahlprogramm{dezentrale Energieversorgung}
\antrag{unbekannt}\version{03:14, 20. Jun. 2010}

\subsubsection{dezentrale Energieversorgung}
\abstimmung
Um eine dezentrale Energieversorgung zu ermöglichen, setzt die Piratenpartei auch auf effiziente, kleine Kraftwerke, wie etwa Blockheizkraftwerke. Diese Kraftwerke wollen wir zu einem möglichst großen Teil mit vor Ort auf ökologischem Weg gewonnen Brennstoffen betreiben.
 
\wahlprogramm{Dezentrale Energieversorgung}
\antrag{Silberpappel}\version{03:14, 20. Jun. 2010}

\subsubsection{Dezentrale Energieversorgung}
\abstimmung
Ein wichtiger Aspekt moderner Energiepolitik ist die zunehmender Dezentralisierung der Energieerzeugung. Die damit einhergehende Unabhängigkeit von Großkraftwerken kann durch regionale / kommunale Energiegewinnung aus umweltfreundlichen Quellen (Wind, Sonne, Wasser, Geothermie, Biomasse (keine Nahrungsmittel)) erreicht werden.
\subsubsection{Geothermie}
\abstimmung
In Rheinland-Pfalz sehen wir vor allem gute Voraussetzungen für die Nutzung von Geothermie.

\subsubsection{Infrastruktur}
\abstimmung
Da eine stärkere Dezentralisierung der Strom- und Wärmeerzeugung eine angepasste Infrastruktur voraussetzt, sind neue Speicher- und Verteilungstechnologien nötig. Wir werden deren Entwicklung und Einsatz verstärkt fördern.
 
\subsection*{Netzneutralität bei Energienetzen}
\wahlprogramm{Netzneutralität bei Energienetzen}
\antrag{Silberpappel}\version{03:14, 20. Jun. 2010}

\subsubsection{Netzneutralität bei Energienetzen}
\abstimmung
Monopole von Konzernen in der Energienetz-Infrastruktur behindern den Wettbewerb. Deshalb streben wir eine eigentumsrechtliche Entflechtung der Stromnetze an. Dadurch kann ein diskriminierungsfreier Zugang für alle Energieerzeuger garantiert werden, besonders auch für regionale Kleinerzeuger.

%%\section{Inneres und Justiz}
\version{03:19, 20. Jun. 2010}

\subsection*{Einleitung}
\wahlprogramm{Grundlagen einer piratigen Innenpolitik}
\antrag{Silberpappel (abgeschrieben von BaWü)}\version{03:19, 20. Jun. 2010}
\subsubsection{Grundlagen einer piratigen Innenpolitik}
\abstimmung
Wir setzen uns für eine effektive Sicherheitspolitik ein. Ein sicheres Leben ist ein wichtiges und unabdingbares Gut für die Bewohner unseres Landes. Uns geht es um angemessene und vor allem wirksame Methoden. In der Vergangenheit wurden im Bereich der Sicherheitspolitik ineffektive Gesetze erlassen, aber gleichzeitig die Mittel für Polizei und andere Behörden gekürzt. Grundrechte wurden dadurch eingeschränkt und die Überwachung hat zugenommen, dabei soll in einer Demokratie der Bürger den Staat überwachen und nicht umgekehrt.
 
\subsection*{Lockerung der Residenzpflicht}
\wahlprogramm{Asyl- und Flüchtlingspolitik}
\antrag{KV Trier/Trier-Saarburg}\version{03:19, 20. Jun. 2010}
\subsubsection{Lockerung der Residenzpflicht}
\abstimmung
Die allgemeine Erklärung der Menschenrechte garantiert in Artikel 13 das Recht auf Freizügigkeit. Deutschland ist der einzige Staat in Europa, der dieses für Asylsuchende und anerkannte Flüchtlinge einschränkt.

Die Residenzpflicht macht Menschen zu Kriminellen, nur weil sie sich frei bewegen wollen. Polizei, Gerichte und Behörden werden zusätzlich unnötig belastet.

Ähnlich wie bereits in Bayern und Brandenburg wollen wir daher auch in unserem Bundesland die Residenzpflicht lockern.
 

\wahlprogramm{Abschaffung von Ausreisezentren}
\antrag{Thomas Heinen}\version{03:19, 20. Jun. 2010}
\subsubsection{Abschaffung von Ausreisezentren}
\abstimmung
Wir unterstützen die Forderung der Bewohner des Ausreisezentrums (LUFA) in Trier nach einer ersatzlosen Schließung der Einrichtung.

Die Bewohner der LUFA sind Flüchtlinge, die durch starken Druck zu einer "freiwilligen" Ausreise gebracht werden sollen. Sinn und Zweck von Ausreisezentren ist es, den Willen von Menschen zu brechen.

Diese Zielsetzung und die Art und Weise der Unterbringung sind mit dem Menschenrecht nicht vereinbar.

Wir werden uns für eine menschenwürdige Flüchtlingspolitik einsetzen.
 
\subsection*{Polizei im öffentlichen Raum Internet}
\wahlprogramm{Polizei im öffentlichen Raum Internet}
\antrag{KV Trier/Trier-Saarburg}\version{03:19, 20. Jun. 2010}
\subsubsection{Modul 1}
\abstimmung
Auch im Internet müssen die Grundrechte und das Prinzip der Verhältnismäßigkeit gewahrt bleiben. Eingriffe in private Kommunikation, etwa das Mitlesen von E-Mails, dürfen nur nach richterlicher Anordnung möglich sein. Das Einschleusen von Software in private Computer lehnen wir vollständig ab.
 
\subsection*{Keine religiösen Symbole in öffentlichen Gebäuden}
\wahlprogramm{Keine religiösen Symbole in öffentlichen Gebäuden}\label{justiz:symbole1}
\antrag{KV Trier/Trier-Saarburg}\konkurrenz{justiz:symbole2}\version{03:19, 20. Jun. 2010}
\subsubsection{Keine religiösen Symbole in öffentlichen Gebäuden}
\abstimmung
Das Anbringen von religiösen Symbolen in öffentlichen Gebäuden verletzt die Religionsfreiheit von Angehörigen anderer Religionen und Menschen ohne Religion. Dies wurde vom Bundesverfassungsgericht sowie vom Europäischen Gerichtshof für Menschenrechte festgestellt. Wir möchten daher dafür sorgen, dass diese religiösen Symbole aus öffentlichen Gebäuden, vor allem auch den öffentlichen Schulen, entfernt werden.
 
\wahlprogramm{Keine religiösen Symbole in öffentlichen Gebäuden}\label{justiz:symbole2}
\antrag{marcus}\konkurrenz{justiz:symbole1}\version{03:19, 20. Jun. 2010}
\subsubsection{Keine religiösen Symbole in öffentlichen Gebäuden}
Das Anbringen von religiösen Symbolen in öffentlichen Gebäuden kann die religiösen Gefühle von Menschen anderer Religion verletzen. In diesen Fällen ist basisdemokratisch eine Entscheidung über den Verbleib oder die Entfernung dieser Symbole zu entscheiden. In öffentlichen Schulen haben ausserhalb des Religionsunterrichtes religiöse Symbole nichts zu suchen, da wir im Zeitalter der Aufklärung für eine strikte Trennung von Kirche und Staat eintreten.
 
\newpage
\subsection*{Gewaltmonopol}
\wahlprogramm{Gewaltmonopol}
\antrag{KV Trier/Trier-Saarburg}\version{03:19, 20. Jun. 2010}
\subsubsection{Keine Privatisierung hoheitlicher Aufgaben}
Das Gewaltmonopol des Staates darf nicht an Privatfirmen delegiert werden. Polizeiaufgaben, das Beaufsichtigen von Gefängnissen und ähnliches müssen vollständig in staatlicher Hand bleiben.
 
[Bearbeiten] Justiz
Meta
Antragsteller: 	KV Trier/Trier-Saarburg
Thema: 	Justiz
Id: 	17.8
Module
17.8.1 Unabhängigkeit der Staatsanwaltschaften 	Staatsanwälte sind an dienstliche Anweisungen ihrer Vorgesetzten gebunden. Dadurch besteht die Gefahr, dass politisch unerwünschte Strafverfahren beeinträchtigt werden. Um die Unabhängigkeit der Justiz und den Rechtsstaat zu stärken, wollen wir, dass die Landesregierung sich gesetzlich verpflichtet, von ihrem Weisungsrecht gegenüber den Landesstaatsanwälten keinen Gebrauch mehr zu machen. Insbesondere soll es keine Dienstanweisungen mehr geben, die sich auf einzelne Verfahren beziehen.
17.8.2 Öffentlichkeitsarbeit der Staatsanwaltschaften 	Wir sehen mit Sorge, wie durch eine nicht zu verantwortende Öffentlichkeitsarbeit einiger Staatsanwaltschaften die im Rechtsstaat verankerte Unschuldsvermutung zunehmend zu Lasten von Beschuldigten ausgehebelt wird. Deshalb wollen wir dienst- und strafrechtliche Sanktionsmöglichkeiten gegenüber Staatsanwälten bei entsprechenden Verstößen verschärfen.
 
Meta
Antragsteller: 	marcus
Thema: 	Justiz
Id: 	17.9
Module
17.9.1 Unabhängigkeit der Staatsanwaltschaften 	Wir fordern die Unabhängigkeit der Staatsanwälte von der politischen Führung und deshalb ein Verbot des Weisungsrechtes der vorgesetzten Behörden auf Einstellung von Verfahren.
17.9.2 Unabhängigkeit der Staatsanwaltschaften 	Alle Verfahren, die aufgrund von Weisungsrechten übergeordneter Behörden gegenüber Staatsanwälten von diesen eingestellt werden, sind jährlich unter Namensnennung des Anweisenden zu veröffentlichen. Nur sofern die Verfahren gegen Privatpersonen liefen, sind deren Daten zu anonymisieren.
 
Meta
Antragsteller: 	Limbo
Thema: 	Strafmaß
Id: 	17.10
Module
17.10.1 Strafmaßverschärfung 	Die Würde jedes Menschen ist bekanntermaßen unantastbar. Deswegen sollte jeder, der diese Würde wissentlich (psychologisch geprüft) zerstört (z.B. durch Vergewaltigung, Misshandlung, Freiheitsberaubung) mit lebenslanger Strafe rechnen. Dies ist nicht immer der Fall, da die Sicherungsverwahrung oftmals nachträglich auf Antrag gändert werden. Dadurch verliert diese Strafe an abschreckender Wirkung. Deswegen wäre eine Möglichkeit Anträge zu streichen durch verpflichtende regelmäßge Gutachten über Gefahrepotential zu ersetzen. Darauf sollte die weitere Haftplanung fußen.
 
[Bearbeiten] Polizei
Meta
Antragsteller: 	Silberpappel
Thema: 	Unabhängige Kontrolle der Polizei
Id: 	17.11
Module
17.11.1 who watches the watchmen? 	Wir treten für eine demokratische, transparente und unabhängige Kontrolle der Polizei ein. Das wollen wir erreichen durch die Einführung einer unabhängigen Polizeikommission. Die Polizeikommission soll sicherstellen, dass die Exekutivorgane nach rechtsstaatlichen Prinzipien arbeiten. Die Kommission soll als Ansprechpartner von Beschwerdeführern und aktiver Ermittler tätig werden. Sie soll polizeiinterne Beschwerden und Beschwerden von Bürgern entgegennehmen. Um die Unabhängigkeit der Kommission zu gewährleisten, sollte sie nicht den Innenministerien, sondern einem anderen Ressort unterstellt sein.
%%\section{Verbraucherschutz}

\subsection*{Mehr Transparenz beim Einkauf}
\wahlprogramm{Mehr Transparenz beim Einkauf}
\antrag{unbekannt}\version{03:16, 20. Jun. 2010}
\subsubsection{Mehr Transparenz beim Einkauf}
\abstimmung
Die Piratenpartei setzt sich für einen besseren Verbraucherschutz und mehr Transparenz, nicht nur in der Politik und Verwaltung, sondern auch im Alltag ein.
 
\wahlprogramm{Mehr Transparenz beim Einkauf}
\antrag{unbekannt}\version{03:16, 20. Jun. 2010}
\subsubsection{Mehr Transparenz beim Einkauf}
\abstimmung
Nach dem Scheitern der Lebensmittelampel will die Piratenpartei einen alternativen Weg testen, um für mehr Transparenz beim Einkauf zu sorgen. Die Piratenpartei will ein System testen, bei dem die Verbraucher ihre individuellen Kriterien an ein Produkt, also zum Beispiel auch Unverträglichkeit gegen bestimmte Stoffe, in ein individuelles Profil auf einer Karte speichern können. An Barcodescannern soll so jeder Kunde schnell erkennen können, ob das Produkt seinen individuellen Anforderungen entspricht. Der Datenschutz hat für uns in Verbindung mit diesem Projekt natürlich höchste Priorität.
 
\subsection*{Verbraucherschutz}
\wahlprogramm{Verbraucherinformationsgesetz stärken}
\antrag{KV Trier/Trier-Saarburg}\version{03:16, 20. Jun. 2010}
\subsubsection{Verbraucherinformationsgesetz stärken}
\abstimmung
Wir wollen das Landesgesetz zur Ausführung des Verbraucherinformationsgesetzes (AGVIG) so stärken, dass Verbraucher Informationen, beispielsweise zu belasteten Lebensmitteln, auf gut zugänglichen Plattformen rasch und einfach auffinden können, ohne sie erst in langen Auskunftsprozessen anfordern zu müssen.
 
\subsection*{Verbraucherzentrale stärken}
\antrag{KV Trier/Trier-Saarburg}\version{03:16, 20. Jun. 2010}
\subsubsection{Verbraucherzentrale stärken}
\abstimmung
Verbraucherzentralen spielen eine wichtige Rolle in der Beratung von Verbrauchern und im Schutz von Verbraucherinteressen. Die Einschränkung der Arbeit der Verbraucherzentrale Rheinland-Pfalz durch restriktive Mittelzuweisungen lehnen wir ab.

Wir unterstützen insbesondere den Einsatz der Verbraucherzentralen für den Datenschutz der Verbraucher und ihren Kampf gegen das Modell des „Gläsernen Konsumenten“.

Wir wollen einen Verbraucherschutz, der das Recht auf umfassende Information verbindet mit einem Verbandsklagerecht zur Durchsetzung von Verbraucherinteressen.
 
\subsection*{Verbraucherinformation vor Ort durch Smiley-System}
\antrag{KV Trier/Trier-Saarburg}\version{03:16, 20. Jun. 2010}
\subsubsection{Verbraucherinformation vor Ort durch Smiley-System}
\abstimmung
Die Ergebnisse von Lebensmittelkontrollen werden anhand unterschiedlicher Smileys zeitnah und gut sichtbar an der Eingangstür angebracht, um den Verbraucher zusätzlich zum Informationssystem im Internet direkt vor Ort zu informieren. Das in Dänemark etablierte und sehr erfolgreiche Smiley-System soll auch in Rheinland-Pfalz eingeführt werden. So ist für den Kunden direkt, beispielsweise vor Restaurants, Eisdielen oder Supermärkten, ersichtlich, ob Hygienevorschriften und Lebensmittelgesetze eingehalten werden. Auf Hygienesünder kann reagiert werden, was bisher meistens nicht möglich ist. Negativ bewertete Betriebe haben durch die Kundenreaktion und Folgekontrollen die Möglichkeit und vor allem die Motivation, Mängel zu beseitigen und sich positive Smileys zu verdienen.
 
\subsection*{Transparente Kennzeichnung von Lebensmitteln}
\antrag{KV Trier/Trier-Saarburg}\version{03:16, 20. Jun. 2010}
\subsubsection{Transparente Kennzeichnung von Lebensmitteln}
\abstimmung
Auf der Vorderseite von Verpackungen muss statt Prozentangaben und beliebig wählbarer Portionsgrößen eine einheitliche, differenzierte und transparente Kennzeichnung dem Verbraucher eine schnelle und verlässliche Orientierung geben. Die von der Lebensmittelindustrie auf der Vorderseite von Verpackungen bevorzugte Nährwertkennzeichnung trägt nicht dazu bei, dem Verbraucher sinnvolle Informationen an die Hand zu geben. Besonders irreführend ist die Angabe des prozentualen Anteils am Tagesbedarf. Da sie prinzipiell vielen Personengruppen wie zum Beispiel Kindern nicht gerecht werden kann, ist sie durch eine sinnvolle, verpflichtende Kennzeichnung zu ersetzen. Diese muss sich auf feste Portionsgrößen von 100g/ml entsprechend der Nährwertangaben auf der Rückseite beziehen.

%%\section{Nachwort}
\wahlprogramm{Nachwort}\label{nachwort:unglow}
\antrag{Unglow}\konkurrenz{nachwort:trier}\version{12:50, 13. Jun. 2010}\\
Die in diesem Programm festgelegten Grundsätze und Forderungen stellen einen Zwischenstand der Ausarbeitung dieser Themen seitens unserer Partei dar, der als Programm dient, mit dem wir in den Landtagswahlkampf 2011 ziehen. In vielen Bereichen gibt es einen nahezu unbegrenzten Spielraum für Forderungen, die sich aus dem hier gesagten direkt ableiten lassen. Diese weitere Ausarbeitung und Anwendung unserer programmatischen Grundsätze ist erwünscht und wird von uns als natürlicher demokratischer Prozess der Weiterentwicklung und Vertiefung begriffen.

\wahlprogramm{Nachwort}\label{nachwort:trier}
\antrag{KV Trier/Trier-Saarburg}\konkurrenz{nachwort:unglow}\version{12:50, 13. Jun. 2010}\\
Die Piraten sind, auch gemessen am Alter ihrer Mitglieder, eine junge Partei und möchten neue Ideen in die Politik einbringen. Wir laden alle Menschen, alle Generationen und alle gesellschaftlichen Gruppen im Land dazu ein, mit uns diese neue Politik zu gestalten.

KLARMACHEN ZUM ÄNDERN!



\end{document}
