\section{Gesundheit}

\wahlprogramm{Elektronische Gesundheitskarte}\label{wp:gesundheit:egk1}
\antrag{Niemand13}\konkurrenz{wp:gesundheit:egk1}\version{03:29, 20. Jun. 2010}

\subsubsection{Elektronische Gesundheitskarte stoppen!}
\abstimmung
Gerade im Gesundheitswesen ist der Schutz der Privatsphäre von besonderer Wichtigkeit. Deshalb fordert die Piratenpartei einen effektiven Schutz der Patientendaten und wirksame Kontrollmechanismen. Mit der geplanten Elektronischen Gesundheitskarte laufen wir Gefahr, ein System einzuführen in dem umfangreich und unkontrolliert Patientendaten zentral gespeichert werden. Patientinnen und Patientinnen können ihr Recht auf informationelle Selbstbestimmung nicht mehr wahrnehmen. Die vielen Datenskandale der letzten Zeit zeigen eindrücklich die Risiken solcher Systeme. Gerade die Gesundheitsdaten der Bürgerinnen und Bürger werden große Begehrlichkeiten wecken. Die Piratenpartei fordert deshalb den Stopp der elektronischen Gesundheitskarte.
 

\wahlprogramm{Elektronische Gesundheitskarte}\label{wp:gesundheit:egk2}
\antrag{Unbekannt}\konkurrenz{wp:gesundheit:egk1}\version{03:29, 20. Jun. 2010}

\subsubsection{Elektronische Gesundheitskarte umgestalten}
\abstimmung
Gerade im Gesundheitswesen ist der Schutz der Privatsphäre von besondere Wichtigkeit. Gleichzeitig muss aber auch gerade das Gesundheitssystem für den Patienten möglichst transparent sein, damit er selbst die für seine Gesundheit besten Entscheidungen treffen kann. Dazu muss der Patienten auch mit möglichst vielen Rechten ausgestattet sein. Deshalb fordert die Piratenpartei sowohl einen effektiven Schutz der Patientendaten als auch transparentes Gesundheitssystem und ausreichend Recht für die Patienten. Mit der geplanten Elektronischen Gesundheitskarte laufen wir Gefahr, ein System einzuführen in dem umfangreich Patientendaten gespeichert werden. Die vielen Datenskandale der letzten Zeit zeigen die Risiken eines solchen System. Gerade die Gesundheitsdaten der Bürgerinnen und Bürger werden große Begehrlichkeiten Wecken. Die elektronische Gesundheitskarte bietet aber viele Vorteile, um die Effizienz des Gesundheitssystems zu steigern und die Versorgung der Patienten zu verbessern. Während der Patient durch die Gesundheitskarte transparenter werden würde ist die Arbeit der Ärzte relativ intransparent. Die Piratenpartei fordert deshalb, dass auf der elektronischen Gesundheitskarte nicht mehr Daten gespeichert werden sollen als auf der bisherigen Krankenkarte. Die Karte soll aber die Möglichkeit bieten, mehr Daten zu erfassen. Jeder Patient soll, wenn er will, selbst die Speicherung zusätzlicher auf der Gesundheitskarte zulassen können.
 
\subsubsection{Bewertung von Ärzten}
\antrag{Unbekannt}\version{03:29, 20. Jun. 2010}

\subsubsection{Bewertung von Ärzten}
\abstimmung
Aufgrund des Schutzes des Patienten ist die Erhebung von Daten zur Kontrolle der Leistung der Ärzte nur schwer möglich. Aber gerade solche Informationen währen auch für die Patienten wichtig. Es muss also auch ein Weg gefunden um dem Patienten möglichst viel Informationen zur Verfügung zu stellen. Um eine besser Kontrolle über die Leistung der Ärzte zu haben und um den Patienten mehr Informationen zur Verfügung zu stellen fordern wird, dass auch jeder Kassenpatient immer eine Rechnung bekommt auf der alle Leistungen die der Arzt erbracht hat aufgelistet sind. Zu jeder Rechnung ist dem Patient auch ein anonymer Fragebogen auszuhändigen, mit dem er bestätigen kann, dass er alle auf der Rechnung aufgeführten Leistungen auch wirklich erhalten hat. Zudem soll der Patient auf dem Fragebogen auch eine Bewertung über seine Zufriedenheit mit der Behandlung und der Beratung des Arztes abgeben können. Die Abgabe der Fragebögen soll möglichst einfach gestaltet werden. Die Daten aus solchen Erhebungen sollen allen Bürgern frei zugänglich gemacht werden.
 
\wahlprogramm{Pflege}
\antrag{Silberpappel}\version{03:29, 20. Jun. 2010}

\subsubsection{Menschenwürde}
\abstimmung
Kostendruck und Gewinnstreben haben in vielen Pflegeeinrichtungen dazu geführt, dass die Pflegebedürftigen unter Umständen leben müssen, die ihre Menschenwürde verletzen.

Wir wollen dafür sorgen, dass ruhigstellende Medikamente nur verabreicht werden, wenn dies dem Wohl des Pflegebedürftigen dient, oder zum Schutz der Pflegenden absolut notwendig ist.

Auch das Fesseln ans Bett (''Fixierung'') soll nur zulässig sein, wenn es zum Schutz des Pflegebedürftigen oder der Pflegenden unumgänglich ist.

\subsubsection{Wege}
\abstimmung
Um dies zu erreichen, setzen wir uns für eine ausreichende Personalausstattung in der Pflege ein, für effektivere Kontrollen und dafür, dass dabei nicht nur Zahlen geprüft, sondern auch Bewohner der Pflegeeinrichtung befragt werden.

\subsubsection{Demokratie}
\abstimmung
Angehörigenbeiräte sehen wir als weiteres sinnvolles Mittel, Qualität und Menschlichkeit in der Pflege zu fördern.

\subsubsection{Nachsatz}
\abstimmung
Die Würde des Menschen ist das höchste Gut in unseren Grundgesetz und muss auch in der Pflege oberstes Gebot sein.

\subsubsection{Transparenz}
\abstimmung
Der Medizinische Dienst der Krankenkassen (MDK) prüft Pflegeeinrichtungen und erstellt die sogenannten ''Einrichtungsbezogenen Pflegeberichte''. Diese dürfen nach derzeitiger Gesetzeslage nicht veröffentlicht werden.

\subsubsection{Transparenz herstellen Alternative I}
\abstimmung
Wir wollen dagegen eine Pflicht zur Veröffentlichung einführen.

\subsubsection{Transparenz herstellen Alternative II}
\abstimmung
Wir wollen die Veröffentlichung erlauben.

\subsubsection{Auswirkungen Transparenz}
\abstimmung
Durch solche Informationen können sich die Verbaucher ein Bild von der Qualität einzelner Pflegeeinrichtungen machen. So entsteht Druck auf die Pflegeeinrichtungen, Missstände zu beseitigen und Qualität zu erhöhen.
 
\newpage
\wahlprogramm{Gesundheit und Freiheit}
\antrag{Unbekannt}\version{03:29, 20. Jun. 2010}

\subsubsection{ }
\abstimmung
Die Gesundheit ist unser höchstes Gut.

Bewusst wird dies vielen Menschen erst dann, wenn die Abwesenheit der Gesundheit, also eine Erkrankung ins Spiel kommt. Eine Erkrankung kann uns mehr oder minder, zeitweise oder auch für immer in unseren geliebten Freiheiten einschränken. Die Erkrankung kann uns in eine Pflegebedürftigkeit bringen, bei der wir womöglich für den Rest unseres Lebens, vielleicht auch nur für eine kürzere Zeitspanne, auf die Hilfe anderer angewiesen sind. Erst in dieser Zeit, der Abwesenheit der Gesundheit merkt der Betroffene die Wertschätzung dieses Gutes "Gesundheit".

Die Gesundheit und die Freiheit eines jeden Menschen ist ein hohes zu schützende Gut. Erhalten Sie sich Ihre Freiheit und Ihre Gesundheit.

Zur Erhaltung unserer Gesundheit bedarf es auch einer großen Eigenverantwortung und Selbstdisziplin. Angefangen bei einer gesunden Ernährung, regelmäßigem aber nicht übertriebenem Sport, dem vermeiden von Gesundheitsrisiken durch z.B. den Konsumverzicht von tolerierten Drogen wie Nikotin und Alkohol oder nicht tolerierten Drogen wie LSD, Marihuana, Kokain und vielen anderen natürlichen und chemisch hergestellten Drogen.

Wie wichtig ist Ihnen Ihre Gesundheit oder auch die Gesundheit Ihrer Kinder? Wie frei sind Sie in Ihrer Entscheidung sich gesund zu ernähren? Wie viel Eigenverantwortung tragen Sie selbst zur Erhaltung Ihrer Gesundheit? Wie viel ist Ihnen /uns unsere Gesundheit Wert? Wie kann oder sollte nach Ihrer Meinung die Politik Einfluss nehmen auf eine gesunde Lebensführung ohne dabei die Freiheit des einzelnen einzuschränken? Wie weit muss die Freiheit einer Marktwirtschaft gesetzlich eingeschränkt werden um die Gesundheit der Menschen nicht zu gefährden, um langfristig das Überleben der Menschheit zu gewährleisten?

Viele Fragen, die weitere Fragen hervorbringen. Wer Fragen stellt ist oft unbequem. Das kann mehrere Ursachen haben? 1. Der Befragte kann nicht antworten weil er keine Antwort auf die Frage hat, sich mit dem Thema noch nicht gedanklich auseinandergesetzt hat. 2. Der Befragte ist an einer ehrlichen Beantwortung nicht interessiert weil er den Fragenden und auch andere in Unkenntnis lassen möchte, um Risiken oder Nebenwirkungen zu vertuschen um Kapital aus einem vermeintlich sicheren oder gesunden Produkt zu schlagen.

Fragen können scharf wie ein Messer, - für den Befragten, manchmal gefährlich sein. Fragen können neue Fragen aufbringen oder Antworten liefern die zumindest kurzfristig akzeptabel sind, bis die Antwort durch neue Erkenntnisse erneut in Frage gestellt werden muss.

Ehrliche Fragen und ehrliche Antworten führen zu einer TRANSPARENZ zu einer Klarheit und Weitsichtigkeit. Wahrheit bringt Klarheit.

Neben der Eigenverantwortung zur Erhaltung der Gesundheit besteht jedoch auch eine gesellschaftliche Verpflichtung für eine humane soziale Grundordnung durch die Wirtschaft. Die Menschen dürfen nicht die Sklaven der Wirtschaft sein sondern die Wirtschaft muss zum Wohle der Menschen da sein. Die Wirtschaft soll das Leben der gesamten Menschheit erleichtern.

Um bei Fragen der gesunden Ernährung zu bleiben wäre zu klären: Wie gesund sind die auf dem Markt angebotenen Lebensmittel? Wie viel Hormone wie viel Antibiotika wird in der Tierzucht eingesetzt? Wie viele Pestizide und Insektizide belasten unser Gemüse tatsächlich?

Andere Fragen im Gesundheitsbereich sind: Was ist mit der elektronischen Gesundheitskarte (EGK)? Wer will sie einführen, verdient damit Geld? Ist die EGK wirklich ein Vorteil für die Kranken? Stecken Versicherungen mit in der Entwicklung? Wie soll es mit den Krankenkassen und den Beiträgen weitergehen?
Ist die Kopfpauschale ein Vorteil oder entlastet Sie nur die Vermögenden?
 
\newpage
\wahlprogramm{Für Aufklärung – gegen Diskriminierung}\label{wp:gesundheit:auf1}
\antrag{Thomas Heinen}\konkurrenz{wp:gesundheit:auf2}\version{03:29, 20. Jun. 2010}

\subsubsection{Für Aufklärung – gegen Diskriminierung}
\abstimmung
Menschen mit Krankheiten, z.B. HIV/AIDS, sind Menschen wie du und ich. Es gibt keinen Grund sie zu diskriminieren. Die Piratenpartei Rheinland-Pfalz will deswegen dabei mithelfen, der Ausgrenzung der Betroffenen in der Gesellschaft ein Ende zu setzen. Dazu werden wir uns dafür einsetzen, dass Projekte in Rheinland-Pfalz gestartet und gefördert werden, welche einen positiven Beitrag zur Aufklärung und gegen Diskriminierung leisten.
 
\wahlprogramm{Für Aufklärung – gegen Diskriminierung}\label{wp:gesundheit:auf2}
\antrag{Piraten aus RLP}\konkurrenz{wp:gesundheit:auf1}\version{03:29, 20. Jun. 2010}

\subsubsection{Für Aufklärung – gegen Diskriminierung}
\abstimmung
Menschen mit Krankheiten sind Menschen wie du und ich. Es gibt keinen Grund sie zu diskriminieren. Die Piratenpartei Rheinland-Pfalz will deswegen dabei mithelfen, der Ausgrenzung der Betroffenen in der Gesellschaft ein Ende zu setzen. Dazu werden wir uns dafür einsetzen, dass Projekte in Rheinland-Pfalz gestartet und gefördert werden, welche einen positiven Beitrag zur Aufklärung und gegen Diskriminierung leisten.