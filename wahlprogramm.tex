\section{Wahlprogramm}
\wahlprogramm{Vergaberegister zur Korruptionsbekämpfung - Positivliste}\label{wpa:vergaberegister1}
\antrag{Stephan Schurig}\\
\version{17:32, 9. Apr. 2012}
\begin{itemize}
\item \konkurrenz{wpa:vergaberegister2}
\end{itemize}

\paragraph{Antragstext}:

Der Landesparteitag möge beschließen, folgende Formulierung im Wahlprogramm zu ändern:

\einruecken{Wir wollen ein Vergaberegister schaffen, mit dessen Hilfe bereits \textbf{positiv} auffällig gewordene Firmen künftig von der Vergabe öffentlicher Aufträge \textbf{bevorzugt} werden können \textbf{(Positivliste). Zu diesem Zwecke werden alle Unternehmen aufgeführt, die sich bei vorangegangenen Projekten als vertrauenswürdig erwiesen haben.} Diese Informationen sollen nicht nur Behörden zur Verfügung stehen, sondern auch der interessierten Öffentlichkeit.}

\paragraph{Alte Fassung}:

\einruecken{Wir wollen ein Vergaberegister schaffen, mit dessen Hilfe bereits \textbf{negativ} auffällig gewordene Firmen künftig von der Vergabe öffentlicher Aufträge \textbf{ausgeschlossen} werden können. Diese Informationen sollen nicht nur Behörden zur Verfügung stehen, sondern auch der interessierten Öffentlichkeit.}

\begruendung{Die aktuelle Formulierung macht keine Aussage über die Form des Vergaberegisters (Blacklist, Whitelist, Greylist etc.).}

% -----

\wahlprogramm{Landärztemangel entgegenwirken}
\antrag{Martin Otto}\\
\version{17:51, 9. Apr. 2012}

\paragraph{Antragstext}:

Der Landesparteitag (LPT) möge beschließen, dem Landärztemangel aktiv gegenzusteuern.

\begruendung{Seit Jahren ist Zahl der praktizierenden Ärzte auf dem Land Rückläufig. Das führt zu einer gravierenden Unterversorgung der fachärztlichen Betreuung in ländlichen Regionen.


Erläuterung

Um diesem Mangel an Ärzten in ländlichen Regionen entgegenzusteuern bedarf es umfangreicher struktureller Maßnahmen. Dazu gehört die Einführung eines nichtrückzahlbaren Zusatzstipendiums. Dieses geht einher mit der Verpflichtung, für Dauer der Zahlung anschließend auf dem Land zu arbeiten. Zusätzlich müssen dafür weitere Anreize geschaffen werden. Die angehenden Landärzte erfahren Unterstützung bei der Einrichtung einer Praxis und erhalten ein Grundgehalt. Des weiteren ist eine Neuordnung des Bereitschaftsdienstes unabdinglich. Die Unterstützung von Familienmitgliedern bei der Erwerbstätigkeit, ist ein weiterer wichtiger Anreiz. Zur Umsetzung dieser Maßnahmen, sind umgehend Kommissionen einzusetzen, bestehend aus Fachleuten vom KVSA, dem Hausärzteverband Sachsen-Anhalt e.V., sowie dem Hartmannbund, die die anstehenden Probleme benennen, damit die Politik die zeitnahe Behebung auf den Weg bringen kann.}

% -----

\wahlprogramm{Vergaberegister zur Korruptionsbekämpfung - Negativliste}\label{wpa:vergaberegister2}
\antrag{Stephan Schurig}\\
\version{17:54, 9. Apr. 2012}
\begin{itemize}
\item \konkurrenz{wpa:vergaberegister1}
\end{itemize}

\paragraph{Antragstext}:

Der Landesparteitag möge beschließen, folgende Formulierung im Wahlprogramm zu ändern:

\einruecken{Wir wollen ein Vergaberegister schaffen, mit dessen Hilfe bereits negativ auffällig gewordene Firmen künftig von der Vergabe öffentlicher Aufträge ausgeschlossen werden können \textbf{(Negativliste). Zu diesem Zwecke werden alle Unternehmen aufgeführt, die sich bei vorangegangenen Projekten als nicht vertrauenswürdig erwiesen haben. Dazu muss im Vorfeld eine Negativliste mit Verhaltensverboten aufgestellt werden, welche sich an der Definition {\Gu}Unzulässige geschäftliche Handlungen{\Go} im Gesetz gegen den unlauteren Wettbewerb (UWG) orientiert. Bei Zuwiderhandlung kommt es zur Eintragung des Unternehmens in eine schwarze Liste. Die Löschung dieser Daten ist erst nach einer bestimmten Zeit möglich.} Diese Informationen sollen nicht nur Behörden zur Verfügung stehen, sondern auch der interessierten Öffentlichkeit.}

\paragraph{Alte Fassung}:

\einruecken{Wir wollen ein Vergaberegister schaffen, mit dessen Hilfe bereits negativ auffällig gewordene Firmen künftig von der Vergabe öffentlicher Aufträge ausgeschlossen werden können. Diese Informationen sollen nicht nur Behörden zur Verfügung stehen, sondern auch der interessierten Öffentlichkeit.}

\begruendung{Die Formulierung macht keine Aussage über die Form des Vergaberegisters (Blacklist, Whitelist, Greylist etc.). Die Definition {\Gu}Unzulässiger geschäftlicher Handlungen{\Go} \href{http://www.gesetze-im-internet.de/uwg_2004/anhang_26.html}{ist im UWG geklärt} und \href{http://anwalt-im-netz.de/archiv/2008/uwg-verhaltensverbote.html}{verständlichere Beispiele sind hier aufgeführt}.}

% -----

\wahlprogramm{Abschaffung der 5\%-Hürde bzw. Sperrklausel auf Landesebene}\label{wpa:prozenthuerde1}
\antrag{Stephan Schurig}\\
\version{18:03, 9. Apr. 2012}
\begin{itemize}
\item \konkurrenz{wpa:prozenthuerde2}
\item \konkurrenz{wpa:prozenthuerde3}
\item \konkurrenz{wpa:prozenthuerde4}
\item \konkurrenz{wpa:prozenthuerde5}
\item \konkurrenz{wpa:prozenthuerde6}
\end{itemize}

\paragraph{Antragstext}:

Der Landesparteitag möge folgenden Punkt an geeigneter Stelle im Wahlprogramm einfügen:

\einruecken{Die Piratenpartei Sachsen-Anhalt fordert die Abschaffung der Sperrklausel bei Landtagswahlen in Sachsen-Anhalt.}

\paragraph{Begründung}:

\begruendung{Zu mehr Bürgerbeteiligung und Mitbestimmung gehört, dass auch kleinere gewählte Fraktionen an den demokratischen Prozessen teilhaben dürfen. Der Nutzen wiegt dabei größer als die Gefahren, durch den Einzug radikaler Parteien, da die Mehrheit der Unter-5\%-Parteien nicht als radikal eingestuft werden kann.

\textit{siehe auch die \href{http://wiki.piratenpartei.de/LSA:Landesverband/Organisation/Mitgliederversammlung/2012.1/Antragsfabrik/Abschaffung_der_5\%25-H\%C3\%BCrde_bzw._Sperrklausel_auf_Landesebene}{Pro- und Contra-Argumente auf der Antragsseite}}}

% -----

\wahlprogramm{Herabsetzung der 5\%-Hürde bzw. Sperrklausel auf Landesebene auf 1\%}\label{wpa:prozenthuerde2}
\antrag{Stephan Schurig}\\
\version{18:06, 9. Apr. 2012}
\begin{itemize}
\item \konkurrenz{wpa:prozenthuerde1}
\item \konkurrenz{wpa:prozenthuerde3}
\item \konkurrenz{wpa:prozenthuerde4}
\item \konkurrenz{wpa:prozenthuerde5}
\item \konkurrenz{wpa:prozenthuerde6}
\end{itemize}

\paragraph{Antragstext}:

Der Landesparteitag möge folgenden Punkt an geeigneter Stelle im Wahlprogramm einfügen:

\einruecken{Die Piratenpartei Sachsen-Anhalt fordert die Herabsetzung der Sperrklausel in Sachsen-Anhalt bei Landtagswahlen auf 1\%.}

\begruendung{Zu mehr Bürgerbeteiligung und Mitbestimmung gehört, dass auch kleinere gewählte Fraktionen an den demokratischen Prozessen teilhaben dürfen. Der Nutzen wiegt dabei größer als die Gefahren, durch den Einzug radikaler Parteien, da die Mehrheit der Unter-5\%-Parteien nicht als radikal eingestuft werden kann.

Die staatliche Parteienfinanzierung aus Steuergeldern erhalten Parteien bereits mit {\Gu}mindestens 1,0 Prozent (Bundestags- oder Europawahl) bzw. 0,5 Prozent (Landtagswahlen) der gültigen Stimmen{\Go} (\href{http://www.bpb.de/themen/513F3I,0,Staatliche_Parteienfinanzierung.html}{Quelle}). 

\textit{siehe auch die \href{http://wiki.piratenpartei.de/LSA:Landesverband/Organisation/Mitgliederversammlung/2012.1/Antragsfabrik/Herabsetzung_der_5\%25-H\%C3\%BCrde_bzw._Sperrklausel_auf_Landesebene_auf_1\%25}{Pro- und Contra-Argumente auf der Antragsseite}}}

% -----

\wahlprogramm{Herabsetzung der 5\%-Hürde bzw. Sperrklausel auf Landesebene auf 2\%}\label{wpa:prozenthuerde3}
\antrag{Stephan Schurig}\\
\version{18:07, 9. Apr. 2012}
\begin{itemize}
\item \konkurrenz{wpa:prozenthuerde1}
\item \konkurrenz{wpa:prozenthuerde2}
\item \konkurrenz{wpa:prozenthuerde4}
\item \konkurrenz{wpa:prozenthuerde5}
\item \konkurrenz{wpa:prozenthuerde6}
\end{itemize}

\paragraph{Antragstext}:

Der Landesparteitag möge folgenden Punkt an geeigneter Stelle im Wahlprogramm einfügen:

\einruecken{Die Piratenpartei Sachsen-Anhalt fordert die Herabsetzung der Sperrklausel in Sachsen-Anhalt bei Landtagswahlen auf 2\%.}

\begruendung{Zu mehr Bürgerbeteiligung und Mitbestimmung gehört, dass auch kleinere gewählte Fraktionen an den demokratischen Prozessen teilhaben dürfen. Der Nutzen wiegt dabei größer als die Gefahren, durch den Einzug radikaler Parteien, da die Mehrheit der Unter-5\%-Parteien nicht als radikal eingestuft werden kann.

Die staatliche Parteienfinanzierung aus Steuergeldern erhalten Parteien bereits mit {\Gu}mindestens 1,0 Prozent (Bundestags- oder Europawahl) bzw. 0,5 Prozent (Landtagswahlen) der gültigen Stimmen{\Go} (\href{http://www.bpb.de/themen/513F3I,0,Staatliche_Parteienfinanzierung.html}{Quelle}). 

Dänemark besitzt auf nationaler Ebene lediglich eine Sperrklausel von 2\%. (\href{https://de.wikipedia.org/wiki/Sperrklausel}{Quelle})

\textit{siehe auch die \href{http://wiki.piratenpartei.de/LSA:Landesverband/Organisation/Mitgliederversammlung/2012.1/Antragsfabrik/Herabsetzung_der_5\%25-H\%C3\%BCrde_bzw._Sperrklausel_auf_Landesebene_auf_2\%25}{Pro- und Contra-Argumente auf der Antragsseite}}}

% -----

\wahlprogramm{Herabsetzung der 5\%-Hürde bzw. Sperrklausel auf Landesebene auf 3\%}\label{wpa:prozenthuerde4}
\antrag{Stephan Schurig}\\
\version{18:07, 9. Apr. 2012}
\begin{itemize}
\item \konkurrenz{wpa:prozenthuerde1}
\item \konkurrenz{wpa:prozenthuerde2}
\item \konkurrenz{wpa:prozenthuerde3}
\item \konkurrenz{wpa:prozenthuerde5}
\item \konkurrenz{wpa:prozenthuerde6}
\end{itemize}

\paragraph{Antragstext}:

Der Landesparteitag möge folgenden Punkt an geeigneter Stelle im Wahlprogramm einfügen:

\einruecken{Die Piratenpartei Sachsen-Anhalt fordert die Herabsetzung der Sperrklausel in Sachsen-Anhalt bei Landtagswahlen auf 3\%.}

\begruendung{Zu mehr Bürgerbeteiligung und Mitbestimmung gehört, dass auch kleinere gewählte Fraktionen an den demokratischen Prozessen teilhaben dürfen. Der Nutzen wiegt dabei größer als die Gefahren, durch den Einzug radikaler Parteien, da die Mehrheit der Unter-5\%-Parteien nicht als radikal eingestuft werden kann.

Die staatliche Parteienfinanzierung aus Steuergeldern erhalten Parteien bereits mit {\Gu}mindestens 1,0 Prozent (Bundestags- oder Europawahl) bzw. 0,5 Prozent (Landtagswahlen) der gültigen Stimmen{\Go} (\href{http://www.bpb.de/themen/513F3I,0,Staatliche_Parteienfinanzierung.html}{Quelle}). 

Dänemark besitzt auf nationaler Ebene lediglich eine Sperrklausel von 2\%. (\href{https://de.wikipedia.org/wiki/Sperrklausel}{Quelle}) 

3\% sind ca. 3 Sitze und gleichzeitig die Mindestgröße einer Fraktion im Landesparlament.

\textit{siehe auch die \href{http://wiki.piratenpartei.de/LSA:Landesverband/Organisation/Mitgliederversammlung/2012.1/Antragsfabrik/Herabsetzung_der_5\%25-H\%C3\%BCrde_bzw._Sperrklausel_auf_Landesebene_auf_3\%25}{Pro- und Contra-Argumente auf der Antragsseite}}}

% -----

\wahlprogramm{Herabsetzung der 5\%-Hürde bzw. Sperrklausel auf Landesebene auf 4\%}\label{wpa:prozenthuerde5}
\antrag{Stephan Schurig}\\
\version{18:10, 9. Apr. 2012}
\begin{itemize}
\item \konkurrenz{wpa:prozenthuerde1}
\item \konkurrenz{wpa:prozenthuerde2}
\item \konkurrenz{wpa:prozenthuerde3}
\item \konkurrenz{wpa:prozenthuerde4}
\item \konkurrenz{wpa:prozenthuerde6}
\end{itemize}

\paragraph{Antragstext}:

Der Landesparteitag möge folgenden Punkt an geeigneter Stelle im Wahlprogramm einfügen:

\einruecken{Die Piratenpartei Sachsen-Anhalt fordert die Herabsetzung der Sperrklausel in Sachsen-Anhalt bei Landtagswahlen auf 4\%.}

\begruendung{Zu mehr Bürgerbeteiligung und Mitbestimmung gehört, dass auch kleinere gewählte Fraktionen an den demokratischen Prozessen teilhaben dürfen. Der Nutzen wiegt dabei größer als die Gefahren, durch den Einzug radikaler Parteien, da die Mehrheit der Unter-5\%-Parteien nicht als radikal eingestuft werden kann.

Die staatliche Parteienfinanzierung aus Steuergeldern erhalten Parteien bereits mit {\Gu}mindestens 1,0 Prozent (Bundestags- oder Europawahl) bzw. 0,5 Prozent (Landtagswahlen) der gültigen Stimmen{\Go} (\href{http://www.bpb.de/themen/513F3I,0,Staatliche_Parteienfinanzierung.html}{Quelle}). 

4\% sind ca. 4 Sitze und gleichzeitig etwas mehr als die Mindestgröße einer Fraktion (3) im Landesparlament. 

\textit{siehe auch die \href{http://wiki.piratenpartei.de/LSA:Landesverband/Organisation/Mitgliederversammlung/2012.1/Antragsfabrik/Herabsetzung_der_5\%25-H\%C3\%BCrde_bzw._Sperrklausel_auf_Landesebene_auf_4\%25}{Pro- und Contra-Argumente auf der Antragsseite}}}

% -----

\wahlprogramm{Beibehalten der 5\%-Hürde bzw. Sperrklausel auf Landesebene}\label{wpa:prozenthuerde6}
\antrag{Stephan Schurig}\\
\version{18:11, 9. Apr. 2012}
\begin{itemize}
\item \konkurrenz{wpa:prozenthuerde1}
\item \konkurrenz{wpa:prozenthuerde2}
\item \konkurrenz{wpa:prozenthuerde3}
\item \konkurrenz{wpa:prozenthuerde4}
\item \konkurrenz{wpa:prozenthuerde5}
\end{itemize}

\paragraph{Antragstext}:

Der Landesparteitag möge folgenden Punkt an geeigneter Stelle im Wahlprogramm einfügen:

\einruecken{Die Piratenpartei Sachsen-Anhalt setzt sich dafür ein, die Sperrklausel bei Landtagswahlen in Sachsen-Anhalt von 5\% beizubehalten.}

\begruendung{Die momentane Regelung ist absolut ausreichend.

\textit{siehe auch die \href{http://wiki.piratenpartei.de/LSA:Landesverband/Organisation/Mitgliederversammlung/2012.1/Antragsfabrik/Beibehalten_der_5\%25-H\%C3\%BCrde_bzw._Sperrklausel_auf_Landesebene}{Pro- und Contra-Argumente auf der Antragsseite}}}

% -----

\wahlprogramm{Herabsetzung des aktiven Wahlalters bei Landtagswahlen auf 0 Jahre}\label{wpa:wahlalter1}
\antrag{Stephan Schurig}\\
\version{18:50, 12. Apr. 2012}
\begin{itemize}
\item \konkurrenz{wpa:wahlalter2}
\end{itemize}

\paragraph{Antragstext}:

Es wird beantragt ins Wahlprogramm folgende Forderung einzufügen:

\einruecken{Die Piratenpartei fordert die vollständige Aufhebung des notwendigen Mindestalters zur Wahrnehmung des aktiven Wahlrechts bei Landtagswahlenund damit eine Anpassung des § 42 Abs. 2 der Verfassung des Landes Sachsen-Anhalt. Das aktive Wahlrecht soll ab der Geburt von jedem Bürger wahrgenommen werden können. Die erstmalige Ausübung dieses Wahlrechts erfordert für Unter-16-Jährige die selbständige Eintragung in eine Wählerliste. Eine Stellvertreterwahl durch Erziehungsberechtigte lehnen wir ab.}

\begruendung{
Das Wahlrecht ist ein fundamentales Menschenrecht, kein freundlicherweise gewährtes Privileg. Dieses Recht ist in Artikel 21 der allgemein Erklärung der Menschenrechte verbrieft. Aus der Rechtsprechung des Bundesverfassungsgerichts geht hervor, dass Kinder ab ihrer Geburt zum Staatsvolk zählen und ihnen die Grund- und Bürgerrechte des Grundgesetzes in vollem Umfang zustehen. Einschränkungen dieser Grundrechte müssen sorgfältig begründet werden. Die Rechtsprechung des Verfassungsgerichts steht in dieser Hinsicht im Einklang mit der UN-Konvention für die Rechte des Kindes$^1$, der Gesetzgeber hinkt diesem Anspruch aber weiterhin hinterher. Für uns ist es nicht nachvollziehbar, warum es zum Schutz der Demokratie notwendig ist, Minderjährige von der Wahl auszuschließen und ihnen ihr Abstimmungsrecht zu nehmen. Im Gegenteil stellt ihre Beteiligung in unseren Augen eine Bereicherung dar. Vor diesem Hintergrund ist die Beschränkung des Wahlrechts in Art. 38 II GG auf Menschen über 18 Jahre nicht hinnehmbar.

Demokratie ist kein Instrument zur Wahrheitsfindung, sondern trägt der Idee Rechnung, dass wir nur dann Macht über Menschen ausüben dürfen, wenn sie darüber mitentscheiden und ihre eigenen Interessen in die Waagschale werfen durften, wer diese Macht wie ausübt. Der Gedanke, z.B. Menschen das Wahlrecht zu entziehen, die im Gespräch Beeinflussbarkeit oder politische Unkenntnis zeigen, erscheint uns daher unangemessen. Ebensowenig dürfen wir daher Kindern und Jugendlichen mit dem Argument, ihnen fehlte es noch an politischer Kenntnis oder sie seien zu beeinflussbar, das Wahlrecht vorenthalten: Dies gilt erstens nicht für alle (und zudem für viele Erwachsene), zweitens geht es bei Demokratie eben um die Berücksichtigung des Willens aller im gleichen Maße und nicht etwa darum, die {\Gu}politische Wahrheit{\Go} herauszufinden. Einen Willen und politische Interessen haben Kinder und Jugendliche aber sehr wohl$^2$. Eine Regierung, die von ca. 20\% derer, über deren Rechte und Pflichten sie bestimmen darf, nicht mitgewählt werden durfte, ist nicht demokratisch legitimiert.

Dass uns ein Kinderwahlrecht auf den ersten Blick merkwürdig vorkommt, ist unserer historischen Situation geschuldet und ging vielen Menschen bezüglich des Frauenwahlrechts einmal ebenso. Die Jungen Piraten behaupten von sich, unvoreingenommen neue Wege zu durchdenken und zu beschreiten, wenn die besten Argumente für sie sprechen. Das ist hier der Fall.

Die Grenzziehung zwischen Kind und Jugendlichen ist wissenschaftlich nicht einheitlich definiert$^3$. {\Gu}Kindheit{\Go} ist eine historische Konstruktion der gesellschaftlichen Verhältnisse während der Industrialisierung. Die Unterscheidung von Gesellschaftsmitgliedern nach ihrem Alter ist kein biologischer Diskurs, sondern ein Erziehungsdiskurs und hängt mit gesellschaftlichen Machtverhältnissen zusammen. Kinder werden nicht als Subjekte anerkannt, deren Interessen in der Gegenwart zu berücksichtigen sind, sondern nur im Hinblick auf ihre Zukunft und ihr Potential, zum vollwertigen Mitglied der Gesellschaft zu werden, betrachtet. Der gesellschaftliche Blick auf Kinder ist damit fast immer erwachsenenzentriert$^4$.

Bei der Bewertung des aktuellen Wahlrechts ab 18 - bzw. in einigen Fällen ab 16 Jahren - gilt es zu bedenken, dass alle Beschränkungen des Wahlrechts historische Relikte sind und eine Koppelung des Wahlrechts an die Volljährigkeit keinesfalls die einzig denkbare Möglichkeit ist. Die ersten {\Gu}Demokratien{\Go} schlossen Frauen, Nichtathener und Sklaven aus. Das Wahlrecht zur ersten Wahl im Deutschen Reich im Jahre 1871 besaßen lediglich Männer ab 25 Jahre, was zur damaligen Zeit den Ausschluss eines hohen Bevölkerungsanteils zur Folge hatte. Im Jahr 1970 wurde das aktive Wahlrecht in der Bundesrepublik Deutschland von 21 Jahren auf 18 Jahre abgesenkt. Das Wahlrecht ist historisch gewachsen und nicht an objektiven Kriterien festgemacht. Die Grenze von 18 Jahren ist willkürlich.

Wer wählen darf, interessiert sich mehr für Politik. Durch das fehlende Wahlrecht werden Kinder und Jugendliche zu spät an der demokratischen Kultur beteiligt und somit die Chance vertan, sie früh für Politik zu begeistern und einzubinden. Es ist daher wünschenswert, Kindern und Jugendlichen eine möglichst frühe Beteiligung an Wahlen zu ermöglichen. Politisches Desinteresse würde nicht mehr 18 Jahre eingeübt, stattdessen könnten sich Kinder und Jugendliche demokratisch einbringen, würden sich mehr informieren und es bestünden mehr Anreize, ihnen politische Informationsangebote zu machen. Die politische Bildung der Bevölkerung würde nachhaltig besser. Den durch eine Senkung des Wahlalters auftretenden politischen Fragen von Kindern und Jugendlichen ist auch durch ein stärkeres Gewicht der politischen Bildung im Schulalltag Rechnung zu tragen. NGOs wie z.B. die Greenpeace-Jugend ermöglichen eine Mitgliedschaft ab 14 Jahren, die Jugendfeuerwehr ab 10 Jahren und das Deutsche Jugendrotkreuz ab 6 Jahren. Bereits im Kindesalter werden Menschen also in gesellschaftlich verantwortungsvolle (zukünftige) Positionen einbezogen und begleitet. Es gibt bereits viele kommunale Beteiligungsprojekte mit Kindern und Jugendlichen, beispielsweise Bürgerhaushalte oder Projekte zur Gestaltung der eigenen Stadt bzw. Gemeinde$^5$. Österreich ermöglichte mit der Wahlrechtsreform 2007 allen Bürgerinnen und Bürgern bereits ab 16 Jahren eine Teilnahme an allen Wahlen im Land$^6$.

Die Nicht-Anerkennung von Kindern und Jugendlichen als politische Subjekte basiert auf mehreren Faktoren, die große Parallelen zum Ausschluss von Frauen aufzeigen$^1$:

\begin{itemize}
\item Kinder und Jugendliche sind im beruflichen Umfeld als Partner unbekannt und werden dadurch nicht akzeptiert, bzw. es fehlt die Erfahrung, mit ihnen umzugehen und sie in Entscheidungsprozesse einzubinden,
\item es herrscht ein Adultismus (analog zum Sexismus oder Rassismus), der aus der gesellschaftlichen Realität der Erwachsenenherrschaft hervorgeht,
\item Kinder und Jugendliche werden kaum als öffentliche Personen wahrgenommen und vornehmlich der Privatsphäre (Familie) zugeschrieben, mit der Ausnahme, wenn sie ein öffentliches Ärgernis darstellen,
\item Exklusion von der politischen Partizipation wird häufig als {\Gu}Schutz{\Go} vor sich selbst (z.B. wegen Empfänglichkeit für rassistische und totalitäre Positionen) oder Überforderung begründet.
\end{itemize}

Empfänglichkeit für Rassismus und Totalitarismus ist trotz landläufiger Meinung kein Phänomen, das nur unter Jugendlichen und jungen Erwachsenen auftritt. Andererseits kann politische Partizipation hier sogar präventiv wirken$^1$. Über 75\% aller Jugendlichen bezeichnen die Demokratie als geeignetste Staatsform. Sie sprechen sich für das Grundgesetz aus, sind aber mit der Realisierung demokratischer Ideale und Strukturen unzufrieden$^7$. Insgesamt sind die Ansprüche der Jugendlichen gegenüber der Politik hoch, so erwarten sie von Politikern Ehrlichkeit, Kompromissbereitschaft, Mitbestimmungsrechte, die Fähigkeit zur Durchsetzung politischer Entscheidungen und eine stärkere Einbindung der Interessen Jugendlicher$^3$. Nichtsdestotrotz bleiben viele Jugendliche gegenüber dem Parteiensystem skeptisch und Politikern gegenüber misstrauisch, was teilweise ihre generelle Zurückhaltung beim Wählen erklärt. So erklären beispielsweise 35-40\% aller Jugendlichen zwischen 12 und 17 Jahren in einer Umfrage, dass es keine Partei gebe, die ihre Interessen vertrete und sie deswegen auch nicht wählen gehen würden$^7$.

Ein häufig formulierter Einwand gegen die Absenkung oder Aufhebung des Wahlalters ist, vielen Kindern und Jugendlichen fehle die notwendige Reife. Man kann allerding nicht abstreiten, dass Kinder und Jugendliche bereits in der Lage sind, sich eigenständige Gedanken zu vielgestaltigen Problemen zu machen und ihre eigenen Wertungen zu finden. Es ist anmaßend, eine zwar womöglich mit geringer Lebenserfahrung getroffene, aber dennoch durchaus überlegte Entscheidung oder Wertung aus einem erwachsenen Blickwinkel per se als unreif zu deklarieren, zumal das Reifekriterium bei der Wahlentscheidung Erwachsener keine Rolle spielt. Selbst wenn eine Senkung des Wahlalters mitunter zu naiven und unsachgemäßen Entscheidungen führte - angenommen, eine objektive Bewertung wäre hier möglich - muss Kindern und Jugendlichen auch die Möglichkeit eingeräumt werden, Fehler zu machen und aus ihnen zu lernen. Eine Gefahr für die Demokratie wäre aus dieser Möglichkeit nicht abzuleiten, zumal die Unter-18-Jährigen nur einen geringen Teil der gesamten Wählerschaft ausmachen würden. Daher ist die Sorge über die Beschädigung der Demokratie durch massenhaft unreife Wähler unbegründet, zumal sie zu dem gewonnenen rechtlichen Gehör der Betroffenen in keinem Verhältnis stünde.

Teilhaberechte bedeuten immer auch, Macht abzugeben, in diesem Fall aus den Händen der Erwachsenen in die Hände junger Menschen. Der Ausschluss von Kindern und Jugendlichen vom Wahlrecht bedeutet nicht zuletzt, dass es keine Verpflichtung bzw. keine Verantwortlichkeit der politischen Akteure gibt, die Interessen dieser Altersgruppe zu berücksichtigen und sich vor ihr zu rechtfertigen. Artikel 20 GG formuliert, dass alle Staatsgewalt vom Volke ausgeht, Abgeordnete sollen nach Artikel 38 GG Vertreter des ganzen Volkes sein. In der Praxis stellt sich die Situation allerdings anders dar, wenn rund 15 Millionen Unter-18-Jährige keine Möglichkeit besitzen ihre Stimme abzugeben. Solange Kinder und Jugendliche nicht wählen können, werden ihre Interessen weniger berücksichtigt. Generationengerechtigkeit, Klimaschutz etc. können so schlecht erreicht werden und Probleme werden auf die junge Generation abgeschoben.

Die Absenkung des Wahlalters erfordert auch eine besondere Sorgfalt der Wahlämter und Wahlhelfer im Umgang mit den Jungwählern. Um einem potentiellen Mißbrauch vorzubeugen, müssen die zuständigen Sachbearbeiter entsprechend unterwiesen und vorbereitet werden. Eine Missbrauchsgefahr von Rechten besteht in einer Demokratie immer und unabhängig vom Alter, eine wehrhafte und wertstabile Demokratie ficht das aber nicht an.

Erstwähler, die unter 16 Jahre alt sind, müssen selbständig einmalig ihren Willen zu wählen persönlich in dem für Sie zuständigen Wahlamt beurkunden. Sobald sie als Wähler erfasst sind, erhalten sie zu jeder anstehenden Wahl, zu der sie wahlberechtigt sind, eine Einladung. Eine vollautomatische Erfassung aller Erstwähler unter 16 findet nicht statt. Wahlrecht ist keine Wahlpflicht. Dieses Recht wahrzunehmen, ist die Entscheidung des einzelnen Wählers, der damit auch eine Verantwortung übernimmt.

Es ist jedoch klar, dass allein die Herabsetzung des aktiven Wahlrechts nur ein kleines Glied in einer ganzen Kette von Maßnahmen sein kann, um Jugendliche politisch zu involvieren, ihnen damit die Chance zu geben ihre und unsere Gesellschaft von heute und von morgen zu gestalten. Diese Forderung kann damit lediglich als Anfang einer deutlichen Wendung in der Politik dienen. Kinder und Jugendliche brauchen mehr Begleitung und Ansprechpartner als Erwachsene, um ihre Interessen in politisches Wissen zu transformieren und dieses schließlich für politische Partizipation zu verwenden. Dabei müssen auch politische Diskussionen in Schulen geführt werden, demokratische Mitbestimmungsrechte an Schulen ausgebaut werden und Kinder und Jugendliche in allen Lebensbereichen die Chance erhalten, ihre Lebenswelt fair und ihrem Alter entsprechend zu gestalten.

\textbf{Quellen:}

$^1$ Bertelsmann Stiftung (Hrsg.) (2007): Mehr Partizipation wagen. Argumente für eine verstärkte Beteiligung von Kindern und Jugendlichen. 2. Aufl., Gütersloh.

$^2$ van Deth, J. W., Abendschön, S., Rathke, J. \& M. Vollmar (2007): Kinder und Politik. Politische Einstellungen von jungen Kindern im ersten Grundschuljahr. Wiesbaden.

$^3$ Maßlo, J. (2010): Jugendliche in der Politik. Chancen und Probleme einer institutionalisierten Jugendbeteiligung am Beispiel des Kinder- und Jugendbeirats der Stadt Reinbek. Wiesbaden.

$^4$ Abels, H. \& A. König (2010): Sozialisation. Soziologische Antworten auf die Frage, wie wir werden, was wir sind, wie gesellschaftliche Ordnung möglich ist und wie Theorien der Gesellschaft und der Identität ineinander spielen. Wiesbaden.

$^5$ Gernbauer, K. (2008): Geleitwort. Beteiligung von Jugendlichen als politische Herausforderung. In: Ködelpeter, T. \& U. Nitschke (Hrsg.): Jugendliche planen und gestalten Lebenswelten. Partizipation als Antwort auf den gesellschaftlichen Wandel. Wiesbaden.

$^6$ {\Gu}Parlamentskorrespondenz Nr. 510 vom 21.06.2007. \href{http://www.parlinkom.gv.at/PAKT/PR/JAHR_2007/PK0510/index.shtml}{Wahlrechtsreform 2007 passiert den Bundesrat}{\Go} (Abruf am 22.01.2012)

$^7$ Hurrelmann, K. (o.J.): \href{http://gedankensex.de/2011/08/23/jugendliche-an-die-wahlurnen/}{Jugendliche an die Wahlurnen} (Abruf am 22.01.2012)
}

% -----

\wahlprogramm{Herabsetzung des aktiven Wahlalters bei Landtagswahlen auf 12 Jahre}\label{wpa:wahlalter2}
\antrag{Stephan Schurig}\\
\version{18:50, 12. Apr. 2012}
\begin{itemize}
\item \konkurrenz{wpa:wahlalter1}
\end{itemize}

\paragraph{Antragstext}:

Es wird beantragt ins Wahlprogramm folgende Forderung einzufügen:

\einruecken{Die Piratenpartei fordert die Senkung des notwendigen Alters zur Wahrnehmung des aktiven Wahlrechts bei Landtagswahlen auf 12 Jahre und damit eine Anpassung des § 42 Abs. 2 der Verfassung des Landes Sachsen-Anhalt. Die erstmalige Ausübung dieses Wahlrechts erfordert für Unter-16-Jährige die selbständige Eintragung in eine Wählerliste. Eine Stellvertreterwahl durch Erziehungsberechtigte lehnen wir ab.}

\begruendung{
Das Wahlrecht ist in Art. 21 der allgemeinen Erklärung der Menschenrechte als Grundrecht verankert. Nicht die Teilhabe an diesem Recht muss begründet werden, sondern im Gegenteil die Exklusion von Personengruppen von diesem Recht$^1$. Eine Exklusion findet durch die Nicht-Anerkennung von Kindern als Bürger statt, womit sie lediglich passive Empfänger von gesellschaftspolitischen Entscheidungen werden, ohne die Möglichkeit zu haben sie aktiv mitzugestalten. Kinder und Jugendliche werden zwar als Bürger von morgen wahrgenommen, nicht aber als Bürger von heute, die auch heute partizipieren wollen. Das Wahlalter stellt eine Einschränkung der Bürgerrechte von Kindern und Jugendlichen dar und muss deshalb sehr sorgfältig begründet werden. Der Ausschluss der 12 bis 17jährigen vom Wahlrecht auf Basis von Annahmen über deren fehlende Fähigkeit, politische Zusammenhänge zu verstehen, hält aber einer empirischen Überprüfung nicht stand.

Entwicklungspsychologisch gesehen sind Kinder bereits ab einem Alter von 5 Jahren fähig, Präferenzen zu bekunden$^2$. Mit dem Eintritt in die Schule haben sie bereits ein politisches Grundverständnis und -wissen, eine konsistente Werteorientierung und eine Reflexions- und Argumentationskompetenz, die sich innerhalb des ersten Schuljahres zusehends verstärken. So besitzen sie bereits ein starkes Bewusstsein für globale Themen wie Hunger, Arbeitslosigkeit, Umweltverschmutzung, Migration oder Krieg, wenngleich sie diese noch selten in einen Zusammenhang bringen können$^3$. Ebenso entwickelt sich in dieser Zeit das logische Denken bzw. die Fähigkeit logischer Schlussfolgerungen$^4$. {\Gu}Empirische Untersuchungen belegen, dass Jugendliche bereits etwa ab dem 15. Lebensjahr in der Lage sind, formal-logische Denkoperationen durchzuführen. Dies ist die höchste Stufe der kognitiven Entwicklung, auch Erwachsene erreichen also in Bezug auf diese Dimension kein höheres Niveau{\Go}$^5$.

Ab einem Alter von etwa 13 bis 14 Jahren geben Jugendliche den gleichen Grad an politischem Interesse an wie die Altersgruppe der 18 bis 25jährigen$^6$. Jugendliche sind heutzutage außerdem deutlich selbständiger als noch vor einigen Jahrzehnten. Die Emanzipation vom Lebensstil der Eltern setzt meist schon im Alter von 12 bis 13 Jahren ein$^6$. Im Alter von 12 bis 14 Jahren entwickeln viele Jugendliche die nötige Urteilskraft, um abstrakte Probleme zu verstehen, ethische Positionen zu entwickeln und verantwortungsvolle Entscheidungen zu treffen$^6$. Aus anthropologischer Sicht sind Kinder im Alter zwischen 7 bis 12 Jahren so weit entwickelt, dass ihr Überleben nicht mehr von anderen abhängig ist$^7$. Kinder sind bereits mit 7 Jahren beschränkt geschäftsfähig und mit 12 Jahren beschränkt bzw. mit 14 Jahren vollständig religionsmündig$^8$. Im Alter von 13 Jahren dürfen Jugendliche Beschäftigungen in geringem Umfang verrichten$^9$ und ab dem 15. Lebensjahr erlischt das allgemeine Beschäftigungsverbot$^{10}$ sowie die allgemeine Schulpflicht. Bereits jeder dritte Jugendliche besitzt im Alter von 13 Jahren ein eigenes Bankkonto mit Karten-Verfügungsrecht$^6$.

Diese Erkenntnisse aus der Forschung zeigen, dass die meisten Fähigkeiten, die für verantwortungsvolle politische Entscheidungen notwendig sind, bereits lange vor dem 18. Geburtstag entwickelt werden. Um einen möglichst großen Teil der Bevölkerung am Wahlrecht teilhaben zu lassen, ist eine Absenkung des Wahlalters auf 12 Jahre sinnvoll, womit die ganz überwiegende Zahl der Kinder und Jugendlichen, die die Fähigkeiten und das Interesse zur Teilnahme an Wahlen besitzen, eingeschlossen wären. Ob das Wahlalter allerdings überhaupt an ein persönliches Reifekriterium gebunden werden kann oder muss, sollte jedoch weiterhin kritisch hinterfragt werden, da dies in Bezug auf eine ältere Bevölkerungsschicht auch kein Kriterium darstellt.

Bei der Bewertung des aktuellen Wahlrechts ab 18 - bzw. in einigen Fällen ab 16 Jahren - gilt es zu bedenken, dass alle Beschränkungen des Wahlrechts historische Relikte sind und eine Koppelung des Wahlrechts an die Volljährigkeit keinesfalls die einzig denkbare Möglichkeit ist. Die ersten {\Gu}Demokratien{\Go} schlossen Frauen, Nichtathener und Sklaven aus. Das Wahlrecht zur ersten Wahl im Deutschen Reich im Jahre 1871 besaßen lediglich Männer ab 25 Jahre, was zur damaligen Zeit den Ausschluss eines hohen Bevölkerungsanteils zur Folge hatte. Im Jahr 1970 wurde das aktive Wahlrecht in der Bundesrepublik Deutschland von 21 Jahren auf 18 Jahre abgesenkt. Das Wahlrecht ist historisch gewachsen und nicht an objektiven Kriterien festgemacht. Die Grenze von 18 Jahren ist willkürlich.

Wer wählen darf, interessiert sich mehr für Politik. Durch das fehlende Wahlrecht werden Kinder und Jugendliche zu spät an der demokratischen Kultur beteiligt und somit die Chance vertan, sie früh für Politik zu begeistern und einzubinden. Es ist daher wünschenswert, Kindern und Jugendlichen eine möglichst frühe Beteiligung an Wahlen zu ermöglichen. Politisches Desinteresse würde nicht mehr 18 Jahre eingeübt, stattdessen könnten sich Kinder und Jugendliche demokratisch einbringen, würden sich mehr informieren und es bestünden mehr Anreize, ihnen politische Informationsangebote zu machen. Die politische Bildung der Bevölkerung würde nachhaltig besser. Den durch eine Senkung des Wahlalters auftretenden politischen Fragen von Kindern und Jugendlichen ist auch durch ein stärkeres Gewicht der politischen Bildung im Schulalltag Rechnung zu tragen.

NGOs wie z.B. die Greenpeace-Jugend ermöglichen eine Mitgliedschaft ab 14 Jahren, die Jugendfeuerwehr ab 10 Jahren und das Deutsche Jugendrotkreuz ab 6 Jahren. Bereits im Kindesalter werden Menschen also in gesellschaftlich verantwortungsvolle (zukünftige) Positionen einbezogen und begleitet. Es gibt bereits viele kommunale Beteiligungsprojekte mit Kindern und Jugendlichen, beispielsweise Bürgerhaushalte oder Projekte zur Gestaltung der eigenen Stadt bzw. Gemeinde$^{11}$. Österreich ermöglichte mit der Wahlrechtsreform 2007 allen Bürgerinnen und Bürgern bereits ab 16 Jahren eine Teilnahme an allen Wahlen im Land$^{12}$.

Die Nicht-Anerkennung von Kindern und Jugendlichen als politische Subjekte basiert auf mehreren Faktoren, die große Parallelen zum Ausschluss von Frauen aufzeigen$^1$:

\begin{itemize}
\item Kinder und Jugendliche sind im beruflichen Umfeld als Partner unbekannt und werden dadurch nicht akzeptiert, bzw. es fehlt die Erfahrung, mit ihnen umzugehen und sie in Entscheidungsprozesse einzubinden,
\item es herrscht ein Adultismus (analog zum Sexismus oder Rassismus), der aus der gesellschaftlichen Realität der Erwachsenenherrschaft hervorgeht,
\item Kinder und Jugendliche werden kaum als öffentliche Personen wahrgenommen und vornehmlich der Privatsphäre (Familie) zugeschrieben, mit der Ausnahme, wenn sie ein öffentliches Ärgernis darstellen,
\item Exklusion von der politischen Partizipation wird häufig als {\Gu}Schutz{\Go} vor sich selbst (z.B. wegen Empfänglichkeit für rassistische und totalitäre Positionen) oder Überforderung begründet.
\end{itemize}

Empfänglichkeit für Rassismus und Totalitarismus ist trotz landläufiger Meinung kein Phänomen, das nur unter Jugendlichen und jungen Erwachsenen auftritt. Andererseits kann politische Partizipation hier sogar präventiv wirken$^1$. Über 75\% aller Jugendlichen bezeichnen die Demokratie als geeignetste Staatsform. Sie sprechen sich für das Grundgesetz aus, sind aber mit der Realisierung demokratischer Ideale und Strukturen unzufrieden$^6$. Insgesamt sind die Ansprüche der Jugendlichen gegenüber der Politik hoch, so erwarten sie von Politikern Ehrlichkeit, Kompromissbereitschaft, Mitbestimmungsrechte, die Fähigkeit zur Durchsetzung politischer Entscheidungen und eine stärkere Einbindung der Interessen Jugendlicher$^{13}$. Nichtsdestotrotz bleiben viele Jugendliche gegenüber dem Parteiensystem skeptisch und Politikern gegenüber misstrauisch, was teilweise ihre generelle Zurückhaltung beim Wählen erklärt. So erklären beispielsweise 35-40\% aller Jugendlichen zwischen 12 und 17 Jahren in einer Umfrage, dass es keine Partei gebe, die ihre Interessen vertrete und sie deswegen auch nicht wählen gehen würden$^6$.

Ein häufig formulierter Einwand gegen eine Absenkung des Wahlalters ist, vielen Kindern und Jugendlichen fehle die notwendige Reife. Man kann allerding nicht abstreiten, dass Kinder und Jugendliche bereits in der Lage sind, sich eigenständige Gedanken zu vielgestaltigen Problemen zu machen und ihre eigenen Wertungen zu finden. Es ist anmaßend, eine zwar womöglich mit geringer Lebenserfahrung getroffene, aber dennoch durchaus überlegte Entscheidung oder Wertung aus einem erwachsenen Blickwinkel per se als unreif zu deklarieren, zumal das Reifekriterium bei der Wahlentscheidung Erwachsener keine Rolle spielt. Selbst wenn eine Senkung des Wahlalters mitunter zu naiven und unsachgemäßen Entscheidungen führte - angenommen, eine objektive Bewertung wäre hier möglich - muss Kindern und Jugendlichen auch die Möglichkeit eingeräumt werden, Fehler zu machen und aus ihnen zu lernen. Eine Gefahr für die Demokratie wäre aus dieser Möglichkeit nicht abzuleiten, zumal die 12- bis 18-Jährigen nur einen geringen Teil der gesamten Wählerschaft ausmachen würden. Daher ist die Sorge über die Beschädigung der Demokratie durch massenhaft unreife Wähler unbegründet, zumal sie zu dem gewonnenen rechtlichen Gehör der Betroffenen in keinem Verhältnis stünde.

Teilhaberechte bedeuten immer auch, Macht abzugeben, in diesem Fall aus den Händen der Erwachsenen in die Hände junger Menschen. Der Ausschluss von Kindern und Jugendlichen vom Wahlrecht bedeutet nicht zuletzt, dass es keine Verpflichtung bzw. keine Verantwortlichkeit der politischen Akteure gibt, die Interessen dieser Altersgruppe zu berücksichtigen und sich vor ihr zu rechtfertigen. Artikel 20 GG formuliert, dass alle Staatsgewalt vom Volke ausgeht, Abgeordnete sollen nach Artikel 38 GG Vertreter des ganzen Volkes sein. In der Praxis stellt sich die Situation allerdings anders dar, wenn rund 15 Millionen Unter-18-Jährige keine Möglichkeit besitzen ihre Stimme abzugeben. Solange Kinder und Jugendliche nicht wählen können, werden ihre Interessen weniger berücksichtigt. Generationengerechtigkeit, Klimaschutz etc. können so schlecht erreicht werden und Probleme werden auf die junge Generation abgeschoben.

Die Absenkung des Wahlalters erfordert auch eine besondere Sorgfalt der Wahlämter und Wahlhelfer im Umgang mit den Jungwählern. Um einem potentiellen Mißbrauch vorzubeugen, müssen die zuständigen Sachbearbeiter entsprechend unterwiesen und vorbereitet werden. Eine Missbrauchsgefahr von Rechten besteht in einer Demokratie immer und unabhängig vom Alter, eine wehrhafte und wertstabile Demokratie ficht das aber nicht an.

Erstwähler, die unter 16 Jahre alt sind, müssen selbständig einmalig ihren Willen zu wählen persönlich in dem für Sie zuständigen Wahlamt beurkunden. Sobald sie als Wähler erfasst sind, erhalten sie zu jeder anstehenden Wahl, zu der sie wahlberechtigt sind, eine Einladung. Eine vollautomatische Erfassung aller Erstwähler unter 16 findet nicht statt. Wahlrecht ist keine Wahlpflicht. Dieses Recht wahrzunehmen, ist die Entscheidung des einzelnen Wählers, der damit auch eine Verantwortung übernimmt.

Es ist jedoch klar, dass allein die Herabsetzung des aktiven Wahlrechts nur ein kleines Glied in einer ganzen Kette von Maßnahmen sein kann, um Jugendliche politisch zu involvieren, ihnen damit die Chance zu geben ihre und unsere Gesellschaft von heute und von morgen zu gestalten. Diese Forderung kann damit lediglich als Anfang einer deutlichen Wendung in der Politik dienen. Kinder und Jugendliche brauchen mehr Begleitung und Ansprechpartner als Erwachsene, um ihre Interessen in politisches Wissen zu transformieren und dieses schließlich für politische Partizipation zu verwenden. Dabei müssen auch politische Diskussionen in Schulen geführt werden, demokratische Mitbestimmungsrechte an Schulen ausgebaut werden und Kinder und Jugendliche in allen Lebensbereichen die Chance erhalten, ihre Lebenswelt fair und ihrem Alter entsprechend zu gestalten.

\textbf{Quellen:}

$^1$ Bertelsmann Stiftung (Hrsg.) (2007): Mehr Partizipation wagen. Argumente für eine verstärkte Beteiligung von Kindern und Jugendlichen. 2. Aufl., Gütersloh.

$^2$ Tremmel, J. (2008): Die Ausprägung des Wahlwillens und der Wahlfähigkeit aus entwicklungspsychologischer Sicht. In: Stiftung für die Rechte zukünftiger Generationen (Hrsg.): Wahlrecht ohne Altersgrenze? Verfassungsrechtliche, demokratietheoretische und entwicklungspsychologische Aspekte. München.

$^3$ van Deth, J. W., Abendschön, S., Rathke, J. \& M. Vollmar (2007): Kinder und Politik. Politische Einstellungen von jungen Kindern im ersten Grundschuljahr. Wiesbaden.

$^4$ Swiderek, T. (2003): Kinderpolitik und Partizipation von Kindern. In: Arbeit - Technik - Organisation - Soziales. Band 22. Frankfurt am Main.

$^5$ Hoepner-Stamos zit. in Deutscher Bundesjugendring, Landesjugendring Baden-Württemberg und Bayerischer Jugendring (Hrsg.): \href{http://www.waehlen-ab-14.de/wahlalter-14/argumentationshilfen.php}{Wählen ab 14.} (Abruf am 22.01.2012)

$^6$ Hurrelmann, K. (o.J.): \href{http://gedankensex.de/2011/08/23/jugendliche-an-die-wahlurnen/}{Jugendliche an die Wahlurnen!} (Abruf am 22.01.2012)

$^7$ Bogin, B. (1999): Patterns of Human Growth. 2nd ed. In: Cambridge Studies in Biological and Evolutionary Anthropology 23. Cambridge.

$^8$ \href{http://www.gesetze-im-internet.de/kerzg/__5.html}{Gesetz über die religiöse Kindererziehung § 5} (Abruf am 22.01.2012)

$^9$ \href{http://www.gesetze-im-internet.de/kindarbschv/__2.html}{Kinderarbeitsschutzverordnung § 2} (Abruf am 22.01.2012)

$^{10}$ \href{http://bundesrecht.juris.de/jarbschg/__5.html}{Jugendarbeitsschutzgesetz § 5} (Abruf am 22.01.2012)

$^{11}$ Gernbauer, K. (2008): Geleitwort. Beteiligung von Jugendlichen als politische Herausforderung. In: Ködelpeter, T. \& U. Nitschke (Hrsg.): Jugendliche planen und gestalten Lebenswelten. Partizipation als Antwort auf den gesellschaftlichen Wandel. Wiesbaden.

$^{12}$ Parlamentskorrespondenz Nr. 510 vom 21.06.2007. \href{http://www.parlinkom.gv.at/PAKT/PR/JAHR_2007/PK0510/index.shtml}{Wahlrechtsreform 2007 passiert den Bundesrat} (Abruf am 22.01.2012)

$^{13}$ Maßlo, J. (2010): Jugendliche in der Politik. Chancen und Probleme einer institutionalisierten Jugendbeteiligung am Beispiel des Kinder- und Jugendbeirats der Stadt Reinbek. Wiesbaden.
}

% -----

\wahlprogramm{Aufhebung von §5 FeiertG LSA (Tanzverbot u.a. an Feiertagen)}
\antrag{Stephan Schurig}\\
\version{18:15, 9. Apr. 2012}

\paragraph{Antragstext}:

Der Landesparteitag möge beschließen, folgenden Abschnitt an geeigneter Stelle in das Wahlprogramm aufzunehmen:

\einruecken{\textbf{Aufhebung des §5 FeiertG LSA}

Die Piratenpartei Sachsen-Anhalt strebt die Aufhebung des §5 Gesetz über die Sonn- und Feiertage(FeiertG LSA) an. Die Trennung von Religion und Staat bzw. die Selbstbestimmung des Individuums ist höher zu bewerten, als der erhöhte Schutz religiöser Bräuche. Durch Beibehalten von §4 bleibt der besondere Schutz von Gottesdiensten jedoch bestehen.}

\begruendung{In \href{http://st.juris.de/st/FeiertG_ST_P5.htm}{§5 FeiertG} ist festgelegt, dass an speziellen christlichen Feiertagen neben den Einschränkungen nach §4 zusätzlich untersagt sind:

\begin{enumerate}
\item Veranstaltungen in Räumen mit Schankbetrieb, die über den Schank- und Speisebetrieb hinausgehen,
\item öffentliche sportliche Veranstaltungen sowie
\item alle sonstigen öffentlichen Veranstaltungen, außer wenn sie der Würdigung des Feiertages oder der Kunst, Wissenschaft oder Volksbildung dienen und auf den Charakter des Tages Rücksicht nehmen.
\end{enumerate}

Dieser Abschnitt ist zu streichen, da er eine Einschränkung insbesondere für alle Nicht-Christen darstellt. Im Sinne der Trennung von Kirche und Staat ist das Gesetz nicht mehr zeitgemäßg. Christen können allerdings weiterhin ihrem Glauben und Gottesdiensten uneingeschränkt nachgehen, da der §4 bestehen bleibt, welcher sicherstellt, dass keine Veranstaltungen erlaubt sind, die einen Gottesdienst stören.}

% -----

\wahlprogramm{Flächendeckendes barrierefreies Notruf- und Informationssystem per Mobilfunk (SMS-Notruf)}\label{wpa:smsnotruf1}
\antrag{Stephan Schurig}\\
\version{18:50, 12. Apr. 2012}
\begin{itemize}
\item \konkurrenz{wpa:smsnotruf2}
\end{itemize}

\paragraph{Antragstext}:

Der Landesparteitag möge beschließen, in das Wahlprogramm für die kommende Landtagswahl an geeigneter Stelle aufzunehmen:

\einruecken{Die Piratenpartei Sachsen-Anhalt setzt sich für die zeitnahe Einführung eines flächendeckenden barrierefreien Notruf- und Informationssystem per Mobilfunk in Sachsen-Anhalt ein. Weiterhin unterstützen wir nach Möglichkeit alle Bemühungen für eine bundesweite Umsetzung.}

\begruendung{
Bis heute gibt es in Deutschland keine Möglichkeit in einer Notfallallsituation barrierefrei einen Notruf abzusenden. Besonders behinderte Menschen sind davon betroffen, aber auch wenn ein Handyakku nicht mehr für einen Notruf per Sprache ausreicht, ist ein non-verbaler Notruf notwendig. Ein bundesweiter barrierefreier non-verbaler Notruf für Polizei, Feuerwehr oder Krankenwagen existiert derzeit nicht.

Bisher gibt es lediglich in Berlin und in Köln spezielle Notrufnummern für SMS, die technisch gesehen relativ problemlos auf ganz Deutschland ausweiten werden könnte. Österreich hat bereits sehr gute Erfahrungen mit einer SMS-Notrufnummer für Gehörlose gemacht, die von allen großen Netzbetreibern unterstützt wird.

Es gibt derzeit zwar eine Notfall-Fax-Einrichtung (z.B. in Münster) für Gehörlose und Schwerhörige Menschen, aber dieser ist höchst umständlich und erfordert ein Faxgerät, welches heutzutage nur noch wenige Menschen besitzen oder bei vielen Notfallsituationen nicht verfügbar ist. Ein weiterer Kritikpunkt ist, dass Fax-Notrufe nicht immer so ernst genommen werden, wie es vonnöten wäre oder erst viele Stunden später Hilfe eintrifft und es dann möglicherweise schon zu spät ist.

Länder in denen schon ein Notruf per SMS möglich ist sind:

\begin{itemize}
\item England (\url{http://www.emergencysms.org.uk})
\item USA
\item Australien (\url{http://www.ewn.com.au} - \url{http://www.emergencyalert.gov.au})
\item Irland (\url{http://www.112.ie/Pilot_112_SMS_Service/142})
\item Singapore (\url{http://www.spf.gov.sg/sms70999/index.html})
\item Portugal (\url{http://ec.europa.eu/information_society/activities/112/ms/pt/index_en.htm})
\item Schweden (\url{http://www.sosalarm.se/Documents/Nyheter\%20och\%20Media/Bibliotek/Broschyrer/2010/SMS\%20112\%20Systembeskrivning_EN\%20_2_.pdf})
\item Finnland
\item Island
\item Norwegen (\url{http://www.kokom.no/kokomsoek/publikasjonar/Rapportar/SMS_in_EC_2009.pdf}) 
\end{itemize}

Im Falle einer Katastrophe in Deutschland wird die Bevölkerung per Sirene und Lautsprecherdurchsagen informiert, das Radio und/oder TV einzuschalten um weitere Informationen zu erhalten. Die Bevölkerung wird dazu angehalten ihre Nachbarn und vor allem Hilfsbedürftige zu informieren. Hörbehinderte Menschen bekommen davon jedoch nicht sofort etwas mit und sind somit auf die Aufmerksamkeit ihrer Mitmenschen angewiesen. Dies ist ein unhaltbarer Zustand.

Käme es noch zu einer Evakuierung denken Menschen erstmal an sich, ihre Familie, Freunde. Behinderte Mitmenschen wohl im seltesten Fall. Auch wenn der Katastrophenschutz dazu aufruft.

Übernommen von der AG Barrierefreiheit (siehe \href{https://ag_barrierefreiheit.piratenpad.de/4}{Pad})

Basierend auf der Hörscreening-Studie von Sohn (2) ergibt sich als fundierte Schätzung:

Von den 13,3, Mill. Hörgeschädigten in Deutschland sind demnach

\begin{itemize}
\item leichtgradig schwerhörig 56,5\% = 7,51 Mill.
\item mittelgradig schwerhörig 35,2\% = 4,68 Mill.
\item hochgradig schwerhörig 7,2\% = 958 000
\item an Taubheit grenzend schwerhörig 1,6\% = 213 000
\end{itemize}

Ca. 80.000 Menschen sind von Geburt an taub/gehörlos oder in früher Kindheit ertaubt sind. Übertragen auf Sachsen-Anhalt dürften dies bei 0,1\% immerhin über 2.200 Personen betreffen.
}

% -----

\wahlprogramm{Verbandsklagerecht}
\antrag{Alexander Magnus}\\
\version{18:21, 9. Apr. 2012}

\paragraph{Antragstext}:

Wir setzen uns für die Einführung eines Verbandsklagerechtes für anerkannte Tierschutzorganisationen im Sachsen-Anhalt ein. Tiere können als Lebewesen nicht selbst für ihre Rechte eintreten bzw. diese verteidigen. Daher sind sie auf Vertreter in Form von Verbänden angewiesen. Obwohl Tier- und Umweltschutz nach Art. 20a GG denselben Verfassungsrang haben, werden die beiden Staatsziele ungleich behandelt, wenn es um das Verbandsklagerecht geht. Erfahrungen in Bremen, wo es die Tierschutzverbandsklage inzwischen gibt, zeigen zudem, dass die von den Gegnern der Verbandsklage befürchtete Klageflut ausgeblieben ist. Da auf Bundesebene keine Lösung in Sicht ist, ist die Einführung des Verbandsklagerechts auf Landesebene geboten.

\begruendung{Aus dem Antragsportal LTW2012 des Saarlandes übernommen}

% -----

\wahlprogramm{Mehr Polizeibeamte, weniger Überwachung}
\antrag{Alexander Magnus}\\
\version{18:23, 9. Apr. 2012}

\paragraph{Antragstext}:

Statt den Bürgern Sicherheit durch mehr Überwachungsmaßnahmen vorzuspiegeln, sollten die Gelder dafür in die Beschäftigung von mehr Polizeibeamten investiert werden. Eine Kamera kann - sofern sie überhaupt von einem Beamten überwacht wird - keine Hilfe leisten oder herbeirufen. Ein vor Ort patrouillierender Polizei erhöht die subjektive und die tatsächliche Sicherheit, er kennt die Bewohner {\Gu}seines{\Go} Stadtteiles und kann, noch vor der Notwendigkeit von Sanktionen, auf Mitglieder der Gesellschaft einwirken, die auf die schiefe Bahn zu geraten drohen.

Allerdings lehnen wir einen Polizeistaat ab. Mehr Personal sollte lediglich in problematischen Regionen, Orten bzw. Plätzen bereit gestellt werden, oder dort, wo laufende Ermittlungen durch mangelndes Personal behindert oder gar unmöglich gemacht werden.

% -----

\wahlprogramm{Verbesserte Ausstattung der Polizei}
\antrag{Alexander Magnus}\\
\version{18:24, 9. Apr. 2012}

\paragraph{Antragstext}:

Um der Polizei die Erfüllung ihrer Aufgaben in einem vernünftigen Maße zu ermöglichen, muss die materielle und personelle Ausstattung verbessert werden. Die Anschaffung von Ausrüstung wie z. B. Schutzwesten darf nicht dem einzelnen Polizisten aufgebürdet werden. Gleichzeitig müssen ausreichend Beamte beschäftigt werden, um die Polizeiarbeit angemessen bewältigen zu können. 

\begruendung{Eine entsprechende Präsenz einer gut ausgerüsteten Polizei auf unseren Straßen erhöht die Sicherheit des Einzelnen weit mehr als jede Videoüberwachung.

Quelle: \href{http://www.piratenpartei-bw.de/wahl/wahlprogramm/inneres-und-justiz/}{Wahlprogramm BW} und \href{http://wiki.piratenpartei.de/SH:Landtagswahl_2012/Wahlprogramm\#Inneres_und_Justiz}{Wahlprogramm SH}}

% -----

\wahlprogramm{Mehr und besser ausgestatte Polizeibeamte statt mehr Überwachung}
\antrag{Alexander Magnus}\\
\version{18:25, 9. Apr. 2012}

\paragraph{Antragstext}:

Statt den Bürgern Sicherheit durch mehr Überwachungsmaßnahmen vorzuspiegeln, sollten die Gelder dafür in die Beschäftigung von mehr Polizeibeamten investiert werden. Eine Kamera kann - sofern sie überhaupt von einem Beamten überwacht wird - keine Hilfe leisten oder herbeirufen. Ein vor Ort patrouillierender Polizist erhöht die subjektive und die tatsächliche Sicherheit, er kennt die Bewohner {\Gu}seines{\Go} Stadtteiles und kann, noch vor der Notwendigkeit von Sanktionen, auf Mitglieder der Gesellschaft einwirken, die auf die schiefe Bahn zu geraten drohen.

Allerdings lehnen wir einen Polizeistaat ab. Mehr Personal sollte lediglich in problematischen Regionen, Orten bzw. Plätzen bereit gestellt werden, oder dort, wo laufende Ermittlungen durch mangelndes Personal behindert oder gar unmöglich gemacht werden.

Um der Polizei die Erfüllung ihrer Aufgaben in einem vernünftigen Maße zu ermöglichen, muss die materielle und personelle Ausstattung verbessert werden. Die Anschaffung von Ausrüstung wie z. B. Schutzwesten darf nicht dem einzelnen Polizisten aufgebürdet werden. 

\begruendung{Zusammenlegung von {\Gu}Mehr Polizeibeamte, weniger Überwachung{\Go} und {\Gu}Verbesserte Ausstattung der Polizei{\Go} Begründung des Antrages zweite Zeile etc.}

% -----

\wahlprogramm{Flächendeckendes barrierefreies Notruf- und Informationssystem per Mobilfunk (SMS-Notruf) - Zielgruppe präzisiert}\label{wpa:smsnotruf2}
\antrag{Stephan Schurig}\\
\version{18:50, 12. Apr. 2012}
\begin{itemize}
\item \konkurrenz{wpa:smsnotruf1}
\end{itemize}

\paragraph{Antragstext}:

Der Landesparteitag möge beschließen, in das Wahlprogramm für die kommende Landtagswahl an geeigneter Stelle aufzunehmen: 

\einruecken{Die Piratenpartei Sachsen-Anhalt setzt sich für die zeitnahe Einführung eines flächendeckenden barrierefreien Notruf- und Informationssystem per Mobilfunk in Sachsen-Anhalt ein. Davon profitieren insbesondere gehörlose und schwerhörige Menschen in Gefahrensituationen. Weiterhin unterstützen wir nach Möglichkeit alle Bemühungen für eine bundesweite Umsetzung.}

\begruendung{
Bis heute gibt es in Deutschland keine Möglichkeit in einer Notfallallsituation barrierefrei einen Notruf abzusenden. Besonders behinderte Menschen sind davon betroffen, aber auch wenn ein Handyakku nicht mehr für einen Notruf per Sprache ausreicht, ist ein non-verbaler Notruf notwendig. Ein bundesweiter barrierefreier non-verbaler Notruf für Polizei, Feuerwehr oder Krankenwagen existiert derzeit nicht.

Bisher gibt es lediglich in Berlin und in Köln spezielle Notrufnummern für SMS, die technisch gesehen relativ problemlos auf ganz Deutschland ausweiten werden könnte. Österreich hat bereits sehr gute Erfahrungen mit einer SMS-Notrufnummer für Gehörlose gemacht, die von allen großen Netzbetreibern unterstützt wird.

Es gibt derzeit zwar eine Notfall-Fax-Einrichtung (z.B. in Münster) für Gehörlose und Schwerhörige Menschen, aber dieser ist höchst umständlich und erfordert ein Faxgerät, welches heutzutage nur noch wenige Menschen besitzen oder bei vielen Notfallsituationen nicht verfügbar ist. Ein weiterer Kritikpunkt ist, dass Fax-Notrufe nicht immer so ernst genommen werden, wie es vonnöten wäre oder erst viele Stunden später Hilfe eintrifft und es dann möglicherweise schon zu spät ist.

Länder in denen schon ein Notruf per SMS möglich ist sind:

\begin{itemize}
\item England (\url{http://www.emergencysms.org.uk})
\item USA
\item Australien (\url{http://www.ewn.com.au} - \url{http://www.emergencyalert.gov.au})
\item Irland (\url{http://www.112.ie/Pilot_112_SMS_Service/142})
\item Singapore (\url{http://www.spf.gov.sg/sms70999/index.html})
\item Portugal (\url{http://ec.europa.eu/information_society/activities/112/ms/pt/index_en.htm})
\item Schweden (\url{http://www.sosalarm.se/Documents/Nyheter\%20och\%20Media/Bibliotek/Broschyrer/2010/SMS\%20112\%20Systembeskrivning_EN\%20_2_.pdf})
\item Finnland
\item Island
\item Norwegen (\url{http://www.kokom.no/kokomsoek/publikasjonar/Rapportar/SMS_in_EC_2009.pdf})
\end{itemize}

Im Falle einer Katastrophe in Deutschland wird die Bevölkerung per Sirene und Lautsprecherdurchsagen informiert, das Radio und/oder TV einzuschalten um weitere Informationen zu erhalten. Die Bevölkerung wird dazu angehalten ihre Nachbarn und vor allem Hilfsbedürftige zu informieren. Hörbehinderte Menschen bekommen davon jedoch nicht sofort etwas mit und sind somit auf die Aufmerksamkeit ihrer Mitmenschen angewiesen. Dies ist ein unhaltbarer Zustand.

Käme es noch zu einer Evakuierung denken Menschen erstmal an sich, ihre Familie, Freunde. Behinderte Mitmenschen wohl im seltesten Fall. Auch wenn der Katastrophenschutz dazu aufruft.


Übernommen von der AG BArrierefreiheit (siehe \href{https://ag_barrierefreiheit.piratenpad.de/4}{Pad}) 
}

% -----

\wahlprogramm{Klare Trennung von Kirche und Staat}
\antrag{Prof. Dr. Michael Rost, Biederitz}\\
\version{18:29, 9. Apr. 2012}

\paragraph{Antragstext}:

Die Piratenpartei setzt sich für eine klare Trennung von Kirche und Staat ein. Die Piratenpartei ist für Religionsfreiheit und Gleichberechtigung aller Religionen. Jeder Mensch hat das Recht eine Religion auszuüben, aber jede Religion ist reine Privatsache jedes Menschen. Die Piratenpartei ist gegen weitere Alimentierung der Kirchen und Religionsgemeinschaften vom Staat, gegen das Eintreiben der Kirchensteuer durch den Staat, gegen vom Staat alimentierte kirchliche Hochschulen, gegen finanzielle Zuschüsse an Kirchen und Religionsgemeinschaften, gegen Religionsunterricht an staatlichen Schulen, gegen religiöse Zeichen in Schulen. Im Sinne eines evolutionären Humanismus dürfen Menschen ohne Religionsbindung nicht gegenüber anderen Menschen benachteiligt werden und umgekehrt. Die Piratenpartei setzt sich insbesondere auch für die Ablösung der historisch bedingten Finanztransfers an die Kirchen ein.

\begruendung{37,20\% der Deutschen Bevölkerung, und fast 81\% der Bevölkerung Sachsen Anhalts ist konfessionsfrei, es ist deshalb in höchstem Maße ungerecht, wenn dieser überwiegende Teil der Bevölkerung über Steuern und Abgaben für jene aufkommen muss die einer der großen Religionsgemeinschaften angehören, zumal kleine Religionsgemeinschaft dabei ohnehin benachteiligt werden.}

% -----

\wahlprogramm{Änderung der öffentlichen Vergabepraxis}
\antrag{Andreas Rieger}\\
\version{20:53, 14. Apr. 2012}

\paragraph{Antragstext}:

Die Piratenpartei tritt für eine Änderung der Vergabepraxis für öffentliche Aufträge im Land Sachsen-Anhalt ein. Dabei soll in Zukunft nicht das günstigste Angebot, sondern das Angebot, das am nächsten an den im Planfeststellungsverfahren oder der Haushaltsplanung kalkulierten Kosten liegt, verwendet werden. Vergabeverträge sollen weiterhin immer so geschlossen werden, dass Änderung bei den Kosten zu Lasten der Firmen geht, die die Auftäge erhalten haben.

\begruendung{die bisherige Praxis zeigt dass bei öffentlichen Aufträgen am Ende häufig die 2 -3 fachen Kosten erreicht werden als in den Planfeststellungverfahren geplant. Dies ist vor allem deshalb der Fall, weil die Vergabepraxis über das Günstigkeitsprinzip häufig dazu führt, das kartelleartige Strukturen mit Sub-Subfirmen die Auftragsvergabeverfahren gewinnen und anschließende Kostensteigerungen über erpressungartige Verfahren durchgedrückt werden ( bewirkte Pleite von Firmenteilen oder der Vertragsfirma anschließende Erpressung der Poltik nach dem Motto {\Gu}wenn Ihr nicht mehr zahlt wirds nie fertig{\Go}), Hinzu kommen Häufig Korruption und Ungerechtfertigte Einsichtnahme der Vergabeunterlagen.}

% -----

\wahlprogramm{Rechtsextremismus}\label{sa:rechtsex1}
\antrag{Alexander Magnus}\\
\version{18:30, 9. Apr. 2012}
\begin{itemize}
\item \konkurrenz{sa:rechtsex2}
\end{itemize}

\paragraph{Antragstext}:

In unserer Gesellschaft darf kein Platz für Rechtsextremismus, Rassismus und Antisemitismus sein. Rechtsextreme Propaganda muss als solche bloßgestellt und unsere demokratischen Werte ihr gegenübergestellt werden. Die Morde der sich selbst als {\Gu}Nationalsozialistischer Untergrund{\Go} bezeichnenden Vereinigung haben auf besonders erschreckende Art und Weise verdeutlicht, wie groß das Problem des Rechtsextremismus und die von ihm ausgehende Gefahr ist. In den vergangenen Jahren wurde dieses Problem allzu oft verkannt, ignoriert oder kleingeredet. Präventionsarbeit in diesen Bereichen wurde durch Budgetkürzungen erschwert und mitunter unmöglich gemacht. Diese Schritte müssen rückgängig gemacht werden, sodass diese Programme nicht nur ihre alte Stärke zurückgewinnen, sondern darüber hinaus weiter ausgebaut werden können.

\begruendung{Übernommen von \href{http://jandoerrenhaus.de/2012/04/06/distanz.zu.rechts.punkt/}{Jan Doerrenhaus/NRW}

Ich möchte, dass sich der Landesverband klar zum Selbstverständnis der Partei und den auf dem letzten BPT angenommenen Anträgen gegen Rechtsextremismus bekennt. Dabei geht es nicht(!) um eine generelle Abkehr von Extremismus - die ich befürworte! - sondern ganz speziell und insbesondere um Rechtsextremismus. Wer der Meinung ist, dass wir auch ebenso klar und deutlich gegen andere Formen von Extremismus Stellung beziehen sollten, darf gern einen solchen Antrag stellen. Ich bitte daher von Kommentaren, die eine Änderung des Antrages in diese Richtung vorschlagen, abzusehen. Das ist NICHT Thema dieses Antrages.

Zur weiteren Argumentationsunterstützung sei auf das oben verlinkte Blog sowie das von \href{http://tarzun.de/archives/423-Das-Problem-heisst-nicht-Kevin.html}{Tarzun} verwiesen.}

% -----

\wahlprogramm{Geschlechter- und Familienpolitik}
\antrag{Stephan Schurig}\\
\version{13:24, 11. Apr. 2012}

\paragraph{Antragstext}:

Der Landesparteitag möge beschließen folgenden Abschnitt im Wahlprogramm unter dem Punkt {\Gu}Geschlechter- und Familienpolitik{\Go} einzufügen:

\einruecken{Die Piratenpartei steht für eine zeitgemäße Geschlechter- und Familienpolitik. Diese basiert auf dem Prinzip der freien Selbstbestimmung über Angelegenheiten des persönlichen Lebens. Die Piraten setzen sich dafür ein, dass Politik der Vielfalt der Lebensstile gerecht wird. Jeder Mensch muß sich frei für den selbstgewählten Lebensentwurf und für die individuell von ihm gewünschte Form gleichberechtigten Zusammenlebens entscheiden können. Das Zusammenleben von Menschen darf nicht auf der Vorteilnahme oder Ausbeutung Einzelner gründen.

\textbf{Freie Selbstbestimmung von geschlechtlicher und sexueller Identität bzw. Orientierung}

Die Piratenpartei steht für eine Politik, die die freie Selbstbestimmung von geschlechtlicher und sexueller Identität bzw. Orientierung respektiert und fördert. Fremdbestimmte Zuordnungen zu einem Geschlecht oder zu Geschlechterrollen lehnen wir ab. Diskriminierung aufgrund des Geschlechts, der Geschlechterrolle, der sexuellen Identität oder Orientierung ist Unrecht. Gesellschaftsstrukturen, die sich aus Geschlechterrollenbildern ergeben, werden dem Individuum nicht gerecht und sind zu überwinden.

Die Piratenpartei lehnt die Erfassung des Merkmals {\Go}Geschlecht” durch staatliche Behörden ab. Übergangsweise kann die Erfassung seitens des Staates durch eine von den Individuen selbst vorgenommene Einordnung erfolgen.

\textbf{Freie Selbstbestimmung des Zusammenlebens}

Die Piraten bekennen sich zum Pluralismus des Zusammenlebens. Politik muss der Vielfalt der Lebensstile gerecht werden und eine wirklich freie Entscheidung für die individuell gewünschte Form des Zusammenlebens ermöglichen. Eine bloß historisch gewachsene strukturelle und finanzielle Bevorzugung ausgewählter Modelle lehnen wir ab.

\textbf{Freie Selbstbestimmung und Familienförderung}

Die Piratenpartei setzt sich für die gleichwertige Anerkennung von Lebensmodellen ein, in denen Menschen füreinander Verantwortung übernehmen. Unabhängig vom gewählten Lebensmodell genießen Lebensgemeinschaften, in denen Kinder aufwachsen oder schwache Menschen versorgt werden, einen besonderen Schutz. Unsere Familienpolitik ist dadurch bestimmt, dass solche Lebensgemeinschaften als gleichwertig und als vor dem Gesetz gleich angesehen werden müssen.}

\begruendung{Übernahme aus dem \href{http://berlin.piratenpartei.de/wp-content/uploads/2011/08/PP-BE-wahlprogramm-v1screen.pdf}{Wahlprogramm der Berliner Piraten}}

% -----

\wahlprogramm{Ablehnung von Fracking}
\antrag{Stephan Schurig}\\
\version{13:30, 11. Apr. 2012}

\paragraph{Antragstext}:

Der Landesparteitag möge beschließen folgenden Abschnitt an geeigneter Stelle in das Wahlprogramm aufzunehmen:

\einruecken{\textbf{Ablehnung von Fracking}

Die Piratenpartei Sachsen-Anhalt lehnt Hydraulic Fracturing, auch Fracking genannt, als Gasfördermethode ab. Durch diese Methode werden wir und zukünftige Generationen einem kaum kalkulierbaren Risiko ausgesetzt. Das Einbringen zahlreicher, zum Teil hochtoxischer Stoffe mit unkontrollierter Ausbreitung ist abzulehnen. Daher setzen wir uns für ein Verbot von Fracking auf allen politischen Ebenen ein. Um den Energiebedarf zu decken, setzen wir stattdessen auf Effizienzverbesserungen, Einsparungen und generative Energien mit modernen Speichertechniken zum Ausgleich von Fluktuationen bei Energieproduktion und -verbrauch.}

\begruendung{Übernommen von den \href{https://wiki.piratenpartei.de/NRW-Web:Grundsatzprogramm\#Ablehnung_von_Fracking}{Piraten NRW} bzw. aus dem \href{https://lqfb.piratenpartei.de/pp/initiative/show/2104.html}{Bundes-LQFB}, Text korrigiert und leicht abgeändert

siehe dortige Begründungen}

% -----

\wahlprogramm{Kulturerhalt und -förderung (inkl. kulturelle Vielfalt vs. Prestigeobjekte)}
\antrag{Stephan Schurig}\\
\version{13:31, 11. Apr. 2012}

\paragraph{Antragstext}:

Der Landesparteitag möge beschließen folgenden Abschnitt in das Wahlprogramm aufzunehmen: 

\einruecken{\textbf{Kulturerhalt und -förderung}

Wie ein demokratisches Gemeinwesen verfasst ist, wird treffend durch die Worte Friedrich Schillers beschrieben: {\Gu}Die Kunst ist eine Tochter der Freiheit.{\Go} Durch die Kulturförderung werden nicht nur die Kreativen geschützt, sondern auch unsere Haltung und Freiheitsrechte. Eine verantwortliche, transparente, anregende und nachhaltig gestaltende Kulturpolitik kräftigt eine zukunftsorientierte, vielfältige und humane Gesellschaft. Diese Politik muss die notwendigen Rahmenbedingungen für eine freie Entfaltung von Kunst und Kultur schaffen - sie darf diese nicht bewerten oder vereinnahmen.

Die kulturelle Freizügigkeit und Vielfalt sollen durch geförderten Freiraum und unter Berücksichtigung der Rechte der Anwohner verteidigt werden. Behörden sollen ihre Ermessensspielräume nutzen, um zugunsten von Kunst- und Kulturinitiativen zu entscheiden. Das Kulturleben soll sich auch als Wirtschaftsfaktor und Vernetzungsplattform lebendig weiterentwickeln. Kulturentwicklungsplanung ist vielschichtig und muss die kulturelle Bildung, Betätigung und Mitwirkung des Bürgers sowie die Künste und die Kulturwirtschaft aufeinander abstimmen und die dafür notwendigen Ressourcen und Verfahren definieren. Die Piratenpartei ist bestrebt, die Förderstruktur von Kunst und Kultur möglichst stabil zu halten. Bei einzelnen Sparten sollte auch in Wirtschaftskrisen nicht so stark gekürzt werden, dass ihre jeweilige Existenz gefährdet ist, denn im Gegensatz zu materiellen Werten kann eine verlorene kulturelle Infrastruktur nur langsam wieder aufgebaut werden.

Für die PIRATEN steht die Förderung kultureller Vielfalt über der einzelner Prestigeobjekte. Kleine Kulturprojekte sind meist ehrenamtlich organisiert, erreichen und beziehen in ihrer Gesamtheit aber deutlich mehr Menschen mit ein.

Der Zugang zu Kultureinrichtungen muss für alle Gesellschaftsschichten offen gehalten werden, damit diese Institutionen gesellschaftlich verankert sind. Des Weiteren müssen größtenteils öffentlich finanzierte Einrichtungen auch für die gesamte Bevölkerung zugänglich sein.}

\begruendung{\begin{itemize}
\item übernommen aus dem Grundsatzprogramm LV Berlin (Dank an Alex)
\item Vorletzter Abschnitt zur Inititative von alexkid hinzugefügt ({\Gu}Für die PIRATEN...{\Go})
\item Dank an Lennstar und zig fürs Korrekturlesen!
\end{itemize}}

% -----

\wahlprogramm{Aufhebung von §11 FeiertG LSA (Einschränkung der Versammlungsfreiheit)}\label{wpa:feiertg1}
\antrag{Stephan Schurig}\\
\version{18:14, 13. Apr. 2012}
\begin{itemize}
\item \konkurrenz{wpa:feiertg2}
\end{itemize}

\paragraph{Antragstext}:

Der Landesparteitag möge beschließen, folgenden Abschnitt an geeigneter Stelle in das Wahlprogramm aufzunehmen:

\einruecken{\textbf{Aufhebung des §11 FeiertG LSA}

Die Piratenpartei Sachsen-Anhalt strebt die Aufhebung des §11 Gesetz über die Sonn- und Feiertage (FeiertG LSA) an. Für die PIRATEN ist die Einschränkung des Grundrechts auf Versammlungsfreiheit an religiösen Feiertagen inakzeptabel. Die Trennung von Religion und Staat bzw. die Selbstbestimmung des Individuums ist höher zu bewerten, als der erhöhte Schutz religiöser Bräuche. Durch Beibehalten von §4 bleibt der besondere Schutz von Gottesdiensten jedoch bestehen.}

\begruendung{In \href{http://st.juris.de/st/gesamt/FeiertG_ST.htm\#FeiertG_ST_P11}{§11 FeiertG} ist festgelegt, dass an speziellen christlichen Feiertagen das Grundrecht der Versammlungsfreiheit nach Artikel 8 des Grundgesetzes und Artikel 12 der Landesverfassung durch die §§ 4 und 5 eingeschränkt sind.

Dieser Abschnitt ist zu streichen, da er eine Einschränkung insbesondere für alle Nicht-Christen darstellt. Im Sinne der Trennung von Kirche und Staat ist das Gesetz nicht mehr zeitgemäßg. Christen können allerdings weiterhin ihrem Glauben und Gottesdiensten uneingeschränkt nachgehen, da der §4 bestehen bleibt, welcher sicherstellt, dass keine Veranstaltungen erlaubt sind, die einen Gottesdienst stören.

\paragraph{Diskussion \& Informationen}:

\begin{itemize}
\item \url{https://de.wikipedia.org/wiki/Tanzverbot}
\item \url{http://www.laizisten.de/index.php?option=com_content&task=view&id=154}
\item \url{http://www.faz.net/aktuell/rhein-main/kommentar-zum-tanzverbot-gegen-die-gleichgueltigkeit-an-karfreitag-1626823.html}
\item Ablehnung der Petition in Hessen zur Abschaffung inkl. Begründung \url{http://www.dropbox.com/gallery/14236556/1/Entscheidung\%20Petition\%20Tanzverbot?h=8777ba}
\item \url{http://www.piraten-giessen.de/Hintergrund-zu-Tanzen-gegen-das-Tanz-Verbot}
\item \url{http://www.piratenpartei-hessen.de/pressemitteilung/2012-04-06-bundesverfassungsgericht-haelt-demonstrationsverbot-aufrecht-piraten-ber}
\item \url{https://www.facebook.com/profile.php?id=100001862882029}
\end{itemize}}

% -----

\wahlprogramm{Aufhebung von §§5,11 FeiertG LSA (Veranstaltungsverbot und Einschränkung der Versammlungsfreiheit an christlichen Feiertagen) (überarbeitet)}\label{wpa:feiertg2}
\antrag{Stephan Schurig}\\
\version{18:25, 13. Apr. 2012}
\begin{itemize}
\item \konkurrenz{wpa:feiertg1}
\end{itemize}

\paragraph{Antragstext}:

Der Landesparteitag möge beschließen, folgenden Abschnitt an geeigneter Stelle in das Wahlprogramm aufzunehmen:

\einruecken{\textbf{Aufhebung des Veranstaltungsverbots und der Einschränkung der Versammlungsfreiheit an christlichen Feiertagen durch Abschaffung der §§ 5, 11 FeiertG LSA}

Die Piratenpartei Sachsen-Anhalt strebt die Aufhebung der §§ 5 (Erhöhter Schutz) und 11 (Einschränkung von Grundrechten) des Gesetzes über die Sonn- und Feiertage (FeiertG LSA) an. Das Verbot von öffentlichen Veranstaltungen, die nicht der Würdigung des Feiertages oder der Kunst, Wissenschaft oder Volksbildung dienen ist abzuschaffen. Weiterhin ist die Einschränkung des Grundrechts auf Versammlungsfreiheit an religiösen Feiertagen aufzuheben. Die Trennung von Religion und Staat bzw. die Selbstbestimmung des Individuums ist höher zu bewerten, als der erhöhte Schutz religiöser Bräuche. Durch Beibehalten von § 4 (Schutz der Gottesdienste) bleibt der besondere Schutz von Gottesdiensten jedoch bestehen.}

\begruendung{In \href{http://st.juris.de/st/FeiertG_ST_P5.htm}{§ 5 FeiertG} ist festgelegt, dass an speziellen christlichen Feiertagen neben den Einschränkungen nach § 4 zusätzlich untersagt sind:

1. Veranstaltungen in Räumen mit Schankbetrieb, die über den Schank- und Speisebetrieb hinausgehen, 2. öffentliche sportliche Veranstaltungen sowie 3. alle sonstigen öffentlichen Veranstaltungen, außer wenn sie der Würdigung des Feiertages oder der Kunst, Wissenschaft oder Volksbildung dienen und auf den Charakter des Tages Rücksicht nehmen.

In \href{http://st.juris.de/st/gesamt/FeiertG_ST.htm\#FeiertG_ST_P11}{§ 11 FeiertG} ist festgelegt, dass an speziellen christlichen Feiertagen das Grundrecht der Versammlungsfreiheit nach Artikel 8 des Grundgesetzes und Artikel 12 der Landesverfassung durch die §§ 4 und 5 eingeschränkt sind.

Diese Abschnitte sind zu streichen, da sie eine Einschränkung insbesondere für alle Nicht-Christen darstellt. Im Sinne der Trennung von Kirche und Staat ist das Gesetz nicht mehr zeitgemäßg. Christen können allerdings weiterhin ihrem Glauben und Gottesdiensten uneingeschränkt nachgehen, da der § 4 bestehen bleibt, welcher sicherstellt, dass keine Veranstaltungen erlaubt sind, die einen Gottesdienst stören.

\paragraph{Diskussion \& Informationen}:

\begin{itemize}
\item \url{https://de.wikipedia.org/wiki/Tanzverbot}
\item \url{http://www.laizisten.de/index.php?option=com_content&task=view&id=154}
\item \url{http://www.faz.net/aktuell/rhein-main/kommentar-zum-tanzverbot-gegen-die-gleichgueltigkeit-an-karfreitag-1626823.html}
\item Ablehnung der Petition in Hessen zur Abschaffung inkl. Begründung \url{http://www.dropbox.com/gallery/14236556/1/Entscheidung\%20Petition\%20Tanzverbot?h=8777ba}
\item \url{http://www.piraten-giessen.de/Hintergrund-zu-Tanzen-gegen-das-Tanz-Verbot}
\item \url{http://www.piratenpartei-hessen.de/pressemitteilung/2012-04-06-bundesverfassungsgericht-haelt-demonstrationsverbot-aufrecht-piraten-ber}
\item \url{https://www.facebook.com/profile.php?id=100001862882029}
\end{itemize}}

% -----

\wahlprogramm{Ungehinderter Zugang zu Verwaltungsdaten}
\antrag{Christoph Giesel}\\
\version{18:29, 13. Apr. 2012}

\paragraph{Antragstext}:

Der Landesverband möge folgenden Abschnitt an geeigneter Stelle in das Wahlprogramm einfügen:

\einruecken{\textbf{Ungehinderter Zugang zu Verwaltungsdaten}

Die PIRATEN setzen sich für den ungehinderten Zugang zu Protokollen und allen entscheidungsrelevanten Unterlagen aller Gremien auf kommunaler Ebene ein. Dies umfasst auch öffentlich-rechtliche Körperschaften wie die kommunalen Zweckverbände sowie Verträge zwischen staatlichen Stellen und privaten Unternehmen. 

Hierzu soll das Land Sachsen-Anhalt eine geeignete Infrastruktur bzw. Software bereitstellen, die die unkomplizierte Veröffentlichung der öffentlichen Daten im Internet ermöglicht. Die Daten müssen auch maschinenlesbar in freien Formaten zur Verfügung gestellt werden. Die Weiterverarbeitung, Aufbereitung und Auswertung durch Dritte ist ausdrücklich erwünscht. Auf einen barrierefreien Zugang muss besonderer Wert gelegt werden. Die Erstellung einer freien Software zur Veröffentlichung soll geprüft werden. 

Das Informationszugangsgesetz und das Verwaltungskostengesetz des Landes sollen so erweitert werden, dass die kostenlose Erstellung und Versendung von Kopien vorgeschrieben werden, wenn die Daten nicht auf die beschriebene Weise veröffentlicht werden.}

\begruendung{Es kann nicht sein, dass man sich nicht ungehindert die Protokolle der Gremien anschauen kann bzw. für Kopien noch dann noch Geld bezahlen muss.

Mir ist natürlich klar, dass die Kommunen selber nicht genügend Geld für so etwas haben. Daher soll das Land diese unterstützen.}

% -----

\wahlprogramm{Flüchtlinge und Migranten\_innen - Asylpolitik}
\antrag{Karl}\\
\version{0:10, 14. Apr. 2012}

\paragraph{Antragstext}:

Der Landesparteitag möge beschließen folgenden Text mit dem neuen Bereich {\Gu}Asylpolitik{\Go} in das Parteiprogramm aufzunehmen: 

\einruecken{\textbf{Bleiberecht}

Forderung:

Es muss eine umfassende Bleiberechtsregelung mit realistischen Erteilungsvoraussetzungen geben. Das aktive Bemühen von Menschen mit prekärem Aufenthalt muss durch die Behörden anerkannt werden. Außerdem müssen die Fristen zur Beantragung von acht Jahren Aufenthalt in Deutschland gesenkt werden um mehr Menschen neue Möglichkeiten zur selbständigen Lebensunterhaltssicherung zu ermöglichen.

\textbf{Arbeit}

Forderungen:

Um eine gesellschaftliche Teilhabe aller Flüchtlinge und Migranten zu ermöglichen, sollen alle in Deutschland lebenden Menschen eine Arbeitserlaubnis erteilt bekommen. Dieses ermöglicht eine selbstständige Lebensunterhaltssicherung und bereichert den Arbeitsmarkt durch die bisher ungenutzten Qualifikationen der Menschen ohne Arbeitserlaubnis.

\textbf{Ausbildung / Studium}

Forderung:

Der Zugang zu Ausbildung und Studium für Flüchtlinge und Migranten muss gleichberechtigt ermöglicht werden um gerade in einer alternden Gesellschaft wie der Deutschlands die Chancen durch Migration zu nutzen und Perspektiven für alle zu entwickeln. Außerdem müssen ausländische Schulabschlüsse einfacher anerkannt werden. Im Schulbereich müssen bundesweit verbindliche Strukturen und Kapazitäten für Flüchtlingskinder geschaffen werden. Hierzu zählt auch die Sprachförderung und die Einschulung bis zum 18. Geburtstag.

\textbf{Residenzpflicht}

Forderung:

Diese in Europa einzigartige Regelung muss bundesweit für alle Menschen abgeschafft werden. Niemand soll in seinem Recht auf freie Bewegungsfreiheit beschränkt werden. Die Kriminalisierung und Diskriminierung von Flüchtlingen und Migranten muss aufhören.

\textbf{Medizinische Versorgung}

Forderung:

Der Zugang zu umfassender, unbürokratischer medizinischer Versorgung muss ermöglicht werden. Das diskriminierende Asylbewerberleitungsgesetz muss abgeschafft werden und die Menschen müssen Mitglied einer gesetzlichen Krankenkasse werden. Ein erfolgreiches Modell findet sich in Bremen.

\textbf{Unterbringung}

Forderungen:

Unbürokratische Zusicherungen der Mietkostenübernahme durch das Sozialamt in Verbindung mit einer generellen Übernahme der Mietkaution als zinslosem Kredit.

\textbf{Ausländerbehörde}

Forderungen:

Die soziale, fachliche und sprachliche Kompetenzen der Sachbearbeitern muss ausgebaut werden. Die ABH soll nicht nur restriktiv agieren, sondern die Menschen fördern und Teilhabe ermöglichen.}

\begruendung{\textbf{Bleiberecht}

Viele Menschen leben in Deutschland ohne aufenthaltsrechtliche Perspektive und können weder vor noch zurück. Allein in Berlin leben 5.965 Menschen mit Duldung, 2.427 {\Gu}sonstige Ausreisepflichtige{\Go} und 2.207 Asylbewerber\_innen. Diese Menschen werden Deutschland nicht aufgrund von Schikanen der Ausländerbehörden verlassen.

\textbf{Arbeit}

Um in Deutschland arbeiten zu können braucht es eine Arbeitserlaubnis. Das Arbeitserlaubnisrecht grenzt Flüchtlinge und viele Migranten aus. Eine selbstständige Lebensunterhaltssicherung ist hierdurch oft nicht möglich. Eine Mehrheit der Menschen ohne Arbeitserlaubnis möchte arbeiten, hat aber keine Berechtigung.

\textbf{Ausbildung / Studium}

Viele Migrant\_innen unterliegen einem Verbot des Studiums und der Ausbildung. Gerade Jugendliche leiden unter diesen Restriktionen, die i.d.R. ihren Eltern gelten. Ein weiteres Problem ist der Zugang zu schulischer Bildung für Flüchtlingskinder.

\textbf{Residenzpflicht}

Die räumliche Beschränkung des Aufenthalts von Asylbewerberen und Menschen mit Duldung, umgangssprachlich Residenzpflicht, führt zur Kriminalisierung dieser Menschen. Ihnen wird ihr Grundrecht auf Bewegungsfreiheit genommen. Bei Verstößen ahndet die Ausländerbehörden das Vergehen mit Bußgeldern oder Strafbefehlen, welche in vielen Fällen sogar zu Ausweisungen führen können.

\textbf{Medizinische Versorgung}

Der Zugang zu medizinischer Versorgung ist für viele Flüchtlinge und Migrant\_innen durch das Asylbewerberleitungsgesetzes (AsylbLG) eingeschränkt und deckt nur akute und schmerzhafte Erkrankungen ab. Durch diese Regelung sind chronisch Kranke und traumatisierte Menschen exkludiert. Ein weiteres Problem stellen die Krankenscheine vom Sozialamt dar. Sie sind für die betroffenen Menschen umständlich und für die Kommunen teuer.

\textbf{Unterbringung}

Viele Flüchtlinge und Menschen mit Duldungen müssen in Flüchtlingswohnheimen leben. Der Zugang zu regulären Wohnungen ist ihnen häufig gesetzlich verwehrt oder durch hohe Mieten und bürokratische Hürden (Schufa, Kaution, Courtage) unmöglich.

\textbf{Ausländerbehörde}

Die Ausländerbehörde (ABH) ist als {\Gu}Service-Behörde{\Go} dem Staat, aber auch den Kunden verpflichtet. Untersuchungen bei der ABH Berlin lassen aber genau das bezweifeln. Die Wahrnehmung ist eher die einer respektlosen Ordnungsbehörde die als ihr Handlungsziel die Migrationskontrolle, bzw. -verhinderung sieht.}

% -----

\wahlprogramm{Neofaschismus}\label{sa:rechtsex2}
\antrag{Karl}\\
\version{0:10, 14. Apr. 2012}
\begin{itemize}
\item \konkurrenz{sa:rechtsex1}
\end{itemize}

\paragraph{Antragstext}:

Der Landesparteitag möge beschließen den folgenden Text an passender Stelle in das Parteiprogramm aufzunehmen:

\einruecken{Die Piratenpartei Sachsen-Anhalt strebt eine vielfältige Gesellschaft an, in der sich alle Menschen im Rahmen eines friedlichen Zusammenlebens frei nach ihren eigenen Bedürfnissen entfalten können. Deswegen lehnen wir Meinungen, die andere Menschen erniedrigen, grundsätzlich ab. Außerdem erkennen wir an, dass diskriminierendes Gedankengut, wie z.B. Rassismus, Sexismus und Antisemitismus, kein Randproblem darstellt, sondern in der ganzen Gesellschaft zu finden ist. Daher beziehen wir explizit Stellung gegen menschenverachtende Weltanschauungen in organisierter, nicht organisierter als auch in alltäglicher Form.}

\begruendung{Die Freiheit des Einzelnen hört da auf, wo die Freiheit des Anderen anfängt. Deshalb sind menschenverachtende Ansichten abzulehnen und nicht von dem Recht auf Meinungsfreiheit gedeckt.

Dieser Antrag spricht bewusst nicht vom Rechtsextremismus, da der Extremismusbegriff irreführend ist und das gesamtgesellschaftliche Problem verschleiert.

Dieser Antrag spricht auch bewusst nicht von totalitären Weltanschauungen. Ähnlich wie beim Begriff Extremismus wird mit dem Totalitarismusbegriff lediglich eine Form von Herrschaft beschrieben. Es geht also vielmehr darum wie etwas passiert, als darum welche Ideologie hinter dem Handeln steckt. Dieser Antrag richtet sich jedoch explizit gegen menschenverachtende Weltanschauungen und wird diese Begrifflichkeiten weder benutzen, noch an anderer Stelle in diesem Zusammenhang befürworten.}

% -----

\wahlprogramm{Fachärztemangel}
\antrag{Martin Otto}\\
\version{16:15, 14. Apr. 2012}

\paragraph{Antragstext}:

Der Landesparteitag möge beschließen, folgenden Text in ein dafür geeignetes Programm aufzunehmen:

\einruecken{Der Landesverband Sachsen-Anhalt setzt sich dafür ein, dem Landärztemangel gegenzusteuern. Seit Jahren ist die Zahl der praktizierenden Ärzte auf dem Land rückläufig. Das führt zu einer gravierenden Unterversorgung der gesundheitlichen Betreuung in ländlichen Regionen. Um diesem Mangel an Ärzten entgegenzusteuern, bedarf es umfangreicher struktureller Maßnahmen. Dazu gehört:

\begin{itemize}
\item der Beruf \textbf{Praktischer Arzt} muss wieder eingeführt werden
\item der bürokratische Aufwand für Hausärzte muss erheblich erleichtert werden
\item die angehenden Landärzte erfahren finanzielle und materielle Unterstützung bei der Einrichtung einer Praxis und erhalten ein permanentes Grundgehalt, welches gleich oder höher des regionalen Durchschnittsgehaltes eines Facharztes ist.
\item der ärztliche Leistungskatalog muss zugunsten der ärztlichen Grundversorgung überarbeitet werden 
\end{itemize}

Zur Umsetzung dieser Maßnahmen sind umgehend Kommissionen einzusetzen, bestehend aus Fachleuten vom KVSA, dem Hausärzteverband Sachsen-Anhalt e.V., sowie dem Hartmannbund, die die anstehenden Probleme benennen, damit die Politik die zeitnahe Behebung auf den Weg bringen kann.

Als weitere Maßnahmen wird empfohlen:

\begin{itemize}
\item die Einführung eines nichtrückzahlbaren Zusatzstipendiums. Dieses geht einher mit der Verpflichtung, für die Dauer der Zahlung anschließend auf dem Land zu arbeiten. 
\item die Neuordnung des Bereitschaftsdienstes sowie
\item die Unterstützung von Familienmitgliedern bei der Erwerbstätigkeit
\end{itemize}}

% -----

\wahlprogramm{Präventionsmaßnahmen Neofaschismus}
\antrag{Alexander Magnus}\\
\version{16:22, 14. Apr. 2012}

\paragraph{Antragstext}:

Der Landesparteitag möge beschließen, folgenden Text in ein dafür geeignetes Programm aufzunehmen, wenn der Antrag WPA29 - Neofaschismus angenommen wurde:

\einruecken{Die Präventions- und Aufklärungsarbeit, wie zum Beispiel Aussteigerprogramme, schulische Informationsveranstaltungen und Förderung von Bildungsreisen zu Gedenkstätten, wurde in den letzten Jahren durch Budgetkürzungen, unter anderem seiten des Bundesfamilienministeriums, erheblich erschwert und mitunter unmöglich gemacht. Die Piratenpartei Sachsen-Anhalt setzt sich dafür ein, dass diese Schritte rückgängig gemacht werden, damit diese Programme nicht nur ihre alte Stärke zurückgewinnen, sondern darüber hinaus weiter ausgebaut werden können.}

\begruendung{Ausgehend von Antrag WPA29 - Neofaschismus formuliert dieser Antrag konkrete Maßnahmen zur Prävention von menschenverachtenden Weltanschauungen.}

% -----

\wahlprogramm{Zweistufiges Schulsystem}
\antrag{Tom Westendorf}\\
\version{16:25, 14. Apr. 2012}

\paragraph{Antragstext}:

Der Landesparteitag möge beschließen, ein zweistufiges Schulsystem in ein dafür geeignetes Programm aufzunehmen.

\einruecken{Das zweistufige Bildungssystem beinhaltet:

\begin{itemize}
\item gemeinsames Lernen von der ersten bis zur zehnten Klasse mit abschließender Realschulabschlussprüfung für alle
\item anschließend wird die Möglichkeit gegeben, das weiterführende Gymnasium bis zur zwölften Klasse zu besuchen mit einer abschließenden Abiturprüfung
\end{itemize}}

\begruendung{
\begin{itemize}
\item im aktuellen existierenden Bildungssystem werden Schüler zu früh in verschiedene Schulsysteme selektiert (Hauptschule, Realschule, Gymnasium) ohne ihre möglichen Entwicklungspotentiale zu berücksichtigen
\item nach dem jeweiligen Schulabschluss existiert ein enormer Konkurrenzdruck für Lehrstellen zwischen den verschiedenen Schulabgängern
\item ein längeres gemeinsames Lernen fördert das allgemeine Bildungsniveau und verhindert Diskriminierungen innerhalb der Schülerschaft
\item als weiterführendes Ziel muss dieses Schulsystem bundesweit eingeführt werden 
\end{itemize}
}

% -----

\wahlprogramm{Bildungspolitik ist Bundespolitik}
\antrag{Andreas Rieger}\\
\version{16:32, 14. Apr. 2012}

\paragraph{Antragstext}:

Die Piratenpartei Sachsen Anhalt setzt sich dafür ein, dass das Land Sachsen Anhalt in der Kultusministerkonferenz für eine einheitliche Bildungspolitik auf Bundesebene eintritt, und das Kooperationsverbot aufgehoben wird.

\begruendung{Die Bildungspolitik, welche in Deutschland auf Landesebene geregelt ist, wurde zu Lasten der Schüler in letzter Zeit immer mehr zum Spielball der Politiker. Um dem entgegen zu wirken, sollten die Entscheidungsmöglichkeiten der Länder im Schulsystem eingeschränkt werden, indem der Bund verbindliche Vorgaben macht. In Zeiten immer höherer Anforderungen an die Standortflexibilität sind einheitlichere Richtlinien im Schulsystem ein großer Gewinn für Schüler, die den Standort länderübergreifend wechseln müssen. Zudem ist Bildungspolitik auch Integrations- und die bessere Wirtschaftspolitik}

% -----

\wahlprogramm{Bildungsversorgung in der Fläche}
\antrag{Andreas Rieger}\\
\version{16:40, 14. Apr. 2012}

\paragraph{Antragstext}:

Die Piraten Sachsen-Anhalt treten für die Errichtung von Landschulen ohne Klassensystem in Bevölkerungsarmen Regionen ein , in der zumindest Mathematik und Deutsch als Grundfächer gelehrt werden und der restliche Fachunterricht unter Aufsicht dann in virtuellen Klassenräumen stattfindet.

\begruendung{Die Piratenpartei erkennt an, dass der demographische Wandel insbesondere in ländlichen Regionen zu einer massiven Veringerung der Anzahl der schulpflichtigen Kinder führt. Gleichwohl kann das Schließen von Schulen und die damit verbundenen langen Schulwege selbst für Grundschüler nicht die Lösung des Problems sein.}

% -----

\wahlprogramm{Einführung einer einheitlichen Schulbekleidung}
\antrag{Tom Westendorf}\\
\version{16:46, 14. Apr. 2012}

\paragraph{Antragstext}:

Der Landesparteitag möge beschließen, eine einheitliche Schulbekleidung innerhalb innerhalb der Schulen zu fördern. Text des Antrages zweite Zeile etc.

\begruendung{Um Kindern aus verschiedenen sozialen Herkünften ein unvoreingenommenes Miteinander zu ermöglichen, befürworten wir die Einführung einer einheitlichen Schulbekleidung für Schüler während des Schulalltages. Die Kosten dabei sind zur Hälfte vom Land zu tragen. In verschiedenen Staaten, z.B. Japan, China, USA wurden bereits Schuluniformen mit positiver Resonanz eingeführt. Begründung des Antrages zweite Zeile etc.}

% -----

\wahlprogramm{Öffentliche Infrastruktur}
\antrag{MAoAm}\\
\version{20:44, 14. Apr. 2012}

\paragraph{Antragstext}:

Der Landesparteitag möge in sein Grundsatzprogramm folgenden Abschnitt einfügen: 

\einruecken{Die Piratenpartei Sachsen-Anhalt setzt sich für die Erhaltung und Verbesserung der öffentlichen Infrastruktur ein.

\textbf{Erreichbarkeit öffentlicher Verwaltung}

Die Verwaltung einzelner Gebietsgliederungen soll für den Bürger möglichst barrierefrei erreichbar sein. Neben der persönlichen Erreichbarkeit, beinhaltet dies auch die Bereistellung von Dienstleistungen auf elektronischem Weg.

\textbf{Gesundheitswesen}

Die medizinische Grundversorgung ist von zentralen öffentlichem Interesse. Daher darf es nicht durch Gewinnstreben dominiert werden. Sie soll und muss daher durch die öffentliche Hand flächendeckend sichergestellt werden. Privatisierungen von Kliniken oder Universitätskliniken lehnen wir ab.

\textbf{Strom-, Gas-, Wasser-, Abwasser- und Telekommunikationsversorgung, Straßen- und Schienennetz}

Jeder Bürger und jedes Unternehmen muss gleichberechtigten Zugang zu den Versogungs-, Entsorgungs- und Verkehrsnetzen erhalten. Dazu soll die um sich greifende Privatisierung gestoppt und wenn möglich privatisierte Netze in die öffentliche Hand zurückgeführt werden.

\textbf{ÖPNV}

Da der freie Zugang zu Bildung, Wissen, Information, Kultur und Verwaltung gewährleistet sein muss, setzt sich die Piratenpartei Sachsen-Anhalt für einen flächendeckenden ÖPNV ein.

\textbf{Bildung}

Der freie Zugang zu Bildung ist den Piraten ein essentielles Anliegen. Die Qualität und Quantität von Bildungseinrichtungen muss flächendeckend sichergestellt werden.}
