\section{Demokratie und Teilhabe}

\subsection*{Einleitung zum Thema: Modernisierung der Demokratie}
\wahlprogramm{Modernisierung der Demokratie}
\antrag{Unbekannt}\version{03:43, 20. Jun. 2010}

\subsubsection{Einleitung zum Thema: Modernisierung der Demokratie}
\abstimmung
Die Art und Weise wie sich Bürger in unserer Demokratie engagieren hat sich über die letzten Jahrzehnte zunehmend verändert. Statt sich in Parteien zu organisieren und am Ende jeder Legislaturperiode einmal zur Wahl zu gehen, bringen sich die Bürger zunehmend mit Hilfe von Organisationen und Bürgerinitiativen direkt in den demokratischen Prozess ein. Es reicht also nicht mehr, nur alle vier oder fünf Jahre eine Wahl zu veranstalten, um dem Verlangen der Bürger nach politischer Teilhabe gerecht zu werden. Um dieser Veränderung gerecht zu werden, müssen mehr Möglichkeiten geschaffen werden, wie sich die Bürger auch auf Landesebene direkt einbringen können.
 
\subsection*{Transparenz in der Demokratie}
\wahlprogramm{Transparenz in der Demokratie}
\antrag{Unbekannt}\version{03:43, 20. Jun. 2010}

\subsubsection{Transparenz in der Demokratie}
\abstimmung
Um diese Möglichkeiten zu schaffen will die Piratenpartei die Transparenz in der Erarbeitungsphase von Gesetzen auf Landesebene verbessern. Informationen, die die Bürger benötigen, um sich in die politischen Prozesse einzubringen, sollen schnell, übersichtlich und einfach zugänglich gemacht werden.
 
\subsection*{Deliberative Demokratie}
\wahlprogramm{Deliberative Demokratie}
\antrag{Unbekannt}\version{03:43, 20. Jun. 2010}

\subsubsection{Möglichkeiten der deliberativen Demokratie}
\abstimmung
Die Piratenpartei will die Möglichkeiten der deliberativen Demokratie, also Bürgerbeteiligung, in der Erarbeitungsphase von Gesetzen in Rheinland-Pfalz verstärkt nutzen. Das Veranstalten von Bürgerkongressen, Planungszellen oder auch das Nutzen von elektronischen Beteiligungsmöglichkeiten über das Internet darf keine Ausnahme sein, sondern muss bei wichtigen Gesetzen zur Regel werden.
 
\subsection*{Niedrigere Quoren für die direkte Demokratie und obligatorische Volksentscheide}
\wahlprogramm{Niedrigere Quoren für die direkte Demokratie und obligatorische Volksentscheide}\label{wp:demokratie:niedrig}
\antrag{Unbekannt}\konkurrenz{wp:demokratie:mehr}\version{03:43, 20. Jun. 2010}

\subsubsection{Möglichkeiten von Volksbegehren und Volksentscheiden auf Landesebene ausbauen}
\abstimmung
Die Piratenpartei will die Möglichkeiten von Volksbegehren und Volksentscheiden auf Landesebene ausbauen. Wir fordern die Absenkung der Quoren für die direkt demokratischen Instrumente in Rheinland-Pfalz. Bei großen landespoltischen Themen wie der Kommunalreform wird die Piratenpartei immer die Bürger selbst in einen Volksentscheid die endgültige Entscheidung treffen lassen. Zudem wollen wir einen obligatorischen Volksentscheid bei Änderungen der Landesverfassung.
 
\wahlprogramm{Mehr Bürgerbeteiligung}\label{wp:demokratie:mehr}
\antrag{Thomas Heinen}\konkurrenz{wp:demokratie:niedrig}\version{03:43, 20. Jun. 2010}

\subsubsection{Mehr Bürgerbeteiligung - weniger Hürden bei Volksbegehren}
\abstimmung
Die Piratenpartei steht für mehr direkte Beteiligung an öffentlichen Entscheidungen. Neben weiterreichenden Konzepten für die direkte Demokratie setzt sich die Piratenpartei auch ganz konkret für eine Förderung von Volksabstimmungen und eine Vereinfachung von Volksbegehren ein.

Um die bislang nahezu unüberwindbaren Hürden für direktdemokratische Mitbestimmung in Rheinland-Pfalz herabzusetzen fordert der Verein Mehr Demokratie e.V. die Senkung des Unterschriftenquorums und die Abschaffung des Zustimmungsquorums. Zudem wollen sie, dass Bürger über mehr Themen begehren können, beispielsweise auch über Bebauungspläne.

Wir schließen uns den Forderungen des Vereins an und setzen uns für folgende Neuregelungen ein: Die Sammelfrist soll von zwei auf sechs Monate ausgedehnt und die Anzahl der benötigten Unterschriften von ca. 10\% auf 5\% gesenkt werden. Neben dem Auslegen in Amtsräumen soll auch ein freies Sammeln gestattet sein. Wir setzen uns dafür ein, jedes zugelassene Volksbegehren grundsätzlich öffentlich im Landtag zu behandeln.

Weiterhin wollen wir bei Volksabstimmungen die Abschaffung oder zumindest die Senkung der Mindestzahl an Ja-Stimmen (Zustimmungsquoren).
 
\wahlprogramm{Förderung von Volksabstimmungen}
\antrag{Silberpappel}\zusatz{wp:demokratie:mehr}\version{03:43, 20. Jun. 2010}

\subsubsection{Ergänzung 'Mehr Bürgerbeteiligung - weniger Hürden bei Volksbegehren'}
\abstimmung
\textit{Im Abschnitt 'Mehr Bürgerbeteiligung - weniger Hürden bei Volksbegehren' soll 'Förderung von Volksabstimmungen und eine Vereinfachung von Volksbegehren ein.' so ergänzt werden:}

Förderung von Volksabstimmungen / Bürgerentscheiden und eine Vereinfachung von Volksbegehren / Bürgerbegehren ein. 

\textit{(Volksabstimmungen und Volksbegehren sind Landesebene, Bürgerentscheide und Bürgerbegehren sind Stadt- / Gemeindeebene)}

\subsubsection{Ergänzung 'Mehr Bürgerbeteiligung - weniger Hürden bei Volksbegehren'}
\abstimmung
\textit{Im Abschnitt 'Mehr Bürgerbeteiligung - weniger Hürden bei Volksbegehren" soll zwischen "Sammeln gestattet sein.' und 'Wir setzen uns' ergänzt werden:}

Die Themenbegrenzung soll auf ein Mindestmaß reduziert werden.

 
\subsection*{Strikte Gewaltenteilung}
\wahlprogramm{Strikte Gewaltenteilung}
\antrag{Piraten aus RLP}\version{03:43, 20. Jun. 2010}

\subsubsection{Strikte Gewaltenteilung}
\abstimmung
Die strikte Gewaltenteilung soll gesetzlich verankert werden. Insbesondere soll die gleichzeitige Ausübung von Amt und Mandat verboten werden.
 
\subsection*{Förderalismus stärken}
\wahlprogramm{Förderalismus stärken}
\antrag{Unbekannt}\version{03:43, 20. Jun. 2010}

\subsubsection{Förderalismus stärken}
\abstimmung
Die Piratenpartei bekennt sich zum Föderalismus und setzt sich für eine Stärkung des Föderalismus ein. Der Föderalismus gibt den Bürgern in Deutschland wesentlich mehr Einflussmöglichkeiten als in zentralistischen Systemen. Durch den Föderalismus ist es für Verbände, Bürgerinitiativen aber auch für einzelne Bürger in vielen Fällen wesentlich einfacher Politik zu beeinflussen. Nach Ansicht der Piraten sollten Entscheidungen immer auf der niedrigst möglichen Ebene getroffen werden.

\subsubsection{Für einen transparenten Förderalismus}
\abstimmung
Gleichzeitig will die Piratenpartei den Föderalismus aber klarer und transparenter machen. Es muss für die Bürger klar erkennbar sein, welche Ebene eine Entscheidung getroffen hat. Zudem setzen wir uns für eine Entflechtung der Finanzbeziehungen zwischen Bund und Ländern ein.
 
\subsection*{Kein Religionsbezug in der Landesverfassung}
\wahlprogramm{Kein Religionsbezug in der Landesverfassung}
\antrag{KV Trier/Trier-Saarburg}\version{03:43, 20. Jun. 2010}

\subsubsection{Kein Religionsbezug in der Landesverfassung}
\abstimmung
Ein weltlicher und demokratischer Staat steht für die Achtung von Menschen unabhängig ihrer religiösen Ansichten. Statt spezifischem Religionsbezug fordern wir ein Bekenntnis zu allgemein gültigen Werten, aufdenen die Gesellschaft aufbaut. Deutschland garantiert als weltlicher Staat Religionsfreiheit. Religiöse und religionsfreie Weltanschauungen sind Privatsache und die Freiheit der Wahl sowie Gleichbehandlung ist durch eine Verfassung ohne Bezüge zu einem Gott oder einer bestimmten Religion zu garantieren.
 
\subsection*{Öffentliche Petitionen nach Bundesvorbild}
\wahlprogramm{Öffentliche Petitionen nach Bundesvorbild}
\antrag{KV Trier/Trier-Saarburg}\version{03:43, 20. Jun. 2010}

\subsubsection{Öffentliche Petitionen nach Bundesvorbild}
\abstimmung
\textit{Stärkere Bürgerbeteiligung im Gesetzgebungsverfahren durch öffentliche Petitionen unter Einsatz von neuen Kommunikationsverfahren, dadurch Förderung des gesellschaftlichen Diskurses.}

Jedermann hat das Recht, sich mit Bitten und Beschwerden an die Volksvertretung zu wenden. Der Petitionsausschuss des Landtags vermittelt jedes Jahr bei über tausend Petitionen. Diese werden von Betroffenen vorwiegend gegen Behörden- und Gerichtsentscheidungen eingereicht.

Zusätzlich möchten wir den Bürgern Wege ermöglichen, an der Gesetzgebung mitzuwirken. Dazu gehören auch öffentliche Petitionen, die über ein ePetitions-Portal (nach Vorbild des Bundestages) zum gesellschaftlichen Diskurs einladen. Petitionen und Mitzeichnerunterschriften sollen online und offline gesammelt werden können.Petenten mit einer nicht unerheblichen Anzahl von Mitzeichnern sollen dabei ein Anhörungsrecht im Landtag erhalten.
 
\subsection*{Wahlalter für Landtags und Kommunalwahlen}
\wahlprogramm{Wahlalter für Landtags und Kommunalwahlen}\label{wp:demokratie:wahlalter1}
\antrag{Piraten aus RLP}\konkurrenz{wp:demokratie:wahlalter2}\version{03:43, 20. Jun. 2010}

\subsubsection{Wahlalter abschaffen - Mitbestimmungsrecht für alle}
\abstimmung
Die Piratenpartei kämpft für ein Menschenbild, indem der Mensch nicht erst ab 18 Jahren als politisch interessiert und mündig deklariert wird. Wahlreife definiert sich darüber, einen politischen Willen zu haben und diesen artikulieren zu können. Menschen können nur selbst entscheiden, wann sie ihrem politischen Willen Ausdruck verleihen können - unabhängig ihres Alters. Die Piratenpartei verlangt, dass dieses Menschenbild sich auch im Wahlsystem widerspiegelt und fordert daher die Abschaffung des Wahlalters. Wir erachten jegliche Altersgrenzen beim Wahlrecht als willkürlich. Um eine konkret spürbare Verbesserung schnell zu realisieren, soll als Übergangslösung kurzfristig das Wahlalter auf allen Ebenen auf 14 Jahre gesenkt werden.

\subsubsection{PIRATEN lehnen Familienwahlrecht ab}
\abstimmung
Die Piratenpartei lehnt ein Familienwahlrecht ab, da die Unmündigkeit der Kinder und Jugendlichen damit nicht abgeschafft, sondern noch verstärkt wird. Der von uns angestrebten Selbstbestimmung und Emanzipation steht ein Familienwahlrecht im Wege. Jeder Mensch soll selbst frei wählen und mitbestimmen können ohne Bevormundung durch Eltern oder andere Authoritäten.

\subsubsection{Positive Impulse durch Mitbestimmungsrecht für alle}
\abstimmung
Die Abschaffung des Wahlalters stellt einen immensen demokratischen und gesellschaftlichen Fortschritt dar und wird positive Veränderungen auf unsere Gesellschaft haben. Politik wird aus neuen Perspektiven gesehen werden und demokratische Entscheidungen werden sich stärker an einer politischen Nachhaltigkeit für die nachfolgenden Generationen ausrichten. Gleichsam wird das politische Interesse schon früh gefördert und demokratisches Miteinander erlernt.

\subsubsection{Politische Bildung ausbauen}
\abstimmung
Die Piratenpartei fordert begleitend zur Abschaffung des Wahlalters eine Reform der politischen Bildung. Kinder und Jugendliche müssen zusätzlich zum Politikunterricht frühestmöglich an demokratische Entscheidungsverfahren herangeführt werden und selbst mitbestimmen können. Schulen müssen in demokratische Bildungseinrichtungen verwandelt werden, in denen Schüler und Schülerinnen gleichberechtigt mit Eltern und Lehrern entscheiden. Nur so können Kinder und Jugendliche Demokratie erfahren und politisches Interesse und Gespür für politische Teilhabe entwickeln.
 
\wahlprogramm{Wahlalter für Landtags und Kommunalwahlen}\label{wp:demokratie:wahlalter2}
\antrag{Unbekannt}\konkurrenz{wp:demokratie:wahlalter1}\version{03:43, 20. Jun. 2010}

\subsubsection{Wahlalter absenken}
\abstimmung

Gerade die Themen auf Landesebene und Kommunaleben sind Themen, die Jugendliche in hohem Maße betreffen. So wird auf diesen Ebenen zum Beispiel über die Themen Bildung und den öffentlichen Nahverkehr diskutiert. Deshalb fordert die Piratenpartei eine Herabsetzung des Wahlalters für Landtags und Kommunalwahlen auf 16 Jahre, damit auch die Betroffenen selbst die Möglichkeit der demokratischen Teilhabe haben.

\subsubsection{Wahlalter absenken (Variante 2)}
\abstimmung
Gerade die Themen auf Landesebene und Kommunaleben sind Themen, die Jugendliche in hohem Maße betreffen. So wird auf diesen Ebenen zum Beispiel über die Themen Bildung und den öffentlichen Nahverkehr diskutiert. Deshalb fordert die Piratenpartei kurzfristig eine Herabsetzung des Wahlalters für Landtagswahlen auf 16 Jahre und für Kommunalwahlen auf 14 Jahre (langfristig für Landtagswahlen auf 14 Jahre und für Kommunalwahlen auf 12 Jahre), damit auch die Betroffenen selbst die Möglichkeit der demokratischen Teilhabe haben.
