\section{Immaterialgüterrechte}

\subsection*{Urheberrecht und Nutzungsrechte}
\wahlprogramm{Urheberrecht und Nutzungsrechte}
\antrag{Unglow}\version{03:40, 20. Jun. 2010}
\subsubsection{Modul 1}
\abstimmung
Das Nutzungsrecht entfernt sich immer weiter vom Urheber und entwickelt sich hin zum Verwerterrecht. Musik- und Filmindustrie profitieren, während Nutzer kriminalisiert werden. Wir PIRATEN fordern für Privatleute ohne kommerzielle Interessen das Recht, Werke frei verwenden und kopieren zu dürfen. Der Einsatz von Maßnahmen, wie die DRM-Technologie oder ähnliche Kopierschutzmechanismen, die diese und andere rechtmäßige Nutzungen einseitig verhindern, soll untersagt werden. Abgeleitete Werke sind neue künstlerische Schöpfungen und müssen dem Kreativen grundsätzlich erlaubt sein. Dies wird durch eine Anpassung des Urheberrechts gewährleistet, für die wir uns im Bundesrat einsetzen werden.

\subsubsection{Modul 2}
\abstimmung
Der Künstler soll für jedes einzelne Werk die Lizenz frei wählen können.

\subsubsection{Modul 3}
\abstimmung
Die für eine internationale Neuausrichtung des Urheberrechts zu verhandelnden Themen müssen der öffentlichen Debatte gestellt werden und dürfen nicht einseitig durch die Lobbyinteressen der Rechteverwerter geprägt sein.

\subsubsection{Modul 4}
\abstimmung
Wir PIRATEN setzen uns für die Veröffentlichung von Lehrmaterialien unter freien Lizenzen und die bevorzugte Nutzung von freien Lehrmaterialien in der Bildung ein. Dies beinhaltet die Erstellung von Lehrmaterialien durch Lehrkräfte oder beauftragte Personen unter freien Lizenzen.

\subsubsection{Modul 5}
\abstimmung
Wir müssen zumindest folgendes am Urheberrecht ändern:
 
\wahlprogramm{Medien- oder Hardwareabgaben}
\antrag{Unglow}\version{03:40, 20. Jun. 2010}

\subsubsection{Modul 1}
\abstimmung
Eine Neubewertung der Pauschalabgaben ist nötig. Bis zu dieser Neubewertung wird im Sinne des Transparenzgebotes angestrebt, sowohl das resultierende Aufkommen nach Medien/Geräteart als auch seine Verteilung nach Empfänger öffentlich zu machen.
 
\wahlprogramm{Parlamente schreiben die Urheberrecht-Gesetze, nicht die Lobby}
\antrag{Unglow}\version{03:40, 20. Jun. 2010}

\subsubsection{Modul 1}
\abstimmung
Technische Maßnahmen, die verhindern, dass Kunden Kultur im Rahmen des Gesetzes nutzen, wie die sog. DRM-Technologie, werden wir verbieten.
 
\wahlprogramm{Neue Geschäftsmodelle fördern}
\antrag{Unglow}\version{03:40, 20. Jun. 2010}

\subsubsection{Modul 1}
\abstimmung
Für viele Künstler, Schriftsteller, Journalisten, Programmierer und andere Kulturarbeiter stellt heutzutage das Urheberrecht eine wesentliche Grundlage ihrer Geschäftsmodelle und Verdienstmöglichkeiten dar. Die Möglichkeiten der digitalen Vernetzung und Kommunikation und die in oft digitaler Form vorliegenden Werke verändern die Grundlagen für diese Geschäftsmodelle zum Teil radikal.

\subsubsection{Modul 2}
\abstimmung
Anstatt den alten Geschäftsmodellen nachzutrauern und sie mit unzumutbaren Eingriffen in die Privatsphäre der Bürger künstlich am Leben erhalten zu wollen, fordern die PIRATEN dazu auf, neue Geschäftsmodelle zu entwickeln. Diese Geschäftsmodelle sollen den Urhebern der digitalen Kulturgesellschaft ermöglichen, auf marktwirtschaftliche Art und Weise Erlöse aus der Verwertung ihrer Werke oder deren Umfeld zu erzielen, wenn sie dies anstreben.

\subsubsection{Modul 3}
\abstimmung
Überholte Vermittlerfunktionen von Rechteverwertern, die in der Vergangenheit z.B. in der Unterhaltungsmusikindustrie zu hohen Renditen geführt haben, sind größtenteils nicht mehr zeitgemäß und werden in diesem Umfang keinen Bestand haben. Die Ausschaltung von Zwischenhändlern ermöglicht es, dass den Künstlern vom Erlös ihrer Werke ein größerer Teil verbleibt und direkter zufließt. Außerdem wird damit das Spektrum der Kulturszene deutlich erweitert.

\subsubsection{Modul 4}
\abstimmung
Insbesondere die Verwendung von CreativeCommons-Lizenzen erlaubt heutzutage bereits die erfolgreiche wirtschaftliche Verwertung von Werken ohne jegliche Einschränkung bei der digitalen Privatkopie und deren Verbreitung.
 
\wahlprogramm{Keine Kulturflatrate}
\antrag{Piraten aus RLP}\version{03:40, 20. Jun. 2010}

\subsubsection{Keine Kulturflatrate!}
\abstimmung
Pauschalabgabesysteme unter staatlicher Aufsicht wie z.B. die so genannte ''Kulturflatrate'' lehnen wir ab. Wir sind davon überzeugt, dass solche Subventionen technischen Fortschritt und Innovation behindern. Es ist in unseren Augen nicht Aufgabe des Staates, bestimmte Geschäftsmodelle zu sichern oder gar zu subventionieren. Wir sehen in der freien Kopierbarkeit und Verfügbarkeit von immateriellen Kulturgütern einen Gewinn für unsere Gesellschaft.
 
\subsubsection{Patentrecht}
\wahlprogramm{Patentrecht}\label{wp:ip:patent1}
\antrag{Unglow}\konkurrenz{wp:ip:patent2}\version{03:40, 20. Jun. 2010}

\subsubsection{Modul 1}
\abstimmung
Das heutige Patentsystem erfüllt in vielerlei Hinsicht nicht mehr seinen ursprünglichen Zweck, Innovationen zu fördern. Im Gegenteil: Es erweist sich immer öfter als Innovationshemmnis und behindert den technischen und ökonomischen Fortschritt in vielen Bereichen.

\subsubsection{Modul 2}
\abstimmung
Wirtschaftlicher Erfolg ist in der Informationsgesellschaft zunehmend nicht mehr von technischen Erfindungen, sondern von Wissen und Information und deren Erschließung abhängig. Das Bestreben, diese Faktoren nun ebenso mittels des Patentsystems zu regulieren, steht unserer Forderung nach Freiheit des Wissens und Kultur der Menschheit diametral entgegen.

\subsubsection{Modul 3}
\abstimmung
Wir PIRATEN lehnen Patente auf Software und Geschäftsideen ab, weil sie die Entwicklung der Wissensgesellschaft behindern, weil sie gemeine Güter ohne Gegenleistung und ohne Not privatisieren und weil sie kein Erfindungspotential im ursprünglichen Sinne enthalten. Die gute Entwicklung klein- und mittelständischer IT-Unternehmen in Deutschland und ganz Europa hat beispielsweise gezeigt, dass auf dem Softwaresektor Patente völlig unnötig sind.

\subsubsection{Modul 4}
\abstimmung
Aus den gleichen Gründen dürfen Patente auf das Leben, inklusive der Patente auf Saatgut und Gene, nicht erteilt werden. Der Privatisierung der Biodiversität oder der Grundlage menschlichen, tierischen und pflanzlichen Lebens ist mit aller Entschiedenheit entgegenzutreten.

\subsubsection{Modul 5}
\abstimmung
Bei Saatgut und Tieren fordern wir hilfsweise kurzfristig die Formulierung eines uneingeschränkten 'Nachbaurechtes', damit die bisherige Patentierungspraxis nicht weiterhin die natürlichen Verhältnisse auf den Kopf stellt und Bauern ab sofort von solchen Klagen verschont werden. Vertragsbestimmungen, die dem widersprechen, sind für nichtig zu befinden.

\subsubsection{Modul 6}
\abstimmung
Pharmazeutische Patente erzeugen viele ethische Bedenken, nicht zuletzt in Verbindung mit Menschen aus Entwicklungsländern. Sie sind auch eine treibende Kraft für die steigenden Kosten im öffentlich finanzierten Gesundheitssystem. Wir verlangen die Initiierung einer Studie über den ökonomischen Einfluß pharmazeutischer Patente, verglichen mit andern Systemen zur Finanzierung medizinischer Forschung und Alternativen zum gegenwärtigen System.
 
\subsubsection{Patentrecht}
\wahlprogramm{Patentrecht}\label{wp:ip:patent2}
\antrag{KV Trier/Trier-Saarburg}\konkurrenz{wp:ip:patent1}\version{03:40, 20. Jun. 2010}

\subsubsection{Ablehnung von Patenten auf Pflanzen und Tiere}
\abstimmung
Naturressourcen gehören allen. Patente auf Pflanzen und Tiere blockieren die Entwicklung der Wirtschaft, die Einheit des Wissens und den allgemeinen Fortschritt der Menschheit zugunsten von Einzelinteressen und übermäßiger Ansammlung von Macht und Kapital. Wir setzen uns für die Sammlung, Pflege und Weiterentwicklung tradierter Genbestände im Einklang mit den Prinzipien fortschrittlicher Ressourcenentwickung in der Landwirtschaft ein.

\subsubsection{Stellungnahme zur Gentechnik}
\abstimmung
Wir fordern eine gentechnikfreie Landwirtschaft in Rheinland-Pfalz und das Verbot von Freisetzungsversuchen.
