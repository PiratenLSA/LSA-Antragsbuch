\section{Landesfinanzen}

\subsection*{Landesfinanzen}
\wahlprogramm{Landesfinanzen}\label{wp:finanzen:land1}
\antrag{unbekannt}\konkurrenz{wp:finanzen:land2}\version{03:22, 20. Jun. 2010}
\subsubsection{Modul 1}
\abstimmung
Die Piratenpartei strebt einen ausgeglichenen Landeshaushalt an. Um dies zu erreichen wollen wir in erster Linie in dem Bereich der Kulturförderung und der Straßeninfrastruktur sparen.

\subsubsection{Modul 2}
\abstimmung
Zudem wollen wir auch im Bereich der Subventionen Geld einsparen.

\subsubsection{Modul 3}
\abstimmung
Auch durch Datensparsamkeit und den damit verbunden Bürokratieabbau wollen wir Geld einsparen.

\subsubsection{Modul 4}
Für die Bildungspolitik wollen wir hingegen mehr Geld ausgeben.
 
\wahlprogramm{Landesfinanzen}\label{wp:finanzen:land2}
\antrag{Acamir}\konkurrenz{wp:finanzen:land1}\version{03:22, 20. Jun. 2010}

\subsubsection{Ausgaben}
Die Piratenpartei strebt einen ausgeglichenen Landeshaushalt an. Durch den Abbau von Subventionen und den Verzicht auf teure Prestigeprojekte wie den Nürburgring-Freizeitpark wollen wir Geld einsparen. Für die Bildungspolitik wollen wir hingegen mehr Geld ausgeben.

\subsubsection{Einnahmen}
Wir wollen dass sich Rheinland-Pfalz im Bundesrat für eine Erhöhung der Erbschaftssteuer einsetzt um zusätzliche Einnahmen zu generieren.
 
\wahlprogramm{Landesfinanzen}
\antrag{Salorta}\version{03:22, 20. Jun. 2010}

\subsubsection{Modul 1: Ausgeglichener Haushalt}
\abstimmung
Die Piratenpartei strebt einen ausgeglichenen Landeshaushalt an.

\subsubsection{Modul 2: keine schuldenfinanzierten Ausgaben}
Die auf Dauer unverantwortliche Finanzierung von Landesausgaben über Schulden muss gestoppt werden.

\subsubsection{Modul 3: Datensparsamkeit/Bürokratieabbau}
\abstimmung
Durch Datensparsamkeit und den damit verbundenen Bürokratieabbau kann ein Beitrag zu den nötigen Einsparungen geleistet werden.

\subsubsection{Modul 4: Subventionsabbau/keine Prestigeprojekte}
\abstimmung
Große Sparpotentiale sehen wir im Abbau von Subventionen und dem Verzicht auf teure Prestigeprojekte wie den Nürburgring-Freizeitpark.

\subsubsection{Modul 5a: Neubau von Straßen 1} 
\abstimmung
Beim Neubau von Straßen sehen wir ebenfalls Möglichkeiten zur Kürzung von Ausgaben.

\subsubsection{Modul 5b: Neubau von Straßen 2}
\abstimmung
Beim Neubau von Straßen sehen wir Möglichkeiten zur Kürzung von Ausgaben.

\subsubsection{Modul 6a: Kulturförderung 1 (Ergänzung 5a)}
\abstimmung
Beim Neubau von Straßen und der Kulturförderung sehen wir ebenfalls Möglichkeiten zur Kürzung von Ausgaben.

\subsubsection{Modul 6b: Kulturförderung 2 (Ergänzung 5b)}
\abstimmung
Beim Neubau von Straßen und der Kulturförderung sehen wir Möglichkeiten zur Kürzung von Ausgaben.

\subsubsection{Modul 6c: Kulturförderung 3}
\abstimmung
Bei der Kulturförderung sehen wir ebenfalls Möglichkeiten zur Kürzung von Ausgaben

\subsubsection{Modul 6d: Kulturförderung 4}
\abstimmung
Bei der Kulturförderung sehen wir Möglichkeiten zur Kürzung von Ausgaben

\subsubsection{Modul 7: mehr Geld für Bildung}
\abstimmung
Bedarf für Mehrausgaben erkennen wir dagegen im Bereich der Bildungspolitik.

\subsubsection{Modul 8: Transparenz}
\abstimmung
Durch mehr Transparenz bei der staatlichen Auftragsvergabe bietet sich die Chance, Mauscheleien zu lasten der Steuerzahler zu verhindern.

\subsubsection{Modul 9: Ausgabenkontrolle durch direkte Demokratie}
\abstimmung
Mit Hilfe direktdemokratischer Elemente wie Bürgerbegehren und Volksentscheiden wollen wir eine größere Ausgabenkontrolle durch die Bürger erreichen.
 
\paragraph{Ergänzende Erklärung}: Von den Varianten der Module 5 und 6 sollte jeweils nur eine abgestimmt werden, abhängig davon welche Module vorher bereits angenommen wurden.

\subsection*{Einnahmen}
\wahlprogramm{Einnahmenseite}\label{wp:finanzen:einnahme1}
\antrag{Silberpappel}\konkurrenz{wp:finanzen:einnahme2}\version{03:22, 20. Jun. 2010}

\subsubsection{Erbschaftssteuer}
\abstimmung
Wir wollen, dass sich Rheinland-Pfalz im Bundesrat für eine Erhöhung der Erbschaftssteuer einsetzt, um zusätzliche Einnahmen zu generieren.

\subsubsection{Vermögenssteuer}
\abstimmung
Wir wollen, dass sich Rheinland-Pfalz im Bundesrat für die Wiedereinführung einer Vermögenssteuer einsetzt, um zusätzliche Einnahmen zu generieren.

\subsubsection{Erbschaftssteuer + Vermögenssteuer}
\abstimmung
Wir wollen, dass sich Rheinland-Pfalz im Bundesrat für eine Erhöhung der Erbschaftssteuer und die Wiedereinführung einer Vermögenssteuer einsetzt, um zusätzliche Einnahmen zu generieren.

\subsubsection{Familienunternehmen}
\abstimmung
Bei der Gestaltung ist darauf zu achten, dass Familienunternehmen, die Arbeitsplätze schaffen und erhalten, nicht zusätzlich belastet werden.
 
\wahlprogramm{Einnahmenseite}\label{wp:finanzen:einnahme2}
\antrag{marcus}\konkurrenz{wp:finanzen:einnahme1}\version{03:22, 20. Jun. 2010}

\subsubsection{Vermögenssteuer}
\abstimmung
Wir wollen, dass sich Rheinland-Pfalz im Bundesrat für die Wiedereinführung einer Vermögenssteuer unter Beachtung der v.BGH bedungenen Gleichbehandlung von immobilem und mobilem Vermögen, einsetzt. Begründung: 1996 brachte die letztmalig erhobene Vermögensteuer umgerechnet ca 4,6 Mrd Euro Einnahmen ein. Die Vermögenssteuer ist auf alle Vermögen über Euro 500.000 zu erheben, mit Ausnahme von Land- + Forstwirtschaft, da dort eine schleichende Enteignung drohen könnte.

\subsubsection{Erbschaftssteuer}
\abstimmung
Die Erbschaftssteuer ist auf alle Erbfälle, auch Betriebe auszudehnen. Bei betrieblichen Erbschaften ist den Erben, sofern der Betrieb weitergeführt wird, eine Ratenzahlung der Erbschaftssteuer auf begründeten Antrag (um die Existenz des Betriebes nicht zu gefährden) bis zu 20 Jahren zu ermöglichen. Ausnahmen: Land- und Forstwirtschaft.

\subsubsection{Börsenumsatzsteuer}
\abstimmung
Das Land Rheinland-Pfalz möge sich im Bundesrat für eine Wiedereinführung der Börsenumsatzsteuer v. 0,25\% auf Börsenumsätze einsetzen. Begründung: Diese Steuer bringt mehrere 100 Millionen Euro Einnahmen im Jahr, und wird in gleicher Höhe wieder vom größten Finanzplatz Europas, London erhoben.
 
\subsection*{Vereinfachung des Steuersystems}
\wahlprogramm{Vereinfachung des Steuersystems}
\antrag{Silberpappel}\version{03:22, 20. Jun. 2010}

\subsubsection{Erbschaftssteuer}
\abstimmung
Die Piratenpartei Rheinland-Pfalz setzt sich für eine deutliche Vereinfachung des Steuersystems ein. Nur ein einfaches, transparentes Steuersystem kann für jeden Bürger verständlich und damit gerecht sein.

\subsubsection{Ausnahmen verringern}
\abstimmung
Ausnahmen im Steuersystem müssen deutlich reduziert werden.

\subsubsection{Steuersparmodelle}
\abstimmung
Paragraf 15b des Einkommensteuergesetzes verbietet "Steuerstundungsmodelle" nach einem vorgefertigten Konzept. Davon sind hauptsächlich standardisierte Finanzprodukte für Kleinanleger betroffen, nicht aber maßgeschneiderte Steuersparmodelle für außergewöhnlich vermögende Bürger. Wir wollen uns über den Bundesrat dafür einsetzen, das Verbot auch auf maßgeschneiderte Steuersparmodelle zu erweitern.

\subsubsection{Umleiten von Gewinnen}
\abstimmung
Wir wollen erreichen, dass Tricks zur Steuerersparnis, wie das Umleiten von Unternehmensgewinnen in Steueroasen, verboten oder durch geeignete Maßnahmen uninteressant gemacht werden.
 
\textbf{Infos auf der Diskussionsseite}

\subsection*{Verbesserte Steuerprüfung}
\wahlprogramm{Verbesserte Steuerprüfung}\label{wp:finanzen:steuerpruefung}
\antrag{Silberpappel}\version{03:22, 20. Jun. 2010}

\subsubsection{Einleitung}
\abstimmung
Den öffentlichen Haushalten gehen durch Steuerbetrug Milliarden an Einnahmen verloren. Neben dem Personalmangel bei der Bekämpfung von Steuerhinterziehung sind beispielsweise Betriebsprüfer zu sehr kurzen Prüfzeiten bei den Betrieben angehalten, mit der Folge, dass Steuerhinterziehung häufig nicht aufgedeckt und somit geahndet werden kann.

\subsubsection{mehr Steuerprüfer einstellen}
\abstimmung
Jeder Steuerprüfer bringt ein Vielfaches an Einnahmen, verglichen mit dem, was er kostet. Deshalb wollen wir die Zahl der Steuerprüfer in Rheinland-Pfalz erhöhen.

\subsubsection{Steuerprüfer für Steuergerechtigkeit}
\abstimmung
Dies dient auch der Steuergerechtigkeit.

\subsubsection{Prüfzeiten in Großbetrieben}
\abstimmung
Die Prüfzeiten sollen in Großbetrieben ausgeweitet werden, um eine ausreichende Prüfung zu gewährleisten.

\subsubsection{Umsatzsteuerprüfungen}
\abstimmung
Die Umsatzsteuerprüfungen sollen durch Bereitstellung von Steuerprüfern des Landes gestärkt werden. Bereits existierende Zusagen und Vereinbarungen mit dem Bund sollen konsequent umgesetzt werden.

\subsubsection{Unabhängigkeit der Steuerprüfer}
\abstimmung
Wir setzen uns dafür ein, dass Steuerprüfer wirklich unabhängig arbeiten können.
 
\wahlprogramm{Verbesserte Steuerprüfung}
\antrag{marcus}\zusatz{wp:finanzen:steuerpruefung}\version{03:22, 20. Jun. 2010}

\subsubsection{Prüfungszeiträume für Einkommensmillionäre}
\abstimmung
Wir fordern, dass Einkommensmillionäre (Einkommen > 500.000 Euro) regelmäßig und vollständig geprüft werden. Durchschnittlich muss jeder geprüfte Einkommensmillionär Euro 135.000 Steuer nachzahlen (Quelle \href{http://www.welt.de/wirtschaft/article128891/Milliarden_Verlust_durch_fehlende_Steuerpruefung.html}{[1]}) Das gibt bei ca 15.000 Einkommensmillionären ca Euro 2 Mrd Mehreinnahmen für den Zeitraum v. 3 Jahren = ca 733 Millionen pro Jahr.
 
\subsection*{Staatsleistungen an Kirchen beenden}
\wahlprogramm{Staatsleistungen an Kirchen beenden}
\antrag{KV Trier/Trier-Saarburg}\version{03:22, 20. Jun. 2010}

\subsubsection{Staatsleistungen an Kirchen beenden}
\abstimmung
Die Länder zahlen jährlich ca. 400-500 Millionen Euro an die Kirchen, hauptsächlich für die Gehälter von Bischöfen und anderen Geistlichen. In Rheinland-Pfalz wurden dafür im aktuellen Landeshaushalt etwa 50 Millionen Euro veranschlagt. Viele Kommunen in Rheinland-Pfalz müssen darüber hinaus aufgrund jahrhundertealter Verträge eigene Zahlungen an Kirchengemeinden leisten. Wir möchten diese Zahlungsverpflichtungen von Land und Kommunen gesetzlich beenden und die Mittel in anderen Bereichen einsetzen.
