\section{Grundsatzprogramm	}
\grundsatzprogramm{Öffentliche Infrastruktur}
\antrag{MAoAm}\\
\version{23:40, 14. Apr. 2012}

\paragraph{Antragstext}:

Der Landesparteitag möge in sein Grundsatzprogramm folgenden Abschnitt einfügen: 

\einruecken{Die Piratenpartei Sachsen-Anhalt setzt sich für die Erhaltung und Verbesserung der öffentlichen Infrastruktur ein.

\textbf{Erreichbarkeit öffentlicher Verwaltung}

Die Verwaltung einzelner Gebietsgliederungen soll für den Bürger möglichst barrierefrei erreichbar sein. Neben der persönlichen Erreichbarkeit, beinhaltet dies auch die Bereistellung von Dienstleistungen auf elektronischem Weg.

\textbf{Gesundheitswesen}

Die medizinische Grundversorgung ist von zentralen öffentlichem Interesse. Daher darf es nicht durch Gewinnstreben dominiert werden. Sie soll und muss daher durch die öffentliche Hand flächendeckend sichergestellt werden. Privatisierungen von Kliniken oder Universitätskliniken lehnen wir ab.

\textbf{Strom-, Gas-, Wasser-, Abwasser- und Telekommunikationsversorgung, Straßen- und Schienennetz}

Jeder Bürger und jedes Unternehmen muss gleichberechtigten Zugang zu den Versogungs-, Entsorgungs- und Verkehrsnetzen erhalten. Dazu soll die um sich greifende Privatisierung gestoppt und wenn möglich privatisierte Netze in die öffentliche Hand zurückgeführt werden.

\textbf{ÖPNV}

Da der freie Zugang zu Bildung, Wissen, Information, Kultur und Verwaltung gewährleistet sein muss, setzt sich die Piratenpartei Sachsen-Anhalt für einen flächendeckenden ÖPNV ein.

\textbf{Bildung}

Der freie Zugang zu Bildung ist den Piraten ein essentielles Anliegen. Die Qualität und Quantität von Bildungseinrichtungen muss flächendeckend sichergestellt werden.}

% -----

\grundsatzprogramm{Grundsatz Position zum Wettbewerb zwischen Hochschulen }
\antrag{Ren\'{e} Meye und Michel Vorsprach }\\
\version{23:40, 14. Apr. 2012}

\paragraph{Antragstext}:

Der Landesverband möge folgenden Abschnitt an geeigneter Stelle in das Grundsatzprogramm einfügen:

\einruecken{Es gibt immer wieder Bestrebungen Hochschulen {\Gu}effizienter{\Go}, {\Gu}leistungsstärker{\Go} und {\Gu}wettbewerbsfähiger{\Go} zu gestalten. Dabei wird betont, Hochschulen sollen sich am vorgehen von Unternehmen orientieren. Ziel dieser Bestrebungen: Es sollen immer mehr Studenten in immer kürzerer Zeit {\Gu}ausgebildet{\Go} werden. Es sollen immer mehr (in der Wirtschaft direkt verwertbare) Forschungsergebnisse erzeugt werden.

\textbf{Unsere Position:} Wir lehnen diese Bestrebungen ab. Hochschulen und vor allem Universitäten sollen sich frei entfalten können. Diese Wettbewerbsorientierung führt von Bildung der Studenten zu einer reinen Ausbildung der Studenten. Studium ist mehr als fachliche Inhalte lernen. Diese Wettbewerbsfähigkeit führt zu einer reinen Erarbeitung von verwertbaren (meist ingeneurswissenschaftlichen / für Firmen erarbeiteten) Forschungsergebnissen und schränkt Grundlagenforschung und geisteswissenschaftliches vordenken ein.}

\begruendung{Plakativ: Wenn Universitäten wie FIrmen funktionieren. Wozu gibt es dann Universitäten?}
