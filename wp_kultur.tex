\section{Kultur}

\subsection*{Freiräume für Kultur}
Kultur und Kulturgüter schaffen Unterhaltung, Wissen und Identität. Durch Kultur wird auch eine individuelle Entfaltung der Menschen ermöglicht. Darum sollte sich die Kultur möglichst frei entwickeln können, sowie möglichst vielen Menschen zugänglich sein.

Die Piratenpartei möchte deshalb in Rheinland-Pfalz Freiräume für Kultur schaffen. Wir wollen eine Infrastruktur zur Verfügung stellen, die von möglichst vielen Menschen zur Entfaltung ihrer Kreativität genutzt werden kann. Die gezielte Förderung einzelner Kunstprojekte lehnen wir allerdings ab, da dies eine staatliche Beeinflussung der Entwicklung von Kunst und Kultur darstellen würde.

\subsection*{Kostenloser Zugang zu Kulturgütern}
Der Zugang zu Kultur und Kulturgütern soll möglichst vielen Menschen ermöglicht werden. Dazu gehört, dass die Eintrittspreise zu staatlichen Kulturgütern möglichst niedrig sein sollen. Die Piratenpartei will auch den Zugang zu möglichst vielen Kulturgütern kostenlos gestalten.

\subsection*{Rundfunk}
\textbf{(ACHTUNG: vgl. auch Anträge unter OpenAccess!)}

\wahlprogramm{Rundfunk}\label{wp:kultur:rundfunk1}
\antrag{KV Trier/Trier-Saarburg}\version{03:24, 20. Jun. 2010}

\subsubsection{Unabhängigkeit der öffentlichen Rundfunkanstalten}
\abstimmung
Artikel 5 des Grundgesetzes schützt die Pressefreiheit. Derzeit sind die öffentlich-rechtlichen Medienanstalten aber alles andere als unpolitisch. Etwa die Hälfte der Mitglieder des Verwaltungsrates und ein Großteil der Mitglieder des Rundfunkrates des SWR werden direkt von den Landesregierungen von Baden-Württemberg und Rheinland-Pfalz bestimmt. Auch im ZDF sind knapp die Hälfte der Verwaltungs- und Fernsehräte von der Politik bestimmt.

\subsubsection{Rundfunkgebühreneinzug reformieren (GEZ abschaffen)}
\abstimmung
Wir setzen uns im Land und über den Bundesrat dafür ein, die GEZ mittelfristig überflüssig zu machen und abzuschaffen. Der öffentlich-rechtliche Rundfunk soll stattdessen über eine Abgabe für alle steuerpflichtigen Privatpersonen und Unternehmen finanziert werden, die von den Finanzämtern eingezogen wird. Auch hat sich der Umgang der GEZ mit persönlichen Daten als problematisch erwiesen. Deshalb soll sie, solange sie noch besteht, durch die Datenschutzbeauftragten des Landes und Bundes überwacht werden.
 
\newpage
\wahlprogramm{Rundfunk}
\antrag{Silberpappel}\zusatz{wp:kultur:rundfunk1}\version{03:24, 20. Jun. 2010}

\subsubsection{Unabhängigkeit der öffentlichen Rundfunkanstalten}
\abstimmung
Wir wollen die öffentlich-rechtlichen Medienanstalten unabhängiger von der Politik machen.

Soll an das Ende des Moduls ''Unabhängigkeit der öffentlichen Rundfunkanstalten'' gehängt werden.

 
\wahlprogramm{Rundfunk}\label{wp:kultur:rundfunk4}
\antrag{KV Trier/Trier-Saarburg}\version{03:24, 20. Jun. 2010}

\subsubsection{Dauerhafte Verfügbarkeit von durch öffentlich-rechtliche Rundfunkanstalten erstellten Inhalten}
\abstimmung
Vom gebührenfinanzierten öffentlich-rechtlichen Rundfunk erstellte Inhalte sind seit Umsetzung des 12. Rundfunkänderungsstaatsvertrags nur kurze Zeit in den Mediatheken der Rundfunkanstalten abrufbar, obwohl sie auch dauerhaft von öffentlichem Interesse sind, da sie beispielsweise als Quelle für die politische Diskussion dienen. Sie sollten deshalb zeitlich unbegrenzt zur Verfügung gestellt werden. Dafür werden wir uns einsetzen.
 
\wahlprogramm{Rundfunk}
\antrag{Thomas Heinen}\zusatz{wp:kultur:rundfunk4}\version{03:24, 20. Jun. 2010}

\subsubsection{ }
\abstimmung
Wir fordern die sofortige Überarbeitung des Staatsvertrages mit dem Ziel, die Inhalte, die durch die Bürger finanziert werden, langfristig für jeden Menschen frei verfügbar zu machen. Jeder Bürger hat einen Anspruch auf diese Inhalte. Die gesetzlichen Verweildauerregelungen müssen daher genauso wie der Drei-Stufen-Test umgehend auf den Prüfstand.
 
\newpage
\wahlprogramm{Rundfunk}
\antrag{KV Trier/Trier-Saarburg}\version{03:24, 20. Jun. 2010}

\subsubsection{Freie Lizenzen für Inhalte der öffentlich-rechtlichen Rundfunkanstalten}
\abstimmung
Wenn die Allgemeinheit Fernseh- und Rundfunkprogramme bezahlt, soll sie diese auch uneingeschränkt nutzen können. Aus deutschen Rundfunkgebühren finanzierte Inhalte sollen deshalb unter freie Lizenzen gestellt werden.
 
\wahlprogramm{Rundfunk}
\antrag{Pirat aus RLP}\version{03:24, 20. Jun. 2010}

\subsubsection{Öffentlich-rechtliche Programme werbefrei gestalten} 	\abstimmung
Um die Neutralität der öffentlich-rechtlichen Programme sicherzustellen, müssen diese komplett werbefrei sein. Die bisherige Regelung führt dazu, dass gerade vor 20 Uhr versucht wird quotenträchtige Formate einzusetzen, um Werbeeinnahmen zu generieren. Dies ist nicht im Sinne des öffentlich-rechtlichen Informationsauftrags. Durch ein komplett werbefreies Angebot kann zudem die Akzeptanz einer Abgabe für die öffentlich-rechtlichen Programme erhöht werden.

\subsubsection{Finanzierung (Vorschlag 1)}
\abstimmung
Die Ausfälle der Werbeeinnahmen sollen durch strukturelle Einsparungen erreicht werden. Dabei ist sicherzustellen, dass diese nicht auf Kosten der Informationsqualität erreicht werden.

\subsubsection{Finanzierung (Vorschlag 2)}
\abstimmung
Die Ausfälle der Werbeeinnahmen sollen durch eine geringfügige Erhöhung der Gebühren finanziert werden.

\subsubsection{Finanzierung (Vorschlag 3)}
\abstimmung
Die Ausfälle der Werbeeinnahmen sollen sowohl durch strukturelle Einsparungen bei den öffentlich-rechtlichen Sendern selbst, als auch durch eine geringfügige Erhöhung der Gebühren finanziert werden. Die Programm- und Informationqualität muss trotz Einsparungen jederzeit sicher gestellt sein.
 
\subsection*{Reform des Landesgesetzes über den Schutz der Sonn- und Feiertage}
\wahlprogramm{Abschaffung des Tanzverbots}
\antrag{Silberpappel}\version{03:24, 20. Jun. 2010}

\subsubsection{Abschaffung des Tanzverbots}
Das Tanzverbot in Rheinland-Pfalz wird durch das „Landesgesetz über den Schutz der Sonn- und Feiertage“ geregelt. Wir wollen die Paragrafen 6, 7 und 8 streichen (Verbot von Versammlungen und Veranstaltungen, Verbot von Sportveranstaltungen, Verbot von Tanzveranstaltungen).

\subsubsection{ }
\abstimmung
Der Staat soll hier nicht in die Freiheit des Einzelnen eingreifen.

\subsubsection{ }
\abstimmung
Wir setzen uns dafür ein, das Tanzverbot aufzuheben.
 
\wahlprogramm{Teilnahme am kulturellen Leben für alle}
\antrag{KV Trier/Trier-Saarburg}\version{03:24, 20. Jun. 2010}

\subsubsection{Teilnahme am kulturellen Leben für alle}
\abstimmung
Wir wollen, dass alle Menschen am kulturellen Leben teilhaben können. Bei der Förderung kultureller Einrichtungen soll darauf geachtet werden, dass diese auf Barrierearmut achten und Angebote für sozial schwache Besucher bieten, zum Beispiel deutlich reduzierte Eintrittspreise.
 
\wahlprogramm{Öffentlicher Raum für alle}
\antrag{Silberpappel}\version{03:24, 20. Jun. 2010}

\subsubsection{Öffentlicher Raum für alle}
\abstimmung
Die Nutzungsmöglichkeiten des öffentlichen Raums für alle müssen verbessert werden. Die Innenstädte gehören auch spielenden Kindern und skatenden Jugendlichen. Wir möchten den Gebrauch öffentlicher Gebäude durch Bürgervereinigungen, Vereine und Kulturgruppen fördern und setzen uns für entsprechende Verbesserungen in Nutzungs- und Haftungsregelungen ein.
