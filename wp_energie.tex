\section{Energieversorgung}

\subsection*{Umbau der Energieversorgung}

\wahlprogramm{Ökologische Energieversorgung}
\antrag{unbekannt}\version{03:14, 20. Jun. 2010}

\subsubsection{Ökologische Energieversorgung} 
\abstimmung
Die Piratenpartei strebt in Rheinland-Pfalz eine möglichst dezentrale, nachhaltige und ökologische Energieversorgung an.

\subsubsection{Abschaltung Großkraftwerke}
Langfristig streben wir die Abschaltung aller Großkraftwerke in Rheinland-Pfalz an.

\subsubsection{Erneuerbare Energien}
Dazu sollen alle Arten der erneuerbaren Energien, die in Rheinland-Pfalz sinnvoll eingesetzt werden können, auch eingesetzt und ausgebaut werden.
 
\subsection*{Priorität für eine ökologische Energieversorgung}
\wahlprogramm{Umbau der Rheinland-Pfälzischen Energieversorgung}
\antrag{unbekannt}\version{03:14, 20. Jun. 2010}

\subsubsection{Umbau der Rheinland-Pfälzischen Energieversorgung}
\abstimmung
Die Piratenpartei strebt diesen Umbau der Rheinland-Pfälzischen Energieversorgung an, auch wenn dabei lokal Nachteile entstehen, wie zum Beispiel optische Beeinträchtigungen durch Windräder, oder die Gefahr leichter Erdbeben durch Geothermiekraftwerke.
 
\wahlprogramm{Priorität für eine ökologische Energieversorgung}
\antrag{Limbo}\version{03:14, 20. Jun. 2010}

\subsubsection{Priorität im Staat}
\abstimmung
Um dem Bürger und jedem Menschen in Rheinland-Pfalz und der Bundesrepublik mit bestem Beisspiel vorran zu gehen, sollte das Land jedes öffentliche Gebäude und wo immer möglich Strom aus ausschließlich regenerativen Energien beziehen. Dies wird natürlich die Landeskasse belasten, sollte aber als Investition in die Zukunft gesehen werden und könnte versucht werden durch höhere Parkgebühren in Innenstädten teilweise gegenzufinanzieren. Nebeneffekt wäre evtl. ein Wechsel der Autofahrer auf den ÖPNV.
 
\subsection*{Klein- statt Großkraftwerke}
\wahlprogramm{dezentrale Energieversorgung}
\antrag{unbekannt}\version{03:14, 20. Jun. 2010}

\subsubsection{dezentrale Energieversorgung}
\abstimmung
Um eine dezentrale Energieversorgung zu ermöglichen, setzt die Piratenpartei auch auf effiziente, kleine Kraftwerke, wie etwa Blockheizkraftwerke. Diese Kraftwerke wollen wir zu einem möglichst großen Teil mit vor Ort auf ökologischem Weg gewonnen Brennstoffen betreiben.
 
\wahlprogramm{Dezentrale Energieversorgung}
\antrag{Silberpappel}\version{03:14, 20. Jun. 2010}

\subsubsection{Dezentrale Energieversorgung}
\abstimmung
Ein wichtiger Aspekt moderner Energiepolitik ist die zunehmender Dezentralisierung der Energieerzeugung. Die damit einhergehende Unabhängigkeit von Großkraftwerken kann durch regionale / kommunale Energiegewinnung aus umweltfreundlichen Quellen (Wind, Sonne, Wasser, Geothermie, Biomasse (keine Nahrungsmittel)) erreicht werden.
\subsubsection{Geothermie}
\abstimmung
In Rheinland-Pfalz sehen wir vor allem gute Voraussetzungen für die Nutzung von Geothermie.

\subsubsection{Infrastruktur}
\abstimmung
Da eine stärkere Dezentralisierung der Strom- und Wärmeerzeugung eine angepasste Infrastruktur voraussetzt, sind neue Speicher- und Verteilungstechnologien nötig. Wir werden deren Entwicklung und Einsatz verstärkt fördern.
 
\subsection*{Netzneutralität bei Energienetzen}
\wahlprogramm{Netzneutralität bei Energienetzen}
\antrag{Silberpappel}\version{03:14, 20. Jun. 2010}

\subsubsection{Netzneutralität bei Energienetzen}
\abstimmung
Monopole von Konzernen in der Energienetz-Infrastruktur behindern den Wettbewerb. Deshalb streben wir eine eigentumsrechtliche Entflechtung der Stromnetze an. Dadurch kann ein diskriminierungsfreier Zugang für alle Energieerzeuger garantiert werden, besonders auch für regionale Kleinerzeuger.
