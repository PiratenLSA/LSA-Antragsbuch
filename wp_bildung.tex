\section{Bildung}

\subsection*{Präambel: Wert von Bildung und finanzielle Mittel}
\wahlprogramm{Präambel: Wert von Bildung und finanzielle Mittel}
\antrag{Piraten aus RLP}\version{03:56, 20. Jun. 2010}\\
\abstimmung

\subsubsection{Präambel: Wert von Bildung und finanzielle Mittel}
Wir sehen Bildung als unabdingbares Menschenrecht und fordern Chancengleichheit und den freien Zugang zu Informationen und Bildung für alle Menschen sowie eine demokratische Organisation der Lehr- und Lerneinrichtungen. Wir fordern einen massiven Ausbau der Investitionen ins Bildungssystem und die Gewährleistung freien, selbstbestimmten Lernens im gesamten Bildungsweg. In einer global vernetzten Wissensgesellschaft ist Bildung die wichtigste Ressource eines jeden Menschen und Voraussetzung für die freie Entfaltung seiner Persönlichkeit. Sie garantiert seine Entwicklung zum freien und mündigen Bürger. Die kulturellen und persönlichen Entfaltungsmöglichkeiten der Menschen basieren auf dem allgemeinen Bildungsniveau sowie der persönlichen Qualifizierung jedes Bürgers. Um ein hohes Bildungsniveau erreichen und halten zu können, müssen die finanziellen Mittel des Bildungssystems erhöht werden und Priorität vor allen anderen Ausgaben erhalten. Alle anderen Aufgaben haben hinter der Bildung zurückzustehen.

\wahlprogramm{Präambel: Wert von Bildung und finanzielle Mittel}
\antrag{Niemand13}\version{03:56, 20. Jun. 2010}

\subsubsection{Investitionen in Bildung aufstocken}
\abstimmung
Der prozentuale Anteil der Ausgaben für den Bereich Bildung am gesamten Bruttoinlandsprodukt sinkt jährlich. Wir fordern drastische Investitionssteigerungen, um gute Bildung für jedermann zu ermöglichen. Wir fordern die Einstellung neuen Lehrpersonals an Hochschulen in ausreichender Zahl, um sowohl allen Studieninteressierten einen Platz in dem von Ihnen gewünschten Fach und Abschluss zur Verfügung stellen zu können, als auch allen Studierenden eine individuelle Betreuung durch die Leiter der jeweiligen Lehrveranstaltungen zu gewährleisten. Die an Forschung und Lehre Beteiligten müssen besser entlohnt werden. Die Entwicklung rückläufiger Investitionen in Universitäten wollen wir stoppen. Schlechte Lernbedingungen und prekäre Beschäftigung werden wir nicht dulden. Im Bereich der Schulen fordern wir die Einstellung von mehr Lehrern und kleinere Klassen von maximal 20 Schülern pro Klasse. Wir fordern die Abschaffung des Studienkontenmodells, das finanziell Schwächere in der Durchführung und am erfolgreichen Abschluss eines Studiums effektiv benachteiligt. Die verfassungswidrige Landeskinderregelung muss ersatzlos aus dem Landeshochschulgesetz gestrichen werden. Durch ausreichende Möglichkeiten für Teilzeit- und Abendstudien wollen wir auch Berufstätigen und anderweitig zeitlich Belasteten ein Studium ermöglichen.

\newpage
\subsubsection{Gebührenfreiheit sichern}
\abstimmung
Gebühren jeglicher Art sowie finanzielle und personelle Engpässe – gerade an den Hochschulen – schränken den Zugang zu Bildung ein und werden deshalb von uns kategorisch abgelehnt. Ein Studium ohne Abhängigkeit von Krediten und ohne Schuldenberg nach Studienabschluss muss gewährleistet sein. Wir fordern daher die gesetzlich verankerte Gebührenfreiheit und einen drastischen Ausbau der Investitionen in Schule und Hochschule: Das Bildungsangebot darf sich nicht weiter den knappen Ausgaben anpassen, sondern wir wollen die Ausgaben im Bildungsbereich an die Notwendigkeiten angleichen! Ein Studium ohne Abhängigkeit von Krediten muss gewährleistet sein. Eine private Finanzierung öffentlicher Bildungseinrichtungen muss stets kritisch hinterfragt werden. Ein Einfluss auf Lehrinhalte muss ausgeschlossen sein. Einer Kommerzialisierung von Schulen und Hochschulen stellen wir uns entschieden entgegen. Exzellenzinitiativen wollen wir kritisch überprüfen, damit sich nicht in Konkurrenz um Fördergelder nur noch wenige Hochschulen gute Lehre und Forschung leisten können.
 
\subsection*{Lehr- und Lernmittelfreiheit und Open Access für Rheinland-Pfalz!}
\wahlprogramm{Lehr- und Lernmittelfreiheit und Open Access für Rheinland-Pfalz!}\label{wp:bildung:freiheit1}
\antrag{Niemand13}\zusatz{wp:bildung:freiheit2}\version{03:56, 20. Jun. 2010}

\subsubsection{Lehr- und Lernmittelfreiheit und Open Access für Rheinland-Pfalz!}
\abstimmung
Zum ersten Mal in der Geschichte der Menschheit besteht die Möglichkeit unser komplettes Wissen zu sammeln, zu speichern und für die Allgemeinheit zugänglich zu machen. Gerade im Bereich der Forschung und Lehre bieten sich hier ungeahnte Möglichkeiten. Leider werden diese stark beschnitten. Wir fordern eine vollständige Lern- und Lehrmittelfreiheit für Rheinland-Pfalz. Die Verwendung und das Schaffen von freien Werken zur Vermittlung von Wissen müssen vom Land unterstützt und ausgebaut werden. Freie Werke sind nicht nur kostenfrei im Unterricht einsetzbar, sondern ermöglichen dazu dem Lehrenden ohne rechtliche Hürden die Lernmittel auf seinen Unterricht anzupassen. Alle in den Bibliotheken bereitstehenden Bücher und Zeitschriften sollen, auch in digitaler Form, für die Studierenden und Mitarbeiter frei zugänglich und verfügbar sein. Das Problem nicht bereitstehender oder auch nicht auffindbarer Bücher würde damit gelöst. Aufwendige Fernleihen müssen der Vergangenheit angehören. Die Publikationen aus staatlich finanzierter oder geförderter Forschung und Lehre werden oft in kommerziellen Verlagen publiziert, deren Qualitätssicherung von ebenfalls meist staatlich bezahlten Wissenschaftlern im Peer-Review-Prozess übernommen wird. Die Publikationen werden jedoch nicht einmal Bibliotheken der Forschungseinrichtungen kostenlos zur Verfügung gestellt. Wir dulden nicht, dass der Steuerzahler für Produktion, Qualitätssicherung und Nutzung insgesamt dreifach für die Kosten der Publikationen im Milliardenbereich aufkommt. Wir fordern, dass alle wisschenschaftlichen Publikationen, die aus öffentlich geförderter Forschung hervorgehen, auch allen Bürgern kostenfrei zur Verfügung stehen.

\subsubsection{Open Access-Offensive für Rheinland-Pfalz!}
\abstimmung
Wir unterstützen die Berliner Erklärung der Open-Access-Bewegung und fordern die Zugänglichmachung des wissenschaftlichen und kulturellen Erbes der Menschheit über das Internet nach dem Prinzip des Open Access. Wir sehen es als Aufgabe des Staates an, dieses Prinzip an den von ihm finanzierten und geförderten Einrichtungen durchzusetzen. Wir fordern den Einsatz offener Software in Forschung und Lehre. Software ist Wissen und wir wollen nicht länger Millionen an Steuergeldern für geschlossene und intransparente Systeme ausgeben. Mit der Förderung und dem Einsatz von offener Software wollen wir für Transparenz an den Hochschulen, für Erweiterbarkeit der System durch Interessierte und für die Förderung von kleinen und mittelständischen Unternehmen im Land sorgen. Software muss Studierenden und Mitarbeitern an jedem Hochschulrechner, zumindest als Alternative, angeboten werden. Auch bei der Neuanschaffung von Programmen oder dem Neuaufbau von Systemen und Datenbanken wollen wir, dass Open-Source-Lösungen eingesetzt werden. Wir lehnen die Anschaffung proprietärer Software bei existierenden Open-Source- Alternativen grundsätzlich ab. Studierende dürfen im Rahmen ihres Studiums nicht zur Nutzung oder gar zur Anschaffung bestimmter proprietärer Software genötigt werden, genauso wenig wie Mitarbeiter. Umfassende Kooperationsverträge mit Software- Monopolisten lehnen wir ab. Im Rahmen des „Open Date“ sollen Hochschulen all ihre Daten über offene, standardisierte Schnittstellen allen Interessierten kostenlos zur Verfügung stellen.

\wahlprogramm{Lernmittelfreiheit - Für eine kostenlose Schulbildung}\label{wp:bildung:freiheit2}
\antrag{Piraten aus RLP}\zusatz{wp:bildung:freiheit1}\version{03:56, 20. Jun. 2010}

\subsubsection{Lernmittelfreiheit - Für eine kostenlose Schulbildung}
\abstimmung
Damit auch sozial schwache Kinder nicht benachteiligt werden und da nach unsere Überzeugung Schule kostenlos sein muss, fordern wir eine komplette Lernmittelfreiheit für Rheinland-Pfalz. Des weiten wollen wir die Erstellung von kostenlosen Lernmaterialien als Alternative zu den kommerziellen Lernmaterialien fördern.
 
\wahlprogramm{Lehrmittel}\label{wp:bildung:lehrmittel1}
\antrag{KV Trier/Trier-Saarburg}\konkurrenz{wp:bildung:lehrmittel2}\version{03:56, 20. Jun. 2010}

\subsubsection{Einsatz von Lehrmitteln unter freien Lizenzen}
\abstimmung
Wir wollen, dass an Bildungseinrichtungen Lehrmittel mit freien Lizenzen verwendet werden. Dies trägt zur Kostensenkung bei.

\subsubsection{Mehr Nutzung von freier Software}
\abstimmung
Freie Software ist kostengünstiger für Schulen und Eltern. Der Zugang ist damit in jedem Haushalt mit Computer gesichert.
 
\newpage
\wahlprogramm{Lehrmittel}\label{wp:bildung:lehrmittel2}
\antrag{Piraten aus RLP}\konkurrenz{wp:bildung:lehrmittel1}

\subsubsection{Lizenzfreies Unterrichtsmaterial}
\abstimmung
Die Veröffentlichung von Unterrichtsmaterialien und Unterrichtsentwürfen unter freien Lizenzen und via Internet soll gefördert werden. Dies vereinfacht den Lehrkräften die Verwendung bestehender und die Erarbeitung neuer Unterrichtsmaterialien. Auf einer staatlich finanzierten Plattform soll den Lehrern der leichte Austausch und die gegenseitige Qualitätssicherung (beispielsweise durch eine Begutachtung seitens mehrerer Kollegen (peer-review)) ermöglicht werden.

\subsubsection{Schulbücher unter offner Lizenz}
\abstimmung
\begin{enumerate}
\item Die Erstellung von Schulbüchern unter freier Lizenz (z.b. GPL) soll staatlich gefördert werden.
\item Die Autorenleistungen, für die jeweilige Erstellung und Aktualisierung, werden hierbei jeweils einmalig durch das Land finanziert, sodass eine jeweilige dauerhafte Vergütung pro Medium entfällt.
\item Interessierte haben die Möglichkeit an den freien Produkten mitzuarbeiten und sie nach Belieben zu verändern und zu verbessern.
\item Die Qualität der Einsendungen wird durch eine Begutachtung seitens mehrerer Kollegen (peer-review) sichergestellt. Auf Qualität geprüfte Versionen werden für alle Nutzer erkennbar zertifiziert.
\item Eine Veröffentlichung soll immer sowohl in Digital-, als auch als Papierform erfolgen. Druckversionen der Medien werden zum Selbstkostenpreis angeboten. Sofern das Schulbuch von einer Klasse verwendet wird, muss dieses den jeweiligen Schülern als kostenfreies Printexemplar zur Verfügung gestellt werden.
\end{enumerate} 

\subsection*{Bildungseinrichtungen demokratisieren!}
\wahlprogramm{Bildungseinrichtungen demokratisieren!}\label{wp:bildung:demokratie1}
\antrag{Niemand13}\version{03:56, 20. Jun. 2010}
\begin{itemize}
\item \konkurrenz{wp:bildung:demokratie2}
\item \konkurrenz{wp:bildung:demokratie3}
\end{itemize}

\subsubsection{Bildungseinrichtungen demokratisieren!}
\abstimmung
Bildungseinrichtungen sind für SchülerInnen und StudentInnen ein prägender und umfassender Teil des Lebens. Sie sind deswegen als Lebensraum der Lernenden zu begreifen, der durch sie mitbestimmt werden muss. In Schulen müssen SchülerInnen ein Mitspracherecht bei der Gestaltung ihres Schulalltags haben. Demokratische Werte müssen vermittelt und vor gelebt werden, um die Akzeptanz der Entscheidungen zu erhöhen, das Gemeinschaftsgefühl zu stärken und selbstbestimmtes Lernen im ausreichenden Maße zu ermöglichen. Wir fordern eine grundlegende demokratische Organisation von Schule und Hochschule.

\subsubsection{Hochschulrat abschaffen - Hochschulen demokratisch gestalten!}
\abstimmung
Bei den Universitäten stellen sowohl das bestehende Ungleichgewicht zugunsten des Hochschulrats, als auch die geplante Novelle des Landeshochschulgesetzes eine Entmündigung der breiten Mehrheit zugunsten nicht gewählter Gremienvertreter und des Präsidialamts dar. Was als „Autonomie der Hochschule“ angepriesen wurde, verkehrt sich in ihr Gegenteil: Hochschulen verlieren die Unabhängigkeit, welche für die Erfüllung ihrer Aufgaben, die eine freiheitlich-demokratische Gesellschaft ihnen übertragen hat, unentbehrlich ist. Demokratische Entscheidungsstrukturen dürfen nicht weiter durch wirtschaftliche Einflüsse oder die Etablierung autoritärer Strukturen beeinträchtigt und unterwandert werden. Wir fordern die Abschaffung des Hochschulrates und die Übertragung aller Kompetenzen auf den Senat. Die unabhängige Mitwirkung aller Interessengruppen in den demokratischen Willensbildungsprozessen der Hochschulen muss gesichert werden und sich im Hochschulgesetz widerspiegeln. Studentischen VertreterInnen sollen aufgrund der Größe der Studierendenschaft mit einer Drittelparität in allen entscheidungsbefugten Gremien vertreten sein.

\subsubsection{Beabsichtigtes Landeshochschulgesetz stoppen!}
\abstimmung
Verschärft wird die oben aufgezeigte Entwicklung durch das neue Landeshochschulgesetz (LHG), das unter dem Deckmantel der Autonomie der Hochschule demokratische Grundstrukturen unterminiert: Die Entmachtung demokratischer Gremien und der Ausbau präsidialer Entscheidungskompetenzen, die Begünstigung der Trennung von Forschung und Lehre sowohl durch die Einrichtung von Forschungskollegs, als auch durch die Möglichkeit der Freistellung von ProfessorInnen von der Lehre für bis zu 10 Jahre, und die Schaffung von Einfallstoren für Unternehmen durch die Gründung von Hochschulverbünden und außeruniversitären Betrieben, die auch Privatunternehmen offen stehen. Wir dagegen fordern, dass VertreterInnen der Studierendenschaft in den entscheidungsbefugten, universitären Gremien nicht länger untervertreten sind und lehnen die beabsichtigte Novelle des LHG in der derzeitigen Form ab.
 
\subsection*{Bildungseinrichtungen demokratisieren!}
\wahlprogramm{Bildungseinrichtungen demokratisieren!}\label{wp:bildung:demokratie2}
\antrag{Piraten aus RLP}\version{03:56, 20. Jun. 2010}
\begin{itemize}
\item \konkurrenz{wp:bildung:demokratie1}
\item \konkurrenz{wp:bildung:demokratie3}
\end{itemize}

\subsubsection{Mehr Demokratie an Schulen wagen, Schülervertretungen stärken (Variante 1)}
\abstimmung
Um als demokratiekomptenter Bürger aufzuwachsen ist es wichtig, dass Schüler schon frühzeitig Demokratie-Lernen und erfahren. Die Schule sollte sie dabei unterstützen. Wir fordern die Ausweitung der Demokratie an Schulen und die verstärkte Mitbestimmung der Schülerschaft (bspw. durch die flächendeckende Einführung von Schülerparlamenten). Schülervertretungen müssen besser über ihre Rechte informiert und geschult werden.

\subsubsection{Mehr Demokratie an Schulen wagen, Schülervertretungen stärken (Variante 2 [aus BW-Programm])}
\abstimmung
Die gelebte Vermittlung der Grundprinzipien unserer demokratischen Staats- und Gesellschaftsform ist eine der Aufgaben staatlicher Bildungseinrichtungen. An allen rheinland-pfälzischen Schulen sollen deshalb schrittweise Klassenräte und Schülerparlamente eingeführt werden. Durch die frühe Möglichkeit, sich an (schul-)politischen Entscheidungen zu beteiligen und Themen zu erarbeiten, soll unter anderem der Politikverdrossenheit unter Jugendlichen vorgebeugt werden. Außerdem können Kinder und Jugendliche demokratische Prinzipien und Werte auf diese Art und Weise kennen und schätzen lernen, wodurch sie kritischer mit extremistischem Gedankengut umgehen können.
 
\subsection*{Bildungseinrichtungen demokratisieren!}
\wahlprogramm{Demokratisierung der Bildung}\label{wp:bildung:demokratie3}
\antrag{KV Trier/Trier-Saarburg}\version{03:56, 20. Jun. 2010}
\begin{itemize}
\item \konkurrenz{wp:bildung:demokratie1}
\item \konkurrenz{wp:bildung:demokratie2}
\end{itemize}

\subsubsection{Demokratisierung der Bildung}
\abstimmung
Wir setzen uns für eine Demokratisierung der Schul- und Bildungslandschaft ein. Wir wollen die Demokratisierung des Bildungsbereichs unter anderem durch weitergehende Rechte für die Schülermitverwaltungen und die Studentenschaften erreichen.
 
\subsection*{Freies, individuelles Lernen ermöglichen!}
\wahlprogramm{Freies, individuelles Lernen ermöglichen!}\label{wp:bildung:frei1}
\antrag{Niemand13}\konkurrenz{wp:bildung:frei2}\version{03:56, 20. Jun. 2010}

\subsubsection{Schutz vor Überwachung}
\abstimmung
Jeder Mensch ist ein Individuum mit persönlichen Neigungen, Stärken und Schwächen. Institutionelle Bildung soll daher den Einzelnen unterstützen seine Begabungen zu entfalten, Schwächen abzubauen und neue Interessen und Fähigkeiten zu entdecken. Neben starren Lehr- und Stundenplänen, werden vor allem einige Formen der Leistungsbewertung diesen Forderungen nicht gerecht. Insbesondere die Bewertung von Verhalten nach einem vorgegebenen Normenraster z.B. durch Kopfnoten lehnen wir ab. Für ein freies Lernen und Lehren ist der Schutz vor Überwachung und Zensur unabdingbare Voraussetzung. Wer sich beobachtet fühlt oder nicht mehr sicher weiß, wer was über ihn weiß, der wird sein Verhalten anpassen und sich in seinem Lehr- und Lernprozess nicht frei entfalten.

\subsubsection{Schutz vor Zensur und Informationskontrolle}
\abstimmung
Eine Zensur behindert den Zugang zu Information, zu Wissen und zu Demokratie und wird von uns daher aufs Schärfste bekämpft. Wir fordern den uneingeschränkten Zugang zu allen Informationen.

\subsubsection{Kontrolle von Datenverarbeitung der Lernenden}
Ü\abstimmung
berwachung - auch in Form von Data-Warehousing-Systemen, in denen massenhaft Studierendendaten gespeichert, gesammelt und ausgewertet werden - lehnen wir ab. Für alle Systeme, die personenbezogene Daten von Lernenden oder Lehrenden verarbeiten, fordern wir maximale Transparenz, Nachvollziehbarkeit bzgl. der Datenabfragen und wirksame organisatorische und technische Maßnahmen zum Schutz vor Missbrauch. Verwaltungssysteme müssen auch stets die Lehre unterstützen und dürfen keinesfalls von sich aus Auswirkungen auf die Gestaltung des Lehrbetriebs nehmen.

\subsubsection{Barrierefreiheit}
\abstimmung
Eine Barrierefreiheit setzen wir für alle Systeme
 
\wahlprogramm{Freies, individuelles Lernen ermöglichen!}\label{wp:bildung:frei2}
\antrag{KV Trier/Trier-Saarburg}\konkurrenz{wp:bildung:frei1}\version{03:56, 20. Jun. 2010}

\subsubsection{Bildung als Teil der individuellen Entwicklung}
- zurückgezogen, da gleich \ref{wp:bildung:frei1} -
 
\wahlprogramm{Persönlichkeitsrechte von Schülern und Lehrern achten}
\antrag{KV Trier/Trier-Saarburg}\version{03:56, 20. Jun. 2010}

\subsubsection{Persönlichkeitsrechte von Schülern und Lehrern achten}
\abstimmung
Die Privat- und Intimsphäre sowie das Recht auf informationelle Selbstbestimmung von Schülern und Lehrern müssen gewahrt bleiben. Videoüberwachung und private Sicherheitsdienste haben keinen Platz in Schulen. Präventive Durchsuchungen und Kontrollen oder Urinuntersuchungen sind zu unterlassen. Die Unschuldsvermutung gilt auch für Schüler. Diese unter Generalverdacht zu stellen, zerstört das Vertrauen zu Schule und Lehrern, ohne welches Unterricht und Erziehung aber nicht möglich sind.
 
\wahlprogramm{Freies, individuelles Lernen ermöglichen!}
\antrag{Piraten aus RLP}\version{03:56, 20. Jun. 2010}

\subsubsection{Individuelle Bildung}
\abstimmung
Derzeit ist das Bildungsangebot in vielen Hinsichten stark eingeschränkt und umfasst wenig Spielraum für die optimale Entfaltung der eigenen Fähigkeiten. Daher sollen Maßnahmen gefördert werden, die die Auswahl von Bildungsangeboten erhöht.

\subsubsection{Lebenslanges Recht auf Bildung}
\abstimmung
Das Recht auf Bildung soll sich auf das gesamte Lebensalter erstrecken um die Möglichkeiten der Bürger für freie Selbstentfaltung und Lebensgestaltung zu ermöglichen. Bisher beschränkt sich die Ausbildung fast ausschließlich auf die jüngeren Generationen, älteren Menschen wird die Möglichkeit der Aus- und Weiterbildung derzeit nicht in dem selbem Maße zugestanden wie den Jüngeren.

\subsubsection{Individuelle Förderung}
\abstimmung
Jeder Schüler hat seine Individuellen Stärken, Schwächen und Bedürfnisse, werden diese nicht erkannt und gefördert verschlechtert sich das allgemeine Schulklima und die individuelle Leistungsfähigkeit wird nicht voll ausgeschöpft. Wir möchten eine bessere Förderung einzelner Schüler und deren Interessen. Dies kann durch Angebote wie Arbeitsgruppen Wahlpflichtfächer und Förderuntericht erreicht werden.
 
\newpage
\subsection*{Mehr Geld für Bildung}
\wahlprogramm{Mehr Geld für Bildung}\label{wp:bildung:geld1}
\antrag{Piraten aus RLP}\konkurrenz{wp:bildung:geld2}\version{03:56, 20. Jun. 2010}

\subsubsection{Mehr Geld für Bildung}
\abstimmung
Im Vergleich der Bundesländer ist das Land Rheinland-Pfalz eines der Bundesländer mit den niedrigsten Ausgaben für den Bereich Bildung. Wir fordern, dass der Bildung im Land Rheinland-Pfalz ein höherer Stellenwert zukommt. Darum wollen die finanzielle Ausstattung von Schulen und Universitäten verbessern.
 
\wahlprogramm{Finanzierung von Bildung und Forschung}\label{wp:bildung:geld2}
\antrag{KV Trier/Trier-Saarburg}\konkurrenz{wp:bildung:geld1}\version{03:56, 20. Jun. 2010}

\subsection*{Finanzierung von Bildung und Forschung}
\abstimmung
Für eine reiche Industrienation wie Deutschland ist es unverständlich, dass hier nur ein im internationalen Vergleich verschwindend geringer Teil der öffentlichen Mittel in Bildung und Forschung investiert wird. Bildung und Forschung sind eine Investition in die Zukunft unserer Gesellschaft und in jeden Menschen. Wir fordern daher eine bessere finanzielle Ausstattung von Bildungsstätten aller Art und gleichermaßen der Forschung mit staatlichen Mitteln. Schönrechnereien – wie die Einbeziehung von Lehrerpensionen – lehnen wir dabei ab.
 
\wahlprogramm{Wirtschaftlicher Bildungssoli}
\antrag{Piraten aus RLP}\version{03:56, 20. Jun. 2010}

\subsubsection{Wirtschaftlicher Bildungssoli}
\abstimmung
Bei der Finanzierung des Bildungssystems ist unsere komplette Gesellschaft gefordert. Vor allem unsere Wirtschaft profitiert maßgeblich von gut ausgebildeten Kräften. Wir möchten deshalb, dass sich die wirtschaftlichen Unternehmen durch eine zweckgebundene Bildungsabgabe an der Finanzierung unseres Bildungssystems noch stärker beteiligen und so ihren positiven Beitrag zur Zukunft Deutschlands leisten. Für Unternehmen wird sich diese Investition lohnen, da sie von den besser ausgebildeten, leistungsfähigeren Kräften später stark profitieren werden.
 
\newpage
\subsection*{Vereinbarkeit von Familie und Beruf}
\wahlprogramm{Vereinbarkeit von Familie und Beruf}
\antrag{Piraten aus RLP}\version{03:56, 20. Jun. 2010}

\subsubsection{Veränderungen im Umfeld der Kinder}
\abstimmung
Das Umfeld, in dem Kinder aufwachsen, verändert sich. Da in immer mehr Familien beide Elternteile berufstätig sind oder die Eltern der Kinder getrennt leben, muss sich das Bildungssystem an diese Verhältnisse anpassen. Es müssen zusätzliche Angebote geschaffen werden, welche die Eltern unterstützen und entlasten.

\subsubsection{Vereinbarkeit von Familie und Beruf}
\abstimmung
Darum muss das staatliche Betreuungsangebot für Kinder und Jugendliche ausgebaut werden, so dass Familie und Beruf vereinbar werden. Dabei darf es nicht vorkommen, dass die Betreuung lediglich eine beaufsichtigte Verwahrung der Kinder und Jugendlichen ist. Vielmehr müssen Möglichkeiten geschaffen werden, wie sich die Kinder und Jugendlichen entwickeln und entfalten können.
 
\wahlprogramm{Vereinbarkeit von Familie und Beruf}
\antrag{KV Trier/Trier-Saarburg}\version{03:56, 20. Jun. 2010}

\subsubsection{Familienfreundliche Ganztagesbetreuung an Schulen}
\abstimmung
Staatliche Bildungseinrichtungen sollen den Familien dabei helfen, die notwendige Flexibilität zu erreichen, den Anforderungen des Familien- und Berufslebens gerecht zu werden. Dafür soll an allen Schulen ein Angebot zur Ganztagesbetreuung geschaffen werden. Das Betreuungsangebot ergänzt den Unterricht um zusätzliche Bildungsmöglichkeiten und außerschulische Aktivitäten. Neben Wahlfächern, Hausaufgabenbetreuung und Nachhilfe soll ein möglichst breites Angebot an kulturellen oder sportlichen Tätigkeiten ermöglicht werden. Dabei ist die Zusammenarbeit mit Vereinen ausdrücklich erwünscht und zu beiderseitigem Vorteil.

\subsubsection{Schulspeisung}
\abstimmung
Eine gesunde Ernährung ist aus Gründen der körperlichen und geistigen Entwicklung und der Konzentrationsfähigkeit der Kinder wichtig.

Schulspeisungen können dabei helfen, dass sich Kinder ausgewogen und gesund ernähren. Wir fordern daher die Einführung gesunder und ausgewogener Schulspeisungen an allen Schulen und Kindertagesstätten.

Die Finanzierung dieser Schulspeisungen ist dabei so zu gestalten, dass alle Schüler unabhängig von der sozialen oder finanziellen Lage der Familie daran teilnehmen können. Zur Vermeidung sozialer Ausgrenzung sollen finanzielle Erleichterungen so gestaltet sein, dass andere Schüler nicht erfahren, wer gefördert wird.

Bei der Planung sollte auch berücksichtigt werden, ob die Verwaltungskosten für die Essensgebühren die Einnahmen übersteigen oder eine vollständig kostenlose Schulspeisung günstiger wäre.
 
\subsection*{Organisationsstruktur / Ausstattung}
\wahlprogramm{Gleichbehandlung der Träger}
\antrag{Piraten aus RLP}\version{03:56, 20. Jun. 2010}

\subsubsection{Gleichbehandlung der Träger}
\abstimmung
Konfessionelle, soziale, kulturelle oder sonstige Zugangsbeschränkungen sind in Einrichtungen, die (auch zu Teilen) öffentlich finanziert werden, nicht zulässig. Bei der öffentlichen Finanzierung von Einrichtungen sind alle Träger gleich zu stellen.
 
\wahlprogramm{Kein Schulsponsoring}
\antrag{Piraten aus RLP}\version{03:56, 20. Jun. 2010}

\subsubsection{Kein Schulsponsoring}
\abstimmung
Die Schule sollte ein neutraler Raum sein, indem sich die Schüler frei entwickeln können und der dementsprechend von der Gemeinschaft finanziert wird. Wir lehnen deshalb Schulsponsoring durch die Privatwirtschaft ab.
 
\wahlprogramm{Bessere Unterstützung der Kommunen beim Ausbau und Erhalt der Schulinfrastruktur}
\antrag{Piraten aus RLP}\version{03:56, 20. Jun. 2010}

\subsubsection{Bessere Unterstützung der Kommunen beim Ausbau und Erhalt der Schulinfrastruktur}
\abstimmung
Schulen sind ein Platz, an dem unsere Heranwachsenden viele Stunden ihres Lebens verbringen. Die Leistung der Heranwachsenden und ihr Verhältnis zur Institution Schule hängt nicht unmaßgeblich davon ab, ob sie sich dort wohlfühlen. Ein guter Zustand des Gebäudes und die bestmögliche Ausstattung können dazu maßgeblich beitragen. Wir fordern, dass das Land die Kommunen beim Ausbau und Erhalt der Schulinfrastruktur optimal unterstützt.
 
\wahlprogramm{Ein Laptop für jeden Schüler / Internet in jedem Klassensaal}
\antrag{Pirat aus RLP}\version{03:56, 20. Jun. 2010}

\subsubsection{Ein Laptop für jeden Schüler / Internet in jedem Klassensaal}
\abstimmung
Die Ausstattung mit digitalen Arbeitsmitteln und ein Internetzugang für alle Lernenden ist eine Grundvoraussetzung für den Zugang zur Informations- und Wissensgesellschaft und einer aktiven Teilhabe an dieser. Die Schulen müssen deshalb an die technischen Gegebenheiten unserer Zeit angepasst werden. Schüler und Lehrer sollten die Möglichkeit haben spontan und flexibel das Internet als ergänzendes Mittel des Unterrichts (bspw. bei Gruppenarbeiten) zu nutzen. Computerräume sind in ihrer Verfügbarkeit und Flexibilität stark begrenzt. Wir fordern möchten deshalb jeden Schüler bei seiner Einschulung mit einem schulischen Leih-Laptop bzw. Netbook ausstatten, welcher bei der Einschulung in die weiterführende Schule dann durch ein angepasstes Modell ersetzt wird.

\subsubsection{Gemeinsame Ausarbeitung der Laptopausstattung mit Schülern, Lehrern und Eltern}
\abstimmung
Die Laptop bzw. Netbookausstattung, insbesondere die Wahl der passenden Softwarepakete, findet gemeinsam mit den Schülern, Lehrern und Eltern statt, um sie optimal an die Wünsche und Gegegebenheiten anzupassen

\subsubsection{Internet in jedem Klassensaal}
\abstimmung
Um die Laptops umfassend nutzen zu können fordern wir die Ausstattung der kompletten Schulgebäude mit drahtlosem Internet.

\subsubsection{ }
\abstimmung
Für Lehrkräfte sollen Schulungen angeboten werden, mit denen sie sich bei Bedarf in diesem Bereich methodisch weiterbilden können.
 
\subsection*{Bessere Ausstattung von öffentlichen Bibliotheken}
\wahlprogramm{Bessere Ausstattung von öffentlichen Bibliotheken}
\antrag{Piraten aus RLP}\version{03:56, 20. Jun. 2010}

\subsubsection{Bessere Ausstattung von öffentlichen Bibliotheken}
\abstimmung
Obwohl zahlreiche Bibliotheken bereits erste Schritte auf dem Weg zu umfassenden Medien- und Informationszentren unternommen haben, sollten insbesondere Computerarbeitsplätze, Internetzugänge, Zugänge zu Datenbanken und umfangreiche Bestände mit neuen Informations-, Bildungs- und Unterhaltungsträgern weiter ausgebaut und effektiv finanziert werden, vor allem im ländlichen Raum.
 
\subsection*{Reduzierung des Unterrichtsausfalls}
\wahlprogramm{Reduzierung des Unterrichtsausfalls}
\antrag{Piraten aus RLP}\version{03:56, 20. Jun. 2010}

\subsubsection{Reduzierung des Unterrichtsausfalls}
\abstimmung
Eine besondere Aufgabe stellt die Reduzierung des Unterrichtsausfalls dar. In der Zukunft ist aufgrund des demographischen Wandels mit einem Sinken der Schülerzahlen zu rechnen. Es ist jedoch nicht im Sinne unserer Kinder, dass wir die Zeit bis dahin mit Unterrichtsausfall und Aushilfslehrern einfach aussitzen. Der Unterrichtsausfall kurzfristig reduziert werden. Jeder Schüler hat ein Anrecht darauf, dass der für ihn vorgesehene Unterricht stattfindet.

\subsubsection{Einstellen neuer Lehrkräfte und bessere Optimierung des Systems (Vorschlag 1 / ACHTUNG: Abhängig von beschlossener Klassengröße)}
\abstimmung
Wir möchten das System analysieren und organisatorische Optimierungsmöglichkeiten zur Reduzierung des Unterrichtsaufalls nutzen. Darüber hinaus soll der Mangel an Lehrern durch Neueinstellungen bei Bedarf ausgeglichen werden. Durch unser zukünftiges Ziel die Klassengröße auf maximal 20 Schüler zu begrenzen, werden wir auch in Zukunft eine hohe Anzahl an Lehrkräften benötigen.

\subsubsection{Einstellen neuer Lehrkräfte und bessere Optimierung des Systems (Vorschlag 2 / ACHTUNG: Abhängig von beschlossener Klassengröße)}
\abstimmung
Wir möchten das System analysieren und organisatorische Optimierungsmöglichkeiten zur Reduzierung des Unterrichtsaufalls nutzen. Darüber hinaus soll der Mangel an Lehrern durch Neueinstellungen bei Bedarf ausgeglichen werden. Durch unser Ziel die Klassengröße kurzfristig auf maximal 20 Schüler und längerfristig auf maximal 15 Schüler zu begrenzen, werden wir auch in Zukunft eine hohe Anzahl an Lehrkräften benötigen.

\subsection*{Säkularisierung der Bildung}
\wahlprogramm{Säkularisierung der Bildung}
\antrag{KV Trier/Trier-Saarburg}\version{03:56, 20. Jun. 2010}

\subsubsection{Säkularisierung der Bildung}
\abstimmung
Wo Menschen unterschiedlichen Glaubens zusammenleben, müssen staatliche Bildungseinrichtungen weltanschaulich neutral sein. Der bisher in Landesverfassung und Schulgesetz vorhandene Religions- und Gottesbezug sollte deswegen gestrichen werden.
 
\wahlprogramm{Ethik-Unterricht}
\antrag{KV Trier/Trier-Saarburg}\version{03:56, 20. Jun. 2010}

\subsubsection{Ethik-Unterricht}
\abstimmung
Wir möchten für alle Schüler Ethikunterricht flächendeckend bereits ab der ersten Klasse anbieten.
 
\newpage
\subsubsection{Bildungsstandards}
\wahlprogramm{Bildungsstandards}
\antrag{KV Trier/Trier-Saarburg}\version{03:56, 20. Jun. 2010}

\subsubsection{Bildungsstandards}
\abstimmung
Auf Basis bildungspolitischer Erkenntnisse und der Diskrepanz zu derzeit herrschenden Bildungs-Missständen in Deutschland fordern wir die zügige Umsetzung der Bildungsempfehlungen (vom Institut zur Qualitätsentwicklung im Bildungswesen, HU Berlin und der Kultusministerkonferenz der Länder) nach festgesetzten Bildungsstandards auf Bundes- und Länderebene. Zur Gewährleistung bundeseinheitlicher Bildungsstandards in allen Bundesländern übernimmt das ausführende Organ der Bundesregierung die qualitätsführende Kontrolle und Evaluation.

\subsubsection{Vergleichbarkeit und bundesweiter Rahmen}
\abstimmung
Um die Vorteile des föderativen Schulsystems mit den Vorteilen eines zentral geregelten Bildungssystems zu verbinden, fordern wir mehr Richtlinienkompetenzen für den Bund. Dies betrifft insbesondere die Bereiche Vergleichbarkeit von Abschlüssen, Strukturausgleich und Freizügigkeit.
 
\subsubsection{Kindergärten/Kindertagesstätten u. Vorschulen}
\wahlprogramm{Kindergärten}
\antrag{Piraten aus RLP}\version{03:56, 20. Jun. 2010}

\subsubsection{Freier Zugang zu Kindergärten und Kindertagesstätten}
\abstimmung
Eltern müssen Kindergärten bzw. Kindertagesstätten für ihre Kinder frei wählen können. Jedem Kind muss dazu bis zum Schuleintritt ein kostenloser Kindergartenplatz in einem staatlichen Kindergarten in der Nähe zur Verfügung stehen.

\subsubsection{Ganztagsbetreuung / private Kinderbetreuung}
\abstimmung
Außerdem muss eine staatliche Ganztagsbetreung unter gleichen Bedingungen z.B. in Kindertagesstätten gewährleistet sein. Auch alternative Betreuungsangebote wie private Kinderbetreuung in Kleingruppen müssen staatlich finanziert werden. Eltern müssen über das Angebot ausreichend informiert werden.
 
\wahlprogramm{Bessere Ausbildung und Bezahlung von Erziehern}
\antrag{Piraten aus RLP}\version{03:56, 20. Jun. 2010}

\subsubsection{Bessere Ausbildung und Bezahlung von Erziehern}
\abstimmung
Von Erziehern und Betreuern im vorschulischen Bereich wird immer mehr gefordert. Wir planen daher, Bezahlung sowie Aus- und Fortbildung der Arbeitenden den neuen Anforderungen anzupassen, um die stärkere Belastung zu berücksichtigen.
 
\wahlprogramm{Vorschulische Förderung}
\antrag{Piraten aus RLP}\version{03:56, 20. Jun. 2010}

\subsubsection{Vorschulische Förderung}
\abstimmung
Der vorschulischen Förderung von Kindern kommt eine zentrale Bedeutung zu. Sie muss gewährleisten, dass alle Kinder unabhängig von ihrer sozialen, finanziellen und kulturellen Herkunft mit guten Grundvoraussetzungen ihre Schullaufbahn beginnen können. Alle Fördermöglichkeiten müssen für Kinder und Eltern kostenlos und frei zugänglich angeboten werden.
 
\subsection*{Grundschulen}
\wahlprogramm{Grundschulen}
\antrag{Piraten aus RLP}\version{03:56, 20. Jun. 2010}

\subsubsection{Verbesserung der Zusammenarbeit zwischen Grundschulen und weiterführenden Schulen}
\abstimmung
Wir wollen die Zusammenarbeit zwischen Grundschulen und weiterführenden Schulen verbessern, um Schülern gerade den Übergang von der 4. zur 5. Klasse zu erleichtern. Den Grundschulen wird es so ermöglicht die Schüler noch besser auf die Anforderungen der weiterführenden Schulen vorzubereiten. Auch die weiterführenden Schulen profitieren davon, wenn sie sehen welche Fähigkeiten und Arbeitstechniken Schüler in der Grundschule schon lernen und somit in der 5. Klasse mitbringen.
 
\newpage
\subsection*{Förderschulen}
\wahlprogramm{Förderschulen}
\antrag{unbekannt}\version{03:56, 20. Jun. 2010}

\subsubsection{Die Zukunft der Förderschulen (Variante 1)}
\abstimmung
In diesem Zusammenhang muss sich gleichermaßen über die Zukunft der Förderschulen und zudem verstärkt über bessere Integrationsmodelle Gedanken gemacht werden. Auch dieses Thema möchten wir nach pragmatischen Gesichtspunkten angehen und zusammen mit wissenschaftlichen Experten ein zukunftsfähiges Modell für diesen Bereich entwickeln.

\subsubsection{Die Zukunft der Förderschulen (Variante 2)}
\abstimmung
In Rheinland-Pfalz ist für lernbehinderte, körperbehinderte oder sonstige Kinder mit Förderbedarf das Risiko einer Förderschuleinstufung und der daraus folgenden Ausgrenzung aus dem Regelschulbetrieb im internationalen Vergleich besonders hoch. Der gemeinsame Unterricht von Kindern mit und ohne Behinderung wirkt sich, wie internationale Studien beweisen, auf den Lernerfolg beider Gruppen positiv aus. Deshalb wollen wir das hierzulande betriebene Modell der Förderschule soweit möglich verlassen und eine Schule für alle ermöglichen. Dies erfordert unter anderem bauliche Maßnahmen zum barrierefreien Zugang an Schulen.
 
\subsection*{Weiterführende Schulen}
\wahlprogramm{Die zukünftige Struktur unseres Bildungssystems}\label{wp:bildung:struktur}
\antrag{Piraten aus RLP}\konkurrenz{wp:bildung:eingliedrig}\version{03:56, 20. Jun. 2010}

\subsubsection{Die zukünftige Struktur unseres Bildungssystems}
\abstimmung
Die zukünftige Struktur unseres Bildungssystems soll sich an dem Wohl unserer Kinder ausrichten und nicht an einer ideologischen Diskussion. Wir möchten deshalb die verstärkte Erforschung von verschiedenen Schultypen und Schulformen fördern und die Diskussion ganz pragmatisch im Sinne zukünftiger Generationen gestalten. Es soll sich letztendlich für ein Modell entschieden werden, welches aus wissenschaftlich fundierter Sicht für unsere Kinder die besten Bildungs- und Zukunftschancen bringt.
 
\newpage
\wahlprogramm{Die zukünftige Struktur unseres Bildungssystems}\label{wp:bildung:eingliedrig}
\antrag{Piraten aus RLP}\konkurrenz{wp:bildung:struktur}\version{03:56, 20. Jun. 2010}

\subsubsection{Eingliedriges Schulsystem (Vorschlag 1)}
\abstimmung
In den letzten Jahren wurden zahlreiche Schwächen im deutschen Bildungssystem aufgedeckt. Die PISA-Studien haben gezeigt, dass es im deutschen Bildungssystem deutliche Schwächen gibt. In Deutschland hängt Bildung immer noch stark von dem sozialen Status der Eltern ab.

Die Chancen auf einen guten Schulabschluss, dürfen nicht von dem sozialen Status der Eltern abhängen. Dies stellt nicht nur eine Benachteiligung sozial schwacher Kinder da, Deutschland kann es sich auch nicht leisten nur einen Teil seiner Kinder gut auszubilden. Das Schulsystem muss dementsprechend umgebaut werden, dass die Schüler individuell gefördert werden und dass ihnen während ihrer gesamten Schullaufbahn alle Wege offen stehen.

Wir wollen die Reform, die mit der Realschule-Plus begonnen wurde, zu ihrem logischen Ende bringen. Die Piratenpartei befürwortet deshalb den Ausbau von Gesamtschulen in Rheinland-Pfalz. Langfristig soll an jeder Schule in Rheinland-Pfalz jeder Schulabschluss erreicht werden können. Dadurch soll vermieden werden, dass Schüler schon in der fünften Klasse eine Schullaufbahn einschlagen, die auf einen bestimmten Abschluss hinausläuft. Wie sich die Schulen intern organisieren, wollen wir den Schulen überlassen.

\subsubsection{Eingliedriges Schulsystem (Vorschlag 2)}
\abstimmung
Nicht nur die PISA-Studien haben gezeigt, dass es im deutschen Bildungssystem deutliche Schwächen gibt. In Deutschland hängt Bildung immer noch stark von dem sozialen Status der Eltern ab.

Die Chancen auf einen guten Schulabschluss dürfen nicht von dem sozialen Status der Eltern abhängen. Dies stellt nicht nur eine Benachteiligung sozial schwacher Kinder da, Deutschland kann es sich auch nicht leisten nur einen Teil seiner Kinder gut auszubilden. Das Schulsystem muss dementsprechend umgebaut werden, dass die Schüler individuell gefördert werden und ihnen während ihrer gesamten Schullaufbahn alle Wege offen stehen. Wir wollen die Reform, die mit der Realschule-Plus begonnen wurde, zu ihrem logischen Ende bringen.

Die Piratenpartei befürwortet deshalb den Ausbau von Gesamtschulen in Rheinland-Pfalz. Langfristig soll an jeder weiterführenden Schule in Rheinland-Pfalz jeder Schulabschluss erreicht werden können. So kann vermieden werden, dass Schüler schon in der fünften Klasse eine Schullaufbahn einschlagen, die auf einen bestimmten Abschluss hinausläuft. Die interne Organisation soll den jeweiligen Schulen überlassen werden.

\wahlprogramm{Die zukünftige Struktur unseres Bildungssystems}
\antrag{Piraten aus RLP}\version{03:56, 20. Jun. 2010}

\subsubsection{Unterstützung der Schulen durch nicht lehrendes Personal}
\abstimmung
Schulen müssen nach eigenem Ermessen auch mit ausreichend nicht-lehrendem Personal und finanziellen Ressourcen ausgestattet werden. Dazu zählen beispielsweise technische Assistenten, Sozialarbeiter und Personal für administrative Angelegenheiten.
 
\wahlprogramm{Die zukünftige Struktur unseres Bildungssystems}
\antrag{Piraten aus RLP}\version{03:56, 20. Jun. 2010}

\subsubsection{Kleinere Klassen (Vorschlag 1)}
\abstimmung
Um eine individuelle Förderung zu ermöglichen wollen wir schnellstmöglich den Klassenteiler an Rheinland-Pfälzischen Schulen auf 20 absenken und die Schulen finanziell besser ausstatten.

\subsubsection{Kleinere Klassen (Vorschlag 2)}
\abstimmung
Um eine individuelle Förderung zu ermöglichen wollen wir schnellstmöglich den Klassenteiler an Rheinland-Pfälzischen Schulen auf 15 absenken und die Schulen finanziell besser ausstatten.

\subsubsection{Kleinere Klassen (Vorschlag 3)}
\abstimmung
Um eine individuelle Förderung zu ermöglichen wollen wir schnellstmöglich den Klassenteiler an Rheinland-Pfälzischen Schulen kurzfristig auf 20 absenken und die Schulen finanziell besser ausstatten. Langfristig soll die größe eines Klassenverbandes maximal 15 Schüler betragen.
 
\wahlprogramm{Benotung und Bewertungskriterien}
\antrag{Piraten aus RLP}\version{03:56, 20. Jun. 2010}

\subsubsection{Benotung und Bewertungskriterien}
\abstimmung 
Die Aussagekraft einer Note außerhalb der Rahmenbedingungen, in der sie erhoben wurde, ist sehr gering. Eine Bewertung der Leistung kann nur als Orientierungshilfe für Schüler, Eltern und Lehrer innerhalb der Schullaufbahn dienen. Um diesen Zweck zu erfüllen, sollte die Bewertung von Schülern differenzierter als durch Noten erfolgen. Dazu gibt es zahlreiche Ansätze, die in der täglichen Praxis stärker umgesetzt werden müssen. Insbesondere sind detailliert aufgeschlüsselte fachliche Bewertungen wünschenswert. Kopfnoten lehnen die Piraten grundsätzlich ab.
 
\wahlprogramm{Zusätzliche Förderung und Betreuung in den Klassen 5 und 6 durch Doppelbesetzung}
\antrag{Piraten aus RLP}\version{03:56, 20. Jun. 2010}

\subsubsection{Zusätzliche Förderung und Betreuung in den Klassen 5 und 6 durch Doppelbesetzung}
\abstimmung 
Inbesondere die Schüle der Orientierungsstufe benötigen zusätzliche Förderung und Betreuung, um sich schnellstmöglich auf das neue Umfeld, die Arbeitsmethoden und das Leistungsniveau der weiterführenden Schule einzustellen. Gerade in den Bereichen Deutsch und Mathematik weisen die Schüler oft einen sehr großen Unterschied in ihrer Entwicklung und bei ihrem bisherigen Lernfortschritt auf. Ein einzelner Lehrer kann die große Anzahl der Schüler nicht gleichermaßen fördern. Wir fordern deshalb in den Klassen 5 und 6 jeweils eine Doppelbesetzung der Unterrichtsstunden.

\subsubsection{Ergänzung}
\abstimmung 
Die zweite Lehrkraft könnte hierbei auch ein Student in Ausbildung sein.
 
\wahlprogramm{Einbeziehung von Fachleuten in den Schulunterricht}
\antrag{Piraten aus RLP}\version{03:56, 20. Jun. 2010}

\subsubsection{Einbeziehung von Fachleuten in den Schulunterricht}
\abstimmung 
In stärkerem Maße als bisher sollen Fachleute auch in anderen Schularten als in den Berufsschulen in den Schulunterricht einbezogen werden – nicht nur für Gastvorträge, sondern als quereinsteigende Fachleute mit pädagogischer Eignung und entsprechender Zusatzausbildung. Bei Auswahl und Fortbildung dieser Experten ist darauf zu achten, dass der Unterricht in der Schule weltanschaulich neutral gehalten werden muss.

\newpage
\wahlprogramm{Reduzierung von Leistungsdruck und Stress}\label{wp:bildung:stress1}
\antrag{Piraten aus RLP}\konkurrenz{wp:bildung:stress2}\version{03:56, 20. Jun. 2010}

\subsubsection{Reduzierung von Leistungsdruck und Stress}
\abstimmung 
Gerade junge Menschen benötigen Zeit zur Entwicklung. Sie sollten ihre Kindheit und Jugend aktiv erleben, mit Freunden etwas unternehmen, Vereinsaktivitäten betreiben können und vieles mehr. Es ist nicht zweckmäßig, dass die Schüler dabei einem forlaufenden Leistungsdruck und Stress durch die Schule ausgesetzt sind. Wir fordern deshalb die Struktur und Inhalte der Bildungseinrichtungen in der Art zu überarbeiten, dass der forlaufende Leistungsdruck und Stress durch die Schule reduziert wird.

\subsubsection{Eine Schulzeit bis zum Abitur von mind. 12 1/2 Jahren}
\abstimmung 
Die Verkürzung der Schulzeit bei fast unverändertem Lehrplaninhalt ist unzweckmäßig. Menschen durchlaufen ihre Schulzeit dadurch möglicherweise schneller, doch persönliche Reife und das Erwerben von Lebenserfahrung benötigen dennoch ihre Zeit. Wir fordern deshalb eine Schulzeit bis zum Abitur von mind. 12 1/2 Jahren.

\subsubsection{Kein G8}
\abstimmung 
Das sogenannte G8-Gymnasium lehnen wir ab.
 
\wahlprogramm{Reduzierung von Leistungsdruck und Stress}\label{wp:bildung:stress2}
\antrag{KV Trier/Trier-Saarburg}\konkurrenz{wp:bildung:stress1}\version{03:56, 20. Jun. 2010}

\subsubsection{Leistungsdruck und Schulstress verringern}
\abstimmung 
Überfüllte Lehrpläne und Lernstandserhebungen sind hohe Stressfaktoren und setzen die Schüler unnötig unter Druck. Die Bildungspläne müssen angepasst werden, besonders der Bildungsplan des Gymnasiums an die zwölfjährige Schullaufbahn. Statt Lernstandserhebungen wie PISA oder VERA, die nur den Wissensstand messen, sollen langfristige Evaluationsverfahren eingesetzt werden, die auch die Selbstreflexion der Schüler einbeziehen und somit die Lernprozesse unterstützen.
 
\wahlprogramm{Politische Bildung}
\antrag{Piraten aus RLP}\version{03:56, 20. Jun. 2010}

\subsubsection{Ausweitung der Politischen Bildung (Vorschlag 1)}
\abstimmung 
Demokratische und politische Bildung ist ein elementarer Bestandteil, um heranwachsende Menschen auf das Leben und die Beteiligung in unserer freiheitlichen, demokratischen Gesellschaft vorzubereiten. Nur wenn sie die Demokratie umfassend begreifen, können sie sich an ihr aktiv beteiligen und sie dadurch erhalten. Wir fordern deshalb den landesweiten Ausbau der politischen Bildung

\subsubsection{Ausweitung der politischen Bildung (Vorschlag 2)}
\abstimmung 
Um eine Demokratie umfassend mitzugestalten und vor allem kontrollieren zu können, benötigen Menschen umfassende demokratische Handlungskompetenzen. Wir fordern deshalb eine Ausweitung der demokratischen und gleichermaßen der politischen Bildung und dementsprechend auch mehr Zeitkontingente für den Sozialkundeunterricht im schulischen Alltag.

\subsubsection{Ausweitung des Sozialkundeunterrichts (Vorschlag 1)}
\abstimmung 
Dem Sozialkundeunterricht steht an rheinland-pfälzischen Schulen ein sehr geringes Stundenkontingent zu (an Gymnasien bspw. zwei Unterrichtsstunden pro Woche in Klasse 9 und eine Unterrichtsstunde pro Woche in Klasse 10). Dies ist eindeutig zu wenig, um den Schülern angemessene soziale, politische und wirtschaftliche Kompetenzen und Fähigkeiten zu vermitteln. Wir fordern, unabhängig von der Schulform, einen Sozialkundeunterricht ab Klasse 7 mit einem Stundenkontigent von mindestens drei Stunden pro Woche.

\subsubsection{Ausweitung des Sozialkundeunterrichts (Vorschlag 2)}
\abstimmung 
Um als mündiger Bürger an der demokratischen Willensbildung mitzuwirken wird ein Grundverständnis unseres politischen Systems benötigt. Aus diesem Grund fordern wir eine verbesserte politische Bildung und einen Ausbau des Sozialkundeunterrichts.
 
\newpage
\wahlprogramm{Mehr Zeit für Projekte}
\antrag{Piraten aus RLP}\version{03:56, 20. Jun. 2010}

\subsubsection{Mehr Zeit für Projekte}
\abstimmung 
Wir alle wissen, am besten lernt man, indem man viel selbst ausprobiert und übt. Handlungsorientierung sollte dementsprechend in der Schule im Vordergrund stehen. Wir fordern mehr Zeit für Projekte an Schulen und vor allem auch mehr Zeit für die Zusammenarbeit zwischen mehreren Schulen. Die Schüler erhalten so die Möglichkeit sich mit Themen intensiver auseinanderzusetzen und selbst aktiv zu werden.
 
\wahlprogramm{Mehr Zeit für Projekte}
\antrag{Piraten aus RLP}\version{03:56, 20. Jun. 2010}

\subsubsection{Flächendeckende Einführung von Informatik im Leistungskursangebot der Oberstufe}
\abstimmung
Wir fordern die flächendeckende Einführung von Informatik im Leistungskursangebot an allen rheinland-pfälzischen Schulen mit Oberstufe.
 
\wahlprogramm{Mehr Zeit für Projekte}
\antrag{Piraten aus RLP}\version{03:56, 20. Jun. 2010}

\subsubsection{Lernen fürs Leben}
\abstimmung
Viele Aspekte des alltäglichen Lebens werden in der Schule nicht aufgegriffen. Wir fordern, dass die Aspekte Ernährung, Gesundheit, Medienkompetenz, Verbraucherkompetenz und Kritisches Denken in Projektgruppen mit praktischer Ausrichtung die Schüler auf ein mündiges, selbstbestimmtes und informiertes Leben vorbereiten. Wichtig hierbei ist, dass Schüler aktiv involviert werden und praktisch arbeiten.

\subsubsection{Ernährung, Bewegung, Gesundheit}
\abstimmung
Wir setzen uns dafür ein, dass die Themen Gesundheit, Ernährung und Bewegung unter aktuellen wissenschaftlichen Erkenntnissen und in ausreichendem Maß an Schulen gelehrt werden. Erklärtes Ziel ist es, Schülern eine ausgewogene Lebensweise zu vermitteln. Dies kann gefördert werden, indem theoretische Überlegungen praktisch angewandt werden.

\subsubsection{Ernährung}
\abstimmung
Durch gemeinsames Kochen und Essen, bei gleichzeitiger Erläuterung der theoretischen Hintergründe, werden die Schüler zu einer ausgewogenen Ernähung angeregt.

\subsubsection{Bewegung}
\abstimmung
Der Spass an der Bewegung sollte gefördert werden. Statt des üblichen Rahmenlehrplans, sollten Sportarten einzeln angeboten werden. Ob sich ein Schüler letztendlich für Leichtathletik, Teamsport oder Kraftsport entscheidet soll seine persönliche Entscheidung sein.

\subsubsection{Gesundheit}
\abstimmung
Die Schüler sollen über die Bereiche Sexualität, Gewalt und Suchtprävention ausgiebig aufgeklärt werden.
 
\wahlprogramm{Kritisches Denken}
\antrag{Piraten aus RLP}\version{03:56, 20. Jun. 2010}

\subsubsection{Kritisches Denken, Medienkompetenz und Umgang mit Verbraucherrechten}
\abstimmung
Der Umgang mit Medien und das kritische Hinterfragen von aktuellen Begebenheiten ist eine wichtige Kernkompetenz des Lebens. Die Komplexität des heutigen Informations- Dienstleistungs-, Medien- und Produktangebots erfordert oft die kritische Auseinandersetzung mit sozialen, wirtschaftlichen, gesellschaftlichen und rechtlichen Aspekten.

\subsubsection{(Ergänzung) Projektgruppen}
\abstimmung
1. In Projektgruppen sollen daher praktische Erfahrungen zu folgenden Bereichen gesammelt werden:

2. Informationsbeschaffung,-Selektion und -Diskussion 3. Mediengestaltung, Medienkompetenz 4. Datenschutz und verantwortlicher Umgang mit Daten 5. Auseinandersetzung mit Verträgen und Verbindlichkeiten 6. Haushaltsplanung, Finanzierung, Umgang mit Geld

7. Diskussion von Nachrichten, Religion und politischem Tagesgeschehen

\subsubsection{(Ergänzung) Einrichtung eines neuen Unterrichtsfachs}
\abstimmung
Wir fordern deshalb, dass für diese Erfordernisse ein neues Unterrichtsfach eingerichtet wird. Das Vermitteln dieser Fähigkeiten kann in einem Frontalunterricht jedoch nicht funktionieren. Deshalb müssen sie im Rahmen eines offenen Unterrichtes vermittelt werden, der aktive Teilhabe und praktisches Arbeiten fordert und fördert.

\wahlprogramm{Erweitertes Angebot an Fremdsprachen}
\antrag{Piraten aus RLP}\version{03:56, 20. Jun. 2010}

\subsubsection{Erweitertes Angebot an Fremdsprachen}
\abstimmung
Derzeit werden Synergieeffekte, die sich beim Lernen bestimmter Sprachkombinationen ergeben nicht sinnvoll genutzt. Dies liegt vorallem an dem stark eingeschränkten Angebot an Sprachen, das in der Regel derzeit nur Latein/Französich/Englisch umfasst. Als zweite Fremdsprache würde sich zum Beispiel Spanisch eignen da es hier große Überschneidungen mit dem Französichen und auch dem Latein gibt. Wir kämpfen daher für ein größeres Angebot von Sprachkursen an Schulen.Unterrichtsgeschehen integriert.
 
\wahlprogramm{Bundeswehr}
\antrag{Thomas Heinen}\zusatz{wp:bildung:planspiele}\version{03:56, 20. Jun. 2010}

\subsubsection{Keine Bundeswehr an Schulen}
\abstimmung
Wir sehen die Entsendung von Jugendoffizieren der Bundeswehr für Lehrzwecke in Klassenzimmer und zur Aus- bzw. Weiterbildung von Lehrkräften sehr kritisch. Klassenzimmer sollen nicht zu Rekrutierungsbüros werden.

\subsubsection{ }
\abstimmung
Von der Bundeswehr ausgebildete Referendare, einseitiges Unterrichtsmaterial, Bundeswehrbesuche und von Soldaten gestaltete Unterrichtseinheiten mit Abiturprüfungsinhalten dienen der Manipulation und Rekrutierung, nicht der Erziehung zur eigenständigen Auseinandersetzung mit der Problematik.

\subsubsection{Kooperation des Landes mit Bundeswehr auflösen}
\abstimmung
Die Kooperationsvereinbarung des Landes RLP mit der Bundeswehr zum Einsatz von Jugendoffizieren im Unterricht an rheinland-pfälzischen Schulen lehnen wir ab und fordern deren Aufkündigung. Einseitige Information und Bundeswehrplanspiele haben im Unterricht nichts verloren. Wir fordern einen ausgewogenen Unterricht und die kontroverse Darstellung und Diskussion von Themen, die in der Öffentlichkeit umstritten erscheinen. Die Bundeswehr darf an Schulen nur informieren, wenn gleichzeitig auch Kritiker zu Wort kommen.
 
\wahlprogramm{Planspiele}\label{wp:bildung:planspiele}
\antrag{unbekannt}\zusatz{wp:bildung:bundeswehr}\version{03:56, 20. Jun. 2010}

\subsubsection{Mehr staatliche geförderte Planspiele im Unterricht}
\abstimmung
Planspiele sind, gerade im politischen Unterricht, ein hervoragend geeignetes Mittel, um Schülern selbst komplexe Sachverhalte spielend und handlungsorientiert näher zu bringen und sie für Demokratie und Politik zu begeistern. Wir fordern deshalb mehr Zeit und mehr Angebote für Planspiele im Unterricht. Die Weiterentwicklung und Durchführung von Planspielen ist allerdings oft auch mit starken finanziellen Kosten verbunden, diese sollen vom Staat getragen werden, damit alle Schüler davon partizipieren können.
 
\newpage
\subsection*{Hochschulen}
\wahlprogramm{Zugang zu Bildung}\label{wp:bildung:zugang}
\antrag{Niemand13}\konkurrenz{wp:bildung:beschraenkung}\version{03:56, 20. Jun. 2010}

\subsubsection{Zugang zu Bildung verbessern!}
\abstimmung
Der freie Zugang zu Information, Bildung, Ausbildung und Weiterbildung ist für die Gesellschaft und eine starke Demokratie dringend notwendig und eine der wichtigsten Ressourcen und Investitionen in die Zukunft. Er ist daher im Interesse aller und es ist vordergründige staatliche Aufgabe eine gute und moderne Bildungsinfrastruktur zu finanzieren und jederman frei und kostenlos zur Verfügung zu stellen. Der Zugang zur Hochschule ist aktuell entgegen aller Lippenbekenntnisse stark eingerschränkt und für viele Menschen unmöglich. Viele Menschen können ihr Recht auf ein Studium nicht wahrnehmen, viele müssen ihr Studium vorzeitig abbrechen. Bewerber werden durch hohe NC-Hürden daran gehindert, überhaupt erst ein Studium zu beginnen. Gründe verschiedenster Art, wie Kindererziehung, soziales Engagement, Studiengangwechsel, Selbstfinanzierung und/oder familiäre, bzw. persönliche Schwierigkeiten erschweren die Durchführung des Studiums in Regelstudienzeit. Unserer Auffassung nach ist viel zu wenig Lehrpersonal vorhanden, um allen Studieninteressierten die Möglichkeit zur Aufnahme eines von ihnen gewünschten Studiums zu geben, oder auch nur den schon Studierenden gute Lernbedingung zu bieten und deren individuelle Betreuung zu ermöglichen. Die Konsequenz ist aktuell der Ausschluss Interessierter von einem Studium ihrer Wahl und überfüllte Veranstaltungen.

\subsubsection{Bildungsbarrieren abbauen!}
\abstimmung
Arbeitsaufwand und Anzahl Studierender pro Veranstaltung sind zu hoch, so dass Dozierende sich zwischen Vernachlässigung der Lehre und damit der Verpflichtung gegenüber den Studierenden oder Vernachlässigung der Forschung und damit der eigenen wisschenschaftlichen Karriere entscheiden müssen. Wir fordern die Gewährleistung des in der Verfassung verbrieften Rechts auf Bildung für alle Menschen und wollen die Hochschulen so ausstatten, dass dies uneingeschränkt wahrgenommen werden kann. Körperliche, soziale und finanzielle Beeinträchtigungen dürfen kein Hindernis für die Zulassung zu einem Studium und dessen erfolgreicher Durchführung und Beendigung sein. Eine ausreichende Finanzierung und Ausstattung der Hochschulen wollen wir sicherstellen. Die deutliche Erhöhung des BAFöG-Satzes sehen wir als dringend notwendig an und messen ihre hohe Priotität bei. Wir fordern die Abschaffung von Zulassungsbeschränkungen für alle Studiengänge. Mit einem an der Anzahl der Studieninteressierten orientierten Ausbau von Studienplätzen wollen wir jegliche Zulassungsbeschränkung obsolet machen. Die Wahl des Studienganges muss auf Grund des Interesses und nicht auf Grund von hohen NC-Hürden getroffen werden.
 
\wahlprogramm{Gegen Zulassungsbeschränkungen}\label{wp:bildung:beschraenkung}
\antrag{Piraten aus RLP}\konkurrenz{wp:bildung:zugang}\version{03:56, 20. Jun. 2010}

\subsubsection{Gegen Zulassungsbeschränkungen}
\abstimmung
Angehende Studenten werden durch Zulassungsbeschränkungen in ihrer Freiheit eingeschränkt. Wir fordern deshalb langfristig einen weiteren Ausbau der Universitäten, so dass Zulassungsbeschränkungen abgeschafft werden können.
 
\wahlprogramm{Keine Studiengebühren}\label{wp:bildung:gebuehren1}
\antrag{Piraten aus RLP}\konkurrenz{wp:bildung:gebuehren2}\version{03:56, 20. Jun. 2010}

\subsubsection{Keine Studiengebühren}
\abstimmung
Die Steigerung der Qualität und den Ausbau der Universitäten wollen wir dabei nicht durch Studiengebühren finanzieren. Die Piratenpartei lehnt Studiengebühren für das Erststudium generell ab.
 

\wahlprogramm{Abschaffung Studiengebühren}\label{wp:bildung:gebuehren2}
\antrag{Thomas Heinen}\konkurrenz{wp:bildung:gebuehren1}\version{03:56, 20. Jun. 2010}

\subsubsection{Abschaffung Studiengebühren}
\abstimmung
Jeder Mensch hat das Recht auf die Teilhabe an der Gesellschaft, auf Bildung und kulturelle Betätigung. Studiengebühren und andere finanzielle Zusatzbelastungen im Studium halten Menschen aber vom Studieren ab. Wir fordern daher die Abschaffung der Studiengebühren und weiterer finanzieller Zusatzbelastungen für Studierende wie Verwaltungsgebühren, um barriere- und kostenfreie Bildung für alle zu realisieren. Das Land muss dafür Sorge tragen, dass den Universitäten und studentischen Organisationen die finanziellen Ausfälle ersetzt werden.
 
\wahlprogramm{Bologna-Reform}
\antrag{Niemand13}\version{03:56, 20. Jun. 2010}

\subsubsection{Bologna-Reform reformieren!}
\abstimmung
Wir fordern ein freies und selbstbestimmtes Studium ohne bürokratische Hürden, ohne stetigen Leistungsdruck und starren vorgegebenen Stundenplan, wie sie heute Studierenden- Alltag sind. Durch hohen Leistungsdruck, Dauerüberprüfung und eine rigorose Modularisierung bleibt kein Freiraum mehr für individuelle Schwerpunktsetzung. Wir wollen die Regelstudienzeit der Bachelorgänge prüfen und die Prüfungslast mit dem Ziel der Reduzierung evaluieren. Den permanten Prüfungsdruck sowie den Einfluss von Einzelleistungen auf die Gesamtnote wollen wir herabsetzen. Wir wollen eine Ausweitung der Kombinationsmöglichkeiten der Fächer untereinander, so dass eine breit gefächerte, freie Bildung möglich wird. Dabei müssen auch die Fächer gleichwertig berücksichtigt werden, die abseits des jeweilig üblichen Fächerkanons liegen oder aus fachbezogenen Studiengängen stammen. Um die durch den Bachelor zu erzielende Erleichterung von Auslandsaufenthalten zu erreichen, müssen zukünftig auch sämtliche, bei Auslandsaufenthalten in den eigenen Fächern erbrachten Leistungen, anerkannt werden.

\subsubsection{Vielfalt bewahren!}
\abstimmung
Kleine und ohnehin schon untervertretene Studienfächer wollen wir am Leben erhalten: Lehre und Forschung in solchen Fächern darf nicht aus mangelnder Popularität eingestellt werden! Wir fordern die Umsetzung der eigentlichen Ziele, die die Bologna-Reform mit ihrem aktuellen Konzept für Bachelor und Master verfehlt hat: Die Schaffung einfach verständlicher und gut vergleichbarer Abschlüsse, die Erhöhung der internationalen Mobilität und die Reduzierung der Zahl der StudienabbrecherInnen durch ein verkürztes, überschaubares Studium. Wir fordern einen massiven Ausbau der Master-Studienplätze! Derzeit ist nur einem Bruchteil der BachelorabsolventInnen ein Platz sicher. Dies führt zu neuen Bildungshürden und die Abschlussnote wird den persönlichen Fähigkeiten vorangestellt. Jedem Interessenten muss ein Masterstudium ermöglicht werden! Zulassungsquoten lehnen wir ab.
 
\subsubsection{Familienfreundlichere Hochschulen}
\antrag{Piraten aus RLP}\version{03:56, 20. Jun. 2010}

\subsubsection{Familienfreundlichere Hochschulen}
\abstimmung
Hochschulen sollen familienfreundlicher gestaltet werden. Dies betrifft sowohl die Arbeit in Forschung, Lehre und Verwaltung als auch das Studium. Eine akademische Karriere muss parallel zur Kindererziehung möglich sein. Hierzu sollen (gerade auch für Professoren, Doktoranden und den wissenschaftlichen Nachwuchs) verstärkt Teilzeitstellen angeboten werden. Gleichzeitig muss die Kinderbetreuung an Hochschulen ausgebaut werden, so dass für alle Kinder von Studenten oder Angestellten der Universität Betreuungsplätze zur Verfügung stehen.
 
\subsubsection{Höhere Qualität der Bildung}
\antrag{Piraten aus RLP}\version{03:56, 20. Jun. 2010}

\subsubsection{Höhere Qualität der Bildung an Universitäten}
\abstimmung
Aufgrund von Überfüllung und starker Unterfinanzierung sind an den Universitäten erhebliche qualitative Mängel in der Ausbildung der Studenten entstanden. Die Qualität des Studiums sinkt durch Überfüllung und eine schlechte Betreuung. Aus den gleichen Gründen wird es zunehmend schwerer, das Studium schnell zu beenden. Wir fordern eine bessere finanzielle Ausstattung der Universitäten, die Einstellung zusätzlicher Dozenten und den Ausbau der Universitäten, so dass genügend Platz für die Studierenden zur Verfügung steht.

\newpage
\wahlprogramm{Freies und selbstbestimmtes Studieren}
\antrag{Piraten aus RLP}\version{03:56, 20. Jun. 2010}

\subsubsection{Freies und selbstbestimmtes Studieren}
\abstimmung
Auch an den Universitäten in Rheinland-Pfalz gibt es massive Probleme, dies haben die zahlreichen Studentenproteste gezeigt. Das Studium wurde durch die Bachelor- und Masterstudiengänge zunehmend verschult, die Studenten werden immer mehr eingeengt. Die Piratenpartei fordert deshalb eine schnelle Reform der Bologna-Reformen, damit wieder ein individuelles und selbstbestimmtes Studieren ermöglicht wird. Als erste Maßnahme sollen dabei auch die entsprechenden Studienordnungen für die Umsetzung der Bologna-Refom im Land Rheinland-Pfalz überprüft und mit Blick auf ein freies und selbstbestimmtes Studieren überarbeitet werden.
 
\subsection*{Lehrer / Lehrerausbildung}
\wahlprogramm{Praxiserfahrung}
\antrag{Piraten aus RLP}\version{03:56, 20. Jun. 2010}

\subsubsection{Praxiserfahrung für Lehramtsstudenten ab dem 1. Semester}
\abstimmung
Wir setzen uns für eine Erhöhung der frühzeitigen praktischen Tätigkeiten der Lehramtsstudierenden ein.
 
\wahlprogramm{Duales Studium für angehende Lehrkräfte}
\antrag{Piraten aus RLP}\version{03:56, 20. Jun. 2010}

\subsubsection{Duales Studium für angehende Lehrkräfte}
\abstimmung
Wir möchten, dass das Lehramtsstudium in ein duales Studium, ähnlich einer Berufsakademie, umgewandelt wird. Die Studierenden erhalten während ihrer Ausbildung schon ein Gehalt und werden regelmäßig ins Unterrichtsgeschehen integriert.

\subsubsection{Ergänzung}
\abstimmung
Hierdurch lassen sich Engpässe im Personal hervorragend ausgleichen.
 
\wahlprogramm{Neutralität, Unabhängigkeit und Gerechtigkeit der Lehre}\label{wp:bildung:neutralitaet}
\antrag{Pirat aus RLP}\konkurrenz{wp:bildung:beamten}\version{03:56, 20. Jun. 2010}

\subsubsection{Neutralität, Unabhängigkeit und Gerechtigkeit der Lehre sichern, Beamtenstatus bewahren}
\abstimmung
Neutralität, Unabhängig und Gerechtigkeit sind für ein demokratisches Bildungssystem von elementarer Bedeutung. Nur ein Lehrer, der nicht politischem, gesellschaftlichem und wirtschaftlichem Druck ausgesetzt ist, kann Inhalte objektiv und kritisch vermitteln und so bei Schülern die kritischen Denkprozesse anstoßen, welche in einer Demokratie so dringend benötigt werden.

\subsubsection{Modul2}
\abstimmung
Sollte der Beamtenstatus abgeschafft werden, währe im Bildungsbereich befristeten Arbeitsverträgen und einer Entlassung, bspw. über die Ferien, Tür und Tor geöffnet. Die dringend benötigte Motivation der Lehrkräfte für ihre gesellschaftlich elementare Aufgabe wird durch solche prekären Arbeitsverhältnisse stark reduziert. In den Ferien soll ein Lehrer die Möglichkeit haben sich auf das neue Unterrichtsjahr vorzubereiten. Eine drohende Arbeitslosigkeit mit bürokratischen Maßnahmen des Arbeitsamtes ist in dieser Zeit absolut unakzeptabel.

\subsubsection{Modul3}
\abstimmung
Wir fordern deshalb den Erhalt und die Fortführung des Beamtenstatus.

\wahlprogramm{Anreize für Lehrer}
\antrag{unbekannt}\version{03:56, 20. Jun. 2010}

\subsubsection{Anreize für Lehrer schaffen, Demotivation und Burn-Out rechtzeitig verhindern}
\abstimmung
Die stark überwiegende Mehrheit der Lehrer im Land setzt sich für einen guten Unterricht und ihre Schüler ein. Gerade aber diese stark engagierten Lehrkräfte sind früher oder später durch Frustration von Demotivation und Burn-Out-Syndrom betroffen. Zudem sind die Aufstiegschancen in der Beamtenlaufbahn als Lehrer sehr begrenzt. Nicht der Beamtenstatus, sondern fehlende Anreize für engagierte Lehrkräfte und vor allem die fehlende Ausbildung im Umgang mit Frustration und der zunehmenden Belastung im Schulalltag sind hier das Problem. Dieses löst man nicht, indem man den Beamtenstatus abschafft und engagierten Lehrkräften zusätzlich die berufliche Sicherheit entzieht. Wir fordern stattdessen die Beamtenlaufbahn mit mehr Anreizen und Aufstiegschancen zu gestalten und vor allem mehr Hilfen für engagierte Lehrer anzubieten, um Symptome eines Born-Outs rechtzeitiger zu erkennen und zu vermeiden.

\wahlprogramm{Beamtenstatus abschaffen}\label{wp:bildung:beamten}
\antrag{KV Trier/Trier-Saarburg}\version{03:56, 20. Jun. 2010}
\begin{itemize}
\item \konkurrenz{wp:bildung:neutralitaet}
\item \konkurrenz{wp:bildung:prekaer}
\end{itemize} 

\subsubsection{Beamtenstatus abschaffen}
\abstimmung
Wir setzen uns dafür ein, den Beamtenstatus im Bildungsbereich abzuschaffen und auf gleichberechtigte und faire Arbeitsbedingungen für alle Lehrenden im Schul- und Hochschulbereich hinzuwirken.
 

\wahlprogramm{Prekäre Beschäftigungssituationen}\label{wp:bildung:prekaer}
\antrag{unbekannt}\konkurrenz{wp:bildung:beamten}\version{03:56, 20. Jun. 2010}

\subsubsection{Prekäre Beschäftigungssituationen im Bildungssektor verhindern!}
\abstimmung
Durch die Abschaffung des Beamtenstatus darf im Bildungssektor prekären Beschäftigungsverhältnissen nicht Tür und Tor geöffnet werden. Prekäre Beschäftigungsverhältnisse senken die Motivation und sorgen für Verunsicherung bei den Angestellten. Sie sind für die Erfüllung der hohen Erwartungen, welche an Bildungseinrichtungen gestellt werden, nicht förderlich. Engagierten Lehrern muss eine langfristige Perspektive geboten werden. Entlassungen über die Ferien aus Kostengründen und stark befristete Verträge (Ausnahme: Vertretungslehrkräfte) schließen wird definitiv aus. Was für Lehrer gilt, muss auch für Erzieher gelten. Auch ihnen soll eine langfristige Perspektive geboten werden.
 
\wahlprogramm{Zweiklassengesellschaft}
\antrag{KV Trier/Trier-Saarburg}\version{03:56, 20. Jun. 2010}

\subsubsection{Gegen eine Zweiklassengesellschaft im Lehrer- und Dozentenbereich}
\abstimmung
Von der Öffentlichkeit weitgehend unbemerkt schleicht sich eine Zweiteilung im Bereich der Bildungsvermittler ein: Auf der einen Seite stehen gut abgesicherte Beamte auf Lebenszeit, auf der anderen Seite billige Honorarkräfte, die in den Schulen große Teile des Nachmittagsunterrichts und der Betreuung übernehmen beziehungsweise die an den Hochschulen als Lehrbeauftragte in vielen Bereichen dafür sorgen, dass überhaupt noch ein ausreichendes Lehr- und Betreuungsangebot vorhanden ist. Die Piratenpartei Rheinland-Pfalz wird sich dafür einsetzen, neue unbefristete Hochschulstellen vor allem im Bereich der wissenschaftlichen Mitarbeiter einzurichten. Bestehende Lehraufträge an Schulen und Hochschulen wollen wir angemessener als bisher vergüten und befristete in unbefristete Arbeitsverträge umwandeln.
 
\subsection*{Erwachsenenbildung}
\wahlprogramm{Erwachsenenbildung}
\antrag{Pirat aus RLP}\version{03:56, 20. Jun. 2010}

\subsubsection{Erwachsenenbildung}
\abstimmung
Die Erwachsenenbildung ist ein weites Feld. Sie reicht von Alphabetisierungskursen und Sprachkursen im Rahmen der Integration von Zuwanderern über die betriebliche Fortbildung und privatwirtschaftliche Qualifizierung bis hin zu einem Zweit- oder Drittstudium an einer Hochschule. Die Landesregierung sollte den Aufbau von frei zugänglichem Lehr- und Unterrichtsmaterialien in diesen Bereichen finanziell fördern, um den Zugang zu Bildung auch für Erwachsene zu vereinfachen.

\subsubsection{Ausbau des Volkshochschulangebots}
\abstimmung
Die Piratenpartei regt daher an das System der Volkshochschulen durch den Ausbau zertifizierter Fortbildungsmöglichkeiten zu stärken. Dies kann parallel zu den existierenden privatwirtschaftlich geführten Bildungsunternehmen und -initiativen vonstatten gehen. Dazu sollen die Volkshochschulen durch die Einführung von Summerschools, Kursen und Curricula in Kooperation mit den Berufsakademien, Fachhochschulen und Universitäten noch effizienter als bisher in unsere Bildungslandschaft integriert werden. Hierzu ist die Bereitstellung von Online-Werkzeugen, die ein orts- und zeitunabhängiges Lernen fördern und ermöglichen, unerlässlich. Angeregt wird daher die staatlich finanzierte beziehungsweise staatlich geförderte Bereitstellung von Lernplattformen zum integrierten Lernen als flankierende Maßnahme.

\subsubsection{ }
\abstimmung
Wir wollen ein integratives Konzept „Lebenslanges Lernen“ aufbauen, das Volkshochschulen mit Schulen, Fachhochschulen, Berufsschulen, Universitäten und andere Bildungseinrichtungen zu einem Verbund der Erwachsenenbildung effektiv zusammenführt.

\newpage
\subsection*{Pluralismus vs. Extremismusprävention}
\wahlprogramm{Förderung von Projekten zur Meinungsvielfalt}\label{wp:bildung:projekte}
\antrag{Pirat aus RLP}\konkurrenz{wp:bildung:demokratie}\version{03:56, 20. Jun. 2010}

\subsubsection{Förderung von Projekten zur Meinungsvielfalt}
\abstimmung
Meinungsvielfalt stellt eine Grundlage unserer Demokratie dar. Mit so genannten "Anti-Extremismus-Projekten" und "Anti-Extremismus-Prävention" wird versucht, Meinungsvielfalt zu unterdrücken und gesellschaftlich radikale Positionen zu diskreditieren. Wir setzen uns daher gegen pauschale "Extremismusprävention" ein an deren Ende zwangsläufig eine uniforme Gesellschaft stehen muss. Wir PIRATEN stehen für Meinungsvielfalt und demokratische Diskussion auch radikaler und innovativer Positionen.
 
\wahlprogramm{Demokratie schützen}\label{wp:bildung:demokratie}
\antrag{Pirat aus RLP}\konkurrenz{wp:bildung:projekte}\version{03:56, 20. Jun. 2010}

\subsubsection{Demokratie schützen - Förderung der Extremismusprävention}
\abstimmung
Die verschiedenen Formen des Extremismus stellen eine fortlaufende Gefahr für unsere freiheitlich demokratische Grundordnung dar. Die Menschenrechte, allen voran die unantastbare Würde des Menschen, sind die essentielle Grundlage für unsere demokratische Gesellschaft. Offene Denkmuster wie Rassismus, Fremdenfeindlichkeit oder auch Antisemitismus können und dürfen wir nicht dulden. Wo die würde des Menschen (bspw. durch rassitischen Hass) angegriffen wird, sind der Meinungsfreiheit deutliche Grenzen gesetzt. Wir setzen uns deshalb für die verstärkte Förderung der Extremismusprävention (insbesondere der Rechtsextremismusprävention) ein. Gerade ehrenamtlichen Projekten soll hierbei die bestmögliche, insbesondere finanzielle, Unterstützung zu Teil werden.
 
\subsection*{Nachwort}
\wahlprogramm{Nachwort}
\antrag{Niemand13}\version{03:56, 20. Jun. 2010}
\subsubsection{Nachwort - freie Bildung ist Sauerstoff für die Demokratie}
\abstimmung
Die Piratenpartei Rheinland-Pfalz will die obigen Forderungen auf allen Ebenen konsequent vertreten und umsetzen und so ein freies Lernen sowie Bildungsgerechtigkeit und Chancengleichheit etablieren und soziale Ungleichheit beseitigen. Wir sehen Bildung als Schlüsselfaktor zur gesellschaftlichen Teilhabe in der Informationsgesellschaft und als Grundlage für Frieden und Demokratie.
