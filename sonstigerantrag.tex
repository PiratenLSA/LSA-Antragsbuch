\section{Sonstiger Antrag}
\sonstigerantrag{Umbenennung des Landesverbandes}
\antrag{Stephan Schurig}
\lqfb{http://lqfb.piraten-lsa.de/lsa/initiative/show/231.html}{-}{-}{-}

\paragraph{Antragstext}:

Der {\Gu}Landesverband Piratenpartei Sachsen-Anhalt{\Go} möge in {\Gu}Landespflaster Piratenpartei Sachsen-Anhalt{\Go} umbenannt werden.

\begruendung{Verbände müssen unter hohem Druck angelegt werden, während Pflaster nahezu von allein kleben. Der Aufwand eines Verbandswechsels ist sehr viel höher, als eines Pflasterwechsels und letzterer benötigt weniger Resourcen.

Inspiriert durch \url{https://twitter.com/beapirate/statuses/224490108374614020}}

% -----

\sonstigerantrag{LiquidFeedback-Support für Anträge}
\antrag{Stephan Schurig}
\lqfb{http://lqfb.piraten-lsa.de/lsa/initiative/show/236.html}{10}{1}{1}

\paragraph{Antragstext}:

Der Landesverband möge es einrichten, dass Mitglieder, die (noch) nicht über einen Liquid-Feedback(LQFB)-Zugang verfügen, oder diesen nicht nutzen können (technische oder Verständnisbarrieren), eine Möglichkeit bekommen, Anträge beim Vorstand für das LQFB einzureichen. Der Vorstand oder von ihm beauftragte Personen sollen diese dann stellvertretend einstellen können, insofern sie nicht gegen die Grundsätze der Piratenpartei Deutschland verstoßen.

Weiterhin wäre es denkbar, dass Nicht-Mitglieder Anträge über ein Formular einreichen können und diese dann von einem Mitglied/den beauftragten Personen/dem Vorstand des Landesverbandes übernommen und ins LQFB eingestellt werden können.

\begruendung{Momentan gibt es ziemlich hohe Hürden für Menschen, die nicht technikaffin sind. Anstatt sich mühsam durch LQFB arbeiten zu müssen, sollte es weitere einfache Möglichkeiten der Beteiligung geben. Ansprechpartner bzw. ein Support (z.B. eine Servicegruppe auf Landesebene) wäre dafür sinnvoll, um solche Anträge zu übernehmen und für andere Mitglieder einzustellen.

Anträge von Nicht-Mitgliedern könnten damit ebenso einfließen, insofern sie von einem Mitglied übernommen werden. Tut dies kein Mitglied, passiert nichts weiter.}

% -----

\sonstigerantrag{Transparenzstandards für Vorstandsmitglieder und Mandatsträger}
\antrag{Christoph Giesel}
\lqfb{http://lqfb.piraten-lsa.de/lsa/initiative/show/248.html}{13}{1}{1}

\paragraph{Antragstext}:

Der Landesparteitag möge folgende Tranzparenzstandards als Empfehlung für alle derzeit und zukünftig amtierenden Vorstandsmitglieder und Mandatsträger der Piratenpartei Sachsen-Anhalt sowie dessen Gliederungen beschließen:

\einruecken{Vorstandsmitglieder sämtlicher Gliederungsebenen sowie Abgeordnete/Mandatsträger der Piratenpartei Deutschland bzw. ihrer Fraktionen in Volksvertretungen verpflichten sich an geeigneter Stelle im Internet mindestens folgende Informationen über ihre Tätigkeit zu veröffentlichen und aktuell zu halten: 

\textbf{Bezüge}

\begin{itemize}
\item Bezüge, die sich auf Grund des Amts/Mandats ergeben
\item Nebeneinkünfte, Höhe, sowie durch welche Tätigkeit (nicht bei Vorstandsmitgliedern, oder bei kommunalen Mandatsträgern, die lediglich Aufwandsentschädigung erhalten)
\item Ausstattung, die aufgrund des Amts/Mandats bezahlt wird
\item Sonderzahlungen, die sich aus dem Amts/Mandat ergeben (z.B. Reisekostenerstattungen, Bezuschussung der Krankenkasse usw.) 
\end{itemize}

\textit{Bei Ausstattungen, Sonderzahlungen und Nebeneinkünften einmaliger Natur gilt eine Bagatellgrenze von \EUR{500} (pro Monat). Hierbei reicht die Angabe eines kumulierten Überschlagswerts (z.B. Büromaterial für \EUR{100}).}

\textbf{Parlamentarische/Politische Arbeit und Lobby}

\begin{itemize}
\item Treffen mit Lobbyisten und Interessenvertretern, hier insbesondere
\begin{itemize}
\item Datum
\item Personen
\item Organisation sowie
\item Thema des Gesprächs (ggf. mit inhaltlichem Überblick des Gespräches) 
\end{itemize}
\item Gesellschaftliche Anlässe, Empfänge und Politische Abende, an denen man aufgrund seines Amts/Mandats teilgenommen hat
\item Parlamentarische/Politische Initiativen, an deren Ausarbeitung sie beteiligt waren
\end{itemize}
}

\begruendung{Die Piratenpartei fordert Transparenz. Wir können nur etwas fordern, was wir es selbst vorleben.

Dieser Antrag unterscheidet bei Nebeneinkünften zwischen Amts- und Mandatsträger. Dies wurde vor allem deshalb gemacht, weil unsere Partei basisdemokratisch sein soll und somit die Vorstände weniger zu sagen haben - sie sind mehr oder weniger nur für die Organisation und Administration zuständig. Da sie selber nur soviel Einfluss auf die Programmentwicklung wie jeder andere Pirat haben, bringt eine Beeinflussung bzw. Bestechung dieser wenig etwas. Sofern Vorstände auf Grund des Amts Geld von Firmen erhalten, ist dies schon mit dem Punkt "Bezüge, die sich auf Grund des Amts ergeben" abgedeckt. Außerdem muss man es sich bewusst machen, dass Vorstände nur ehrenamtlich tätig sind. Daher sehe ich es nicht für notwendig an, dass unnötig in die Persönlichkeitsrechte der Vorstände eingegriffen werden soll.

Zudem wurde eine Bagatellgrenze angeregt, die in ein wenig veränderter Form im Antrag eingeflossen ist. Kleinbeiträge können kumuliert und deren Wert überschlagen werden, um den Amts- bzw. Mandatsträger Zeit zu sparen.}

% -----

\sonstigerantrag{GTFO 1337 - Sportdynamitfischen}
\antrag{Alexander Zinser}

\paragraph{Antragstext}:

Die Piratenpartei Sachsen-Anhalt erkennt an, dass Sportdynamitfischen unter Verwendung thermonuklearer Sprengsätze nicht zur Vorbereitung von Fischfilet geeignet ist.

\begruendung{Inspiriert durch die \href{http://wiki.piratenpartei.de/AG\_Sportdynamitfischen}{AG Sportdynamitfischen} muss dies einfach mal gesagt werden.}

% -----

\sonstigerantrag{Audioprotokoll des LPT}
\antrag{Stephan Schurig}

\paragraph{Antragstext}:

Der Landesparteitag möge beschließen, ein Audioprotokoll zu erstellen. Dies soll alle Diskussionen auch nachträglich nachvollziehbar machen sowie die Schrift-Protokollierenden entlasten.

Inwieweit die Risiken durch Privatfu einzuschätzen sind, möge der Landesparteitag abwägen. Ggf. können nachträglich private Informationen aus der Aufnahme gelöscht werden, insofern sie in die Privatsphäre Anderer eingreift.

\begruendung{Weil das gut ist. Oder wenigstens sein könnte.}

% -----