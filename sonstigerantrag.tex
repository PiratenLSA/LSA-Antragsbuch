\section{Sonstiger Antrag}
\sonstigerantrag{Ablehnung von gegenderte Sprache in offiziellen Texten}\label{sa:gender1}
\antrag{Christoph Giesel}\\
\version{20:57, 12. Apr. 2012}
\begin{itemize}
\item \konkurrenz{sa:gender2}
\item \konkurrenz{sa:gender3}
\end{itemize}

\paragraph{Antragstext}:

Der Landesparteitag möge beschließen:

\einruecken{Der Landesverband erkennt an, dass es in der Deutschen Sprache eine Diskrepanz zwischen Sexus und Genus gibt. Mit der {\Gu}Geschlechtergerechten Sprache{\Go} wird der Versuch unternommen, dieses Problem zu lösen. Dennoch konnte bisher keine Lösung gefunden werden, die allen Ansprüchen gerecht wird. Bis eine einfachere und gerechte Lösung gefunden ist, soll zur Vereinheitlichung bei öffentlich Texten keine {\Gu}sprachliche Gleichstellung der Geschlechter{\Go} genutzt werden. Dies beinhaltet beispielsweise Doppelformen (Studentinnen und Studenten), Binnen-I (StudentInnen) oder auch Gender Gap (Student\_innen). 

Zu öffentlichen Texten zählen unter anderem Satzungen, Pressemitteillungen, Webseiten Artikel, offizielle Briefe bzw. Antworten, Flyern und Plakate.}

\begruendung{Die {\Gu}Geschlechtergerechte Sprache{\Go} wird heftig diskutiert. Dennoch möchte ich, dass wir Stellung dazu beziehen und eine einheitliche Sprache bei öffentlichen Texten verwenden. Dies soll nicht frauenfeindlich oder ähnliches sein, sondern es wird von einem {\Gu}generischen Maskulinum{\Go} ausgegangen.}

% -----

\sonstigerantrag{Sprachliche Gleichstellung}\label{sa:gender2}
\antrag{Karl}\\
\version{20:57, 12. Apr. 2012}
\begin{itemize}
\item \konkurrenz{sa:gender1}
\item \konkurrenz{sa:gender3}
\end{itemize}

\paragraph{Antragstext}:

Der Landesparteitag möge beschließen:

\einruecken{Der Landesverband Sachsen-Anhalt empfiehlt allen seinen Mitgliedern sich bei offiziellen Texte z.B. mithilfe des sogenannten Gender Gaps (Student\_innen), um eine sprachliche Gleichstellung aller Menschen jedweden gefühlten oder biologischen Geschlechts zu bemühen.}

\begruendung{Ich erkenne an, dass das Gendern von Texten keine optimale Lösung ist, jedoch ist es aus meiner Sicht der aktuellen sprachlichen Situation auf jeden Fall vorzuziehen.

Der Konkurrenzantrag geht mit seiner Forderung vom \href{https://de.wikipedia.org/wiki/Generisches_Maskulinum}{generischen Maskulinum} aus, d.h. von der Annahme, dass beim Gebrauch von der männlichen grammatischen Form bei allgemeinen Begrifflichkeiten die weibliche Form natürlicher Weise mit eingeschlossen sei. Sprache ist jedoch immer eine Abbildung realer Zustände und so ist das bewusste Weglassen der weiblichen Form (die es nun mal im Deutschen gibt) stets ein unterschwelliges Statement dafür, dass andere Geschlechter in unserer Partei nicht willkommen sind. Ich empfehle zu diesem Thema auch noch \href{http://www.scilogs.de/wblogs/blog/sprachlog/sprachstruktur/2011-12-14/frauen-natuerlich-ausgenommen}{diesen} Blogeintrag.}

% -----

\sonstigerantrag{Keine bindende geschlechtersprachlichen Regelung}\label{sa:gender3}
\antrag{Stephan Schurig}\\
\version{18:32, 13. Apr. 2012}
\begin{itemize}
\item \konkurrenz{sa:gender1}
\item \konkurrenz{sa:gender2}
\end{itemize}

\paragraph{Antragstext}:

Der Landesparteitag möge beschließen:

\einruecken{Die (Nicht-)Anwendung und Wahl einer geschlechterspezifische Sprache in öffentlichen Texten soll den einzelnen Untergliederungen, Arbeitsgruppen, Crews, Piraten etc. selbst überlassen werden. Eine bindene Regelung für den gesamten LV ist abzulehnen, da diese in keinem anderen Bezug auf Sprache und Diskriminierung existieren bzw. gefordert werden.

Zu öffentlichen Texten zählen unter anderem Satzungen, Pressemitteillungen, Webseiten Artikel, offizielle Briefe bzw. Antworten, Flyern und Plakate.}

\begruendung{Wenn auch unüblich, zitiere ich hier einmal, Lena Rohrbach aus der Mailingliste des Kegelklubs:

{\Gu}Mein wichtigstes Gegenargument wäre eines, das gar nicht auf die (in der Partei umstrittenen) Vorteile einer Sprache, die die weibliche u.a. Formen nennt, abzielt, sondern auf Freiheit und Pluralismus: Dass es denjenigen, die einen Text schreiben, obliegen sollte, wie sie sich ausdrücken, gerade weil die Form des (nicht-)genderns sehr viel mit den eigenen Positionen und der eigenen Identität zu tun hat. Wir arbeiten alle ehrenamtlich und das mindeste ist, denen, die eine Arbeit machen, dabei nicht auch noch ihren Ausdruck unnötig und von außen vorzuschreiben. Das ist einer Partei, die auf Freiheit und die Rechte des Individuums setzt, nicht würdig. Wer möchte, dass offizielle Texte anders aussehen, soll in die entsprechende Arbeitsgruppe gehen und die Text selbst - und anders - schreiben. Für Vorschriften von außen braucht es schon verdammt gute Gründe, z.B., dass eine Aussage sonst rassistisch wäre.{\Go}}

% ------


% -----

\sonstigerantrag{Laufendes Parteiprogramm}\label{sa:parteiprogramm1}
\antrag{MAoAm}\\
\version{16:42, 14. Apr. 2012}
\begin{itemize}
\item \konkurrenz{sa:parteiprogramm2}
\end{itemize}

\paragraph{Antragstext}:

Der Landesparteitag möge beschließen, ein laufendes Parteiprogramm zu eröffnen.

\begruendung{mündlich}

% -----

\sonstigerantrag{Eröffnung des Wahlprogramms 2016}\label{sa:parteiprogramm2}
\antrag{Karl}
\begin{itemize}
\item \konkurrenz{sa:parteiprogramm1}
\end{itemize}

\paragraph{Antragstext}:

Der Landesparteitag möge beschließen, das Landesprogramm 2016 zu eröffnen.

% -----

\sonstigerantrag{Vorstandbeschluss 2012-04-09-006 validieren}\label{sa:vorstandsbeschluss1}
\antrag{Kevin Oelze}
\begin{itemize}
\item \konkurrenz{sa:vorstandsbeschluss2}
\end{itemize}

\paragraph{Antragstext}:

Der Landesparteitag möge den Vorstandsbeschluss 2012-04-09-006 bestätigen oder ablehnen:

\einruecken{Wir sind eine globale Gemeinschaft von Menschen, unabhängig von Alter, Geschlecht und Abstammung sowie gesellschaftlicher Stellung, offen für alle mit neuen Ideen.

Wer jedoch mit Ideen von Rassismus, Sexismus, Homophobie, Ableismus, Transphobie und anderen Diskriminierungsformen und damit verbundener struktureller und körperlicher Gewalt auf uns zukommt, hat sich vom Dialog verabschiedet und ist jenseits der Akzeptanzgrenze.

Wer es darauf anlegt, das Zusammenleben in dieser Gesellschaft zu zerstören und auf eine alternative Gesellschaft hinarbeitet, deren Grundsätze auf Chauvinismus und Nationalismus beruht, arbeitet gegen die moralischen Grundsätze, die uns als Piraten verbinden.

Die unterzeichnenden Piraten erklären das Vertreten von Rassismus und von der Verharmlosung der historischen und aktuellen faschistischen Gewalt für unvereinbar mit einer Mitgliedschaft.}

% -----

\sonstigerantrag{Vorstandbeschluss 2012-04-09-006 validieren (Alternative)}\label{sa:vorstandsbeschluss2}
\antrag{Kevin Oelze}
\begin{itemize}
\item \konkurrenz{sa:vorstandsbeschluss1}
\end{itemize}

\paragraph{Antragstext}:

Der Landesparteitag möge den folgenden Text beschließen:

\einruecken{Wir sind eine globale Gemeinschaft von Menschen, unabhängig von Alter, Geschlecht und Abstammung sowie gesellschaftlicher Stellung, offen für alle mit neuen Ideen.

Wer jedoch mit Ideen von Rassismus, Sexismus, Homophobie, Ableismus, Transphobie und anderen Diskriminierungsformen und damit verbundener struktureller und körperlicher Gewalt auf uns zukommt, hat sich nicht grunsätzlich vom Dialog verabschiedet, dennoch vertreten wir diese Ansichten keinesfalls.

Wer es darauf anlegt, das Zusammenleben in dieser Gesellschaft zu zerstören und auf eine alternative Gesellschaft hinarbeitet, deren Grundsätze auf Chauvinismus und Nationalismus beruht, arbeitet gegen die moralischen Grundsätze, die uns als Piraten verbinden.

Die unterzeichnenden Piraten erklären das Vertreten von Rassismus und von der Verharmlosung der historischen und aktuellen faschistischen Gewalt für unvereinbar mit einer Mitgliedschaft.}

% -----

\sonstigerantrag{Sonstiges: Bildungssymposion}
\antrag{Dominik {\Gu}Ineluki{\Go} Wondrousch}

\paragraph{Antragstext}:

Der Landesparteitag möge den Vorstand mit der Organisation eines Symposions zum Thema Bildungspolitik beauftragen.
Hierbei ist es wünschenswert, dass dieses Symposion vor dem nächsten Programmparteitag stattfindet.

\begruendung{Bildung ist eine unser Kernkompetenzen, doch die Anträge auf dem LPT 2012-1 haben gezeigt, dass es in diesem Bereich noch erheblichen Diskussionsbedarf gibt.
Um den Meinungsaustausch und die Vernetzung der Mitglieder auch abseits des LQFB zu ermöglichen, stellt ein solches Real-Life-Treffen ein nützliches Tool dar.}

