\section{Sonstiger Antrag}
\sonstigerantrag{Ablehnung von gegenderte Sprache in offiziellen Texten}\label{sa:gender1}
\antrag{Christoph Giesel}\\
\version{20:57, 12. Apr. 2012}
\begin{itemize}
\item \konkurrenz{sa:gender2}
\end{itemize}

\paragraph{Antragstext}:

Der Landesparteitag möge beschließen:

\einruecken{Der Landesverband erkennt an, dass es in der Deutschen Sprache eine Diskrepanz zwischen Sexus und Genus gibt. Mit der {\Gu}Geschlechtergerechten Sprache{\Go} wird der Versuch unternommen, dieses Problem zu lösen. Dennoch konnte bisher keine Lösung gefunden werden, die allen Ansprüchen gerecht wird. Bis eine einfachere und gerechte Lösung gefunden ist, soll zur Vereinheitlichung bei öffentlich Texten keine {\Gu}sprachliche Gleichstellung der Geschlechter{\Go} genutzt werden. Dies beinhaltet beispielsweise Doppelformen (Studentinnen und Studenten), Binnen-I (StudentInnen) oder auch Gender Gap (Student\_innen). 

Zu öffentlichen Texten zählen unter anderem Satzungen, Pressemitteillungen, Webseiten Artikel, offizielle Briefe bzw. Antworten, Flyern und Plakate.}

\begruendung{Die {\Gu}Geschlechtergerechte Sprache{\Go} wird heftig diskutiert. Dennoch möchte ich, dass wir Stellung dazu beziehen und eine einheitliche Sprache bei öffentlichen Texten verwenden. Dies soll nicht frauenfeindlich oder ähnliches sein, sondern es wird von einem {\Gu}generischen Maskulinum{\Go} ausgegangen.}

% -----

\sonstigerantrag{Sprachliche Gleichstellung}\label{sa:gender2}
\antrag{Karl}\\
\version{20:57, 12. Apr. 2012}
\begin{itemize}
\item \konkurrenz{sa:gender1}
\end{itemize}

\paragraph{Antragstext}:

Der Landesparteitag möge beschließen:

\einruecken{Der Landesverband Sachsen-Anhalt empfiehlt allen seinen Mitgliedern sich bei offiziellen Texte z.B. mithilfe des sogenannten Gender Gaps (Student\_innen), um eine sprachliche Gleichstellung aller Menschen jedweden gefühlten oder biologischen Geschlechts zu bemühen.}

\begruendung{Ich erkenne an, dass das Gendern von Texten keine optimale Lösung ist, jedoch ist es aus meiner Sicht der aktuellen sprachlichen Situation auf jeden Fall vorzuziehen.

Der Konkurrenzantrag geht mit seiner Forderung vom \href{https://de.wikipedia.org/wiki/Generisches_Maskulinum}{generischen Maskulinum} aus, d.h. von der Annahme, dass beim Gebrauch von der männlichen grammatischen Form bei allgemeinen Begrifflichkeiten die weibliche Form natürlicher Weise mit eingeschlossen sei. Sprache ist jedoch immer eine Abbildung realer Zustände und so ist das bewusste Weglassen der weiblichen Form (die es nun mal im Deutschen gibt) stets ein unterschwelliges Statement dafür, dass andere Geschlechter in unserer Partei nicht willkommen sind. Ich empfehle zu diesem Thema auch noch \href{http://www.scilogs.de/wblogs/blog/sprachlog/sprachstruktur/2011-12-14/frauen-natuerlich-ausgenommen}{diesen} Blogeintrag.}
