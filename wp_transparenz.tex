\section{Transparenz}
\subsection*{Informationsfreiheit im 21. Jahrhundert - Offene Daten für mündige Bürger!}

\wahlprogramm{Staatliche Daten veröffentlichen}\label{transparenz:daten}
\antrag{Unglow}\version{04:02, 20. Jun. 2010}
\subsubsection{Modul 1}
\abstimmung
Der Zugang zu Wissen und Information ist die Grundlage für unsere freiheitlich-demokratische Informations- und Wissensgesellschaft. Wir PIRATEN setzen uns daher für eine Stärkung der Informationsfreiheit und einen freien und offenen Zugang zu allen staatlichen und staatlich geförderten Informationsbeständen ein.

\subsubsection{Modul 2}
\abstimmung
Sämtliche staatlichen Daten müssen grundsätzlich der Öffentlichkeit und damit jedermann frei zugänglich gemacht werden. Unter staatlichen Daten verstehen wir alle staatlichen und staatlich finanzierten Informationen, ausgenommen personenbezogene Daten und ggf. wenige klar zu definierende und begründende Ausnahmefälle. Diese Ausnahmeregelungen sind möglichst eng und eindeutig zu formulieren und dürfen nicht pauschal ganze Behörden oder Themengebiete ausgrenzen.
 
\wahlprogramm{Demokratie durch Transparenz}
\antrag{Piraten aus RLP}\version{04:02, 20. Jun. 2010}
\subsubsection{Transparenz des Staatswesens und Lobbyismus}
\abstimmung
Die politische Arbeit wird in Deutschland stark von Lobbyinteressen gesteuert. Unternehmensvertreter nehmen unbemerkt Einfluss auf Politiker und arbeiten sogar an Gesetzen mit. Abhängigkeiten zwischen Unternehmen und Politikern müssen aufgedeckt werden. Abgeordnete sollen ihre Nebentätigkeiten und die gegebenenfalls daraus resultierenden Einkünfte veröffentlichen. Abgeordnete der Piratenpartei werden mit gutem Beispiel vorangehen und dies mit dem Einzug ins Parlament offenlegen. Dem Bürger muss klar ersichtlich sein, welche Interessen hinter Gesetzesinitiativen stecken und wer, wie und wann auf den Gesetzgebungsprozess Einfluss genommen hat.

Zu einem transparenten Staat gehören neben den Regelungen zu Lobby- und Nebentätigkeiten von Parlamentariern und Amtsträgern auch die gelebte Verpflichtung, Entscheidungsfindungsprozesse für den Bürger wahrnehmbar und nachvollziehbar öffentlich zu machen, wie auch Verordnungen, Diskussionspapiere und Vertragswerke so zu gestalten, dass diese so kurz wie nötig, so sprechend wie möglich und für den Bürger verständlich gehalten sind. Wir lehnen geheime Ausschüsse ab.

Gleichzeitig müssen die Interessen der Bürger besser vertreten werden. Zudem sollen Nichtregierungsorganisationen gefördert werden, die für die Rechte und Interessen der Bürger eintreten.

\subsubsection{Höchste demokratische Standards für Deutschland}
\abstimmung
Die PIRATEN treten ein, für eine nachvollziehbare und transparente Politik und Verwaltung in Deutschland. Deutschland sollte sich an die höchsten demokratischen Standards halten und innerhalb Europas eine vorbildliche Rolle diesbezüglich anstreben. Deshalb sollten solche Prinzipien wie transparente Staatsführung, schnelle und gerechte Gerichtsverfahren und die Redefreiheit stets beachtet werden. In diesen Tagen und in dieser Zeit ist es wesentlich, den gesetzlichen Schutz der Bürger vor willkürlichen Staatszugriffen weiterhin durchzusetzen.

\subsubsection{Gläsener Staat statt Gläserner Bürger}
\abstimmung
Eine demokratische Gesellschaft braucht einen transparenten Staat und keine gläsernen Bürger. Die Bürger müssen die Möglichkeit haben, sich frei und unbeobachtet zu versammeln, und ihre Meinung ohne Furcht vor staatlicher Überwachung ausdrücken zu können. Um dies in die Informationsgesellschaft zu übertragen, muss das Recht auf anonyme Kommunikation ausgebaut werden. Deswegen muss das Korrespondenzgeheimnis auf digitale Kommunikation ausgeweitet werden.

\subsubsection{Offenlegung von Nebeneinkünften und Nebentätigkeiten von Amts- und Mandatsträgern}
\abstimmung
Mandatsträger und Ausübende politischer Ämter müssen zur Offenlegung sämtlicher Nebeneinkünfte und Nebentätigkeiten verpflichtet sein. Nur wenn der Bürger weiß von wem die genannten Personen bezahlt werden und für wen sie arbeiten kann er sich ein vollständiges Bild über deren Unabhängigkeit oder ggf. deren Abhängigkeit machen. Die Offenlegungspflicht soll auch für unentgeltliche [nicht private) Nebentätigkeiten, wie Ehrenämter in Vereinen und Verbänden gelten.

\subsubsection{Begrenzung von Neben- und Folgetätigkeiten}
\abstimmung
Amtsträger dürfen neben ihrem Amt und wenigstens 2 Jahre nach Beendigung ihrer Amtstätigkeit nicht in Unternehmen, Vereinen oder Verbänden tätig sein, die direkt durch die Amtstätigkeit betroffen sind. Auf Antrag eines Betroffenen kann die jeweilige Situation von einem Ausschuss gesondert untersucht werden, um unverhältnismäßige Eingriffe in die Berufsfreiheit zu vermeiden.
 
\subsection*{Gläserner Staat}
\wahlprogramm{Gläserner Staat}
\antrag{KV Trier/Trier-Saarburg}\zusatz{transparenz:daten}\version{04:02, 20. Jun. 2010}
\subsubsection{Gläserner Staat}
\abstimmung
Der Anspruch der Gesellschaft auf Wissen endet dort, wo die Privatsphäre beginnt. Persönlichkeitsrechte wie die informationelle Selbstbestimmung sind Grundpfeiler für die freiheitlich demokratische Grundordnung unseres Staates. Damit der Bürger seiner Kontrollpflicht dem Staat gegenüber nachkommen kann, muss dieser offen und transparent aufgestellt sein.

Die Demokratie soll gestärkt werden, indem mehr Mitwirkungsmöglichkeiten und Einblicke in die Abläufe gewährt werden. Durch Einsicht in die Staatsgeschäfte können Korruption, Bürokratie und Lobbyismus erkannt werden. Inkompetenzen und Versäumnisse werden schneller aufgedeckt.
 
\subsection*{Informationsfreiheit ist Bürgerrecht!}
\wahlprogramm{Informationsfreiheit / Prinzip der Öffentlichkeit}\label{transparenz:infofreiheit}
\antrag{Unglow}\version{04:02, 20. Jun. 2010}
\subsubsection{Modul 1}
\abstimmung
Die alte Weisheit „Wissen ist Macht“ gilt in der Informationsgesellschaft mehr denn je. Nur wer umfänglich informiert ist, kann fundierte Entscheidungen fällen. Eine umfassende Information von Bürgern und Bürgerinnen ist auch Voraussetzung für politisches Engagement und demokratische Kontrolle der vom Volk legitimierten Macht. Jeder Bürger kann staatliche Angaben selbst überprüfen, aus neuen Blickwinkeln betrachten und neue, vorher unbekannte Zusammenhänge entdecken. Dies führt zu einer Demokratisierung der Informationskanäle und erhöht die Kontrollmöglichkeiten der Zivilgesellschaft gegenüber dem Staat. Gemäß dem Mehr-Augen-Prinzip können Angaben gemeinschaftlich besser überprüft, Entscheidungen hinterfragt und kritisiert werden. Verbesserungsvorschläge können von Allen erarbeitet werden und die besten Lösungen können umgesetzt werden. Dem Missbrauch und der Willkür Einzelner wird vorgebeugt.

\subsubsection{Modul 2}
\abstimmung
Wir PIRATEN wollen daher Parlamente und Behörden und die rechtlichen Grundlagen so umgestalten, dass sie diesem gesamtgesellschaftlichen Anspruch der Informationsfreiheit für alle Bürger Rechnung tragen. Wir setzen uns dafür ein, dass sich der Staat vom Prinzip der Geheimhaltung abkehrt und ein Prinzip der Öffentlichkeit einführt, welches den mündigen Bürger in den Mittelpunkt staatlichen Handelns und Gestaltens stellt. Dies schafft nach der festen Überzeugung der Piratenpartei die unabdingbaren Voraussetzungen für eine moderne Wissensgesellschaft in einer freiheitlichen und demokratischen Ordnung.
 
\wahlprogramm{Informationsfreiheit / Keine Zensur!}
\antrag{Piraten aus RLP}\version{04:02, 20. Jun. 2010}
\subsubsection{Keine Zensur!}
\abstimmung
Die Bestrebungen etablierter Parteien, eine Inhaltsfilterung im Internet zu etablieren, lehnen wir kategorisch ab. Staatliche Kontrolle des Informationsflusses, also Zensur, ist ein Instrument von totalitären Regimen und hat in einer Demokratie nichts verloren. Der Kampf gegen rechtswidrige Angebote im Internet muss jederzeit mit rechtsstaatlichen und transparenten Mitteln geführt werden. Bereits die Etablierung einer Zensurinfrastruktur ist inakzeptabel. Die Beurteilung der Rechtswidrigkeit muss gemäß der in Deutschland geltenden Gewaltenteilung und Zuständigkeit getroffen werden.

\subsubsection{ZugangsErschwerungsGesetz aufheben!}
\abstimmung
Die PIRATEN werden sich dafür stark machen, den Irrweg des ZugangsErschwerungsGesetzes zu beenden und dieses Zensur-Gesetz aufzuheben.

\subsubsection{ZugangsErschwerungsGesetz aufheben!}\footnote{Richtiger Titel?}
\abstimmung
Auch den Jugendmedienschutzstaatsvertrag (JMStV), den die rheinlandpfälzische Landesregierung vorangetrieben hat, lehnen wir kategorisch ab, da er in unseren Augen einen völlig falschen Weg im Jugendschutz beschreitet. Wir fordern Aufklärung und die Vermittlung von Medienkompetenz an Kinder, Jugendliche und Eltern statt einer Zensur von Inhalten im Rundfunk oder Internet.
 
\wahlprogramm{Transparenz}
\antrag{KV Trier/Trier-Saarburg}\zusatz{transparenz:infofreiheit}\version{04:02, 20. Jun. 2010}
\subsubsection{Wissen ist Macht}
\abstimmung
"Wissen ist Macht" wird bislang eher als Legitimation dafür verwendet, Wissen für sich zu behalten, abzuschotten und zu monopolisieren. Eine erfolgreiche Gesellschaft des 21. Jahrhunderts muss den Satz erweitern zu "Wissen ist Macht – wenn es allen gehört". Denn eingesperrtes Wissen ist gesellschaftlich totes Wissen, nutzt zunächst nur dem, der daraus "Kapital" schlägt, wenn überhaupt. Denn noch viel häufiger liegt das Wissen verschlossen in Tresoren, weil es vergessen oder falsch verstanden wird.

Geteiltes Wissen wächst schneller als isoliertes Wissen. Die Wissenschaftsgemeinschaft weiß das schon langeund bewertet den Rang eines Forschers deshalb nach seinen Publikationen und der Häufigkeit, mit der er zitiert wird. Verbraucherschützer, Umweltschutz-Organisationen, Bündnisse für Verkehrsprojekte und viele andere Organisationen und Initiativen, die die Interessen der Menschen vertreten, warten darauf, dass die öffentliche Verwaltung ihre Informationsschätze teilt und nicht versteckt. Die Piratenpartei versteht sich als Vertreter der Wissensgesellschaft.
 
\wahlprogramm{Transparenz}
\antrag{KV Trier/Trier-Saarburg}\version{04:02, 20. Jun. 2010}
\subsubsection{Transparente Information über Großprojekte}
\abstimmung
Bei der Planung und Umsetzung von Großprojekten wie Nürburgring oder Hochmoselübergang sollen frühzeitig alle relevanten Informationen veröffentlicht werden. Daneben sollen die betroffenen Bürger angemessen und frühzeitig beteiligt werden. Bei einer Verlegung in private Rechtsformen muss diese Veröffentlichungspflicht weiterhin gewährleistet sein. Wir wollen eine offenere Kommunikation bei der Planung und Umsetzung von Großprojekten anstoßen.
 
\subsection*{Moderne Verwaltung mit offenen Daten!}
\wahlprogramm{Offene Daten}
\antrag{Unglow}\version{04:02, 20. Jun. 2010}
\subsubsection{Modul 1}
\abstimmung
Staatliche Daten, wie Wetter- und Geodaten, Verkehrs- und Einwohnerstatistiken, müssen allen Bürgern zur Verfügung stehen und dürfen nicht länger großen Teilen der Gesellschaft vorenthalten werden. Die heutige Informationspolitik schließt wertvolle Daten in Aktenschränken oder nicht allgemein verarbeitbaren Dateiformaten ein. Bürger bekommen wichtige Informationen nur auf Nachfrage. Wir wollen das Potential der weltweiten Vernetzung ausschöpfen und werden deshalb offene Schnittstellen zum Abruf dieser Daten für jedermann einführen.

\subsubsection{Modul 2}
\abstimmung
Die modernen Informationstechnologien machen eine proaktive, zeitnahe Veröffentlichung und Verbreitung von staatlichen Informationen in offenen und strukturierten Datenformaten kostengünstig und schnell möglich. Die Piratenpartei tritt dafür ein, dass alle staatlichen Stellen von diesen Möglichkeiten Gebrauch machen, statt der Verbreitung dieser Informationen Steine in den Weg zu legen. Wir wollen durchsetzen, dass Rohdaten in maschinenlesbaren Formaten bereitgestellt werden, die eine schrankenlose Weiterverarbeitung durch Nicht-Regierungsorganisationen, Forschungseinrichtungen und interessierte Bürger zulassen.

\subsubsection{Modul 3}
\abstimmung
Eine Veröffentlichung von Daten in Rohform und der Zugriff über offene Schnittstellen ermöglicht vielfältige Anwendungen. Die Piratenpartei betrachtet daher die Veröffentlichung von staatlichen Informationen in offenen, strukturierten Formaten als ein wesentliches Merkmal eines demokratischen Informationszeitalters. Open-Data- und Semantic-Web-Initiativen, welche für die Veröffentlichung von strukturierten Daten eintreten, wollen wir deshalb explizit fördern. Ebenso wollen wir den Einsatz freier Software in allen Einrichtungen des Landes forcieren. Langfristige Verträge mit Monopolisten lehnen wir ab.
 
\wahlprogramm{Transparenz}
\antrag{KV Trier/Trier-Saarburg}\version{04:02, 20. Jun. 2010}
\subsubsection{Freie und plattformunabhängige Dateiformate für staatliche Veröffentlichungen}
\abstimmung
Offene Formate garantieren, dass Informationen auch langfristig lesbar sind. Diese müssen möglichst in durchsuchbarer Form zur Verfügung gestellt werden.

Der Zugang zu veröffentlichten Informationen darf nicht davon abhängen, welches Computersystem jemand benutzt, ob spezielle Software installiert oder gekauft wurde. Deshalb ist es erforderlich, Veröffentlichungen in einer Form vorzunehmen, die auf offenen standardisierten Formaten basiert.

\subsubsection{Offene Dateiformate in der Verwaltung}
\abstimmung
Wir werden dafür sorgen, dass die Verwaltungen des Landes und der Kommunen vollständig auf offene Dateiformate umsteigen. Dies vereinfacht den Datenaustausch zwischen den Behörden untereinander und mit den Bürgern.

\subsubsection{Freie Software in Behörden und staatlichen Einrichtungen}
\abstimmung
Sicherheit und langfristige Kosteneinsparungen durch Einsatz von freier Software.

Durch die Offenheit des Quellcodes bei freier Software gibt es keine Abhängigkeit von einem bestimmten Softwarehersteller. Dies verbessert die Möglichkeiten für spätere Anpassungen, wenn sich beispielsweise rechtliche Rahmenbedingungen für Behörden ändern. Bei freier Software entfallen außerdem auf lange Sicht große Summen für Lizenzgebühren. Den kurzfristig höheren Kosten für Einarbeitungsaufwand stehen so mittel- und langfristige Einsparungen gegenüber. Wartungsverträge können mit Firmen vor Ort geschlossen werden, was die regionale Wirtschaft fördert.
 
\wahlprogramm{Transparenz}
\antrag{Piraten aus RLP}\version{04:02, 20. Jun. 2010}
\subsubsection{Kooperation mit Microsoft aufkündigen}
Die Verträge der Landesregierung mit dem Software-Monopolisten Microsoft zum Einsatz von Software in Schulen, Hochschulen und Verwaltung sowie Bereich des Jugendmedienschutzes und der "Medienkompetenzförderung" lehnen wir ab und werden wir aufkündigen.
 
\subsection*{Auskunftsanspruch verbessern!}
\wahlprogramm{Auskunftsanspruch}
\antrag{Unglow}\version{04:02, 20. Jun. 2010}
\subsubsection{Modul 1}
\abstimmung
Wir wollen gewährleisten, dass jeder Bürger unabhängig von der Betroffenheit und ohne den Zwang zur Begründung sein Recht durchsetzen kann, auf allen Ebenen der staatlichen Ordnung Einsicht in die Aktenvorgänge und die den jeweiligen Stellen zur Verfügung stehenden Informationen zu nehmen. Dies gilt für schriftliches Aktenmaterial ebenso wie für digitale oder andere Medien.

\subsubsection{Modul 2}
\abstimmung
Ausnahmeregelungen zum Auskunftsanspruch sind eng und eindeutig zu formulieren und dürfen nicht pauschal ganze Behörden oder Verwaltungsgebiete ausnehmen. Für eine breite und effiziente Nutzung der Daten ist die Auskunftsstelle verpflichtet, Zugang in Form einer Akteneinsicht oder einer Materialkopie zu gewähren. Der Zugang soll zeitnah und mit einer klaren und fairen Kostenregelung erfolgen. Verweigerung des Zugangs muss schriftlich begründet werden und kann vom Antragsteller sowie von betroffenen Dritten gerichtlich überprüft werden lassen, wobei dem Gericht zu diesem Zweck voller Zugang durch die öffentliche Stelle gewährt werden muss.

\subsubsection{Modul 3}
\abstimmung
Wir werden alle öffentlichen Stellen verpflichten, regelmäßig sowohl Organisations- und Aufgabenbeschreibungen zu veröffentlichen, einschließlich Übersichten der Arten von Unterlagen, auf die zugegriffen werden kann, als auch einen jährlichen öffentlichen Bericht über die Handhabung des Auskunftsrechts.
 
\subsection*{Korruption erschweren!}
\wahlprogramm{Korruption}
\antrag{Unglow}\version{04:02, 20. Jun. 2010}
\subsubsection{Lobbyismus aufdecken}
\abstimmung
Damit für die Rheinland-Pfälzischen Bürgerinnen und Bürger klar ersichtlich ist, wer die Politik im Land beeinflusst, werden wir ein vollständiges Lobbyistenregister auf Landesebene einführen, in dem alle Verbände und Vertreter aufgeführt werden, die Einfluss auf Gesetzgebungsprozesse oder deren Ausgestaltung durch Verordnungen haben. In den Ministerien dürfen keine Mitarbeiter von Unternehmen dauerhaft ihre Arbeit verrichten. Lediglich in transparenten Anhörungen dürfen diese als Sachverständige angehört werden. Anhörungen zu Gesetzesinitiativen oder anderen Vorhaben der Landesregierung müssen stets öffentlich angekündigt werden und für jeden zugänglich sein. Insbesondere Verbraucherverbände, Bürgerrechts- und Menschenrechtsorganisationen müssen von Anfang an in Gesetzgebungsprozesse eingeweiht werden und Gelegenheit zur Stellungnahme bekommen. Alle Stellungnahmen von Interessenverbänden müssen öffentlich z.B. über das Internet zugänglich gemacht werden.

\subsubsection{Vergaberegister zur Korruptionsbekämpfung}
\abstimmung
Wir wollen ein Vergaberegister schaffen, mit dessen Hilfe bereits auffällig gewordene Firmen künftig von der Vergabe öffentlicher Aufträge ausgeschlossen werden. Diese Informationen sollen nicht nur Behörden zur Verfügung stehen, sondern auch der interessierten Öffentlichkeit. Das Korruptionsbekämpfungsgesetz von Nordrhein-Westfalen kann hier als Vorlage dienen.

\subsubsection{Offenlegung der Nebeneinkünfte von Landtagsabgeordneten}
\abstimmung
Die Höhe und Herkunft aller Einnahmen aus Nebentätigkeiten müssen einzeln und in vollem Umfang veröffentlicht werden. Dazu werden wir ein Modell erarbeiten, das über die Regelungen auf Bundesebene hinausgeht. Das dreistufige System reicht nicht aus, da die höchste Stufe von 7000 Euro nichts darüber aussagt, wie hoch die Nebeneinkünfte tatsächlich ausfallen. Um mögliche Interessenkonflikte erkennen zu können, müssen die zusätzlichen Einkünfte transparent offengelegt werden.
 
\subsection*{Transparenter Haushalt}
\wahlprogramm{Transparenter Haushalt}
\antrag{KV Trier/Trier-Saarburg}\version{04:02, 20. Jun. 2010}
\subsubsection{Transparenter Haushalt}
\abstimmung
Die Transparenz im Haushalt des Landes und bei der Verwendung von sonstigen Landesmitteln muss dringend verbessert werden. Haushaltswahrheit und Haushaltsklarheit sind nicht im erforderlichen Maß gewährleistet.

Wir werden uns dafür einsetzen, dass die Haushalte der überwiegend aus öffentlichen Mitteln finanzierten Stiftungen unter verstärkte parlamentarische Kontrolle gestellt werden.
 
\subsection*{Veröffentlichungsdienst 2.0 - freier Zugang zum Landesrecht!}
\wahlprogramm{Veröffentlichungsdienst 2.0 - freier Zugang zum Landesrecht!}
\antrag{Unglow}\version{04:02, 20. Jun. 2010}
\subsubsection{Modul 1}
\abstimmung
Unwissenheit schützt vor Strafe nicht. Aber sich über geltendes Recht - also Vorschriften, Erlasse, Verordnungen oder Entscheidungen - zu informieren, könnte heute wesentlich einfacher sein.

Wir wollen deshalb eine zentrale Anlaufstelle im Internet umsetzen, die neben Rechtsprechung und Gesetzgebung auch Verordnungen, Umsetzungsrichtlinien, Berichte, Empfehlungen, Analysen, amtliche Bekanntmachungen, Gesetzesentwürfe und sonstige Drucksachen von Land und Kommunen enthält, komplett mit Suchfunktion, Änderungsverfolgung, Querverweisen und Kommentarmöglichkeit.

\subsubsection{Modul 2}
\abstimmung
Das Material wird, sofern nicht ohnehin gemeinfrei, unter eine liberale Lizenz gestellt, die eine (auch kommerzielle) Weiterverwendung der Texte zulässt. Dabei sollen offene, einheitliche Schnittstellen für die automatische Abfrage und frei zugängliche Datenformate genutzt werden.

\subsubsection{Modul 3}
\abstimmung
Von diesem einfachen Zugriff profitieren alle Bürger und Unternehmen. Auch die Arbeit der staatlichen Stellen (Verwaltung, Gerichte, Landtag) wird durch eine einheitliche Plattform für die Veröffentlichung von Dokumenten und Daten erleichtert.
 
\subsection*{Weitere Maßnahmen für Rheinland-Pfalz}
\wahlprogramm{Weitere Maßnahmen}\label{transparenz:weiter}
\antrag{Unglow}\version{04:02, 20. Jun. 2010}
\subsubsection{Modul 1}
\abstimmung
Um die Informationsfreiheit im obigen Rahmen vollumfänglich zu gewährleisten, wollen die Rheinland-Pfälzischen PIRATEN folgende Maßnahmen ergreifen:
\begin{itemize}
\item die Digitalisierung aller staatlichen Unterlagen, die neu erstellt werden
\item Forschungsprojekte zur Digitalisierung alter Unterlagen sowie die Erforschung von Langzeitarchivierungsstrategien
\item den freien Zugang zu allen Gesetzen und Gesetzesentwürfen, bereits in der Entstehungsphase
\item den freien Zugang zu allen Beschlüssen des Landtages und anderer politischer Gremien
\item die komplette Offenlegung des Abstimmungsverhaltens im Rheinland-Pfälzischen Landtag und seinen Ausschüssen
\item die komplette Offenlegung des Abstimmungsverhaltens der Landesregierung im Bundesrat
\item die komplette Offenlegung der Nebeneinkünfte der Landtagsabgeordneten und Minister
\item den freien Zugang zu allen finanziellen Ausgaben der Landesregierung, der Ministerien, des Landtags und seiner Fraktionen
\item den freien Zugang zu allen Messdaten, die staatlichen Institutionen vorliegen (Wetterdaten, Flugverkehrsdaten, Gewässerdaten, Katasterdaten, Luftbilder, u.v.m)
\item den freien Zugang zu allen statistischen Erhebungen, die durch die Verwaltung oder in deren Auftrag vorgenommen werden
\item das Angebot von offenen Schnittstellen zur automatischen Abfrage der bereitgehaltenen Dokumente, Daten und Informationen in standardisierten, offenen Formaten
\item die Einrichtung einer kostenlosen Beratungsstelle, die den Bürgern und Bürgerinnen offene Fragen und komplexe Sachverhalte erläutert
\item die finanzielle Förderung von Open-Data- und Semantic-Web-Initiativen und Forschung in diesem Bereich
\item die Zusammenarbeit mit Rheinland-Pfälzischen Hochschulen zur Digitalisierung, Aufbereitung und Zurverfügungstellung aller Daten in offenen Formaten
\item die Unabhängigkeit des Landesbeauftragten für Informationsfreiheit und eine bessere finanzielle und personelle Ausstattung der Behörde
\item das Angebot aller Ausschreibungen in einem standardisierten, maschinenlesbaren Datenformat
\item die Einführung einer Meldepflicht für alle Behörden bei Datenpannen und ein standardisiertes Verfahren zur Benachrichtigung der Betroffenen
\item die Veröffentlichung aller Verträge der Landesregierung und der Ministerien mit Unternehmen
\item die Einführung eines vollständigen Lobbyistenregisters auf Landesebene
\item eine klare Kennzeichnung, welche Passagen in Gesetzesentwürfen von wem hinzugefügt wurden
\item die umgehende Bekanntmachung von Art und Umfang aller Abhörmaßnahmen, Observationen oder Datenabfragen inklusive der Information von welcher Polizeibehörde oder welchem Geheimdienst diese auf welcher rechtlichen Grundlage durchgeführt werden, sowie die umfassende Information der Betroffenen sofort nach Ende der Maßnahme
\item die ausschließliche Verwendung quelloffener Software durch die Verwaltung
\end{itemize}
 
\wahlprogramm{Ergänzung zu 'Weitere Maßnahmen'}
\antrag{Silberpappel}\zusatz{transparenz:weiter}\version{04:02, 20. Jun. 2010}
\subsubsection{Protokolle von öffentlichen Gemeinderatssitzungen}
\abstimmung
Verpflichtung der Gemeindeverwaltungen zur Veröffentlichung der Protokolle von öffentlichen Gemeinderatssitzungen im Internet.

\subsubsection {Gemeindesatzungen}
\abstimmung
Verpflichtung der Gemeindeverwaltungen zur Veröffentlichung der Satzungen der Gemeinde im Internet
